%%
%% $Id: ami_reference.tex,v 1.3 1999-06-12 16:18:07 rbarve Exp $
%%
\chapter{AMI Programmers' Reference}

\section{AMI Streams}
\label{sec:ami_stream-ref}

An AMI stream\index{streams!AMI} is nothing more than an abstraction
of an ordered set of objects of any one class which is stored in
secondary memory.  In some applications, the elements of a stream
might be nothing more than integers.  In others, they might be objects
of an arbitrarily complex user defined class.  Creating a stream of
objects of class \verb|foo| called \verb|foo_stream| requires nothing
more than the definition

\begin{verbatim}
AMI_STREAM<foo> foo_stream;
\end{verbatim}

\verb|AMI_STREAM| is a macro that resolves to a class
template declared to match the underlying semantics of the target I/O
architecture by using an appropriate BTE\index{block transfer engine}
(see Section~\ref{sec:ref-bte}) and MM 
(see Section~\ref{sec:ref-mm}).

The \verb|AMI_STREAM| class,
templated on the type T of its items,  consists of the following
public member functions: % defined in the file ``ami\_single.h'';
%some important related enumerated types that the user will be
% concerned with have been defined in ``ami\_base.h''.

\begin{itemize}
\item{Constructors and Destructor} An \verb|AMI_STREAM<T>| can be constructed 
with a filename argument
to the constructor, or with the filename and streamtype
arguments to the constructor, or  without any
arguments given to the constructor (in which case a randomly generated
filename and the default streamtype is used)
with the usual semantics. The corresponding constructors 
are:
\begin{verbatim}
     AMI_STREAM(const char *dev_path, const AMI_stream_type st); 
     AMI_STREAM(const AMI_stream_type st); 
     AMI_STREAM();
     AMI_STREAM(const BTE_STREAM<T> &s);
\end{verbatim}
The last constructor constructs an \verb|AMI_STREAM| from an exisiting
\verb| BTE_STREAM|.
%Here,
%\emph{filename} specifies the name of the Unix file ``backing'' the
%AMI\_STREAM and is also the name of the AMI\_STREAM itself. The
%\emph{streamtype} specifies whether the AMI\_STREAM is to be a 
%read-only or read-write or append-only or write stream (See the header file
%``ami\_base.h'' for semantics.) The default  \emph{streamtype} is read-write. 
%The Unix file backing the AMI\_STREAM is opened by the constructor.


\item{Substreams}

It is possible to construct one AMI stream that is actually a
substream of another.  In fact, if one so desires, one can construct
substreams of substreams of subtreams {\em ad infinitum}.  A substream
is defined as a contiguous range of objects from within the ordered
collection of objects that make up the stream.

In order to create a substream, we use a pseudo-constructor called
\verb|new_substream()|, which is declared as follows:
\begin{verbatim}
AMI_err AMI_stream_single<T>::new_substream(AMI_stream_type st,
                                            off_t sub_begin,
                                            off_t sub_end,
                                            AMI_base_stream<T> **sub_stream)
\end{verbatim}
The stream type is one of \verb|AMI_READ_STREAM|,
\verb|AMI_WRITE_STREAM|, or \verb|AMI_READ_WRITE_STREAM|, which define
what operation will be permitted on this substream.  The offsets
\verb|sub_begin| and \verb|sub_end| define where in the stream the new
substream will begin.  Upon completion, \verb|*substream| is set to
point to the newly allocated and created substream.

The reason we do not use a real constructor is to get around the fact
that constructors can not be virtual.  Because \verb|new_substream()|
is not a constructor, but rather a function each particular
implementation of which calls an appropriate constructor, it can be a
pure virtual function in the stream base class, which forces it to be
defined for all actual stream implementations.  This is discussed in
more detail in Sections~\ref{sec:ref-ami} and~\ref{sec:ref-bte}, which
discuss the implementation of the AMI and BTE in more detail.

\item{Reading and Writing} \verb|AMI\_STREAM|s have a notion of \emph{current
item} in the stream of items. The \verb|read_item(T **elt)| function
returns with *elt pointing to the current item of the \verb|AMI_STREAM<T>|
and the  \verb|write_item(T elt)| function 
returns after setting the current item of \verb|AMI_STREAM<T>| to the value 
elt. 

\begin{verbatim}
    AMI_err read_item(T **elt);
    AMI_err write_item(const T &elt);
\end{verbatim}

Since it was deemed useful to read or write many items at a time
from or to an AMI\_STREAM, TPIE provides the 
\verb|read_array(T *mm_space, off_t *len)| and the \\
\verb|write_array(const T *mm_space, off_t len)| to read
(resp. write) *len (resp. len) items into (resp. from) 
array \verb|mm_space| of items of type T from (resp. into) the
current locations of the \verb|AMI_STREAM<T>|.


\item{Seek:}
In order to seek within a stream, the user can:
\begin{verbatim}
    AMI_err seek(off_t offset);
\end{verbatim}
Seeks to the item offset in the stream.


\item{Truncate:}
\begin{verbatim}
    AMI_err truncate(off_t offset);
\end{verbatim}
Truncate/extend the stream to the specified number of items. The file
pointer will be moved to the end of the stream.


\item{Get main\_memory\_usage}
In order to obtain the total number of bytes taken up by a stream, the user can
query using:
\begin{verbatim}
    AMI_err main_memory_usage(size_t *usage,
                              MM_stream_usage usage_type);
\end{verbatim}


\item{Name and Length}
\verb|AMI\_STREAM<T>| member functions \verb|name()| and \verb|stream_len()|
can be used by the user to get the name and length of the
\verb|AMI_STREAM<T>|.
\begin{verbatim}
    AMI_err name(char **stream_name);
\end{verbatim}
Returns the path name of the file backing the stream. The name will be
stored in newly allocated space.
\begin{verbatim}
    off_t stream_len(void);
\end{verbatim}
Returns the current number of items in the stream.



\item{Available Streams}
\begin{verbatim}
    int available_streams(void);    
\end{verbatim}
Returns the number of currently available streams.

\item{Set persistence}
\begin{verbatim}
    void persist(persistence);
\end{verbatim}
Sets the persistence of the stream to one of the following:
\begin{itemize}
\item \verb|PERSIST_DELETE:| Delete the stream from the disk when it is
  destructed.
\item \verb|PERSIST_PERSISTENT:| Do not delete the stream from the disk when
  it is destructed.
\item \verb|PERSIST_READ_ONCE:| Delete each block of data from the disk as
  it is read.
\end{itemize}

By default, all streams are deleted at destruction time
(\verb|PERSIST_DELETE|).  
\end{itemize}

\section{AMI Errors}
\label{sec:ami-errors}

\index{error|(}
AMI entry points typically return error codes of the enumerated type
\verb|AMI_err|.  Member functions of operation management
objects\index{operation management objects} also typically return this
type.  Possible values for error codes include those listed below.  It
is expected that in future releases of TPIE, many of these error codes
will be replaced by exceptions.  Exceptions are not currently used by
TPIE becasue the {\tt g++} compiler does not fully support them.

\subsection{General Error Codes}

\begin{description}
\item[{\tt AMI\_ERROR\_NO\_ERROR}] No error occured.  The call the the
  entry point returned normally.
\item[{\tt AMI\_ERROR\_IO\_ERROR}] A low level I/O error occured.
\item[{\tt AMI\_ERROR\_END\_OF\_STREAM}] An attempt was made to read
  past the end of a stream or write past the end of a substream.
\item[{\tt AMI\_ERROR\_READ\_ONLY}] An attempt was made to write to a
  read-only stream.
\item[{\tt AMI\_ERROR\_OS\_ERROR}]  An unexpected operating system
  error occurred.  Details should appear in the log file if logging is
  enabled.  See Section~\ref{sec:logging}.
\item[{\tt AMI\_ERROR\_BASE\_METHOD}] An attempt was made to call a
  member function of the virtual base class of {\tt AMI\_STREAM}.  This
  indicates a bug in the implementation of AMI streams.
\item[{\tt AMI\_ERROR\_BTE\_ERROR}] An error occurred at the BTE
  level.  
\item[{\tt AMI\_ERROR\_MM\_ERROR}] An error occurred within the memory
  manager.
\item[{\tt AMI\_ERROR\_OBJECT\_INITIALIZATION}] An AMI entry point was
  not able to properly initialize the operation management object that
  was passed to it.  This generally indicates a bug in the operation
  management object's initialization code.
%\item[{\tt AMI\_ERROR\_PERMISSION\_DENIED}]
\item[{\tt AMI\_ERROR\_INSUFFICIENT\_MAIN\_MEMORY}] The MM could not
  make adequate main memory available to complete the requested
  operation.  Many operations adapt themselves to use whatever main
  memory is available, but in some cases, when memory is extremely
  tight, they may not be able to function.
\item[{\tt AMI\_ERROR\_INSUFFICIENT\_AVAILABLE\_STREAMS}]
  The AMI could not allocate enough intermediate streams to perform
  the requested operation.  Certain operating system restrictions
  limit the number of streams that can be created on certain
  platforms.  Only in unusual circumstances, such as when the
  application itself has a very large number of open streams, will
  this error occur. 
\item[{\tt AMI\_ERROR\_ENV\_UNDEFINED}]
  An environment variable necessary to initialize the AMI was not defined.
\item[{\tt AMI\_ERROR\_BIT\_MATRIX\_BOUNDS}]
  A bit matrix larger than the number of bits in an offset into a
  stream was passed to \verb|ami_gp()|.
\item[{\tt AMI\_ERROR\_NOT\_POWER\_OF\_2}]
  The length of a stream on which a bit permutation was to be
  performed is not a power of two.
\item[{\tt AMI\_MATRIX\_BOUNDS}] An attempt was made to perform a
  matrix operation on matrices whose bounds did not match appropriately.
\end{description}


\subsection{Return Values for Scan Management Objects}

More information on the precise semantics of these values appears in
Section~\ref{sec:ref-ami-scan}.
\begin{description}
\item[{\tt AMI\_SCAN\_CONTINUE}] Tells \verb|AMI_scan()| to continue
  to call the \verb|operate()| member function of the scan management
  object with more data.
\item[{\tt AMI\_SCAN\_DONE}] Tells \verb|AMI_scan()| that the scan is
  complete. 
\end{description}

\subsection{Return Values for Merge Management Objects}

More information on the precise semantics of these values appears in
Section~\ref{sec:ref-ami-merge}.
\begin{description}
\item[{\tt AMI\_MERGE\_CONTINUE}] Tells \verb|AMI_merge()| to continue
  to call the \verb|operate()| member function of the scan management
  object with more data.
\item[{\tt AMI\_MERGE\_DONE}] Tells \verb|AMI_merge()| that the scan is
  complete. 
\item[{\tt AMI\_MERGE\_OUTPUT}]  Tells \verb|AMI_merge()| that the
  last call generated output for the output stream.
\item[{\tt AMI\_MERGE\_READ\_MULTIPLE}]  Tells \verb|AMI_merge()| that
  more than one input object was consumed and thus the input flags
  should be consulted.
\end{description}

\index{error|)}

\section{AMI Entry Points}

\index{AMI entry point}\index{entry point!AMI|see{AMI entry point}}

% This section describes all the commonly used AMI entry points.  

\subsection{Scanning}
\label{sec:ref-ami-scan}

\index{scanning|(}

The simplest polymorph of \verb|AMI_scan()| is
\begin{verbatim}
    AMI_err AMI_scan(ST *smo, AMI_STREAM<T> *outstream);
\end{verbatim}
which generates an output stream of objects of type \verb|T|.
\verb|smo| is a pointer to a scan management object of used defined
class \verb|ST|.  \verb|ST| should provide member functions {\tt
  AMI\_err initialize(void)} and {\tt AMI\_err operate(T *out1,
  AMI\_SCAN\_FLAG *sf)}.

Scanning can also read and process an input stream of objects of type
\verb|T|.  The polymorph that does this looks like
\begin{verbatim}
    AMI_err AMI_scan(AMI_STREAM<T> *instream, ST *smo);
\end{verbatim}
Again, \verb|smo| is a pointer to a scan management object of used
defined class \verb|ST|.  \verb|ST| should provide member functions
{\tt AMI\_err initialize(void)} and {\tt AMI\_err operate(const T
  \&in1, AMI\_SCAN\_FLAG *sf)}.

\verb|AMI_scan()| may also read multiple input streams, each
potentially of a different type, and write multiple output streams,
each potentially of a different type.  This is done using the polymorph
\index{scanning!multi-type}
\begin{verbatim}
    AMI_err AMI_scan(AMI_STREAM<T1> *instream1, 
                     AMI_STREAM<T2> *instream2, ..., 
                     ST *smo, 
                     AMI_STREAM<U1> *os1, 
                     AMI_STREAM<U1> *os2, ...);
\end{verbatim}
The scan management object class \verb|ST| should have an operate
method taking the appropriate number and type of operands.  For
example, {\tt AMI\_err operate(const T1 \&in1, const T2 \&in2, ...,
  AMI\_SCAN\_FLAG *sfin, U1 *out1, U2 *out2, ..., AMI\_SCAN\_FLAG
  *sfout)}.  In the current implementation of TPIE there can be
between zero and four inputs and zero and four outputs.


\subsubsection{Scan Management Objects}  
\index{operation management objects!scan|(}

A scan management object class must provide two member functions for
\verb|AMI_scan()| to call.  The first is
\begin{verbatim}
    AMI_err initialize(void);
\end{verbatim}
\verb|initialize()| initializes the object to prepare it for a scan.
This member function is called once by each call to \verb|AMI_scan()|
in order to initialize the scan managment object before any data
processing takes place.  This function should return
\verb|AMI_ERROR_NO_ERROR| if successful, or an appropriate error
otherwise.  See Section~\ref{sec:ami-errors} for a list of error
codes.

Most of the work done is a scan is typically done in by the scan
management objects's member function
\begin{verbatim}
    AMI_err operate([const T1 &in1, ...,  AMI\_SCAN\_FLAG *sfin, ] 
                    [ U1 *out1, ..., AMI\_SCAN\_FLAG *sfout]);
\end{verbatim}

Either one or more input objects or one or more output parameters must
be specified.  These must correspond in number and type to the streams
passed to the polymorph of \verb|AMI_scan()| with which this scan
management object is to be used.

If there are inputs, \verb|sfin| points to an array of flags, one for
each input parameter.  Flags that are set (non-zero) indicate that the
corresponding input parameters are valid.  If the input flags are left
untouched, it is assumed that the input parameters were processed.  If
one or more are cleared (set to zero) then the it is assumed that the
corresponding inputs were not processed and should be presented again
on the next call to \verb|operate()|.  This permits out of step
scanning\index{scanning!out of step}, as illustrated in
Section~\ref{sec:out-of-step}.

The output parameters are used to store objects to be written to the
output streams.  The output flags are set by \verb|operate()| to
indicate which outputs are valid and should be written to the output
streams.

The return value of \verb|operate()| will normally be either
\verb|AMI_SCAN_CONTINUE|, which indicates that the function should be
called again with any taken input parameters replaced by the next
objects from the streams they came from, of \verb|AMI_SCAN_DONE|,
which indicates that the scan is complete and no more input needs to
be processed.  Note that \verb|operate()| may continue to return
\verb|AMI_SCAN_CONTINUE| even when the input flags indicate that their
is no more input to be processed.  This is useful if the scan
management object maintains some internal state that must be written
out after all input has been processed.

There are numerous examples of scan managment objects in
Section~\ref{sec:scanning} as well as in the test applications that
appear in the TPIE distribution.

\index{operation management objects!scan|)}
\index{scanning|)}

\subsection{Merging}
\label{sec:ref-ami-merge}
\index{merging|(}

To merge a number of streams in a single pass, TPIE provides the entry
point.

\begin{verbatim}
AMI_err AMI_single_merge(AMI_STREAM<T> **instreams, arity_t arity,
                         AMI_STREAM<T> *outstream, M *m_obj);
\end{verbatim}

\verb|instreams| is an array of pointers to the input streams, all of
which are streams of objects of type \verb|T|.  \verb|arity| is the
number of input streams.  \verb|outstream| is the output stream.
Finally, \verb|m_obj| points to a merge management object which will
control the merge.  Merge management objects are described below.
 
If the number of streams is large enough that all cannot be properly
buffered in the available main memory, then this function will fail,
returning \verb|AMI_ERR_INSUFFICIENT_MAIN_MEMORY|.  Most programs are
unlikely to need to call \verb|AMI_single_merge()| directly.  

More commonly, programs will wish to partition a stream into
substreams small enough to fit in main memory, operate on each in main
memory, and then merge them together, possibly in several passes if
low memory conditions dictate.  TPIE provides the entry point to do
this, which is
\begin{verbatim}
AMI_err AMI_partition_and_merge(AMI_STREAM<T> *instream,
                                AMI_STREAM<T> *outstream, M *m_obj);
\end{verbatim}
This entry point takes care of all the details of determining how much
main memory is available, how big the initial substreams can be, how
many streams can be merged at a time, and how many levels of merging
must take place.

\subsubsection{Merge Management Objects}  
\index{operation management objects!merge|(} A merge management object
must provide \verb|initialize()| and \verb|operate()| member
functions, whose functions are analogous to their namesakes for scan
management objects.  They should be declared as
\begin{verbatim}
    AMI_err initialize(arity_t arity, const T * const *in,
                       AMI_merge_flag *taken_flags,
                       int &taken_index);
\end{verbatim}
and
\begin{verbatim}
    AMI_err operate(const T * const *in, AMI_merge_flag *taken_flags,
                    int &taken_index, T *out);
\end{verbatim}
The arguments to \verb|initialize()| are \verb|arity|, the number of
input streams in the merge, \verb|in| a pointer to an array of
pointers to input objects, each of which is the first object appearing
in one of the input streams,  \verb|taken_flags| an array of flags
indicating which of the inputs are present (i.e. which of the input
streams is not empty), and a pointer to an output object.

The typical behavior of an \verb|initialize()| member function for a
merge management object is to place all the input objects into a data
structure and then return \verb|AMI_MERGE_READ_MULTIPLE| to indicate
that it took all the inputs which were indicated to be valid by
\verb|taken_flags|.  \verb|initialize| need not process all inputs; it
can turn off any flags in \verb|taken_flags| corresponding to inputs
that should be presented to \verb|operate()|.  Alternatively, it can
set \verb|taken_index| to the index of a single input it processed and
return \verb|AMI_MERGE_CONTINUE|.

The \verb|operate()| member function for a merge management object is
called repeatedly to process input objects.  Typically, this member
function will choose one input object to process, and set
\verb|taken_index| to the index of the pointer to that object in the
input array.  This object is then typically added to a dynamic data
structure maintained by the merge management object.  If output is
generated, for example by removing an object from the dynamic data
structure, \verb|operate()| should return \verb|AMI_MERGE_OUTPUT|,
otherwise, it returns either \verb|AMI_MERGE_CONTINUE| to indicate that
more input should be presented, or \verb|AMI_MERGE_DONE| to indicate
that the merge has completed.  

Alternatively, \verb|operate()| can clear the elements of
\verb|taken_flags| that correspond to inputs it does not currently
wish to process, and then return \verb|AMI_MERGE_READ_MULTIPLE|.  This
is generally undesirable because, if only one input is taken, it is
far slower than using \verb|taken_index| to indicate which input was
taken.  The merge management object must clear all other flags, and
then TPIE must can all the flags to see which inputs were or were not 
processed.

Any merge management object that is used with the entry point
\verb|AMI_partition_and_merge()| needs to provide a member function to
operate on substreams of data that are small enough to fit into main
memory.  This member function is
\begin{verbatim}
    AMI_err main_mem_operate(T* mm_stream, size_t len);
\end{verbatim}
\verb|mm_stream| is a pointer to an array of objects that have
been read into main memory.  \verb|len| is the number of objects in
the array.

Finally, because merge management objects are allowed to maintain data
structures whose size is linear in the number of input streams being
processed, there must be some mechanism to communicate their memory
requirements to TPIE.  This is done through the following two member
functions:
\begin{verbatim}
    size_t space_usage_overhead(void);
    size_t space_usage_per_stream(void);
\end{verbatim}
The former should return an upper bound on the amount of main memory
the object will allocate in addition to the portion that is linear in
the number of streams.  The latter should return the amount of
additional main memory needed per input stream.

There are examples of merge management objects in
Section~\ref{sec:merging} as well as in the test applications that
appear in the TPIE distribution.
\index{operation management objects!merge|)}
\index{merging|)}


\subsubsection{AMI merging without Merge Management Objects}
\label{sec:ref-miami-merge}
By using appropriate \verb|operate()| and \verb|main_mem_operate()| functions, 
merge management objects allow can be used to conveniently program
operations other than standard comparison based merging used while 
sorting that have a merging like flavor. Sometimes, for the purpose of
the standard external merging operation, it may be preferable to forego
the generality provided by merge management objects and use routines
specifically geared towards standard external merging operations, 
in the interest of potential increased efficiency. The programmer can
use the entry points below in such situations.

The first entry point
\begin{verbatim}

 AMI_err MIAMI_single_merge(AMI_STREAM<T> **instreams, 
                            arity_t arity, 
                            AMI_STREAM<T> *outstream)

\end{verbatim}
provides functionality identical to the \verb|AMI_single_merge()|
routine presented above, except that this routine assumes that the
items of class T have well-defined \verb| < |, \verb| > |, etc. operators 
that are to be used to order the merge. In contrast, the second
entry point 
\begin{verbatim}

 AMI_err MIAMI_single_merge_cmp(AMI_STREAM<T> **instreams, 
                                arity_t arity, 
                                AMI_STREAM<T> *outstream,  
                                int (*cmp)(CONST T&, CONST T&))

\end{verbatim}
which provides identical functionality requires a well-defined 
comparison function \verb| cmp()| that takes two items of type T
and returns a negative, zero or positive number depending on whether
or not the first item is less than, equal to or greater than the second
item in the desired merging order. 

The third entry point is more interesting and 
assumes that the items of type T in the streams being
merged include a member field, KEY, with well-defined (possibly by means of
overloading)  \verb| < | and \verb| > | operators at compile time, and
furthermore, that the ordering of the T type items is to based on the
total order of their KEY fields. If these assumptions hold, the
programmer can use the function,

\begin{verbatim}

AMI_err MIAMI_single_merge_Key(AMI_STREAM<T> **instreams, 
                               arity_t arity, 
                               AMI_STREAM<T> *outstream, 
                               int keyoffset, 
                               KEY dummykey)
\end{verbatim}

to merge {\tt arity} streams of type T. Here, {\tt keyoffset} is the
offset with respect to the start of an item of type T of the field KEY
within T. The motivation for such a function is to 
exploit situations in which the KEY field is often much smaller than the
T type item: For instance, if T is a rectangle, each item may be of 
size approximately 40 bytes, whereas if one is interested in
sorting along the X co-ordinate, the KEY may be of size 4 or 8 bytes.
If such information is known to the programmer, the above entry point 
can exploit it by using appropriately-suited simple internal memory merging 
techniques: Often, the small size of the KEY field can improve the running
time of the operation.

The return
values of the above three functions are  similar to those of
\verb|AMI_single_merge()|. 

We also provide three entry points analogous to the \verb|AMI_partition_and_merge()|
function that do not require merge management objects and 
that use the  \verb|MIAMI_single_merge_*()| routines the way 
\verb|AMI_partition_and_merge()| uses the \verb|AMI_single_merge()|: Each
analogue entry point corresponds to one of the above three  
\verb|MIAMI_single_merge_*()| entry points. 


The first function, based on \verb|MIAMI_single_merge_stream()|,
\begin{verbatim}
template<class T>
AMI_err AMI_partition_and_merge_stream(AMI_STREAM<T> *instream,
                                       AMI_STREAM<T> *outstream)
\end{verbatim}
provides the exact same functionality as the
\verb|AMI_partition_and_merge()| function except that it requires 
the type T to have well defined  \verb| < | and \verb| > | operators.

The second function, based on  \verb|MIAMI_single_merge_cmp()|,
\begin{verbatim}
template<class T>

template<class T>               
AMI_err AMI_partition_and_merge_stream_cmp(AMI_STREAM<T> *instream,
                                           AMI_STREAM<T> *outstream,
                                           int (*cmp)(CONST T&, CONST T&)
                                           )
\end{verbatim}
has the same requirements as the \verb|MIAMI_single_merge_cmp()| function.

The third function,   based on  \verb|MIAMI_single_merge_Key()|,
\begin{verbatim}
template<class T, class KEY>
AMI_err AMI_partition_and_merge_Key(AMI_STREAM<T> *instream,
                                   AMI_STREAM<T> *outstream, 
                                              int keyoffset, 
                                               KEY dummykey)
\end{verbatim}
provides similar functionality as \verb|AMI_partition_and_merge()|
except that it is  uses KEY and {\tt keyoffset}
information while merging and sorting, as indicated above.


%The function
%\begin{verbatim}
%template<class T, class KEY>
%AMI_err AMI_replacement_selection_and_merge_Key(AMI_STREAM<T> *instream,
%                                               AMI_STREAM<T> *outstream, 
%                                               int keyoffset,
%                                               KEY dummykey)
%\end{verbatim}
%is the same as \verb|AMI_partition_and_merge_Key()| above, except that
%it uses a fundamentally different run formation algorithm. The 
%\verb|AMI_replacement_selection_and_merge_Key()| function uses
%the replacement selection (Algorithm R in Knuth's Vol 3, Section 5.4.1)
%technique to form runs. This technique is known to produce runs that
%are twice as long and half as many as the ordinary run formation
%technique used in  \verb|AMI_partition_and_merge_Key()| on average,
%and so has the potential of reducing the number of merge passes.
%But the internal memory performance of replacement selection may
%be a little worse than the internal memory performance of the ordinary
%run formation technique; so what is eventually better could very well
%depend on various parameters.



\subsection{Comparison Sorting}
\label{sec:ref-ami-sort}

\index{sorting!comparison|(}

\subsubsection{External Sorting based on AMI\_partition\_and\_merge()}

In order to provide maximum flaxibility to programmers, comparison
sorting a set of objects in a stream can be done in three different
ways.  The particular choice of method will depend on the exact
context in which the sort is being performed. The three \verb|AMI_sort()| 
polymorphs mentioned below are implemented using the  \verb|AMI_partition_and_merge()| 
function and merge management objects.

All three polymorhps of \verb|AMI_sort()| take an argument
\verb|instream|, which is a pointer to the input stream, and
\verb|outstream|, which is a pointer to the output stream to which the
sorted results are to be written.
The first polymoph takes no additional arguments.  
\begin{verbatim}
AMI_err AMI_sort(AMI_STREAM<T> *instream, AMI_STREAM<T> *outstream)
\end{verbatim}
In order to
determine the relative order of two input objects in the output, it
uses the operator \verb|<|.  Thus, the type \verb|T| of objects in the
stream must have a member \verb|operator<|. 

The second polymorph relies on a user supplied comparison function,
similar to that used in the familiar C library function \verb|qsort()|.
\begin{verbatim}
AMI_err AMI_sort(AMI_STREAM<T> *instream, AMI_STREAM<T> *outstream,
                 int (*cmp)(CONST T&, CONST T&));
\end{verbatim}
When called, \verb|cmp(a,b)| should return a negative value if $a$ is
less that $b$, a positive value if $a$ is greater than $b$, and
\verb|0| if \verb|a| and \verb|b| are equal.

It is often useful to use a comparison function when the type of
object being sorted has several fields, and we sometime wish to sort
on one field and sometimes on another.  Two differenct comparison
functions can be defined, eliminating the need to provide a \verb|<|
operator that is compile-time bound to perform comparisons is a
particular way.

The final polymorph of \verb|AMI_sort()| relise on a comparator class.
\begin{verbatim}
AMI_err AMI_sort(AMI_STREAM<T> *instream, AMI_STREAM<T> *outstream,
                 comparator<T> *cmp)
\end{verbatim}
Instead of defining a comparison function, the programmer defines a
comparator class, which has a member function 
\begin{verbatim}
    int compare(const T& t1, const T& t2);
\end{verbatim}
which behaves exactly as the comparison function described above.

Examples of comparison based sorting appear in
Section~\ref{sec:cmp-sorting} as well as in the test applications that 
appear in the TPIE distribution.  Section~\ref{sec:list-ranking}
illustrates the use of multiple comparison functions to sort streams
of objects in different ways.


\subsubsection{External Sorting based on MIAMI\_single\_merge\_*()}
In situations wherein merge management objects are sought to be avoided 
(see Section~\ref{sec:ref-ami-merge}), the programmer may use the 
\verb|AMI_partition_and_merge_stream()|, \verb|AMI_partition_and_merge_Key()|,
and the \verb|AMI_partition_and_merge_stream_cmp()| routines (with the appropriate
pre-conditions mentioned in Section~\ref{sec:ref-ami-merge} met,) to 
do external sorting.

\index{sorting!comparison|)}

\subsection{Key Bucket Sorting}
\label{sec:ref-ami-kb-sort}

\index{sorting!key bucket|(}

\begin{verbatim}
AMI_err AMI_kb_dist()
\end{verbatim}

\begin{verbatim}
AMI_err AMI_kb_sort()
\end{verbatim}

\begin{verbatim}
AMI_err AMI_kb_dist_key()
\end{verbatim}

\begin{verbatim}
AMI_err AMI_kb_sort_key()
\end{verbatim}
\index{sorting!key bucket|)}

\subsection{General Permuting}
\label{sec:ref-ami-gp}

\index{permutation!general|(}

\begin{verbatim}
AMI_err AMI_general_permute(AMI_STREAM<T> *instream, 
                            AMI_STREAM<T> *outstream, 
                            AMI_gen_perm_object *gpo);
\end{verbatim}
\index{permutation!general|)}

\subsection{Bit Permuting}
\label{sec:ref-ami-bp}

\index{permutation!bit|(}
\begin{verbatim}
AMI_err AMI_BMMC_permute(AMI_STREAM<T> *instream, 
                         AMI_STREAM<T> *outstream, 
                         AMI_bit_perm_object *bpo);
\end{verbatim}
\index{permutation!bit|)}

\subsection{Dense Matrices}
\label{sec:ref-ami-matrix}

\index{matrices!dense|(}
\index{matrices!dense|)}

\subsection{Sparse Matrices}
\label{sec:ref-ami-sm}

\index{matrices!sparse|(}
\index{matrices!sparse|)}

\subsection{Stacks}
\label{sec:ref-ami-stack}

\index{stacks|(}

External stacks are implemented through the templated class
\verb|AMI_stack<T>|, which is a subclass of \verb|AMI_STREAM<T>|.  In
addition to the normal operations operations on streams, an
\verb|AMI_stack<T>| has member functions \verb|AMI_err push(const &T
t)| and \verb|AMI_err pop(T **ppt)|.  The former appends a copy of the
item \verb|t| to the end of the stream, increasing it's length by one.
The latter removes the last item from the stream, decreasing it's
length by one and storing a pointer to the removed object in the space
pointed to by \verb|ppt|.  An example of the use of an external stack
can be seen in the convex hull code in Section~\ref{sec:convex-hull}.

\index{stacks|)}

\subsection{Elementwise and Scalar Elementwise Arithmetic}
\label{sec:ref-ami-arith}

\index{elementwise arithmetic|(} 

Elementwise arithmetic on streams is done through the use of scan
management objects designed specifically for this purpose.  They are
used as parameters to \verb|AMI_scan()|, just like any other scan
management object.  To operate on two input streams and produce a
single output stream, each of whose elements is computed from the
corresponding elements of the input streams, we use a scan management
object of class \verb|AMI_scan_op<T>|, where \verb|op| is one of
\verb|add|, \verb|sub|, \verb|mult|, or \verb|div|.  As long as
\verb|T| is a type that has the necessary operator (i.e. \verb|+|,
\verb|-|, \verb|*|, or \verb|/|) then the corresponding class
\verb|AMI_scan_op<T>| can be instantiated.

If the two input streams are not of the same length, then elementwise
operations will only continue as long as there is data available from
the shorter of the two streams.  Thus, the output stream will be
exactly as long as the shorter of the two input streams.  No attempt
is made to use an identity if one stream runs out before the other.

A related class of scan management object,
\verb|AMI_scan_scalar_op<T>| produces an output object equal to
\verb|t op s| for each input object \verb|t| and some scalar |s|.
The value of \verb|s| is set at the time the object is constructed,
not at initialization time (i.e. when \verb|initialize()| is called.
For example, the declaration
\begin{verbatim}
    AMI_scan_scalar_div<double> ssd(17.0);
\end{verbatim}
declares a scan management object that will produce a stream of output
each element of which is one seventeenth the corresponding input.

Since dense matrices with elements of type \verb|T| are a subclass of
\verb|AMI_STREAM<T>|, elementwise arithmetic operations on dense
matrices are implemented on top of scans using elementwise arithmetic
operations.

\index{elementwise arithmetic|)}
