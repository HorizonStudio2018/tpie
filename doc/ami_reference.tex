%%
%% $Id: ami_reference.tex,v 1.14 2002-06-26 23:17:42 tavi Exp $
%%
\chapter{TPIE Programmer's Reference}
\plabel{cha:reference}

%%%%%%% Memory Manager %%%%%%%%%
\mysection{Registration-based Memory Manager}
\plabel{sec:mm-ref}
\index{memory manager|(}

\subsection{Files}
  \btabb
    \entry{\#include <mm\_register.h>} {Note that there is no need to
include this file when using the AMI entry points, since it is included by
all AMI header files.}
  \etabb

\subsection{Class Declaration}
  \btabb
    \entry{class \textbf{MM\_register};} {}
  \etabb

\subsection{Global Variables}
  \btabb
    \entry{MM\_register \textbf{MM\_manager};} {This is the only instance of
the \myverb{MM\_register} class that should exist in a program.}
  \etabb

\subsection{Description}
The TPIE memory manager \myverb{MM\_manager}, the only instance of class
\myverb{MM\_register}, traps memory allocation and deallocation requests in
order to monitor and enforce memory usage limits. The actual memory
allocation requests are done using the standard C++ operators \myverb{new}
and \myverb{delete}, which have been replaced with in-house versions that
interact with the memory manager.

\subsection{Public Member Functions}
  \btabb

    \entry{MM\_err \textbf{enforce\_memory\_limit}();} {Instruct TPIE to
    abort computation when the memory limit is exceeded.}

    \entry{MM\_err \textbf{ignore\_memory\_limit}();} {Instruct TPIE to
    ignore the memory limit set using \myverb{set\_memory\_limit}.}

    \entry{size\_t \textbf{memory\_available}();} {Return the number of
    bytes of memory which can be allocated before the user-specified limit
    is reached.}

    \entry{size\_t \textbf{memory\_limit}();} {Return the memory limit as
    set by the last call to method \myverb{set\_memory\_limit}.}

    \entry{size\_t \textbf{memory\_used}();} {Return the number of bytes
    of memory currently allocated.}

    \entry{MM\_err \textbf{set\_memory\_limit}(size\_t size);} {Set the
    application's memory limit. The memory limit is set to \myverb{size}
    bytes. If the specified mamory limit is greater than or equal to the
    amount of memory already allocated, \myverb{set\_memory\_limit} returns
    \myverb{MM\_ERROR\_NO\_ERROR}, otherwise it returns
    \myverb{MM\_ERROR\_EXCESSIVE\_ALLOCATION}. By default, successive calls
    to operator \myverb{new} will cause the program to abort if the
    resulting memory usage would exceed \myverb{size} bytes. This behaviour
    can be controlled explicitly by the use of methods
    \myverb{enforce\_memory\_limit}, \myverb{warn\_memory\_limit} and
    \myverb{ignore\_memory\_limit}.}

    \entry{int \textbf{space\_overhead}();} {TPIE imposes a small space
    overhead on each memory allocation request received by operator
    \myverb{new}. This involves increasing each allocation request by a
    fixed number of bytes. The precise size of this increase is machine
    dependent, but typically 8 bytes. Method \myverb{space\_overhead}
    returns the size of this increase.}

    \entry{MM\_err \textbf{warn\_memory\_limit}();} {Instruct TPIE to
    issue a warning when the memory limit is exceeded.}

  \etabb
\index{memory manager|)}

%%%%%%% AMI Stream %%%%%%%%%%
\mysection{Streams}
\index{streams!AMI|(}\plabel{sec:ref-ami-stream}
\index{AMI_STREAM@{\tt AMI\_STREAM}}

\subsection{Files}
  \btabb
    \entry{\#include <ami\_stream.h>} {}
  \etabb

\subsection{Class Declaration}
  \btabb
    \entry{template<class T> class \textbf{AMI\_STREAM};} {}
  \etabb

\subsection{Description}
An \myverb{AMI\_STREAM<T>} object stores an ordered collection of objects of
type {\tt T} on external memory.

\index{AMI_STREAM@{\tt AMI\_STREAM}!stream types|(}
The stream type of an \myverb{AMI\_STREAM} indicates what
operations are permitted on the stream.
An \myverb{AMI\_STREAM<T>} object can have one of four different types:
\begin{itemize}
    
    \item \myverb{AMI\_READ\_STREAM}: Input operations on
    the stream are permitted, but output is not permitted.
    
    \item \myverb{AMI\_WRITE\_STREAM}: Output operations are
    permitted, but input operations are not permitted. 
    
    \item \myverb{AMI\_APPEND\_STREAM}: Output is appended
    to the end of the stream. Input operations are not
    permitted. This is similar to
    \myverb{AMI\_WRITE\_STREAM} except that if the stream is
    constructed on a file containing an existing stream,
    objects written to the stream will be appended at the
    end of the stream.\comment{LA: It this true?
    DH: seems ok}

    \item \myverb{AMI\_READ\_WRITE\_STREAM}: Both input and output
    operations are permitted.
\end{itemize}
\index{AMI_STREAM@{\tt AMI\_STREAM}!stream types|)}

\subsection{Constructors, Destructor and Related Functions}
  \btabb

    \entry{\textbf{AMI\_STREAM}();} {A new stream of type
    \myverb{AMI\_READ\_WRITE\_STREAM} is constructed on a file with a
    randomly generated name.}
 
    \entry{\textbf{AMI\_STREAM}(const char *path\_name);} {A stream of type
    \myverb{AMI\_READ\_WRITE\_STREAM} is constructed on the file whose path
    name is given. If the file does not already exist, a new stream is
    constructed on a newly created file with the specified file name. If
    the file already exists, it is checked if it contains a valid stream,
    and if so, the new stream is constructed on this file. If the file does
    not contain a valid stream, the status flag is set to
    \myverb{AMI\_STREAM\_STATUS\_INVALID}.}

    \entry{\textbf{AMI\_STREAM}(const char *path\_name, AMI\_stream\_type st);} {A
    stream of type \noiv{st} is constructed on the file whose pathname is
    given.}

    \entry{\textbf{AMI\_STREAM}(BTE\_STREAM<T> \&bs);} {A stream is constructed from
    an existing \myverb{BTE\_STREAM} (see Section~\ref{sec:ref-bte}). This
    constructor will not normally be used by a TPIE application
    programmer\comment{LA: Why do we have it then? DH: for
    completeness}. The new \myverb{AMI\_STREAM} gets the same type as the
    \myverb{BTE\_STREAM}.}

    \entry{\textbf{$\sim$AMI\_STREAM}();} {Destructor. Free the memory
    buffer and close the file. IF the persistence flag is
    \myverb{PERSIST\_DELETE}, also remove the file.}

    \index{new_substream()@{\tt new\_substream()}!AMI|)}     
    \entry{AMI\_err \textbf{new\_substream}(AMI\_stream\_type    st,
                            off\_t              sub\_begin,
                            off\_t              sub\_end,
                            AMI\_base\_stream<T> **sub\_stream );} 
    {A substream is an AMI stream that is part of another AMI stream. More
    precisely, a substream $B$ of a stream $A$ is defined as a contiguous
    range of objects from the ordered collection of objects that make up
    the stream $A$.  If desired, one can construct substreams of substreams
    of substreams {\em ad infinitum}. Since a substream is a stream in its
    own right, many of the stream member functions can be applied to a
    substream. A substream can be created via the
    pseudo-constructor\footnote{The reason we do not use a real constructor
    is to get around the fact that constructors can not be virtual.  Please
    see Section~\ref{cha:implementation} for more details.}
    \myverb{new\_substream()}. Here, \myverb{st} specifies the type of the
    substream, and the offsets \myverb{sub\_begin} and \myverb{sub\_end}
    define the positions in the original stream $A$ where the new substream
    $B$ will begin and end. Upon completion, \myverb{*sub\_stream} points
    to the newly created substream.}
    \index{new_substream()@{\tt new\_substream()}!AMI|)} 
  \etabb

\subsection{Public Member Functions}
  \btabb

    \entry{bool \textbf{operator!}() const;} {Return \noiverb{true} if the
    status of the stream is not
    \myverb{AMI\_STREAM\_STATUS\_VALID}, \noiverb{false} otherwise. See
    also \myverb{is\_valid()} and \myverb{status()}.}

    \entry{off\_t chunk\_size() const;} {Return the maximum number of items (of type
    \noiverb{T}) that can be stored in one block.}

    \entry{bool \textbf{is\_valid}() const;} {Return \noiverb{true} if the
    status of the stream is \myverb{AMI\_STREAM\_STATUS\_VALID},
    \noiverb{false} otherwise. See also \myverb{status()}.}

    \index{main_memory_usage()@{\tt main\_memory\_usage()}!AMI|(}
    \entry{AMI\_err \textbf{main\_memory\_usage}(size\_t *usage, MM\_stream\_usage
usage\_type);} {This function is used for obtaining the amount of main memory used by an
\noiv{AMI\_STREAM<T>} object (in bytes). \myverb{usage\_type} is one of the
following:
\begin{description}
  \item[\myverb{MM\_STREAM\_USAGE\_CURRENT}] Total amount of memory currently
used by the stream.
  \item[\myverb{MM\_STREAM\_USAGE\_MAXIMUM}] Max amount of memory that will
ever be used by the stream.
  \item[\myverb{MM\_STREAM\_USAGE\_OVERHEAD}] The amount of memory used by
the object itself, without the data buffer.
  \item[\myverb{MM\_STREAM\_USAGE\_BUFFER}]  The amount of memory used by the
data buffer.
  \item[\myverb{MM\_STREAM\_USAGE\_SUBSTREAM}] The additional amount of
memory that will be used by  each substream created.
\end{description}
}
    \index{main_memory_usage()@{\tt main\_memory\_usage()}!AMI|)}

    \index{name()@{\tt name()}!AMI|(}
    \entry{AMI\_err \textbf{name}(char **stream\_name);} {Store the path to the UNIX
file name holding the stream, in newly allocated memory.}
    \index{name()@{\tt name()}!AMI|)}

    \entry{void \textbf{persist}(persistence p)} {Set the persistence flag to \noiv{p},
    which can have one of two values: \myverb{PERSIST\_DELETE} and
    \myverb{PERSIST\_PERSISTENT}.}

    \index{read_array()@{\tt read\_array()}!AMI|(}
    \entry{AMI\_err \textbf{read\_array}(T *mm\_array, off\_t *len);} {Read \noiv{*len} items from
the current position of the stream into the array \noiv{mm\_array}. The
``current item'' pointer is increased accordingly. }
    \index{read_array()@{\tt read\_array()}!AMI|)}

    \index{read_item()@{\tt read\_item()}!AMI|(} 
    \entry{AMI\_err
    \textbf{read\_item}(T **elt);} {Read the current item from the stream and
    advance the ``current item'' pointer to the next item. The item read is
    pointed to by \noiv{*elt}. If no error has occured, return
\myverb{AMI\_ERROR\_NO\_ERROR}. If the ``current item'' pointer is beyond
the last item in the stream, return \myverb{AMI\_ERROR\_END\_OF\_STREAM}.}
    \index{read_item()@{\tt read\_item()}!AMI|)}
    
    \index{seek()@{\tt seek()}!AMI|(}
    \entry{AMI\_err \textbf{seek}(off\_t off);} {Move the current position to \noiv{off}.}
    \index{seek()@{\tt seek()}!AMI|)}

    \entry{AMI\_stream\_status \textbf{status}() const;} {Return the status
    of the stream instance. The result is either
    \myverb{AMI\_STREAM\_STATUS\_VALID} or
    \myverb{AMI\_STREAM\_STATUS\_INVALID}. The only operation that can
    leave the stream invalid is the constructor (if that happens, the log
    file contains more information). No items should be read from or
    written to an invalid stream.}

    \index{stream_len()@{\tt stream\_len()}!AMI|(}
    \entry{off\_t \textbf{stream\_len}(void);} {Return the number of items stored in
the stream.}
    \index{stream_len()@{\tt stream\_len()}!AMI|)}

    \index{truncate()@{\tt truncate()}!AMI|(}
    \entry{AMI\_err \textbf{truncate}(off\_t off);} {Resize the stream to
\noiv{off} items. If \noiv{off} is less than the number of objects in the
stream, \noiv{truncate()} truncates the stream to
\noiv{off} objects. If \noiv{off} is more than the
number of objects in the stream, \noiv{truncate()} extends
the stream to the specified number of objects. In either
case, the ``current item'' pointer will be moved to the new end of
the stream.}
    \index{truncate()@{\tt truncate()}!AMI|)}

    \index{write_array()@{\tt write\_array()}!AMI|(}
    \entry{AMI\_err \textbf{write\_array}(const T *mm\_array, off\_t len);} {Write
\noiv{len} items from array \noiv{mm\_array} to the stream, starting in the
current position. The ``current item'' pointer is increased accordingly.}
    \index{write_array()@{\tt write\_array()}!AMI|)}

    \index{write_item()@{\tt write\_item()}!AMI|(}
    \entry{AMI\_err \textbf{write\_item}(const T \&elt);} {Write \noiv{elt} to the
stream in the current position. Advance the ``current item'' pointer to the
next item. If no error has occured, return \myverb{AMI\_ERROR\_NO\_ERROR}.}
    \index{write_item()@{\tt write\_item()}!AMI|)}
        
  \etabb
\index{streams!AMI|)}


%%%%%%% Scanning %%%%%%%%
\mysection{Scanning}
\plabel{sec:ref-ami-scan}
\index{scanning|(}
\index{AMI_scan()@{\tt AMI\_scan()}|(}

\subsection{Files}
  \btabb
    \entry{\#include <ami\_scan.h>} {}
  \etabb

\subsection{Function Declaration}
  \btabb
    \entry{template<class T1, class T2, ..., class ST, class U1, class U2,
...> AMI\_err \textbf{AMI\_scan}(AMI\_STREAM<T1> *in1, AMI\_STREAM<T2>
*in2, ..., ST *smo, AMI\_STREAM<U1> *out1, AMI\_STREAM<U2> *out2, ...);} {}
  \etabb

\subsection{Description}

\noiv{AMI\_scan()} reads zero, one or multiple input streams (up to
four), each potentially of a different type, and writes zero, one or
multiple output streams (up to four), each potentially of a different type.
\noiv{smo} is a pointer to a {\em scan management object} of user-defined
class \noiv{ST}, as described below.  
%\noiv{ST} should provide member functions {\tt
%AMI\_err initialize(void)} and {\tt AMI\_err operate(const T1 \&in1,
%const T2 \&in2, ..., AMI\_SCAN\_FLAG *sfin, U1 *out1, U2 *out2, ...,
%AMI\_SCAN\_FLAG  *sfout)}.

\subsection{Scan Management Objects}

\index{operation management objects!scan|(}

A scan management object class must inherit from \myverb{AMI\_scan\_object}:
\begin{verbatim}
template<class T1, class T2,..., class U1, class U2,...>
class ST: public AMI_scan_object;
\end{verbatim}
In addition, it must provide two member
functions for \myverb{AMI\_scan()} to call: \myverb{initialize()} and \myverb{operate()}.
\index{initialize()@{\tt initialize()}|(}
\begin{verbatim}
    AMI_err initialize(void);
\end{verbatim}
    Initializes a scan management object to prepare
    it for a scan.  This member function is called once by
    each call to \myverb{AMI\_scan()} in order to initialize
    the scan management object before any data processing
    takes place.  This function should return
    \myverb{AMI\_ERROR\_NO\_ERROR} if successful, or an
    appropriate error otherwise. See
    Section~\ref{sec:ami-errors} for a list of error codes.
\index{initialize()@{\tt initialize()}|)}
    
\index{operate()@{\tt operate()}|(}
    Most of the work of a scan is typically done in the scan
    management object's \myverb{operate()} member function:
\begin{verbatim}
    AMI_err operate(const T1 &in1, const T2 &in2,...,  AMI_SCAN_FLAG *sfin,
                    U1 *out1, U2 *out2,..., AMI_SCAN_FLAG *sfout);
\end{verbatim}
    
    One or more input objects or one or more output
    parameters must be specified.  These must correspond in
    number and type to the streams passed to the polymorph
    of \myverb{AMI\_scan()} with which this scan management
    object is to be used.
    
    If present, the inputs \noiv{*in1, ...} are application
    data items of type \noiv{T1}, and \myverb{sfin} points
    to an array of flags, one for each input.  On entry to
    \myverb{operate()}, flags that are set (non-zero)
    indicate that the corresponding inputs contain data.  If
    on exit from \myverb{operate()}, the input flags are
    left untouched, \myverb{AMI\_scan()} assumes that the
    corresponding inputs were processed.  If one or more
    input flags are cleared (set to zero) then
    \myverb{AMI\_scan()} assumes that the corresponding
    inputs were not processed and should be presented again
    on the next call to \myverb{operate()}.  This permits
    out of step scanning\index{scanning!out of step}, as
    illustrated in Section~\ref{sec:tut-out-of-step}.
    
    If present, the outputs \noiv{*out1, ...} are
    application data items of type \noiv{U1}, and
    \myverb{sfout} points to an array of flags, one for each
    output. On exit from \myverb{operate()}, the outputs
    should contain any objects to be written to the output
    streams, and the output flags must be set to indicate to
    \myverb{AMI\_scan()} which outputs are valid and should
    be written to the output streams.
    
    The return value of \myverb{operate()} will normally be
    one of the following:
    \begin{itemize}
        \item \myverb{AMI\_SCAN\_CONTINUE}:
        \index{AMI_SCAN_CONTINUE@{\tt AMI\_SCAN\_CONTINUE}} indicates that
        the function should be called again with any
        ``taken'' inputs replaced by the next objects from
        their respective streams
        \item \myverb{AMI\_SCAN\_DONE}: 
        \index{AMI_SCAN_DONE@{\tt AMI\_SCAN\_DONE}}
        indicates that the
        scan is complete and no more input needs to be
        processed.
    \end{itemize}
    
    Note that \myverb{operate()} is permitted to return
    \myverb{AMI\_SCAN\_CONTINUE} even when the input flags
    indicate that there is no more input to be processed.
    This is useful if the scan management object maintains
    some internal state that must be written out after all
    input has been processed.  

\index{operate()@{\tt operate()}|)} 

Examples of the use of scan management objects are given in
Section~\ref{sec:tut-scanning} as well as in the test applications that
appear in the TPIE distribution.

\index{operation management objects!scan|)}
\index{AMI_scan()@{\tt AMI\_scan()}|)}
\index{scanning|)}


%%%%%%%%% Scanning from a C++ stream %%%%%%%%%%
\mysection{Scanning from a C++ stream}
\plabel{sec:ref-cxx-stream-input}

\subsection{Files}
  \btabb
    \entry{\#include <ami\_scan\_utils.h>} {}
  \etabb


\subsection{Class Declaration}
  \btabb
    \entry{template<class T> class \textbf{cxx\_istream\_scan};} {}
  \etabb

\subsection{Description}
A scan management class template for reading the contents of an
ordinary C++ input stream into a TPIE stream.  It works with
streams of any type for which a \myverb{>>} operator is defined for C++
stream input.

\subsection{Constructor}
  \btabb
    \entry{\textbf{cxx\_istream\_scan}(istream *instr = \&cin);} {Create a
scan management object for scanning the contents of C++ stream
\noiv{*instr}. The actual scanning is done using \myverb{AMI\_scan} with
no input streams and one output stream.}
  \etabb


%%%%%%%%% Scanning into a C++ stream %%%%%%%%%%
\mysection{Scanning into a C++ stream}
\plabel{sec:ref-cxx-stream-output}

\subsection{Files}
  \btabb
    \entry{\#include <ami\_scan\_utils.h>} {}
  \etabb


\subsection{Class Declaration}
  \btabb
    \entry{template<class T> class \textbf{cxx\_ostream\_scan};} {}
  \etabb

\subsection{Description}
A scan management class template for writing the contents of a TPIE stream
into an ordinary C++ output stream.  It works with
streams of any type for which a \myverb{<<} operator is defined for C++
stream output.

\subsection{Constructor}
  \btabb
    \entry{\textbf{cxx\_ostream\_scan}(istream *outstr = \&cout);} {Create a
scan management object for scanning into C++ stream
\noiv{*outstr}. The actual scanning is done using \myverb{AMI\_scan} with
one input stream and no output streams.}
  \etabb


%%%%%%%% Stream Merging %%%%%%%%%
\mysection{Stream Merging}
\index{merging|(}
\index{AMI_merge()@{\tt AMI\_merge()}|(}
\plabel{sec:ref-ami-merge}

\subsection{Files}
  \btabb
    \entry{\#include <ami\_merge.h>} {}
  \etabb

\subsection{Function Declarations}
  \btabb
    \entry{template<class T>\\  AMI\_err \textbf{AMI\_merge}(AMI\_STREAM<T> **instreams,
                      arity\_t arity, AMI\_STREAM<T> *outstream);} {}
    \entry{template<class T>\\  AMI\_err \textbf{AMI\_merge}(AMI\_STREAM<T> **instreams,
                      arity\_t arity, AMI\_STREAM<T> * outstream,
                      int (*cmp)(const T\&, const T\&));} {}
    \entry{template<class T>\\ AMI\_err \textbf{AMI\_merge}(AMI\_STREAM<T> **instreams,
                      arity\_t arity, AMI\_STREAM<T> * outstream, CmpObj *co);} {}
    \entry{template<class T, class KEY>\\ AMI\_err \textbf{AMI\_merge}(AMI\_STREAM<T> **instreams,
                   arity\_t arity, AMI\_STREAM<T> *outstream, int keyoff, KEY dummy);} {}
  \etabb

\subsection{Description}

TPIE provides several merge entry points for merging sorted streams to
produce a single, interleaved output stream. \myverb{AMI\_merge} has
four polymorphs, described below. We will refer to these as the (1)
comparison operator, (2) comparison function, (3) comparison class and (4) key-based
versions of \myverb{AMI\_merge}. 
 The comparison operator version tends to
be the fastest and most straightforward to use. The
comparison class version is comparable in speed (maybe
slightly slower), but somewhat more flexible, as it can support
multiple, different merges on the same keys. The comparison
function version is slightly easier to use than the
comparison class version, but typically it is measureably slower.

\index{AMI_merge()@{\tt AMI\_merge()}|)}
\index{merging|)}


%%%%%%%% Generalized Stream Merging %%%%%%%%%
\mysection{Generalized Stream Merging}
\index{merging!generalized|(}
\index{AMI_generalized_merge()@{\tt AMI\_generalized\_merge()}|(}

\subsection{Files}
  \btabb
    \entry{\#include <ami\_merge.h>} {}
  \etabb

\subsection{Function Declaration}
  \btabb
    \entry{template<class T, class MergeMgr>\\ 
           AMI\_err \textbf{AMI\_generalized\_merge}(AMI\_STREAM<T> **instreams, 
           arity\_t arity, AMI\_STREAM<T> *outstream, MergeMgr *mo);} {}
  \etabb

\subsection{Description}
TPIE entry point \myverb{AMI\_generalized\_merge()} allows an
arbitrary number of streams to be merged into one stream in one pass,
subject to the available main memory.  TPIE will attempt to read the
first block of each stream into the internal memory, and will update
the contents of these buffers as the merge progresses. At least one
block buffer is also required for the output stream from the merge.
The function takes four arguments:
\myverb{instreams} is an array of pointers to the input streams, all of
  which are of type \myverb{AMI\_STREAM<T>},
\myverb{arity} is the number of input streams,
\myverb{outstream} is the output stream, of type \myverb{AMI\_STREAM<T>}, and
\myverb{mo} points to a merge management object that controls the
  merge (merge management objects are described below).

If the merge cannot be completed in one pass due to insufficient
memory, the function fails and it returns
\myverb{AMI\_ERROR\_INSUFFICIENT\_MAIN\_MEMORY}. Otherwise, it returns
\myverb{AMI\_ERROR\_NO\_ERROR}.

\index{AMI_generalized_merge()@{\tt AMI\_generalized\_merge()}|)}


\subsection{Merge Management Objects}\label{ssec:mmo}
\index{operation management objects!merge|(} 

A merge management object class must inherit from
\myverb{AMI\_generalized\_merge\_base}:
\begin{verbatim}
template<class T>
class MergeMgr: public AMI_generalized_merge_base;
\end{verbatim}
In addition, 
a merge management object must provide \myverb{initialize()}
and \myverb{operate()} member functions, whose purposes are
analogous to their namesakes for scan management objects.

\index{initialize()@{\tt initialize()}|(}
    The user's \myverb{initialize()} member function is
    called by the merge function once so that
    application-specific data structures (if any) can be
    initialized.
\begin{verbatim}
    AMI_err initialize(arity_t arity, const T * const *in,
                       AMI_merge_flag *taken_flags,
                       int &taken_index);
\end{verbatim}
 
    where
    \begin{itemize}
        \item \myverb{arity} is the number of input streams
        in the merge,
        \item \myverb{in} is a pointer to an array of
        pointers to input objects, each of which is the
        first objects appearing in one of the input streams,    
        \item \myverb{taken\_flags} an array of flags
        indicating which of the inputs are present (i.e.
        which of the input streams is not empty), and a
        pointer to an output object.
    \end{itemize}
    
    The typical behavior of \myverb{initialize()} is to
    place all the input objects into a data structure and
    then return \myverb{AMI\_MERGE\_READ\_MULTIPLE} to
    indicate that it used (and is now finished with) all of
    the inputs which were indicated to be valid by
    \myverb{taken\_flags}.  \myverb{initialize} need not
    process all inputs; it can turn off any flags in
    \myverb{taken\_flags} corresponding to inputs that
    should be presented to \myverb{operate()}.
    Alternatively, it can set \myverb{taken\_index} to the
    index of a single input it processed and return
    \myverb{AMI\_MERGE\_CONTINUE}.
\index{initialize()@{\tt initialize()}|)}

\index{operate()@{\tt operate()}|(}
    When performing a merge, TPIE relies on the application
    programmer to provide code to determine the order of any
    two application data elements, and certain other
    application-specific processing. By convention, TPIE
    expects these decisions to be made by the
    \myverb{operate()} function:
\begin{verbatim}
    AMI_err operate(const T * const *in, AMI_merge_flag *taken_flags,
                    int &taken_index, T *out);
\end{verbatim}
    The \myverb{operate()} member function is called
    repeatedly to process input objects.  Typically,
    \myverb{operate()} will choose a single input object to
    process, and set \myverb{taken\_index} to the index of
    the pointer to that object in the input array.  This
    object is then typically added to a dynamic data
    structure maintained by the merge management object.  If
    output is generated, for example by removing an object
    from the dynamic data structure, \myverb{operate()}
    should return \myverb{AMI\_MERGE\_OUTPUT}, otherwise, it
    returns either \myverb{AMI\_MERGE\_CONTINUE} to indicate
    that more input should be presented, or
    \myverb{AMI\_MERGE\_DONE} to indicate that the merge has
    completed.
    
    Alternatively, \myverb{operate()} can clear the elements
    of \myverb{taken\_flags} that correspond to inputs it
    does not currently wish to process, and then return
    \myverb{AMI\_MERGE\_READ\_MULTIPLE}.  This is generally
    undesirable because, if only one input is taken, it is
    far slower than using \myverb{taken\_index} to indicate
    which input was taken.  The merge management object must
    clear all other flags, and then TPIE must test all the
    flags to see which inputs were or were not processed.
\index{operate()@{\tt operate()}|)}


\index{operation management objects!merge|)}
\index{merging!generalized|)}


%%%%%%%% Stream Partitioning and Merging %%%%%%%%%
\mysection{Stream Partitioning and Merging}
\index{AMI_partition_and_merge()@{\tt AMI\_partition\_and\_merge()}|(}
\plabel{sec:ref-ami-pmerge}

\subsection{Files}
  \btabb
    \entry{\#include <ami\_merge.h>} {}
  \etabb

\subsection{Function Declaration}
  \btabb
    \entry{template<class T>\\  AMI\_err \textbf{AMI\_partition\_and\_merge}(AMI\_STREAM<T> **instreams,
                      arity\_t arity, AMI\_STREAM<T> *outstream);} {}
    \entry{template<class T>\\  AMI\_err \textbf{AMI\_partition\_and\_merge}(AMI\_STREAM<T> **instreams,
                      arity\_t arity, AMI\_STREAM<T> *outstream,
                      int (*cmp)(const T\&, const T\&));} {}
    \entry{template<class T, class KEY>\\ AMI\_err \textbf{AMI\_partition\_and\_merge}(AMI\_STREAM<T> **instreams,
                      arity\_t arity, AMI\_STREAM<T> *outstream, int keyoffset, KEY dummy);} {}
  \etabb

\subsection{Description}
Each of these functions partitions a stream into substreams small
enough to fit in main memory, sorts them in main memory, and then
merges them together, possibly in several passes if low memory conditions
dictate. The difference between the three polymorphs is the comparison
method: in the first polymorph, comparison is done using the comparison
operator of class \myverb{T}; in the second polymorph, comparison is done
using the comparison function \myverb{cmp}; in the third polymorph,
comparison is done using a key of type \myverb{KEY}, extracted from objects
of type \myverb{T} at byte offset \myverb{keyoffset}.

In order to complete the merge successfully, these functions need sufficient
memory for a binary merge. If not enough memory is available, the function
fails and it returns
\myverb{AMI\_ERROR\_INSUFFICIENT\_MAIN\_MEMORY}. Otherwise, it returns
\myverb{AMI\_ERROR\_NO\_ERROR}.
\index{AMI_partition_and_merge()@{\tt AMI\_partition\_and\_merge()}|)}

%%%%%%%% Generalized Stream Partitioning and Merging %%%%%%%%%
\mysection{Generalized Stream Partitioning and Merging}
\index{AMI_generalized_partition_and_merge()@{\tt AMI\_generalized\_partition\_and\_merge()}|(}
\plabel{sec:ref-ami-gpmerge}

\subsection{Files}
  \btabb
    \entry{\#include <ami\_merge.h>} {}
  \etabb

\subsection{Function Declaration}
  \btabb
    \entry{template<class T, class MergeMgr>\\
           AMI\_err \textbf{AMI\_generalized\_partition\_and\_merge}(AMI\_STREAM<T> *instream, AMI\_STREAM<T> *outstream, MergeMgr *mo);} {}
  \etabb

\subsection{Description}
This function partitions a stream into substreams small
enough to fit in main memory, operates on each in main memory, and then
merges them together, possibly in several passes if low memory conditions
dictate. This function takes three arguments:
\myverb{instream} points to the input stream,
\myverb{outstream} points to the output stream, and
\myverb{mo} points to a merge management object that controls the merge.
This function takes care of all the details of determining how much main
memory is available, how big the initial substreams can be, how many
streams can be merged at a time, and how many levels of merging must take
place.

In order to complete the merge successfully, the function needs sufficient
memory for a binary merge. If not enough memory is available, the function
fails and it returns
\myverb{AMI\_ERROR\_INSUFFICIENT\_MAIN\_MEMORY}. Otherwise, it returns
\myverb{AMI\_ERROR\_NO\_ERROR}.
\index{AMI_generalized_partition_and_merge()@{\tt AMI\_generalized\_partition\_and\_merge()}|)}

\subsection{Merge Management Objects}
\index{operation management objects!merge|(} 
The \myverb{AMI\_partition\_and\_generalized\_merge()} entry point requires
a merge management object similar to the one described in
Section~\ref{ssec:mmo}. The following three additional member functions
must also be provided.

\begin{itemize}
    \item \index{main_mem_operate()@{\tt main\_mem\_operate()}}
\begin{verbatim}
    AMI_err main_mem_operate(T* mm_stream, size_t len);
\end{verbatim}
\noindent
where
    \begin{itemize}
        \item \myverb{mm\_stream} is a pointer to an array
        of objects that have been read into main memory,
        \item \myverb{len} is the number of objects in the
        array.
    \end{itemize}
    
    This function is called by
    \myverb{AMI\_partition\_and\_merge()} when a substream of
    the data is small enough to fit into main memory, and
    the (application-specific) processing of this subset of
    the data can therefore be completed in internal memory.

    
    \item \index{space_usage_per_stream()@{\tt space\_usage\_per\_stream()}}
\begin{verbatim}
    size_t space_usage_per_stream(void);
\end{verbatim}
    This function should return the amount of main memory
    that the merge management object will need per per input
    stream. Merge management objects are allowed to maintain
    data structures whose size is linear in the number of
    input streams being processed.

    \item \index{space_usage_overhead()@{\tt space\_usage\_overhead()}}
\begin{verbatim}
    size_t space_usage_overhead(void);
\end{verbatim}
    This function should return an upper bound on the number
    of bytes of main memory\comment{LA: In bytes? DH: yes.}
    the merge management object will allocate in addition to
    the portion that is linear in the number of streams.
    
\end{itemize}

\index{operation management objects!merge|)}


%%%%%%%%% Merge Sorting %%%%%%%%
\mysection{Merge Sorting}\plabel{sec:ref-ami-sort}
\index{sorting!merge|)} 
\index{AMI_key_sort@{\tt AMI\_key\_sort}|(}
\index{AMI_ptr_sort@{\tt AMI\_ptr\_sort}|(}
\index{AMI_sort@{\tt AMI\_sort}|(}

\subsection{Files}
  \btabb
     \entry{\#include <ami\_sort.h>} {}
  \etabb

\subsection{Function Declarations}

  \btabb
     \entry{template<class T>\\
AMI\_err \textbf{AMI\_sort}(AMI\_STREAM<T> *instream, AMI\_STREAM<T> *outstream);} {}     \entry{template<class T>\\
AMI\_err \textbf{AMI\_sort}(AMI\_STREAM<T> *instream, AMI\_STREAM<T> *outstream,
                 int (*cmp)(const T\&, const T\&));} {}
     \entry{template<class T, class CMPR>\\
AMI\_err \textbf{AMI\_sort}(AMI\_STREAM<T> *instream, AMI\_STREAM<T> *outstream,
                 CMPR *cmp);} {}

     \entry{template<class T>\\
AMI\_err \textbf{AMI\_ptr\_sort}(AMI\_STREAM<T> *instream, AMI\_STREAM<T>
*outstream);} {}
     \entry{template<class T>\\
AMI\_err \textbf{AMI\_ptr\_sort}(AMI\_STREAM<T> *instream, AMI\_STREAM<T> *outstream,
                 int (*cmp)(const T\&, const T\&));} {}
     \entry{template<class T, class CMPR>\\
AMI\_err \textbf{AMI\_ptr\_sort}(AMI\_STREAM<T> *instream, AMI\_STREAM<T> *outstream, CMPR *cmp);} {}

     \entry{template<class T, class KEY, class CMPR>\\
AMI\_err \textbf{AMI\_key\_sort}(AMI\_STREAM<T> *instream, AMI\_STREAM<T> *outstream,
         KEY dummykey, CMPR *cmp) ;} {}
  \etabb

\subsection{Description}
TPIE offers several entry points for sorting which use
merging as their underlying paradigm. Please see
Section~\ref{sec:tut-mrg-sorting} for more details of this
approach.

Currently, TPIE offers three merge sorting variants. The
user must decide which variant is most appropriate for their
circumstances.  All accomplish the same goal, but the
performance can vary depending on the situation. They differ
mainly in the way they perform the merge phase of merge
sort, specifically how they maintain their heap data
structure used in the merge phase. The three variants are as
follows:
\begin{itemize}
    \item \myverb{AMI\_sort}: keeps the (entire) first record
    of each sorted run (each is a stream) in a heap. This
    approach is most suitable when the record consists
    entirely of the record key.
    
    \item \myverb{AMI\_ptr\_sort}: keeps a pointer to the
    first record of each stream in the heap. This approach
    works best when records are very long and the key
    field(s) take up a large percentage of the record.

    \item \myverb{AMI\_key\_sort}: keeps the key field(s) and
    a pointer to the first record of each stream in the
    heap. This approach works best when the key field(s) are
    small in comparison to the record size.
\end{itemize}

Any of these variants will accomplish the task of sorting an
input stream in an I/O efficient way, but there can be
noticeable differences in processing time between the
variants. As an example, \myverb{AMI\_key\_sort} appears to be
more cache-efficient than the others in many cases, and
therefore often uses less processor time, despite extra data
movement relative to \myverb{AMI\_ptr\_sort}.

In addition to the three variants discussed above, there are
multiple choices within each variant regarding how the
actual comparison operations are to be performed. These
choices are described in detail for \myverb{AMI\_sort}, below.

\subsubsection{AMI\_sort()}
\myverb{AMI\_sort()} has three polymorphs, described below.
We will refer to these as the (1) comparison operator, (2)
comparison function, and (3) comparison class versions of
\myverb{AMI\_sort}. The comparison operator version tends to
be the fastest and most straightforward to use. The
comparison class version is comparable in speed (maybe
slightly slower), but somewhat more flexible, as it can support
multiple, different sorts on the same keys. The comparison
function version is slightly easier to use than the
comparison object version, but typically it is measureably slower.
Please refer to Section~\ref{sec:tut-mrg-sorting} for examples
of the use of these versions of \myverb{AMI\_sort()}.

\paragraph{Comparison operator version.} This version works on streams of
objects for which the operator \myverb{<} is defined. 

\paragraph{Comparison function version.}
This version uses an explicit function to determine the
relative order of two objects in the input stream. The function takes two
arguments of type \noiv{T} and returns $-1$, $0$, or $1$, if the first
object is less than, equal or greater than the second object in the desired
order. This is useful in cases where we may want to sort a stream of
objects in several different ways.  

\paragraph{Comparison class version.} 
This version of \myverb{AMI\_sort()} is similar to the
comparison function version, except that the comparison
function is now a method of a user-defined comparison
object. This object must have a public member function named
\myverb{compare}, having the following prototype:
\begin{verbatim}
   inline int compare (const KEY & k1, const KEY & k2);
\end{verbatim}

The user-written \myverb{compare} function computes the
order of the two user-defined keys \noiv{k1} and
\noiv{k2}, and returns $-1$, $0$, or $+1$ to indicate that
$k1<k2$, $k1==k2$, or $k1>k2$ respectively.
It will be called by the internals of \noiv{AMI\_key\_sort} to
determine the relative order of records during the sort.


\index{AMI_sort@{\tt AMI\_sort}|)}

\subsubsection{AMI\_ptr\_sort()}

The \myverb{AMI\_ptr\_sort} variant of merge sort in TPIE
keeps only a pointer to each record in the heap used to
perform merging of runs. Similar to \myverb{AMI\_sort}
above, it offers comparison operator, comparison function,
and comparison class polymorphs. The syntax is identical to
that illustrated in the \myverb{AMI\_sort} examples; simply
replace \myverb{AMI\_sort} by \myverb{AMI\_ptr\_sort}.

\index{AMI_ptr_sort@{\tt AMI\_ptr\_sort}|)}

\subsubsection{AMI\_key\_sort()}

The \myverb{AMI\_key\_sort} variant of TPIE merge sort keeps
the key field(s) plus a pointer to the corresponding record
in an internal heap during the merging phase of merge sort.
It requires a sort management object with member functions
\myverb{compare} and \myverb{copy}.
The \noiv{dummyKey} argument of \myverb{AMI\_key\_sort()} is a a
dummy argument having the same type as the user key, and
\noiv{*smo} is the sort management
object, having user-defined \myverb{compare} and
\myverb{copy} member functions as described below.

The \myverb{compare} member function has the following
prototype:
\begin{verbatim}
   inline int compare (const KEY & k1, const KEY & k2);
\end{verbatim}

The user-written \myverb{compare} function computes the
order of the two user-defined keys \noiv{k1} and
\noiv{k2}, and returns $-1$, $0$, or $+1$ to indicate that
$k1<k2$, $k1==k2$, or $k1>k2$ respectively.
It will be called by the internals of \noiv{AMI\_key\_sort} to
determine the relative order of records during the sort.

The \myverb{copy} member function has the following
prototype:
\begin{verbatim}
   inline void copy (KEY *key, const T &record);
\end{verbatim}

The user-written \myverb{copy} function constructs the
user-defined key \noiv{*key} from the contents of the
user-defined record \noiv{record}. It will be called by the
internals of \noiv{AMI\_key\_sort} to make copies of record
keys as necessary during the sort.

\index{AMI_key_sort@{\tt AMI\_key\_sort}|)}
\index{sorting!merge|)} 

%%%%%%%%% Internal Memory Sorting %%%%%%%%
\mysection{Internal Memory Sorting}
\plabel{sec:ref-ami-memsort}
\index{sorting!internal memory|(}

\subsection{Files}
  \btabb
	\entry{\#include <quicksort.h>} {}
  \etabb

\subsection{Function Declarations}

  \btabb
     \entry{template<class T>\\
     void \textbf{quick\_sort\_op}(T *data, size\_t len);} {}

     \entry{template<class T>\\
     void \textbf{quick\_sort\_cmp}(T *data, size\_t len, 
     int (*cmp)(const T\&, const T\&));} {}

     \entry{template<class T>\\
     void \textbf{quick\_sort\_obj}(T *data, size\_t len, 
     CMPR *cmp);} {}
  \etabb

\subsection{Description}

These are internal memory in-place sorting routines that implement the
quicksort\index{quicksort} algorithm (randomized).  These routines are used by the external
memory sorting routines (see Section~\ref{sec:ref-ami-sort}) on streams
that are small enough to fit in memory.
The three polymorphs use different comparison methods:
\noiverb{quick\_sort\_op} uses the comparison operator $<$,
\noiverb{quick\_sort\_cmp} uses the comparison function \noiverb{cmp}, and
\noiverb{quick\_sort\_obj} uses a comparison object of type \noiverb{CMPR}.
\index{sorting!internal memory|)}

%%%%%%%%% Stacks %%%%%%%%%
\mysection{Stacks}
\plabel{sec:ref-ami-stack}
\index{AMI_stack@{\tt AMI\_stack}|(}
\index{stacks|(}

\subsection{Files}
  \btabb
	\entry{\#include <ami\_stack.h>} {}
  \etabb

\subsection{Class Declaration}
   \btabb
	\entry{template<class T> class \textbf{AMI\_stack};} {}
   \etabb

\subsection{Description}
External stacks are implemented through the templated class
\myverb{AMI\_stack<T>}, 
which is a subclass of \myverb{AMI\_STREAM<T>}. As a consequence, it
inherits all public members of \myverb{AMI\_STREAM<T>}, including its
constructors. See Section~\ref{sec:ref-ami-stream}.

\subsection{Public Member Functions}
   \btabb 

      \entry{AMI\_err \textbf{push}(const \&T t);} {Insert a copy of the
      object \myverb{t} to the top of the stack, increasing its length by
      one.}

       \entry{AMI\_err \textbf{pop}(T **ppt);} {Remove the top object from
       the stack, decreasing its length by one and returning the address of
       a pointer to the popped object in \myverb{ppt}.}

   \etabb

\index{stacks|)}
\index{AMI_stack@{\tt AMI\_stack}|)}

%%%%%%%%% Blocks %%%%%%%%%
\mysection{Blocks}
\index{AMI_block@{\tt AMI\_block}|(}

\subsection{Files}
  \btabb
	\entry{\#include <ami\_block.h>} {}
  \etabb

\subsection{Class Declaration}
   \btabb
	\entry{template<class E, class I> class \textbf{AMI\_block}; }{The types
	{\tt E} and {\tt I} should have a default constructor, a copy
	constructor and an assignment operator. The size returned by {\tt
	sizeof(E)} and {\tt sizeof(I)} should be the total size of the
	items copied by the copy constructor/assignment operator.}

   \etabb

\subsection{Description}

An instance of class {\tt AMI\_block<E,I>} is a typed
view of a logical block, which is the unit amount of data transfered
between external storage and main memory. 

The {\tt AMI\_block} class serves a dual purpose: (a) to provide an
interface for seamless transfer of blocks between disk and main memory,
and (b) to provide a structured access to the contents of the block.
The first purpose is achieved through internal mechanisms, transparent
to the user. When creating an instance of class {\tt AMI\_block}, the
constructor is responsible for making the contents of the block
avilable in main memory. When the object is deleted, the destructor is
responsible for writing back the data, if necessary, and freeing the
memory. Consequently, during the life of an {\tt AMI\_block} object, the
contents of the block is available in main memory.
The second purpose is achieved by partitioning the contents of the block
into three fields:
\begin{itemize}
\item[] Links: an array of pointers to other blocks, represented as
block identifiers, of type {\tt AMI\_bid};
\item[] Elements: an array of elements of parameter type {\tt E};
\item[] Info: an info field of parameter type {\tt I}, used to store a 
constant amount of administrative data;
\end{itemize}

The number of elements and links that can be stored is set during
construction: the number of links is passed to the constructor, and the
number of elements is computed using the following formula:
\[\mbox{\rm number\_of\_elements} = \left\lfloor\frac{\mbox{\rm
block\_size} - (\mbox{\tt sizeof}(I) + \mbox{\tt sizeof(AMI\_bid)} *
\mbox{\rm number\_of\_links})}{\mbox{\tt sizeof}(E)}\right\rfloor \]

\subsection{Constructors and Destructor}

   \btabb 

	\entry{\textbf{AMI\_block}(AMI\_COLLECTION *pcoll, unsigned int l, AMI\_bid
	bid);} {{\em Read the block} with id {\tt bid} from block collection
	{\tt *pcoll} in newly allocated memory and format it using the
	template types and the maximum number of links {\tt l}. Persistency
	is set to {\tt PERSIST\_PERSISTENT}.}

	\entry{\textbf{AMI\_block}(AMI\_COLLECTION *pcoll, unsigned int l);} {{\em
	Create a new block} in collection {\tt *pcoll}, allocate memory for
	it, and format it using the template types and the maximum number
	of links {\tt l}. Persistency is set to {\tt
	PERSIST\_PERSISTENT}. The id of the block can be inquired using the
	access member function {\tt bid()}.}

	\entry{\textbf{$\sim$AMI\_block}();} {Destructor. If persistency 
	is {\tt PERSIST\_DELETE}, remove the block from the collection. 
	If it is {\tt PERSIST\_PERSISTENT}, write the block to the 
	collection. Deallocate the memory.}

   \etabb

\subsection{Public Member Objects}

   \btabb

	\entry{b\_vector<E> \textbf{el};} {Access to the elements is done through
	this object, using the public methods of the {\tt b\_vector} class
	(described below).}

	\entry{b\_vector<AMI\_bid> \textbf{lk};} {Access to the links is done
	through this object, using the public methods of the {\tt
	b\_vector} class (described below).}

   \etabb

\subsection{Member Functions}

   \btabb
	
	\entry{I *\textbf{info}();} {Return a pointer to the info element.}

	\entry{AMI\_block<E,I>\& \textbf{operator=}(AMI\_block<E,I>\& B);} {Copy
	block {\tt B} into the current block, if both blocks are associated
	with the same collection. Returns a reference to this block.}

	\entry{AMI\_bid \textbf{bid}() const;} {Return the block id.}

	\entry{char\& \textbf{dirty}();} {Return a reference to the dirty bit. The
	dirty bit is used to optimize writing in some implementations of
	the block collection class. It should be set to $1$ whenever the
	block data is modified. See the implementation details for more.}

	\entry{void \textbf{persist}(persistence p);} {Set the persistency flag to
	{\tt p}. The possible values for {\tt p} are {\tt
	PERSIST\_PERSISTENT} and {\tt PERSIST\_DELETE}.}

	\entry{AMI\_block\_status \textbf{status}() const;} {Return the status of the
	block. The result is either {\tt AMI\_BLOCK\_STATUS\_VALID} or {\tt
	AMI\_BLOCK\_STATUS\_INVALID}. The status of an {\tt AMI\_block}
	instance is set during construction. The methods of an invalid
	block can give erroneous results or fail.}

	\entry{size\_t \textbf{block\_size}() const;} {Return the size of this block
	in bytes.}

	\entry{AMI\_err \textbf{sync}();} {Synchronize the in-memory image of the
	block with the one stored in external storage.}

   \etabb

\subsection{The  b\_vector class}

The {\tt b\_vector} class stores an array of objects of a templated type
{\tt T}. It has a fixed maximum size, or capacity, which is set during
construction (since instances of this class are created only by the
{\tt AMI\_block} class, the constructors are not part of the public
interface). The items stored can be accessed through the array operator.

\subsubsection{Class Declaration}
   \btabb 
	\entry{template<class T> class \textbf{b\_vector};} {The type {\tt T}
	should have a default constructor, as well as copy constructor and
	assignment operator.}
   \etabb

%\subsection{Constructors and Destructor}
%   \begin{tabbing}
%   \hspace*{.3in} \= \hspace{.5in} \= \\ 
%
%      \> {\tt b\_vector(T* p, size\_t cap)}\\
%      \>\>\parbox[t]{5.5in}{Create a b\_vector instance using {\tt p}
%      as the underlying array, of capacity {\tt cap}.}
%
%   \end{tabbing}

\subsubsection{Member Functions}

   \btabb

      \entry{T\& \textbf{operator[]}(size\_t i);}{Return
      a reference to the $i$th item.}

      \entry{const T\& \textbf{operator[]}(size\_t i) const;}{Return a const reference to the $i$th      item.}

      \entry{size\_t \textbf{capacity}() const;}{Return the
      capacity (i.e., maximum number of {\tt T} elements) of this {\tt
      b\_vector}.}

      \entry{size\_t \textbf{copy}(size\_t start, size\_t length,
      b\_vector<T>\& src, size\_t src\_start = 0);}{Copy {\tt length} items from the {\tt src}
      vector, starting with item {\tt src\_start}, to this vector,
      starting with item {\tt start}. Return the number of items
      copied. Source can be {\tt *this}.}

      \entry{size\_t \textbf{copy}(size\_t start, size\_t length, const T* src);}{Copy {\tt length} items from the array {\tt
      src} to this vector, starting in position {\tt start}. Return the
      number of items copied.}

      \entry{void \textbf{insert}(const T\& t, size\_t pos);}{Insert item {\tt t} in position {\tt pos}; all
      items from position {\tt pos} onward are shifted one position higher;
      the last item is lost.}

      \entry{void \textbf{erase}(size\_t pos);}{Erase
      the item in position {\tt pos} and shift all items from position
      {\tt pos+1} onward one position lower; the last item becomes
      identical with the next to last item.}

   \etabb

\index{AMI_block@{\tt AMI\_block}|)}


%%%%%%%% AMI Block Collection %%%%%%%%%
\mysection{Block Collections}
\index{AMI_COLLECTION@{\tt AMI\_COLLECTION}|(}

\subsection{Files}
   \btabb
       \entry{\#include <ami\_coll.h>} {}
   \etabb

\subsection{Class declaration}

   \btabb
	\entry{class \textbf{AMI\_COLLECTION};} {}
   \etabb

\subsection{Description}

A block collection is a set of fixed size blocks. Each block inside the
collection is identified by a block ID, of type \myverb{AMI\_bid}.

\subsection{Constructors and Destructor}

   \btabb
	
	\entry{\textbf{AMI\_COLLECTION}(size\_t lbf = 1);} {Create a {\em new}
	collection with access type {\tt AMI\_WRITE\_COLLECTION} using
	temporary file names. The files are created in a directory given by
	the {\tt AMI\_SINGLE\_DEVICE} environment variable (or {\tt
	``/var/tmp/''} if that variable is not set).  The {\tt lbf}
	(logical block factor) parameter determines the size of the blocks
	stored (the block size is {\tt lbf} times the operating system
	page size). The persistency of the collection is set to {\tt
	PERSIST\_DELETE}.}

	\entry{\textbf{AMI\_COLLECTION}(char *base\_file\_name,
	AMI\_collection\_type t = AMI\_READ\_WRITE\_COLLECTION, size\_t lbf
= 1);} {Create a new or open an 
	existing collection using {\tt base\_file\_name} to find the
	necessary files. The access type is set to {\tt t}. It has one of
	the following values:
	\begin{itemize} 
          \item[]{\tt AMI\_READ\_COLLECTION} Open an existing collection
          read-only;
          \item[]{\tt AMI\_WRITE\_COLLECTION} If the files specified by
          {\tt base\_file\_name} exist, open a collection using those files
          for reading and writing. If the files do not exist, create a new
          collection with read and write acces;
	\end{itemize}
        The {\tt lbf} (logical block factor) parameter determines the size
        of the blocks stored (the block size is {\tt lbf} times the
        operating system page size). The persistency of the collection is
        set to {\tt PERSIST\_PERSISTENT}.}

      \entry{\textbf{$\sim$AMI\_COLLECTION}();} {Destructor.
      Closes all files. If persistency is set to {\tt PERSIST\_DELETE}, it
      also removes the files. There should be no blocks in memory. If the
      destructor detects in-memory blocks, it issues a warning in the TPIE
      log file (if logging is turned on). The memory held by those blocks
      is lost to this program.}

   \etabb

\subsection{Member Functions}

   \btabb

	\entry{bool \textbf{operator!}() const;} {Return \noiverb{true} if
	the status of the collection is not
	\myverb{AMI\_COLLECTION\_STATUS\_VALID}, \noiverb{false}
	otherwise. See also \myverb{is\_valid()} and \myverb{status()}.}

	\entry{size\_t \textbf{block\_factor}() const;} {Return the logical
	block factor. The block size is obtained by multiplying the
	operating system page size by this value.}

	\entry{size\_t \textbf{block\_size}() const;} {Return the size of a block
	stored in this collection, in bytes (all blocks in a collection
	have the same size).}

	\entry{bool \textbf{is\_valid}() const;} {Return \noiverb{true} if the status
	of the collection is \myverb{AMI\_COLLECTION\_STATUS\_VALID}, \noiverb{false}
	otherwise. See also \myverb{status()}.}

	\entry{void \textbf{persist}(persistence p);} {Set the persistency flag to
	{\tt p}. The possible values for {\tt p} are {\tt
	PERSIST\_PERSISTENT} and {\tt PERSIST\_DELETE}.}

	\entry{size\_t \textbf{size}() const;} {Return the number of blocks in the
	collection.}

	\entry{const tpie\_stats\_collection\& \textbf{stats}() const;}
	{Return an object containing the statistics of this instance of the
	collection. The types of statistics computed for a collection are
	tabulated below.\\ \begin{tabular}{|l|l|} \hline \myverb{BLOCK\_GET}
	& Number of block reads\\ \myverb{BLOCK\_PUT} & Number of block
	writes \\ \myverb{BLOCK\_NEW} & Number of block creates\\
	\myverb{BLOCK\_DELETE} & Number of block deletes\\
	\myverb{BLOCK\_SYNC} & Number of block sync operations\\ \hline
	\end{tabular} }

	\entry{AMI\_collection\_status \textbf{status}() const;} {Return
	the status of the collection. The result is either
	\myverb{AMI\_COLLECTION\_STATUS\_VALID} or
	\myverb{AMI\_COLLECTION\_STATUS\_INVALID}. The only operation that
	can leave the collection invalid is the constructor (if that
	happens, the log file contains more information). No blocks should
	be read from or written to an invalid collection.}

	\entry{void *\textbf{user\_data}();} {Return a pointer to a 512-byte array
	stored in the header of the collection. This can be used by the
	application to store initialization information (e.g., the id of
	the block containing the root of a B-tree).}

   \etabb
\index{AMI_COLLECTION@{\tt AMI\_COLLECTION}|)}


%%%%%%%%%% AMI B+-tree %%%%%%%%%%%
\mysection{B+-tree}
\index{AMI_btree@{\tt AMI\_btree}|(}

\subsection{Files}
\btabb
   \entry{\#include <ami\_btree.h>} {}
\etabb

\subsection{Class Declaration}

\btabb
   \entry{template<class Key, class Value, class Compare, class
   KeyOfValue>\\ class \textbf{AMI\_btree};} {}
\etabb

\subsection{Description}

The {\tt AMI\_btree<Key, Value, Compare, KeyOfValue>} class implements the
behavior of a dynamic B+-tree or $(a,b)$-tree storing fixed-size data
items. All data elements (of type {\tt Value}) are stored in the leaves of
the tree, with internal nodes containing keys (of type {\tt Key}) and links
to other nodes. The keys are ordered using the {\tt Compare} function
object, which should define a strict weak ordering (as in the STL sorting
algorithms). Keys are extracted from the {\tt Value} data elements using
the {\tt KeyOfValue} function object.

\subsection{Constructors and Destructor}

\btabb

   \entry{\textbf{AMI\_btree}(const AMI\_btree\_params \&params = btree\_params\_default);}
   {Construct an empty AMI\_btree using temporary files. The tree is stored in a
   directory given by the {\tt AMI\_SINGLE\_DEVICE} environment variable (or {\tt
   "/var/tmp/"} if that variable is not set). The persistency flag is set to
   {\tt PERSIST\_DELETE}. The {\tt params} object contains the
   user-definable parameters (see Appendix for an explanation of the {\tt
   AMI\_btree\_params} class and the default values).}

   \entry{\textbf{AMI\_btree}(const char *bfn, BTE\_collection\_type t =
   BTE\_WRITE\_COLLECTION, const AMI\_btree\_params \&params =
   btree\_params\_default);}{Construct a B-tree
   using the files given by {\tt bfn} (base file name).  The files
   created/used by a Btree instance are outlined in the following
   table.\\[1mm] \begin{tabular}{|l|l|} \hline {\em bfn}{\tt .l.blk} &
   Contains the leaves block collection.\\ \hline {\em bfn}{\tt .l.stk} &
   Contains the free blocks stack for the leaves block collection.\\ \hline
   {\em bfn}{\tt .n.blk} & Contains the nodes block collection.\\ \hline
   {\em bfn}{\tt .n.stk} & Contains the free blocks stack for the nodes
   block collection.\\ \hline \end{tabular}\\[2mm] The persistency flag is
   set to {\tt PERSIST\_PERSISTENT}. The {\tt params} object contains the
   user-definable parameters (see Appendix for an explanation of the {\tt
   AMI\_btree\_params} class and the default values).}

   \entry{\textbf{$\sim$AMI\_btree}();} {Destructor. Either remove or close the supporting
   files, depending on the persistency flag (see method {\tt persist()}).}

\etabb

\subsection{Member functions}

\btabb

   \entry{bool \textbf{erase}(const Key\& k);} {Delete the element with key {\tt k}
   from the tree. Return true if succeded, false otherwise (key not
   found).}

   \entry{bool \textbf{find}(const Key\& k, Value\& v);} {Find an element based on
   its key. If found, store it in {\tt v} and return true.}

   \entry{size\_t \textbf{height}() const;} {Return the height of the tree, including
   the leaf level. A value of $0$ represents an empty tree.}

   \entry{bool \textbf{insert}(const Value\& v);} {Insert an element {\tt v} into the
   tree. Return true if the insertion succeded, false otherwise (duplicate
   key).}

   \entry{bool \textbf{is\_valid}() const;} {Return \noiverb{true} if the status
   of the tree is \myverb{AMI\_BTREE\_STATUS\_VALID}, \noiverb{false}
   otherwise. See also \myverb{status()}.}

   \entry{AMI\_err \textbf{load}(AMI\_STREAM<Value>* is, float lf = 0.7,
   float nf = 0.5)} {Bulk load from the stream {\tt is} of elements. Leaves
   are filled to {\tt lf}$\times$capacity, and nodes are filled to {\tt
   nf}$\times$capacity.}

   \entry{AMI\_err \textbf{load}(AMI\_btree<Key, Value, Compare,
   KeyOfValue>* bt, float leaf\_fill = .7, float node\_fill =
   .5);}{Bulk load from another B-tree. This is a means of reoganizing a
   B-tree after a lot of updates. A newly loaded structure may use less
   space and may answer range queries faster.}

   \entry{AMI\_err \textbf{load\_sorted}(AMI\_STREAM<Value>* is, float lf =
   0.7, float nf = 0.5);} {Same as {\tt load()} above, but bypasses the
   expensive sorting step, by assuming that the stream {\tt is} is sorted.}

   \entry{const AMI\_btree\_params\& \textbf{params}() const;} {Return a const
   reference to the {\tt AMI\_btree\_params} object used by the B-tree. This
   object contains the true values of all parameters (unlike the object
   passed to the constructor, which may contain $0$-valued parameters to
   indicate default behavior; see Section~\ref{ssec:params} below).}

   \entry{void \textbf{persist}(persistence p);} {Set the persistency flag to {\tt
   p}. The persistency flag dictates the behavior of the destructor of
   this AMI\_btree object. If {\tt p} is {\tt PERSIST\_DELETE}, all files
   associated with the tree will be removed, and all the elements stored in
   the tree will be lost after the destruction of this AMI\_btree object. If
   {\tt p} is {\tt PERSIST\_PERSISTENT}, all files associated with the tree
   will be closed during the destruction of this AMI\_btree object, and all the
   information needed to reopen this tree will be saved.}

   \entry{bool \textbf{pred}(const Key\& k, Value\& v);} {Find the highest
   element stored in the tree whose key is lower than \noiverb{k}. If such
   an element exists, return \noiverb{true} and store the result in
   \noiverb{v}. Otherwise, return \noiverb{false}.}

   \entry{void \textbf{range\_query}(const Key\& k1, const Key\& k2,
   AMI\_STREAM<Value>* os);} {Find all elements within the range given by
   keys {\tt k1} and {\tt k2} and write them to stream {\tt os}.}

   \entry{size\_t \textbf{size}() const;} {Return the number of elements
   stored in the leaves of this tree.}

   \entry{AMI\_err \textbf{sort}(AMI\_STREAM<Value>* is,
   AMI\_STREAM<Value>* \&os);} {As a convenience, this function sorts the
   stream {\tt is} and stores the result in {\tt os}. If the value of {\tt
   os} passed to the function is {\tt NULL}, a new stream is created and
   {\tt os} points to it.}

   \entry{AMI\_btree\_status \textbf{status}() const;} {Return the status
   of the collection. The result is either
   \myverb{AMI\_BTREE\_STATUS\_VALID} or
   \myverb{AMI\_BTREE\_STATUS\_INVALID}. The only operation that can leave
   the tree invalid is the constructor (if that happens, the log file
   contains more information).}

   \entry{bool \textbf{succ}(const Key\& k, Value\& v);} {Find the lowest
   element stored in the tree whose key is higher than \noiverb{k}. If such
   an element exists, return \noiverb{true} and store the result in
   \noiverb{v}. Otherwise, return \noiverb{false}.}

   \entry{AMI\_err \textbf{unload}(AMI\_STREAM<Value>* s);} {Write all
   elements stored in this tree to the given stream, in sorted order. No
   changes are performed on the tree.}

%   \> {\tt bool defragment()}\\ \>\>\parbox[t]{5.5in}{Rearrange the nodes
%   and leaves of the tree so that they take the minimum disk space. This is
%   a very time-consuming operation. Return true if completed
%   successfully.}\\[3mm]
\etabb

\subsection{The {\tt AMI\_btree\_params} Class}\label{ssec:params}
\index{AMI_btree_params@{\tt AMI\_btree\_params}|(}
The {\tt AMI\_btree\_params} class encapsulates all user-definable B-tree
parameters. These parameters dictate the layout of the tree and its
behavior under insertions and deletions. An instance of the class created
using the default constructor gives default values to all parameters. Each
paramter can then be changed independently.

\subsubsection{Class Declaration}

\btabb

  \entry{class \textbf{AMI\_btree\_params};} {}

\etabb

\subsubsection{Constructor}

\btabb

  \entry{\textbf{AMI\_btree\_params}()} {Initialize a {\tt Btree\_params} object with
  default values. The default values are given in the following table.\\[1mm]
  \begin{tabular}{|l|c|}
    \hline
    {\em Parameter} & {\em Value} \\ \hline
    {\tt leaf\_size\_min} & 0 \\ \hline
    {\tt node\_size\_min} & 0 \\ \hline
    {\tt leaf\_size\_max} & 0 \\ \hline
    {\tt node\_size\_max} & 0 \\ \hline
    {\tt leaf\_block\_factor} & 1 \\ \hline
    {\tt node\_block\_factor} & 1 \\ \hline
    {\tt leaf\_cache\_size} & 5 \\ \hline
    {\tt node\_cache\_size} & 10 \\ \hline
  \end{tabular}
  }

\etabb

\subsubsection{Public Member Objects}

\btabb

  \entry{size\_t \textbf{leaf\_size\_min}} {Minimum number of elements in a leaf. A
  value of $0$ tells the class to use the default B+-tree behavior. This
  parameter is a guideline. To improve performance, some leaves may have
  fewer elements.}

  \entry{size\_t \textbf{node\_size\_min}} {Minimum number of keys in an internal
  node. A value of $0$ tells the class to use the default B+-tree
  behavior. As above, this parameter is a guideline.}

  \entry{size\_t \textbf{leaf\_size\_max}} {Maximum number of elements in a leaf. A
  value of $0$ tells the class to fill a leaf to capacity. This value is
  strictly enforced.}

  \entry{size\_t \textbf{node\_size\_max}} {Maximum number of keys in an internal
  node. A value of $0$ tells the class to fill a node to capacity. This
  value is strictly enforced.}

  \entry{size\_t \textbf{leaf\_block\_factor}} {The size (in bytes) of a leaf block
  is {\tt leaf\_block\_factor$\times$ os\_block\_size}, where {\tt
  os\_block\_size} is the operating-system specific page size.}

  \entry{size\_t \textbf{node\_block\_factor}} {The size (in bytes) of an internal
  node block is {\tt node\_block\_factor$\times$ os\_block\_size}.}

  \entry{size\_t \textbf{leaf\_cache\_size}} {The size (in number of leaf blocks) of
  the leaf block cache. The cache implements an LRU replacement policy.}

  \entry{size\_t \textbf{node\_cache\_size}} {The size (in number of node blocks) of
  the node block cache. The cache implements an LRU replacement policy.}

\etabb
\index{AMI_btree_params@{\tt AMI\_btree\_params}|)}
\index{AMI_btree@{\tt AMI\_btree}|)}

%%%%%%%% AMI Cache Manager %%%%%%%%%%
\mysection{Cache Manager}

\subsection{Files}
   \btabb
      \entry{\#include <ami\_cache.h>} {}
   \etabb

\subsection{Class Declaration}
   \btabb
      \entry{template<class T, class W> class \textbf{AMI\_CACHE\_MANAGER};} {}
   \etabb

\subsection{Description}

\subsection{Constructors and Destructor}
   \btabb
      \entry{\textbf{AMI\_CACHE\_MANAGER}(size\_t capacity);}{Construct a fully-associative cache manager with the given capacity.}
      \entry{\textbf{AMI\_CACHE\_MANAGER}(size\_t capacity, size\_t assoc);}{Construct a cache manager with the given capacity and associativity.}
      \entry{\textbf{$\sim$AMI\_CACHE\_MANAGER}();}{Destructor. Write out all items still in the cache.}
   \etabb

\subsection{Member Functions}
   \btabb
       \entry{bool \textbf{read}(size\_t k, T \& item);}{Read an item from the cache based on key {\tt k} and store it in {\tt item}. If found, the item is removed from the cache. Return true if the key was found.}
      \entry{bool \textbf{write}(size\_t k, const T \& item);}{Write an item in the cache based on the given key {\tt k}. If the cache was full, the least recently used item is writen out using the {\tt W} function object, and it is removed from the cache.}
      \entry{bool \textbf{erase}(size\_t k);}{Erase an item from the cache based on the given key {\tt k}. Return true if the key was found.}
   \etabb
