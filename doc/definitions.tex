%%
%% $Id: definitions.tex,v 1.6 2005-11-23 15:14:06 adanner Exp $
%%

\hbadness=10000

% For text shading.
\definecolor{lgray}{gray}{.85}
% Section command that displays the section name on a light gray background.
%\newcommand{\mysection}[1]{\penalty-600\vspace*{12mm}\noindent%
%\colorbox{lgray}{\rule{0cm}{4.7mm}\rule{\textwidth}{0cm}}%
%\vspace*{-13.3mm}\section{#1}\penalty+600}
%\newcommand{\mysection}[1]{\section{#1}}
\makeatletter
\newcommand\mysection{\@startsection {section}{1}{\z@}%
                                   {-3.5ex \@plus -1ex \@minus -.2ex}%
                                   {2.3ex \@plus.2ex}%
                                   {\normalfont\Large\bfseries\hspace*{-9mm}\colorbox{lgray}{\rule{0cm}{4.7mm}\rule{15.1cm}{0cm}}\hspace{-15.1cm} }}
\makeatother

% plabel allows the label value to be printed for easier
% writing of the manual
%\newcommand{\plabel}[1]{{\tiny #1}\label{#1}}
\newcommand{\plabel}[1]{\label{#1}}

%%%
%%%  The following macros are obsolete due to switching to
%%%  the listings package
%%% 

%% % use this instead of \verb, which is not allowed as parameter [tavi]
%% %app_config@{\tt app\_config.h}}
%% %\newcommand{\myverb}[1]{\texttt{#1}\index{#1@{\tt #1}}}
%% %\newcommand{\myv}[1]{\texttt{#1}\index{#1@{\tt #1}}}
%% %%\newcommand{\myverb}[1]{\noiv{#1}\index{#1@{\small\tt #1}}}
%%\newcommand{\myv}[1]   {\noiv{#1}\index{#1@{\small\tt #1}}}

%% % use this instead of \myverb if you don't want its
%% % parameter in the index
%% %\newcommand{\noiverb}[1]{\texttt{#1}}
%% %\newcommand{\noiv}[1]{\texttt{#1}}
%% \newcommand{\noiverb}[1]{{\small \tt #1}}
%% \newcommand{\noiv}[1]{{\small \tt #1}}

%%%
%%% End
%%% 

% emphasize and put into index
\newcommand{\emphd}[1]{\emph{#1}\index{#1}}

% used for algorithms
\newcommand{\step}[2] {\begin{enumerate}\item[#1]#2\end{enumerate}}

% for the reference manual [tavi]
\newcommand{\entry}[2]{\> \parbox[t]{6.3in}{{\ttfamily #1}}\\ \>\>\parbox[t]{5.5in}{#2}\\[3mm]}
\newcommand{\btabb}{\begin{tabbing} \hspace*{.3in} \= \hspace{.5in}\=\\ }
\newcommand{\etabb}{\end{tabbing}\vspace*{-12mm}}

%%%
%%%  The following macros are obsolete due to switching to
%%%  the listings package
%%% 

%% \makeatletter    % '@' is now a normal letter for TeX
%% % this makes verbatim text smaller and indented [tavi]
%% \def\verbatim@startline{\small%
%%   \def\verbatim@startline{\hspace*{5mm}\verbatim@line{}}%
%%   \verbatim@startline}
%% % remove some space before the text
%% \addto@hook\every@verbatim{\vspace*{-1mm}}
%% % remove some space after the text
%% \def\verbatim@finish{\def\verbatim@finish{\ifcat$\the\verbatim@line$\else%
%%  \verbatim@processline\fi}\vspace*{-4mm}\verbatim@finish}
%% \makeatother    % '@' is restored as a non-letter character

%%%
%%% End
%%% 

% normal margins for US-size paper
\setlength{\topmargin}{-.5in}   
\setlength{\oddsidemargin}{.175in} % distance from left edge of page to text
\setlength{\evensidemargin}{.175in} % distance from left edge of page to text
\setlength{\textwidth}{6.1in}
\setlength{\textheight}{9in}


%% Macro for writing in the margin comments on what is left to be done.
%% Use \withcomments in the preamble if you want comments to appear.
\def\comment#1{}
\def\withcomments{
% Set \marginparwidth to ensure comment does not runs off the end of the page. 
\setlength{\marginparwidth}{8.5in}
\addtolength{\marginparwidth}{-1.0in}
\addtolength{\marginparwidth}{-\oddsidemargin}
\addtolength{\marginparwidth}{-\textwidth}
\addtolength{\marginparwidth}{-2.0\marginparsep} 
% To get the same space on both sides of the margin text
% because Duke printers use weird margins:
\addtolength{\marginparwidth}{-0.3in} %was -0.125
\newcounter{mycomments}
\def\comment##1{\refstepcounter{mycomments}%
\ifhmode%
\unskip%
{\dimen1=\baselineskip \divide\dimen1 by 2 %
\raise\dimen1\llap{\tiny -\themycomments-}}\fi%
\marginpar{\tiny [\themycomments]: ##1}}%
}




% Additions to makeidx.sty
%\makeatletter
%\@ifundefined{alsoname}%
%   {\def\alsoname{also}}{}

%\def\seealso#1#2{{\em \seename\ \alsoname\/} #1}
%\makeatother

% This manual applies to the following version of TPIE
\newcommand{\edition}{082902}
\newcommand{\version}{082902}
% minimum GNU release required
\newcommand{\gxxversion}{2.95}
% current GNU release we use for development
\newcommand{\gxxcurrent}{2.95}

\newcommand{\tobewritten}{\vspace{\baselineskip}$<$TO BE WRITTEN$>$\vspace{\baselineskip}}
\newcommand{\tobeextended}{\vspace{\baselineskip}$<$TO BE EXTENDED$>$\vspace{\baselineskip}}

%%%
%%% Setup for printing C++ code using the listings package
%%% 

\DeclareFontShape{OT1}{cmtt}{bx}{n}
     {<5><6><7><8>cmbtt8%
      <9>cmbtt9%
      <10><10.95>cmbtt10%
      <12><14.4><17.28><20.74><24.88>cmbtt10%
      }{}

\lstset{language=[ANSI]C++}
\lstset{basicstyle=\ttfamily}
\lstset{showstringspaces=false}
\lstset{numbers=none}
\lstset{numberstyle=\tiny}
\lstset{stepnumber=10}
\lstset{captionpos=b}

\makeatletter
\providecommand{\toclevel@lstlisting}{1}
\makeatother

\newcommand{\CPP}{\texttt{C++}}

%%%
%%% End
%%%

%%% Local Variables: 
%%% mode: latex
%%% TeX-master: "tpie"
%%% End: 
