%%
%% $Id: implementation.tex,v 1.22 2006-01-16 23:37:49 jan Exp $
%%
\chapter{The Implementation of TPIE}
\plabel{cha:implementation}

\tobewritten(Block oriented part of TPIE)\comment{LA:
  Should there be something about progress bar in theis chapter?}

\section{The Structure of TPIE}
\index{structure!of TPIE} \index{components!of TPIE}

TPIE\comment{LA: We should have something like this in beginning of
  tutorial} has three main components, the Access Method Interface
(AMI)\index{access method interface}, a Block Transfer Engine
(BTE)\index{block transfer engine} component, and a Memory Manager
(MM)\index{memory manager} component. Various Block Transfer Engines
(BTEs) can be chosen for handling disk block transfers, perhaps more
than one in a single application.\comment{LA: really?}  The MM
component provides low level memory management services such as
allocating, deallocating, and accounting of internal memory. The AMI
works on top of the Memory Manager and one or more BTEs to provide a
uniform interface for application programs. Applications that use this
interface are portable across hardware platforms, since they never
have to deal with the underlying details of how I/O is performed on a
particular machine. This chapter describes the design decisions,
algorithms, and implementation decisions that were used to build the
MM, BTE and AMI components of TPIE. The Reference Section contains a
description of the AMI and Memory Manager entry points that an
application programmer might normally use. Typically, an application
programmer will not request services from a BTE directly. For this
reason, the BTE services are not presented in the Reference Section of
this manual, but are presented here for those readers who wish to
understand the implementation details of TPIE.


%Although the BTE runs on a single processor, it can support concurrent
%access to multiple disks\index{parallel disks}, allocating and managing
%buffer space for all of them concurrently.

The MM\index{memory manager} manages random access memory on behalf of
TPIE.
%It is the most architecture-dependent component of the
%system. 
Currently, TPIE is distributed with an MM designed for a single
processor, or multiprocessor system with a single global address
space. This MM is relatively simple; its task is to allocate and
manage the physical memory used by the BTE component.

%On a distributed memory system, the MM will have the
%additional task of coordinating communication between processors and memory
%modules in order to support the primitives that the AMI provides.

The AMI\index{access method interface} is an interface layer between
the BTE and user level processes.  It implements fundamental access
methods, such as scanning, permutation routing, merging, and
distribution. It also provides a consistent, object-oriented interface
to application programs.
%The details of how these access methods are implemented depends on the
%hardware on which the system is running.  For example, recursive
%distribution will be done somewhat differently on a parallel disk machine
%than on a single disk machine. The AMI abstracts this fact away, allowing
%an application program that calls a function such as
%\lstinline|AMI_partition_and_merge()|\index{AMI\_partition\_and\_merge@{\tt
%AMI\\_partition\\_and\_merge}} to work correctly regardless of the underlying
%I/O system.
The key to keeping the AMI simple and flexible is the fact that its
user accessible functions serve more as templates for computation than
as actual problem solving functions.  The details of how a computation
proceeds within the template is up to the application programmer, who
is responsible for providing the functions that the template applies
to data.

\section{The Block Transfer Engine (BTE)}
\plabel{sec:ref-bte}
\index{block transfer engine|(}
\index{BTE|see{block transfer engine}}

\comment{LA: Jan please check this section}

\comment{LA: This whole section needs an overhaul - discuss
  implementation much more}

The BTE component is intended to bridge the gap between the I/O
hardware and the rest of the system. It is the layer that is
ultimately responsible for moving blocks of data from physical disk
devices to main memory and back. It works alongside the traditional
buffer cache\index{buffer cache} in a UNIX system.  Unlike the buffer
cache, which must support concurrent access to files from multiple
address spaces, the BTE is specifically designed to support high
throughput processing of data from secondary memory through a single,
user level address space. In order to efficiently support the merging,
distribution, and scanning paradigms, several BTEs support
stream-oriented input and output of data blocks. To further improve
performance, some BTE implementations move data from disk directly
into user space rather than using a kernel-level buffer
cache\index{buffer cache}.  This saves both main memory space and
copying time.\footnote{TPIE also provides a BTE implementation with
support for random-access to disk blocks. Such functionality is useful
when implementing external data structures (indexes). This BTE will be
documented in a future version of this manual.}\comment{JV: Do you mean
the block collection BTE?}
 
%We hope that in most cases it will be possible for the BTE to work with
%device drivers provided by the machine vendor's operating system.  In some
%cases, however, new drivers may have to be written.
%  The BTEs will not, however, be responsible for coordinating the
% actions of multiple CPU's and the disks attached to them. In such cases, a
% separate instance of a BTE will run on each such CPU, and their actions
% will be coordinated by a single multi-threaded MM running at a higher
% level. The reason for the functional split between the two levels is that
% it will likely be advantageous to be able to use a single BTE written for a
% specific piece of hardware with more than one MM, for example, one MM
% written for a homogeneous environment and one for a heterogeneous
% environment. 

Streams in TPIE are implemented as sequentially accessed files, and
each BTE offers public member functions that are analogous to, and
implemented via corresponding functions available in the underlying
\emphd{file access method}. The main difference is that TPIE maintains
a typed view of streams, where user-defined data elements (i.e.
objects) are accessed in a stream, rather than the untyped view of the
data offered by the corresponding file access primitives.\comment{LA:
  Some intro about streams, files, etc - like in AMI reference.  Also
  something later about; block access, read ahead, double buffering,
  logical blocks, file pointer, stream types, persistence, etc.?}


Version \version~of TPIE supports the following three BTE stream
implementations:
\begin{enumerate}
\item \lstinline|BTE_stream_stdio|, which uses the
  \lstinline|stdio| library as its file access method;
    
\item \lstinline|BTE_stream_ufs|, which performs I/O via 
  \lstinline|read()/write()| system calls;
    
\item \lstinline|BTE_stream_mmap|, which uses memory-mapping to
  perform I/Os.
%%  \lstinline|mmap()/munmap()| calls to perform I/O.
\end{enumerate}

The implementation of \lstinline|BTE_stream_stdio| can be regarded as
an external direct-access array, since the position of the $i$-th
element in the stream is given by 
\lstinline|sizeof(BTE_stream_header) + i*sizeof(T)| 
where \lstinline|T| is the type of objects stored in
the stream. In this implementation, blocking disk accessed is
completely at the hands of the operating system. 

The other two implementation use (logically) blocked I/O, i.e., they
allocate a larger chunk of memory (\emph{logical block}), place the
objects to be written into this memory chunk, and then transfer it to
disk with a single I/O. Since elements cannot be cross block
boundaries, some space inside a block may be wasted, if
\lstinline|sizeof(T)| does not divide the logical block size.  The two
implementations differ in that \lstinline|BTE_stream_ufs| uses system
calls which in turn are buffered by the operating system, whereas
\lstinline|BTE_stream_mmap| directly maps files (or blocks thereof)
into main memory thus bypassing the buffer cache.

The choice of BTE in a given application is controlled by compile-time
variables in the \path"app_config.h" file.  The
\lstinline|BTE_stream_stdio| implementation is selected by defining
the variable \lstinline|BTE_STREAM_IMP_STDIO|, the
\lstinline|BTE_stream_ufs| implementation by defining
\lstinline|BTE_STREAM_IMP_UFS|, and the memory mapped implementation
by defining \lstinline|BTE_STREAM_IMP_MMAP|.  For example, the
following code selects the \lstinline|BTE_stream_ufs| Block Transfer
Engine (and does not select the others):

\begin{lstlisting}
//#define BTE_STREAM_IMP_MMAP
//#define BTE_STREAM_IMP_STDIO
#define BTE_STREAM_IMP_UFS
\end{lstlisting}

% If the including module did not explicitly ask for multiple
% implementations but requested more than one implementation,
% issue a warning.  
Multiple implementations are allowed to coexist, with some
restrictions, e.g. declarations of streams must use explicit
subclasses of \lstinline|AMI_stream_base|\comment{LA: BTE?} to specify
what type of streams they are. \comment{LA: This needs to be extended
  somewhere.}  \comment{DH: I am skeptical that it works with multiple
  impl., but I don't see why we need this feature with single disks.
  Multiple impls. in an app may be possible with separate compiles? We
  may need more flexibility with multi disk implementations however.}
The best choice of BTE for a given application is both application and
system dependent.  Section~\ref{sec:choosingbte} discuss how to choose
an appropriate BTE (also refer to the more detailed descriptions of
the BTE's in the next three subsections). If none of the available
BTEs is selected, \lstinline|BTE_STREAM_IMP_UFS| is defined by default
and a warning is generated at compile time.\comment{LA: Which BTE is
  default?}


%The\comment{LA: This is out of context} first block in a TPIE stream,
%referred to as the \emphd{header block}\index{streams!header block},
%contains contextual information for the stream. Currently, the format
%of the header block is BTE-dependent. Please refer to the descriptions
%of the individual BTEs in
%Sections~\ref{sec:ref-bte-stdio},\ref{sec:ref-bte-mmap}, and
%\ref{sec:ref-bte-ufs} for more details.

\subsection{BTE Common Functionality}
\plabel{sec:imp-bte-base}

All BTE stream implementations inherit from class
\lstinline|BTE_stream_base| (in file \path"include/bte_stream_base.h")
and must support the member functions listed below.

%A stream in TPIE is implemented via a file, and the
%individual BTE choices available correspond to different
%access methods for such a file. 

As the details of each access method (i.e.
\lstinline|BTE_stream_ufs|, \lstinline|BTE_stream_stdio| or
\lstinline|BTE_stream_mmap| are different, the BTEs generally have
different member function implementations, including constructors
(construction involves opening the appropriate stream file, etc.).
The implementations of the access methods \lstinline|BTE_stream_ufs|,
\lstinline|BTE_stream_stdio|, and \lstinline|BTE_stream_mmap| are
defined in the respective files \path"include/bte_stream_ufs.h",
\path"include/bte_stream_stdio.h", and
\path"include/bte_stream_mmap.h".  Functionally, the member functions
for each BTE are similar and are outlined in the next subsection.

%The common BTE member functions
%\footnote{These functions can be
%  defined as \texttt{virtual} to assist in debugging a new BTE if the
%  compile-time variable \texttt{BTE\_VIRTUAL\_BASE} is defined to be
%  non-zero (see also file \texttt{app\_config.h}). For best run-time
%  performance, the default value for \texttt{BTE\_VIRTUAL\_BASE} is
%  zero.  }  
%and their usage are outlined below:

\paragraph{Header Information} In addition to the data elements stored
in a stream, a TPIE stream contains a ``header
block''\index{streams!header block} which contains BTE-dependent
information, as well as context information such as the length of the
stream, the size of each item in the stream, etc.  Furthermore, to
facilitate debugging, the stream header also records the type of BTE
that was used to create the stream. The complete listing of all
information stored in a BTE header block is given below:

\lstinputlisting[numbers=left,basicstyle=\ttfamily\small,caption={The
  definition of the header used for BTE streams.}]{../tpie/bte/stream_header.h}

\vspace*{\baselineskip}
\subsubsection{Stream Constructors}

\begin{lstlisting}
    BTE_stream_mmap  ( const char *dev_path, BTE_stream_type st );
    BTE_stream_stdio ( const char *dev_path, BTE_stream_type st );
    BTE_stream_ufs   ( const char *dev_path, BTE_stream_type st );
\end{lstlisting}

\noindent
These constructors each create a stream which is backed by the file
whose path is pointed to by \lstinline|dev_path|.  The stream can be
opened in one of several modes, depending on the value of
\lstinline|st|:

\begin{itemize}
  
\item \lstinline|BTE_READ_STREAM|: the stream can be read, but not
  written. The underlying file must have been created previously in
  this case.
\item \lstinline|BTE_WRITE_STREAM| or
  \lstinline|BTE_WRITEONLY_STREAM|: the stream can be written, but not
  read. This will allow a new stream to be created or overwrite a
  previously-created one.
  
\item \lstinline|BTE_APPEND_STREAM|: the stream is opened, and
  positioned to the end of stream so that new data can be appended.
  This is valid only if the stream was previously created.
\end{itemize}

Each constructor also takes an optional, third argument
\lstinline|lbf| of type \lstinline|TPIE_OS_OFFSET| that denotes the
so-called \emph{logical block factor} and indicates of how many
consecutive (physical) disk blocks a logical block is composed. This
argument defaults to 1; it is not used in the
\lstinline|BTE_stream_stdio| implementation.

\vspace*{\baselineskip}
\subsubsection{new\_substream}\index{new_substream()@{\tt new\_substream()}}
\index{substreams!of BTE streams}

\begin{lstlisting}
    BTE_err new_substream(BTE_stream_type st, 
                          TPIE_OS_OFFSET sub_begin, 
                          TPIE_OS_OFFSET sub_end, 
                          BTE_stream_base<T> **sub_stream);
\end{lstlisting}

\noindent
Constructs a substream of the current stream.  A substream is a
logical contiguous subset of a stream, i.e. from the implementation
point of view, it is a contiguous subset of the data elements in a
file.  The parameters of \lstinline|new_substream()| are as follows:
\begin{itemize}
\item \lstinline|sub_begin| and \lstinline|sub_end| are object offsets
  into the current stream, indicating the limits of the new substream.
    
\item \lstinline|st| is the stream type of the new substream. Please
  refer to the discussion of substream constructors above for a list
  of the valid (sub)stream types.
  
\item \lstinline|sub_stream| will be given the address of a pointer to
  the new substream.
\end{itemize}

While it might seem logical to explicitly define a substream in TPIE
as inheriting (in the \CPP{} sense) from a stream, this would imply a
need for both to have a constructor, and hence for the stream
constructor to be \lstinline|virtual|. This is technically impossible
in \CPP{},\comment{LA: why?}  and so the ``pseudo-constructor''
approach described here is used instead.\comment{LA: Extend like in
  AMI ref.  Stream type?} Note that, in order to provide a generic
interface, that may be used, e.g., from within the AMI, the type of
the ``return value'' \lstinline|sub_stream| needs to be
\lstinline|BTE_stream_base<T> **|, so the user will need to
(dynamically) cast it to the appropriate type.
% Because \lstinline|new_substream()| is not a
%constructor, but rather a function each particular implementation of
%which calls an appropriate constructor, it can be a pure virtual
%function in the stream base class, which forces it to be defined for
%all actual stream implementations. \comment{LA: I don't understand the
%  point of this paragraph.}

\vspace*{\baselineskip}
\subsubsection{read\_item}\index{read_item()@{\tt read\_item()}!BTE}

\begin{lstlisting}
    BTE_err read_item(T **elt);
\end{lstlisting}

\noindent
Update the pointer pointed to by \lstinline|elt| to reference the next
element in the stream. Since the application data elements are
blocked, this often does not require a physical I/O operation. If the
current block has been exhausted, the first element of the next block
will be accessed. In some cases the BTE will try to prefetch the next
block and it may already be in memory. (See descriptions of the
individual BTEs for more details about prefetching or read-ahead.)

\vspace*{\baselineskip}
\subsubsection{write\_item}\index{write_item()@{\tt write\_item()}!BTE}

\begin{lstlisting}
    BTE_err write_item(const T &elt);
\end{lstlisting}

\noindent
Writes the data element \lstinline|elt| to the current position in the
stream. Streams are normally written in sequential order, and so the
current position is usually directly after the previous element
written. The current position (which we will refer to as the value of
the \emphd{file pointer}) can be modified via the \lstinline|seek|
command below.

\vspace*{\baselineskip}
\subsubsection{seek}\index{seek()@{\tt seek()}!BTE}

\begin{lstlisting}
    BTE_err seek(TPIE_OS_OFFSET offset);
\end{lstlisting}

\noindent
Seeks to the object offset \lstinline|offset| in the stream. The BTE
\lstinline|seek| member function is similar to and utilizes the
\lstinline|seek| function supported by the underlying file I/O system
being used. The difference is that TPIE performs a seek to the
requested application data element in the stream rather than to a byte
offset within a file of untyped data.

\vspace*{\baselineskip}
\subsubsection{tell}\index{seek()@{\tt tell()}!BTE}

\begin{lstlisting}
    TPIE_OS_OFFSET tell() const;
\end{lstlisting}

\noindent
Returns the file pointer of the current streams in terms of items.


\vspace*{\baselineskip}
\subsubsection{truncate}\index{truncate()@{\tt truncate()}!BTE}

\begin{lstlisting}
    BTE_err truncate(TPIE_OS_OFFSET offset);
\end{lstlisting}

\noindent
Truncates/extends the stream to the specified number of application
data elements. The file pointer will be moved to the end of the
stream. This is analogous to the \lstinline|truncate| function of the
underlying file access method.  Note: \lstinline|truncate| is not
supported for substreams.

\vspace*{\baselineskip}
\subsubsection{main\_memory\_usage}
\index{main_memory_usage()@{\tt main\_memory\_usage()}!BTE}

\begin{lstlisting}
    BTE_err main_memory_usage(TPIE_OS_SIZE_T *usage,
                              MM_stream_usage usage_type) const;
\end{lstlisting}

\noindent
Queries how much memory is used by the stream for
\lstinline|usage_type| purposes. On return, \lstinline|usage| points
to the value used. The valid values for \lstinline|usage_type| are as
follows:

\begin{itemize}
\item \lstinline|MM_STREAM_USAGE_OVERHEAD|: the size in bytes of the
  stream object plus, if this is not a substream, the stream header
  block.
\item \lstinline|MM_STREAM_USAGE_BUFFER|: the number of bytes consumed
  by block buffers (blocks of elements in main memory).
\item \lstinline|MM_STREAM_USAGE_CURRENT|:
  \lstinline|MM_STREAM_USAGE_OVERHEAD| plus the number of bytes
  currently consumed by block buffers for this stream.
\item \lstinline|MM_STREAM_USAGE_MAXIMUM|:
  \lstinline|MM_STREAM_USAGE_OVERHEAD| plus the maximum number of
  bytes that will be consumed by block buffers for this stream.
\item \lstinline|MM_STREAM_USAGE_SUBSTREAM|: The same as
  \lstinline|MM_STREAM_USAGE_OVERHEAD| but assumes that this is a
  substream.
\end{itemize}

\vspace*{\baselineskip}
\subsubsection{status}\index{status()@{\tt status()}}

\begin{lstlisting}
    BTE_stream_status status() const;
\end{lstlisting}

\noindent
Returns the status of the stream as one of the following
values:  \index{BTE\_STREAM\_STATUS_*@{\tt
    BTE\_STREAM\_STATUS\_*}}

\begin{itemize}
\item \lstinline|BTE_STREAM_STATUS_NO_STATUS|: No status information
  exists for the stream.
\item \lstinline|BTE_STREAM_STATUS_INVALID|: An error was encountered
  while manipulating the stream and the stream can no longer be used.
%UNUSED \item \lstinline|BTE_STREAM_STATUS_EOS_ON_NEXT_CALL|:
\item \lstinline|BTE_STREAM_STATUS_END_OF_STREAM|: The end of stream
  was encountered.
\end{itemize}

%The \lstinline|get_status| member function is one of the few that is
%implemented in \path"include/bte_stream_base.h" and its implementation
%is therefore common to all BTEs. The status is stored in the private
%variable \lstinline|status| of a BTE object.

\vspace*{\baselineskip}
\subsubsection{stream\_len}\index{stream_len()@{\tt stream\_len()}!BTE}

\begin{lstlisting}
    TPIE_OS_OFFSET stream_len() const;
\end{lstlisting}

\noindent
Returns the current number of application data elements in the stream.


\vspace*{\baselineskip}
\subsubsection{name}\index{name()@{\tt name()}!BTE}

\begin{lstlisting}
    BTE_err name(char **stream_name) const;
\end{lstlisting}

\noindent
Returns the path name of the file backing the stream. The name will be
stored in newly allocated space.

\vspace*{\baselineskip}
\subsubsection{read\_only}\index{read_only()@{\tt read\_only()}}

\begin{lstlisting}
  bool read_only() const;
\end{lstlisting}

\noindent
Returns true if the stream mode is \lstinline|BTE_READ_STREAM|. The
file mode is set when creating a stream. See BTE constructors, above
for details.

\vspace*{\baselineskip}
\subsubsection{available\_streams}
\index{available_streams()@{\tt available\_streams()}!BTE}

\begin{lstlisting}
    int available_streams() const;    
\end{lstlisting}

\noindent
Returns the number of additional streams that can be activated (i.e.
created) before an operating system limit on the number of open files
would be exceeded. This reflects only the operating system limit on
number of open files. The TPIE memory requirements for active streams
are not reflected in this limit.

\vspace*{\baselineskip}
\subsubsection{chunk\_size}
\index{chunk_size()@{\tt chunk\_size()}!BTE}

\begin{lstlisting}
    TPIE_OS_OFFSET chunk_size() const;
\end{lstlisting}

%Not clear what this does...
\noindent
Returns the number of application data elements that fit in
a logical block of the current stream.

\vspace*{\baselineskip}
\subsubsection{os\_block\_size}
\index{os_block_size()@{\tt os\_block\_size()}!BTE}

\begin{lstlisting}
    TPIE_OS_OFFSET os\_block\_size() const;
\end{lstlisting}

\noindent
Returns the number of bytes in one physical disk block.

\vspace*{\baselineskip}
\subsubsection{persist}\index{persistence!BTE}
\index{persist()@{\tt persist()}|see{persistence}}

\begin{lstlisting}
    void persist(persistence per);
\end{lstlisting}

\noindent
Sets the persistence attribute of the current stream to the value of
\lstinline|persistence|. This may be one of the following values:
\begin{itemize}
\item \lstinline|PERSIST_DELETE|: Delete the stream from the disk when
  the stream destructor is called.
\item \lstinline|PERSIST_PERSISTENT|: Do not delete the stream from the disk when
  the stream destructor is called.
%\item \lstinline|PERSIST_READ_ONCE|: Delete each block of data from the disk as
%  it is read.
\end{itemize}

By default, all streams are deleted when their destructors are called.

\vspace*{\baselineskip}
\subsubsection{persist}

\begin{lstlisting}
    persistence persist() const;
\end{lstlisting}

\noindent
Returns the persistence attribute of the current stream.


\vspace*{\baselineskip}
\subsubsection{stats}
\index{stats()@{\tt stats()}!BTE}

\begin{lstlisting}
    const tpie_stream_stats& stats() const;
\end{lstlisting}

\noindent
Returns a reference to the internal object that maintains statistical
information such as read and write operations performed.\comment{JV:
  Do we have a discussion of \lstinline|tpie_stats| somewhere?}


\vspace*{\baselineskip}

\subsection{BTE \emphd{stdio}}
\index{BTE stdio|see \tt{BTE\_stream\_stdio}}
\plabel{sec:ref-bte-stdio}

The \emphd{stdio} BTE is implemented by the class
\lstinline|BTE_stream_stdio|, defined in file
\path"include/bte_stdio.h".  \lstinline|BTE_stream_stdio| streams are
stored as ordinary operating system files which are manipulated via
the \emphd{stdio} file access method, i.e., via the standard
\lstinline|C| I/O library. The read/write primitives of
\lstinline|BTE_stream_stdio| streams are implemented using the system
calls \lstinline|fread| and \lstinline|fwrite|. The underlying
operating system blocking and prefetching assure that stream accesses
are done in blocks and so both blocking and prefetching are therefore
automatic and invisible to the TPIE developer.  Note that this means
that a (OS) kernel call is incurred every time a stream object is
accessed, and that every object passes through kernel level buffer
space on its way to user space.  \comment{DH: this seems incorrect to
  me.}

%The \lstinline|BTE_stream_stdio| class inherits from
%\lstinline|BTE_stream_base| (in file
%\path"include/bte_stream_base.h"):\index{BTE_stream_base@{\tt
%    BTE\_stream\_base}}
%\begin{lstlisting}
%  class BTE_stream_stdio : public BTE_stream_base { 
%    private: 
%    FILE *file;
%    BTE_stdio_header header; 
%    ...  
%  }
%\end{lstlisting}

\lstinline|BTE_stream_stdio| implements the public member functions
required for all BTEs---see
Section~\ref{sec:imp-bte-base} for a list of these member functions
and their semantics.
%  \comment{LA: Extend - discuss implementation!}
%Additionally, \lstinline|BTE_stream_stdio| defines its own
%constructors:\comment{LA: Substreams?}

\comment{JV: There used to be a discussion of several constructors
  none of which is in the code anymore. Why did they get deleted?}
%\index{BTE\_stream\_stdio@{\tt BTE\_stream\_stdio}!constructors}

%\begin{lstlisting}
%     BTE_stream_stdio(const char *dev_path, const BTE_stream_type st); 
%     BTE_stream_stdio(const BTE_stream_type st); 
%     BTE_stream_stdio(const BTE_stream_stdio<T> &s);
%\end{lstlisting}

%\comment{JV: The header stuff has changed completely. Someone has to
%  update this section accordingly.}

%The first block of a TPIE \emphd{stdio} stream is a header
%block\index{BTE_stream_stdio@{\tt BTE\_stream\_stdio}!header}
%\index{streams!header block} with the following structure:

%\index{BTE_stdio_header@{\tt BTE\_stdio\_header}}

%\begin{lstlisting}
%typedef struct BTE_stdio_header_v1 { 
%    unsigned int magic_number;  // Set to BTE_STDIO_HEADER_MAGIC_NUMBER
%    unsigned int version;       // Should be 1 for current version.
%    unsigned int length;        // # of bytes in this structure.
%    unsigned int block_length;  // # of bytes in a block.
%    size_t item_size;           // The size of each item in the stream.
%} BTE_stdio_header;
%\end{lstlisting}

\index{BTE_stream_stdio@{\tt BTE\_stream\_stdio}|)}


\index{BTE_stream_mmap@{\tt BTE\_stream\_mmap}|(} 
\subsection{BTE \emphd{mmap}}
\plabel{sec:ref-bte-mmap}
\index{BTE mmap|see {\tt BTE\_stream\_mmap}}

The \emphd{mmap} BTE is implemented by the class
\lstinline|BTE_stream_mmap|, defined in file
\path"include/bte_stream_stdio.h".  \lstinline|BTE_stream_mmap|
streams are stored as ordinary operating system files.  The TPIE
\lstinline|BTE_stream_mmap| implementation uses memory-mapping and
-unmapping mechanisms such as the Unix \lstinline|mmap| and
\lstinline|munmap| calls to perform I/O. A memory-mapping call allows
a part of a file to be associated with a corresponding section of
internal memory.  When the memory is accessed, the corresponding
portion of the file is copied directly into that memory buffer. The
\lstinline|BTE_stream_mmap| primitives explicitly maintain the
currently accessed block of the file mapped into memory.  When an
object outside the current block boundaries is requested, the current
block is unmapped and a new one is mapped from the source file.

The \lstinline|BTE_stream_mmap| is not limited to mapping in blocks of
size equal to the physical block size of the OS (typically 8K bytes).
Often improved performance can be obtained by mapping blocks of much
larger size (for example 256KB).  This is typically due to (track)
buffering and prefetching performed in the disk controller. The
(logical) block size used by \lstinline|BTE_stream_mmap| can be set
using the macro \lstinline|BTE_STREAM_MMAP_BLOCK_FACTOR| in the
\path"app_config.h" file. However, choosing a large (logical) block
size limits the amount of main memory available for an application
program and thus the (logical) block size should be chosen very
carefully in order to obtain maximal performance.

Unlike in \lstinline|BTE_stream_stdio|, where a function call is
incurred every time a stream object is accessed, the
\lstinline|BTE_stream_mmap| only incurs such a call on every (logical)
block. This can lead to improved performance, especially on systems
with a relatively slow CPU compared to the disk.  However, while
prefetching of disk blocks is implicitly done by the operating system
in \lstinline|BTE_stream_stdio|, \lstinline|BTE_stream_mmap| has to
implement its own prefetching scheme. (Note on the other hand that
unlike in \lstinline|BTE_stream_stdio| objects does not pass through
the kernel level buffer space on its way to user space.)\comment{JV:
  If DHs comment above is correct, this should go.}
\lstinline|BTE_stream_mmap| prefetching can be turned on by defining
the macro \lstinline|BTE_MMAP_READ_AHEAD| in the \path"app_config.h"
file. If this compile-time variable is defined,
\lstinline|BTE_stream_mmap| optimize for sequential read speed by
reading (mapping) blocks into main memory before the data they contain
is actually needed. Two methods of read-ahead is provided:\comment{LA:
  Does the asynchronous stuff really work?}
\comment{JV: A comment in the code says, AIO does not work, so we
  should get rid of it for good.}
\begin{itemize}
  
\item If the \lstinline|USE_LIBAIO| macro is undefined (and
  \lstinline|BTE_MMAP_READ_AHEAD| is set), read ahead is done using
  \lstinline|mmap| calls.
  
\item If the \lstinline|USE_LIBAIO| macro is defined (and
  \lstinline|BTE_MMAP_READ_AHEAD| is set), read ahead is done using
  the asynchronous I/O library. This feature requires the asynchronous
  I/O library \lstinline|libaio|.\index{libaio library@{\tt libaio}
    library.}

\end{itemize}
By default \lstinline|BTE_MMAP_READ_AHEAD| is defined,
\lstinline|USE_LIBAIO| is not.\comment{LA: Discuss the two and how
  they differ.}

%\lstinline|BTE_stream_mmap| class inherits from \lstinline|BTE_stream_base|:\index{BTE_stream_base@{\tt BTE\_stream\_base}}
%\begin{lstlisting}
%  class BTE_stream_mmap : public BTE_stream_base { private: //
%    descriptor of the mapped file.  int fd; // A pointer to the mapped
%    in header block for the stream.  mmap_stream_header *header; ...
%  }
%\end{lstlisting}

%\lstinline|BTE_stream_mmap| defines the public member functions
%defined for \lstinline|BTE_stream_base| (please see
%Section~\ref{sec:imp-bte-base} for a list of these member functions
%and their semantics).  In addition to these it defines its own
%constructors: \index{BTE_stream_mmap@{\tt
%    BTE\_stream\_mmap}!constructors} \comment{LA: Extend - discuss
%  implementation.} 
\lstinline|BTE_stream_mmap| implements the public member functions
required for all BTEs---see Section~\ref{sec:imp-bte-base} for a list
of these member functions and their semantics. In addition to that,
\lstinline|BTE_stream_mmap| provides a substream constructor that can
be used to create a stream as a substream of some other stream. 
\begin{lstlisting}
  BTE_stream_mmap(BTE_stream_mmap * super_stream,
                  BTE_stream_type st, 
                  TPIE_OS_OFFSET sub_begin, 
                  TPIE_OS_OFFSET sub_end);
\end{lstlisting}

The parameters (except for the reference to the ``parent'' stream) are
exactly the same as for the \lstinline|new_substream()| method, and
we refer the reader to Section~\ref{sec:imp-bte-base} for their
discussion.\comment{JV: Why don't we have a constructor like this for
  the \lstinline|stdio| BTE? Is it because we need to re-open the
  file? Strange\ldots}

%\begin{lstlisting}
%  BTE_stream_mmap(const char *dev_path, BTE_stream_type st);
%  BTE_stream_mmap(BTE_stream_type st); 
%  BTE_stream_mmap(BTE_stream_mmap<T> &s); 
  
%  // A substream constructor.
%  BTE_stream_mmap(BTE_stream_mmap *super_stream,
%                 BTE_stream_type st,
%                 TPIE_OS_OFFSET sub_begin, TPIE_OS_OFFSET sub_end);
%\end{lstlisting}

%The \lstinline|BTE_stream_mmap| header\index{BTE_stream_mmap@{\tt
%    BTE\_stream\_mmap}!header}\index{streams!header block} structure
%is as follows:

%\begin{lstlisting}
%struct mmap_stream_header { 
%  public:
%    unsigned int magic_number;  // Set to MMAP_HEADER_MAGIC_NUMBER
%    unsigned int version;       // Should be 1 for current version.
%    unsigned int length;        // # of bytes in this structure.
%    TPIE_OS_OFFSET item_logical_eof;     // The number of items in the stream.
%    size_t item_size;           // The size of each item in the stream.
%    size_t block_size;          // The size of a physical block on the
%                                // device where this stream resides.
%    unsigned int items_per_block;
%};
%\end{lstlisting}

\comment{LA: More here?}

\index{BTE_stream_mmap@{\tt BTE\_stream\_mmap}|)}

\index{BTE_stream_ufs@{\tt BTE\_stream\_ufs}|(}

\subsection{BTE \emphd{ufs}}
\plabel{sec:ref-bte-ufs}
\index{BTE ufs|see {\tt BTE\_stream\_ufs}}

The \emphd{ufs} BTE is implemented by the class
\lstinline|BTE_stream_ufs|, defined in file
\path"include/bte_stdio.h".  \lstinline|BTE_stream_ufs| streams are
stored as ordinary operating system files.  \lstinline|BTE_stream_ufs|
streams use the \lstinline|read()| and \lstinline|write()| calls to
implement their I/O.


% The motivation behind implementing \lstinline|BTE_ufs| streams
% is an empirically observed inefficiency\footnote{The
%    inefficiency can occur on account of various reasons. On
%    one system, \lstinline|mmap()| calls were implemented on top
%    of the \lstinline|stdio| interface instead of a direct I/O
%    implementation, resulting in extra overhead.} of
% \lstinline|mmap()| implementations in some systems.\comment{LA:
%    Say that its Solaris?} In such situations, another stream
% implementation that, like \lstinline|BTE_stream_mmap| streams,
% have the potential of exploiting large sized blocks and
% buffers is needed.

% The \lstinline|BTE_ufs| stream implementation simulates the
% \lstinline|BTE_stream_mmap| stream implementation: Whenever the latter
% maps in a new block, the former reads in a new block (via
% \lstinline|read()|) and whenever the latter unmaps a block, the
% latter attains the same result via a \lstinline|write()| call.
% But the \lstinline|BTE_ufs| implementation involves explicitly
% keeping track in the BTE code various things which are
% ``under the hood'' in \lstinline|mmap()|
% implementations.\comment{LA: Like what?} In fact, the code
% in the \lstinline|BTE_ufs| implementation can be said to
% amount to a (very rudimentary) \lstinline|mmap()|
% implementation. 

Like in the case of \lstinline|BTE_stream_mmap| the (logical) block
size can be controlled using the macro
\lstinline|STREAM_UFS_BLOCK_FACTOR| on a global basis or using the
\lstinline|lbf| constructor parameter (see
Section~\ref{sec:imp-bte-base}) on a per-stream basis.
\lstinline|BTE_stream_ufs| also has the same advantage as
\lstinline|BTE_stream_mmap| over \lstinline|BTE_stream_stdio| of only
incurring one kernel call per block. However, unlike
\lstinline|BTE_stream_mmap| (but like \lstinline|BTE_stream_stdio|),
prefetching is done implicitly by the filesystem underlying TPIE (and
objects have to pass through kernel level buffer space).\comment{JV:
  Please double-check against DH's above comment.}
%In fact, nowadays, in the case of sequential accesses, most filesystems
%almost surely implement readahead prefetching, which should suffice for the
%purpose of streaming operations in TPIE. (In the case of non-sequential
%access, the next block to be accessed is more often than not dependent on
%the processing of the contents of the current block, so prefetching is
%difficult to implement or impossible.)
In \lstinline|BTE_stream_ufs| streams, when the asynchronous I/O
library \lstinline|libaio|\index{libaio library@{\tt libaio} library.}
is available, there is a provision to do (user-level) prefetching
within \lstinline|BTE_stream_ufs| streams but we do not recommend its
use on account of the implicit filesystem prefetching.\comment{LA: So
  why there?}\comment{JV: LIBAIO should go, no?}

%\lstinline|BTE_stream_ufs| class inherits from
%\lstinline|BTE_stream_base|:
%\begin{lstlisting}
%class BTE_stream_ufs : public BTE_stream_base { 
%  private:
%     // descriptor of the mapped file.  
%     int fd;
%     // A pointer to the mapped in header block for  the stream.  
%     mmap_stream_header *header; 
%     ...  
%}
%\end{lstlisting}

%\lstinline|BTE_stream_ufs| defines the public member
%functions defined for \lstinline|BTE_stream_base| (please see
%Section~\ref{sec:imp-bte-base} for a list of these member
%functions and their semantics).  In addition to these it
%defines its own constructors: 
%\index{BTE_stream_ufs@{\tt BTE\_stream\_ufs}!constructors}

%\begin{lstlisting}
%  BTE_stream_ufs(const char *dev_path, BTE_stream_type st); 
%  BTE_stream_ufs(BTE_stream_type st); 
%  BTE_stream_ufs(BTE_stream_ufs<T> &s); 
  
%  // A substream constructor.
%  BTE_stream_ufs(BTE_stream_ufs *super_stream,
%                 BTE_stream_type st,
%                 TPIE_OS_OFFSET sub_begin, TPIE_OS_OFFSET sub_end);
%\end{lstlisting}
%The \lstinline|BTE_stream_ufs| header structure is as follows:
%\index{BTE\_stream\_ufs@{\tt BTE\_stream\_ufs}!header}\index{streams!header block}
%\begin{lstlisting}
%struct ufs_stream_header { 
%  public:
%    unsigned int magic_number;  // Set to UFS_HEADER_MAGIC_NUMBER
%    unsigned int version;       // Should be 1 for current version.
%    unsigned int length;        // # of bytes in this structure.
%    TPIE_OS_OFFSET item_logical_eof;     // The number of items in the stream.
%    size_t item_size;           // The size of each item in the stream.
%    size_t block_size;          // The size of a physical block on the
%                                // device where this stream resides.
%    unsigned int items_per_block;
%};
%\end{lstlisting}
\lstinline|BTE_stream_ufs| implements the public member functions
required for all BTEs---see Section~\ref{sec:imp-bte-base} for a list
of these member functions and their semantics. In addition to that,
\lstinline|BTE_stream_ufs| provides a substream constructor that can
be used to create a stream as a substream of some other stream. 
\begin{lstlisting}
  BTE_stream_ufs(BTE_stream_ufs * super_stream,
                  BTE_stream_type st, 
                  TPIE_OS_OFFSET sub_begin, 
                  TPIE_OS_OFFSET sub_end);
\end{lstlisting}

The parameters (except for the reference to the ``parent'' stream) are
exactly the same as for the \lstinline|new_substream()| method, and
we refer the reader to Section~\ref{sec:imp-bte-base} for their
discussion.\comment{JV: Why don't we have a constructor like this for
  the \lstinline|stdio| BTE? Is it because we need to re-open the
  file? Strange\ldots}


\index{BTE_stream_ufs@{\tt BTE\_stream\_ufs}|)}
\comment{LA: Discuss implementation!}


\section{The Memory Manager (MM)}
\plabel{sec:ref-mm}
\index{memory manager|(}
\index{MM|see{memory manager}}
\index{MM_manager@{\tt MM\_manager}|see{memory manager}}

The Memory Manager components of TPIE provide services related to the
management of internal memory:
\begin{itemize}
\item allocation and deallocation of (internal) memory as requested by
  the \lstinline|new| and \lstinline|delete| operators,
\item accounting of memory usage (when required),
\item enforcing of the user-specified internal memory usage limit
  (when required),
\item logging of memory allocation requests (when required).
\end{itemize}

In version \version ~of TPIE, the memory manager
\lstinline|MM_manager|, is built from the source files
\path"lib/src/mm_register.cpp", \path"lib/src/mm_base.cpp",
\path"include/mm_register.h", and \path"include/mm_base.h".  The
memory manager traps memory allocation and deallocation requests in
order to monitor and enforce memory usage limits. It provides a number
of user-callable functions and services, which are documented in
Chapter \ref{cha:reference}.

\tobewritten

\comment{LA: There is some text in here that I commented out.}

%The MM is the layer of TPIE that sits between the AMI interface and the
%BTE.  Its primary role is managing main memory, including memory that
%may be distributed across multiple physical machines.  The performance
%of many of the AMI stream operations, such as sorting, permuting,
%merging, and distribution depend critically on the efficient use of
%main memory.  The first thing the MM will have to do to achieve this
%is bypass the virtual memory system provided by UNIX and related
%operating systems.  The second thing it has to do is bypass the
%traditional UNIX buffer cache and take charge of managing the blocks
%of data provided by the BTE.  In some cases, operating system kernels
%will have to be modified in order for the MM to do its job.  In modern
%micro-kernel operating systems, however, the MM may be able to operate
%entirely as a user level process.

%In multiple CPU environments, the job of the MM will be complicated by
%the need to manage multiple banks of memory.  In tightly coupled
%homogeneous parallel environments, this task is likely to be made far
%simpler by existing hardware and operating system support.  In
%distributed, and in particular in heterogeneous environments, the MM
%will have to work with various network protocols and drivers to
%accomplish its task.

\comment{(Need some comments on the current simple MM that we have and
  some OS issues that come up in attempting to make it more robust.)}
\comment{JV: Who wrote this? What are the issues mentioned?}

\index{memory manager|)}


\section{The Access Method Interface (AMI)}
\index{access method interface|(}
\index{AMI|see{access method interface}}
\plabel{sec:imp-ami}

The\comment{LA: As for other sections - whole AMI sections need many
  more implementation details} AMI-level entry points provided by
TPIE are documented in the Reference section of this manual (see
Section~\ref{cha:reference}), and a number of examples of their use
are given in the Tutorial section (see Section~\ref{ch:tutorial}).  In
this section we examine the TPIE source files which compose the AMI
and provide a brief discussion of their purpose and relationship to to
each other. We also discuss the algorithmic decisions that were made
in constructing the various TPIE services such as creation, scanning,
merging, sorting, and permutation of streams, and the services of the
block collection class. The presentation of this section is organized
by TPIE service (creation of streams, scanning of stream, etc.) and
within each service the relevant source files are itemized
(alphabetically) and discussed. The index of this manual provides a
more direct way to find the documentation for a particular source
file.

\subsection{General Considerations}

The following TPIE files are fundamental and therefore involved in
every TPIE program, no matter which TPIE services are
accessed:\comment{DH: There may be a better place for some of the
  stuff in this intro}

\begin{itemize}
\item \path"include/ami.h": This file should be included in every TPIE
  application program that uses the AMI-level interface. It in turn
  inputs the definitions for the AMI-level services of TPIE. The files
  input by \path"include/ami.h" are itemized below.
    
\item \path"test/app_config.h": This file contains TPIE flags and
  settings that can be customized to an individual application. The
  options available in this file are described in detail in
  Section~\ref{sec:tun-appconfig}.
  
\item \path"include/config.h": This file contains flags and indicators
  that describe the machine and operating system on which TPIE is
  currently running. It is generated automatically by the TPIE
  installation process and is not intended to be modified.
    
\item \path"lib/libtpie.a": This is the static TPIE load library. It
  contains code for the memory manager \lstinline|MM_manager|, TPIE
  logging, BTE statistics, and a small number of other services.
  While most TPIE code is in the form of templates, which generate
  code at compile-time, application programs must also link with
  \path"lib/libtpie.a". This is described further in
  Section~\ref{sec:tut-compiling}.
  
\item \path"include/portability.h": This file should be included in
  every TPIE application program since it provides the type
  definitions and function declarations needed for compiling on
  multiple platforms. Due to this particular importance, this file
  gets automatically included from \path"include/ami.h", so the user
  only should not need to explicitly include it. See
  Section~\ref{sec:portability} for more details.\comment{JV: This
    section needs to be written\ldots{}what is the best place for
    putting it?}

\end{itemize} 

\vspace*{\baselineskip}
\subsubsection{File: \texttt{include/ami.h}} This include
file inputs the various include files required for compilation of an
application program with TPIE.

\lstinputlisting[caption={The AMI configuration file.}]{../tpie/ami.h}

Each of these files is discussed in the sections that follow.

\subsection{Creation of Streams}

AMI\comment{LA: Jan please look at this subsection. JV: Done} stream
objects are created in a TPIE program via the \lstinline|AMI_STREAM|
keyword.  \lstinline|AMI_STREAM| is a macro that resolves to a class
template invocation appropriate to the declared target I/O
architecture. This involves the declared AMI implementation (see
Section~\ref{cha:reference}), BTE implementation\index{block transfer
  engine} (see Section~\ref{sec:ref-bte}) and whether one or multiple
disks are used.  In the current version of TPIE an
\lstinline|AMI_STREAM| is stored in a standard UNIX file on a single
disk. For most applications, an \lstinline|AMI_STREAM| will have type
\lstinline|AMI_stream|, which is the TPIE base class for a stream
backed by a single file (normally on a single disk). The string
\lstinline|AMI_STREAM| is \lstinline|#define|'d to be the string
\lstinline|AMI_stream| in the header file
\path"include/ami_stream_compatibility.h". This arrangement is
intended to permit alternative implementations of
\lstinline|AMI_STREAM| if necessary in the future.

The TPIE code associated with the creation and manipulation of stream
objects is contained mainly within \path"include/ami_stream.h": This
include file defines the class \lstinline|AMI_stream|, which is the
base class for all AMI-level streams backed by a single operating
system file. By means of including \path"include/ami_device.h",
\path"include/ami_err.h", and \path"include/ami_stream_base.h" the
following basic definitions are made:

\begin{itemize}
\item The code in these files defines means for accessing device
  names and reading environment variables.
\item Also, a non-templated base class for all
  \lstinline|AMI_stream|s is defined that maintains basic device
  information.  
\item The code in this file defines the \lstinline|AMI_err| codes
  returned by the AMI-level services (these and their meanings are
  listed in Appendix~\ref{sec:ami-errors}.
\end{itemize}    

Based upon these definitions, \lstinline|AMI_stream<T>| is defined
as follows:

\begin{enumerate}
  
\item The template parameter \lstinline|<T>| represents the type of
  the application data element in the stream.
  
\item The code in this file defines the AMI stream types discussed
  in Section~\ref{cha:reference} (\lstinline|AMI_READ_STREAM|,
  \lstinline|AMI_WRITE_STREAM|, \lstinline|AMI_APPEND_STREAM|, and
  \lstinline|AMI_READ_WRITE_STREAM|).
  
\item At construction time, an \lstinline|AMI_stream<T>| is
  mapped onto a \lstinline|BTE_STREAM<T>| and the AMI stream type is
  mapped as follows:
  
  \begin{itemize}
  \item \lstinline|AMI_READ_STREAM| is mapped to
    \lstinline|BTE_READ_STREAM|
  \item \lstinline|AMI_APPEND_STREAM| is mapped to
    \lstinline|BTE_APPEND_STREAM|
  \item \lstinline|AMI_WRITE_STREAM| and
    \lstinline|AMI_READ_WRITE_STREAM| are mapped to
    \lstinline|BTE_WRITE_STREAM|
  \end{itemize}
  
\item The AMI-level services for streams are implemented by
  corresponding BTE-level services. The AMI member functions
  described in Sections~\ref{sec:ref-ami-stream} (see examples of
  use of some of these in Section~\ref{sec:tut-samplepgm}) are
  implemented by calls to the BTE-level functions documented in
  Section~\ref{sec:ref-bte}. An AMI stream has little internal
  context information as a result. Two exceptions are the following:
  
  \begin{itemize}
  \item \lstinline|r_only|: this flag is \lstinline|true| if the
    stream is only readable, and not writable, corresponding to the
    status \lstinline|AMI_READ_STREAM|. Maintaining this flag allows
    the AMI stream member functions to catch erroneous attempt by
    application-level code to write to a stream whose underlying
    file was opened for reading.
    
  \item \lstinline|destruct_bte|: this flag is \lstinline|true| if
    the destructor of the underlying BTE stream should be called by
    the AMI-level destructor (the normal situation). If the
    AMI-level stream was constructed from a pre-existing BTE stream
    via the constructor
\begin{lstlisting}
    AMI_stream(BTE_STREAM<T> *bs);
 \end{lstlisting}
    then a \lstinline|destruct_bte| value of \lstinline|false| is
    used to prevent the destructor from deleting the BTE stream
    (owned by another AMI stream).
  \end{itemize}
\end{enumerate}


% The public member functions of the \lstinline|AMI_STREAM|
% class are defined in the file ``ami\_single.h''. Some
% important related enumerated types that the user will be
% concerned with are defined in ``ami\_base.h''.
% \lstinline|AMI_STREAM| is a macro that resolves to a class
% template declared to match the underlying semantics of the
% target I/O architecture by using an appropriate AMI
% implementation (see section~\ref{sec:ref-ami}), BTE
% implementation\index{block transfer engine} (see
% Section~\ref{sec:ref-bte}) and MM (see
% Section~\ref{sec:ref-mm}) implementation. 
\comment{LA: There is a lot of text in here I have commented out}

%To use TPIE, the programmer has to create, modify or delete items each of
%which belongs to a stream. From the programmer's perspective,
%computation simply requires her to \emph{choreograph the movement of
%items in various streams}: The I/O operations, the buffering required
%and memory management are performed in TPIE itself. In order to specify
%the choreography of streams, the user can make heavy use of inbuilt
%TPIE templates and functions; it is not often that the user has to write
%special-purpose custom-made TPIE functions. The reason why TPIE's unique
%approach is attractive for external memory computation is the well
%known fact that most external memory algorithms consist of sewing
%together a small number of paradigms such as \emphd{scanning},
%\emphd{merging}, \emphd{distributing}, in a well-coordinated manner
%exploiting all available memory and I/O bandwidth. A large variety of 
%indexing data structures and methods related to those data structures 
%can be implemented using the above paradigms in conjunction with a 
%basic tree template: While indexing data structures have been
%implemented using TPIE streams, work on tree templates and related
%methods to further ease implementation of indexing data structures is
%currently ongoing.

% Each datum in TPIE is an item of a certain type.  The
% programmer normally manipulates the data via the services of
% the \emphd{Access Method Interface (AMI)}, which accesses
% data elements stored in \lstinline|AMI_STREAM|s.  An
% \lstinline|AMI_STREAM| can be viewed as simply a wrapper
% around a \lstinline|BTE_STREAM| (See
% Section~\ref{BTEinterface} for a discussion of the BTE
% interface).

%as a \lstinline|BTE_STREAM|, where BTE stands for \emphd{block transfer
%engine.} Each \lstinline|BTE_STREAM| is a stream built on top
%of a Unix file, with  
%additional features such as typing,  and automatic, efficient,
% I/O and buffering so that the programmer does
%not bear this burden. TPIE can be configured to use one of three
%different \lstinline|BTE_STREAM| implementations: The implementations vary
%fundamentally in the way they do buffering and I/O. The implementation
%of choice depends, to some extent, on the operating system on which 
%TPIE runs. Later we describe the different BTE implementations and
%how to determine the best implementation on any given system; for now,
%we focus on the \lstinline|AMI_STREAM| interface since that
%is most directly 
%relevant to a programmer/user.

%The member functions of the \lstinline|AMI_STREAM| class,
%templated on the type T of its items,  are defined in the file \path"include/ami_single.h";
%some important related enumerated types that the user will be
% concerned with have been defined in \path"include/ami_base.h".

%An important consequence of the construction of an \lstinline|AMI_STREAM| is a
%reduction in the total amount of memory that the user has at her
%disposal since a certain amount of memory is consumed now by buffers
%dedicated to that \lstinline|AMI_STREAM| and its related data
%structures.\footnote{The total amount of memory that the user has
%available, divided by the memory consumed by every AMI\_STREAM thus
%is an upper bound on the maximum number of AMI\_STREAMs that the user
%can have.}
 
%In order to effectively utilize memory, the user needs to keep track
%of the amount of memory she has at her disposal: Sometimes, 
%when one of several algorithms may be usable to solve a given problem,
%it may be useful to choose the algorithm depending on the problem
%size and available memory size. The total amount of memory consumed
%by an \lstinline|AMI_STREAM| often plays an important role while making such 
%a choice. The memory manager \path"include/mm.h" is responsible for keeping
%track of the amount of memory available to the user at any time.

%When an \lstinline|AMI_STREAM| is destroyed using a destructor, that
%amount of memory becomes is added to the available memory for the
%user. 


%Sometimes the
%items in an \lstinline|AMI_STREAM| will never be used  once
%the \lstinline|AMI_STREAM| is 
%destructed whereas sometimes an ``idle''
%\lstinline|AMI_STREAM| may be destroyed  
%to prevent it from hogging memory only to be constructed later on
%when really required. In the former case, it is desirable to 
%destroy or delete the Unix file backing the \lstinline|AMI_STREAM| to save
%disk space whereas in the latter case we need the Unix file backing
%the \lstinline|AMI_STREAM| to exist on disk even when the
%   \lstinline|AMI_STREAM| is 
%destroyed. The  \lstinline|AMI_STREAM| destructor looks at the persistence 
%flag to decide whether it need only close the backing Unix file or
%delete it from disk.

\subsection{Using Multiple BTE Implementations}
\plabel{sec:imp-multi-imp}

For\comment{LA: What does this subsection do/say? Seem to be missing
  something. And does it work??}\comment{JV: Moved it here (it was
  before the previous section).} instance, if a single implementation
is needed, it is sufficient for the application code to create a
stream of \lstinline|int| as follows:

\begin{lstlisting}
AMI_STREAM<int> aStream;
\end{lstlisting}
However, if different implementations, say a \lstinline|ufs| and an
\lstinline|mmap| stream are desired, the code would be similar to the
following:
\begin{lstlisting}
BTE_stream_mmap   firstStream(const char *dev_path, BTE_stream_type st);
BTE_stream_stdio secondStream(const char *dev_path, BTE_stream_type st);
\end{lstlisting}



\subsection{Scanning}
\index{scanning|(}

The file \path"include/ami_scan.h" defines the scanning functionality
provided by TPIE. This file is mechanically generated by
\lstinline|make| when TPIE is compiled and built. It is essentially a
concatenation of the files \path"include/ami_scan.h.head",
\path"include/ami_scan_mac.h", and \path"include/ami_scan.h.tail" --
for more details, see the file \path"include/Makefile".\comment{LA:
  Details needed!!} It is not intended that \path"include/ami_scan.h"
be modified manually.

\index{scanning|)}



\subsection{Merging}
\plabel{sec:imp-ami-merge}
\plabel{sec:imp-ami-generalized-merge}
\index{merging|(}

\subsubsection{Merging}

The file \path"include/ami_merge.h"\ldots \comment{AD: The ``Darren'' version}

\tobeextended\comment{LA: Merge management object etc}

%\subsubsection{Merge Management Objects}
%
%Merge management objects based on the super class \lstinline|AMI_merge_base| 
%defined in the file \lstinline|ami_merge.h" can be used conveniently by
%appropriately defining the member functions of the class.


\subsubsection{Merging sorted runs}


\tobeextended

The \lstinline|AMI_merge_sorted()| polymorphs (defined in file
\path"include/ami_merge_sorted_runs.h") work without a merge management
object and perform standard (total order/comparison based) merging.
The elimination of the merge management object object results in one
less level of function calls and thus in improved efficiency over a
similar comparison based merging based on a merge management object.
The merge is controlled by a priority queue (defined in file
\path"include/mergeheap.h") which exploits the fact that
\lstinline|delete-min| and \lstinline|insert| are always performed
together. This improves efficiency. In \lstinline|AMI_key_merge()| the
heap stores the key and not the entire object. When the object size is
large compared to the key size, this often leads to further
performance improvement.\comment{LA: Extend this stuff so it becomes
  readable}

\subsubsection{Partition and Merge}

\tobeextended

\lstinline|AMI_partition_and_merge()| repeatedly merges together the
maximum possible number of sub-streams using \lstinline|AMI_merge()|.
An important point is that the substreams input to
\lstinline|AMI_merge()| during the execution of
\lstinline|AMI_partition_and_merge()| all originate from the same
underlying parent \lstinline|AMI_STREAM|; thus filesystem accesses are
inherently non-sequential. Throughout the execution of the merging
method \lstinline|AMI_partition_and_merge()|, there are only two
active \lstinline|AMI_STREAM|s at any time: One which stores the
substreams being merged at that stage and one which stores the
substreams output by that stage.\comment{LA: Extend this stuff so that
  it becomes readable! Is it even true?}


\subsection{Comparison Sorting}
\plabel{sec:ref-imp-ami-sort}
\index{sorting!comparison|(}

The file \path"include/ami_sort.h" \ldots

The\comment{LA: Add 2x sorting somewhere. AD: added in user Reference.
  Does progress bar stuff also go here or is it more general? AD: More
  general} TPIE sort implementation is a multi-way merge sort.  Merge
sort consists of two phases: the run formation phase and the merging
phase. During the \emphd{run formation phase}, the $N$ input elements
are read approximately $M$ (one \emphd{memory-load}) at a time, sorted
in memory, and written to disk as sorted ``runs''. In the \emphd{merge
  phase}, the sorted runs are merged together approximately $M/B$ at a
time (where $M$ is the internal memory size and $B$ is the block size)
in a round-robin manner until a single sorted run remains; During each
merge an internal \emph{merge heap} is used to select the next element
to be output among the first element in each of the input runs. Below
we discuss the sort implementation in more detail.

\paragraph{Sort manager.} The sort manager class in
\path"include/sort_manager.h" is responsible for performing the
general external merge sort. It takes as template arguments the type
of object to sort, an \emph{internal sort object} to use for sorting
internal runs, and an \emph{mergeheap object} to use for merging
sorted run.  Given memory and OS restrictions, the sort manager then
computes the size of internal runs to sort as well as the number of
runs to merge at one time.\footnote{Often the number of streams that
  can be merged at one time is not limited by the amount of memory,
  but the number of open file descriptors a running process can open
  at once in a given OS.}\comment{LA: Footnote really true? AD: Yes.
  Number of default open file descriptors per process in Linux is
  typically 1-4K} The
manager then partitions the input data (stream) and use the internal
sort object to sort the partitioned runs (streams). Then it loops over
the sorted runs (streams) and merges runs using the merge heap object
provided until a single output run (stream) is produced.\comment{LA:
  Something somewhere about why not using merge management object
  (AMI\_merge)?}

\paragraph{Internal sorting.} The details of how a set of elements
is sorted in internal memory in the run formation phase is defined
in an internal sort class. Three different
internal sort class objects--one for each comparision type--are defined in
\path"include/internal_sort.h". Each class object has a public
function
\begin{lstlisting}
  sort(AMI_STREAM<T>* in, AMI_STREAM<T>* out, TPIE_OS_OFFSET nItems)
\end{lstlisting}
that reads \lstinline|nItems| from \lstinline|in|, sorts them and
writes the sorted run to \lstinline|out|. They all use an
implementation of quicksort defined in \path"include/quicksort.h".
Optionally STL \lstinline|sort| can be used by defining the flag
\lstinline|TPIE_USE_STL_SORT|. In our experiments both the STL and our
quicksort have similar running times. Neither the quicksort
implementation nor the STL \lstinline|sort| is a stable sort, i.e.,
they do not preserve the relative order of input elements that compare
equal. If users need a stable sort, the sort key should be modified to
include the initial item offset in the comparison.

What distinguishes the different sort class objects is the way two
input elements are compared in the quicksort algorithm. The tree
different variants use a comparison operator, a comparison object, and
a key plus object comparison method, respectively, to compare
elements. The comparison operator based class object uses the operator
\lstinline|<| to compare two elements; The comparison object based
class object compares two elements with a user specified comparison
object containing a method
\lstinline|compare(const T& left, const T& right)|;
The key plus object comparison based class object takes a
user-specified class that has two methods
\begin{lstlisting}
  copy(KEY* key, const T& item)
\end{lstlisting}
and
\begin{lstlisting}
  compare(const KEY& left, const KEY& right)
\end{lstlisting}
The \lstinline|copy| method is used to extract a (presumably) small
\lstinline|KEY| from each input data element of type \lstinline|T| and
the \lstinline|compare| method is then used in to compare such keys in
the quicksort algorithm.\comment{LA: Write more? AD: not in implementation
details, discussed in user ref.} Note that the
(natural) C-style comparison function pointer sorting variant is not
implemented because each comparison would then incur a function call;
the other methods can (and do) use inlined comparison methods and thus
avoid such a function call, something which is not possible with
C-style comparison functions.

\paragraph{Merge heap.} The merge heap classes used in the merge phase
is defined in \path"include/mergeheap.h".  To merge $k$ sorted runs in
$k$ streams, the sort manager repeatedly removes the smallest element
from the merge heap, writes it to a merged sorted output stream (run)
and inserts the next element from the stream (run) that contained the
output element.  Because the removal of an element from a mergeheap is
(almost) always followed by a immediate insertion, each merge heap
class implements a function \lstinline|delete_min_and_insert| that
removes the smallest element and inserts the next element from the
corresponding stream in one atomic operation. This saves one
\lstinline|heapify| operation for each delete min and
insert and greatly improves performance.

Similar to the internal sort class objects, TPIE implements different
mergeheap class objects that
utilize a comparison operator, a comparison object, and a key plus
object comparison method, respectively, to compare elements. In
addition, mergeheap objects are implemented where pointers to elements are
maintained in the heap rather than the elements themselves. While a
pointer must now be dereference each time a comparison or heap
operation is done, these variants may still be faster than non-pointer
variants for large elements, since the large elements do not have to
be copied into and around the heap.\comment{LA: Is this clear/true?}
Two pointer variants are implemented, namely variants that uses
comparison operator and comparison object to perform comparisons;
because (small) keys are extracted from the elements and inserted in
the heap in the key plus object comparison variants of the heapsort
class, a pointer version of this variant is not implemented.

\paragraph{Run storage on disk.} The last remaining implementation detail
to mention is how the sort manager store sorted streams (runs) on
disk.  For a $k$-way merge ($k\approx M/B$), the $N/M$ initial sorted
runs are distributed evenly across $k$ output streams, with the first
$\frac{N}{kM}$ runs in the first output stream, the next
$\frac{N}{kM}$ sorted runs in the second output stream, and so on.
When merging runs, $k$ runs are then merged at a time---one from each
of the $k$ streams---and newly formed runs are distributed across $k$
new output streams.\comment{LA: Is it understandable?} This approach
has several advantages over other methods. For example, the number of
streams (files on disk) during sorting is at most $2k+2$; one for each
for the input and output streams, $k$ for the runs formed in the
previous pass over the data and $k$ for the merged output runs.
Alternatively one stream could have been used per run but since most
file systems have a limit on the number of files in a directory this
could lead to problems. Another alternative would be to put all the
sorted runs into a single stream (file) and use substreams to merge
$k$ runs at a time.  However, this would lead to ``random access'' to
$k$ different locations in the stream (file), making it difficult for
the OS or disk to predict the file access pattern, leading to inferior
performance.  The approach of using a number of streams equal to the
number of merged runs result in sequential access to $k$ separate
streams (files), often leading to improved performance.

  
Various combinations of internal sort class and merge
heap are exposed as easier to use \lstinline|AMI_sort| functions
described in \ref{fig:imp-sort-variants}.
\begin{figure}
\begin{center}
\begin{minipage}[hb]{1.0\linewidth}
\raggedright
\centering{
\begin{tabular}{l|p{2in}|l|l}
\hline
Sorting  & Comparison & Class of Internal & Class of Merge \\
Template & Variant    & Sorter   & Heap    \\
\hline
\lstinline|AMI_sort| & operator & \lstinline|Internal_Sorter_Op| &
\lstinline|merge_heap_op| \\
\lstinline|AMI_sort| & object   & \lstinline|Internal_Sorter_Obj| & \lstinline|merge_heap_obj| \\
\lstinline|AMI_ptr_sort| & operator & \lstinline|Internal_Sorter_Op| &
\lstinline|merge_heap_ptr_op| \\
\lstinline|AMI_ptr_sort| & object   & \lstinline|Internal_Sorter_Obj| &
\lstinline|merge_heap_ptr_obj| \\
\lstinline|AMI_key_sort| & key object  & \lstinline|Internal_Sorter_KObj| & \lstinline|merge_heap_kobj| \\
\hline
\end{tabular}}
\caption{\plabel{fig:imp-sort-variants} Customization of Arguments
   to \lstinline|sort_manager|.}
\end{minipage}
\end{center}
\end{figure}


%\comment{LA: Knuth forecasting stuff? }

\index{sorting!comparison|)}


\subsection{Distribution}
\plabel{sec:ref-imp-ami-distribution}

\tobewritten

\index{Distribution}

\subsection{Key Bucket Sorting}
\plabel{sec:ref-imp-ami-kb-sort}

\tobewritten

\index{sorting!key bucket|(}
\index{sorting!key bucket|)}

\subsection{General Permuting}
\plabel{sec:ref-imp-ami-gp}

The file \path"include/ami_gen_perm.h" \ldots

\tobewritten

\index{permutation!general|(}
\index{permutation!general|)}

\subsection{Bit Permuting}
\plabel{sec:ref-imp-ami-bp}

The file \path"include/ami_bit_permute.h" \ldots \comment{JV: The
  naming is inconsistent.}

\tobewritten

\index{permutation!bit|(}
\index{permutation!bit|)}

\fbox{JV: We're missing \path"include/ami\_coll.h",
  \path"include/ami\_block.h", and \path"include/ami\_btree.h".}
 
\fbox{JV: It's not obvious that they should be included by default,
  though.}



\fbox{JV: None of the following is included from \path"include/ami.h".}

\subsection{Dense Matrices}
\plabel{sec:ref-imp-ami-matrix}

\tobewritten

\index{matrices!dense|(}
\index{matrices!dense|)}

\subsection{Sparse Matrices}
\plabel{sec:ref-imp-ami-sm}

\tobewritten

\index{matrices!sparse|(}
\index{matrices!sparse|)}

\subsection{Stacks}
\plabel{sec:ref-imp-ami-stack}

\tobewritten

\index{stacks|(}
\index{stacks|)}

\subsection{Queues}
\plabel{sec:ref-imp-ami-queue}

\tobewritten

\index{queue|(}
\index{queue|)}


\subsection{Elementwise Arithmetic}
\plabel{sec:ref-imp-ami-arith}

\tobewritten

\index{elementwise arithmetic|(}
\index{elementwise arithmetic|)}

\index{access method interface|)}

%%%
%%% Local Variables: 
%%% mode: latex
%%% TeX-master: "tpie"
%%% End: 
%%%
