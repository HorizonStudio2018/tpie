%% Copyright 2008, The TPIE development team
%% 
%% This file is part of TPIE.
%% 
%% TPIE is free software: you can redistribute it and/or modify it under
%% the terms of the GNU Lesser General Public License as published by the
%% Free Software Foundation, either version 3 of the License, or (at your
%% option) any later version.
%% 
%% TPIE is distributed in the hope that it will be useful, but WITHOUT ANY
%% WARRANTY; without even the implied warranty of MERCHANTABILITY or
%% FITNESS FOR A PARTICULAR PURPOSE.  See the GNU Lesser General Public
%% License for more details.
%% 
%% You should have received a copy of the GNU Lesser General Public License
%% along with TPIE.  If not, see <http:%%www.gnu.org/licenses/>

\documentclass[10pt]{book}
%\usepackage{html} % for inserting html links to be used by latex2html
\ifx\pdfoutput\undefined 
\usepackage[dvips]{hyperref}
\else
\usepackage[pdftex,colorlinks=true]{hyperref}
\fi
\usepackage{path}
\usepackage{makeidx}
\usepackage{verbatim} % new verbatim package.
\usepackage{listings}
%\usepackage{times} % for better pdf output.
\usepackage{color} % for colorbox
%%
%% $Id: definitions.tex,v 1.4 2002-08-30 03:26:00 tavi Exp $
%%

\hbadness=10000

% For text shading.
\definecolor{lgray}{gray}{.85}
% Section command that displays the section name on a light gray background.
%\newcommand{\mysection}[1]{\penalty-600\vspace*{12mm}\noindent%
%\colorbox{lgray}{\rule{0cm}{4.7mm}\rule{\textwidth}{0cm}}%
%\vspace*{-13.3mm}\section{#1}\penalty+600}
%\newcommand{\mysection}[1]{\section{#1}}
\makeatletter
\newcommand\mysection{\@startsection {section}{1}{\z@}%
                                   {-3.5ex \@plus -1ex \@minus -.2ex}%
                                   {2.3ex \@plus.2ex}%
                                   {\normalfont\Large\bfseries\hspace*{-9mm}\colorbox{lgray}{\rule{0cm}{4.7mm}\rule{16.1cm}{0cm}}\hspace{-16.1cm} }}
\makeatother

% plabel allows the label value to be printed for easier
% writing of the manual
%\newcommand{\plabel}[1]{{\tiny #1}\label{#1}}
\newcommand{\plabel}[1]{\label{#1}}

% use this instead of \verb, which is not allowed as parameter [tavi]
%app_config@{\tt app\_config.h}}
%\newcommand{\myverb}[1]{\texttt{#1}\index{#1@{\tt #1}}}
%\newcommand{\myv}[1]{\texttt{#1}\index{#1@{\tt #1}}}
\newcommand{\myverb}[1]{\noiv{#1}\index{#1@{\small\tt #1}}}
\newcommand{\myv}[1]   {\noiv{#1}\index{#1@{\small\tt #1}}}

% use this instead of \myverb if you don't want its
% parameter in the index
%\newcommand{\noiverb}[1]{\texttt{#1}}
%\newcommand{\noiv}[1]{\texttt{#1}}
\newcommand{\noiverb}[1]{{\small \tt #1}}
\newcommand{\noiv}[1]{{\small \tt #1}}

% emphasize and put into index
\newcommand{\emphd}[1]{\emph{#1}\index{#1}}

% used for algorithms
\newcommand{\step}[2] {\begin{enumerate}\item[#1]#2\end{enumerate}}

% for the reference manual [tavi]
\newcommand{\entry}[2]{\> \parbox[t]{6.3in}{\tt #1}\\ \>\>\parbox[t]{5.5in}{#2}\\[3mm]}
\newcommand{\btabb}{\begin{tabbing} \hspace*{.3in} \= \hspace{.5in}\=\\ }
\newcommand{\etabb}{\end{tabbing}\vspace*{-12mm}}

\makeatletter    % '@' is now a normal letter for TeX
% this makes verbatim text smaller and indented [tavi]
\def\verbatim@startline{\small%
  \def\verbatim@startline{\hspace*{5mm}\verbatim@line{}}%
  \verbatim@startline}
% remove some space before the text
\addto@hook\every@verbatim{\vspace*{-1mm}}
% remove some space after the text
\def\verbatim@finish{\def\verbatim@finish{\ifcat$\the\verbatim@line$\else%
  \verbatim@processline\fi}\vspace*{-4mm}\verbatim@finish}
\makeatother    % '@' is restored as a non-letter character

% normal margins for US-size paper
\setlength{\topmargin}{-.5in}   
\setlength{\oddsidemargin}{.1in} % distance from left edge of page to text
\setlength{\evensidemargin}{.1in} % distance from left edge of page to text
\setlength{\textwidth}{6.3in}
\setlength{\textheight}{9in}

%% Macro for writing in the margin comments on what is left to be done.
%% Use \withcomments in the preamble if you want comments to appear.
\def\comment#1{}
\def\withcomments{
% Set \marginparwidth to ensure comment does not runs off the end of the page. 
\setlength{\marginparwidth}{8.5in}
\addtolength{\marginparwidth}{-1.0in}
\addtolength{\marginparwidth}{-\oddsidemargin}
\addtolength{\marginparwidth}{-\textwidth}
\addtolength{\marginparwidth}{-2.0\marginparsep} 
% To get the same space on both sides of the margin text
% because Duke printers use weird margins:
\addtolength{\marginparwidth}{-0.125in}
\newcounter{mycomments}
\def\comment##1{\refstepcounter{mycomments}%
\ifhmode%
\unskip%
{\dimen1=\baselineskip \divide\dimen1 by 2 %
\raise\dimen1\llap{\tiny -\themycomments-}}\fi%
\marginpar{\tiny [\themycomments]: ##1}}%
}

% Additions to makeidx.sty
%\makeatletter
%\@ifundefined{alsoname}%
%   {\def\alsoname{also}}{}

%\def\seealso#1#2{{\em \seename\ \alsoname\/} #1}
%\makeatother

% This manual applies to the following version of TPIE
\newcommand{\edition}{082902}
\newcommand{\version}{082902}
% minimum GNU release required
\newcommand{\gxxversion}{2.95}
% current GNU release we use for development
\newcommand{\gxxcurrent}{2.95}

\newcommand{\tobewritten}{\vspace{\baselineskip}$<$TO BE WRITTEN$>$\vspace{\baselineskip}}
\newcommand{\tobeextended}{\vspace{\baselineskip}$<$TO BE EXTENDED$>$\vspace{\baselineskip}}


\usepackage{graphicx}
%% Comment this out to produce distribution version of manual.
\withcomments

\makeindex

\begin{document}

\title{{\Huge TPIE}\\ User Manual and Reference}
\author{%
Lars Arge \and 
Rakesh Barve \and
Andrew Danner \and
David Hutchinson \and 
Thomas M\o lhave, \and
Octavian Procopiuc \and 
J\"{o}rg Rotthowe \and
Laura Toma \and
Jan Vahrenhold \and
Darren Erik Vengroff \and 
Markus Vogel \and
Rajiv Wickeremesinghe}


\date{{\bf DRAFT} of \today}

\maketitle

\begin{titlepage}
\mbox{ }

\vspace{\fill}

\noindent TPIE User Manual and Reference

\noindent Edition \edition, for TPIE version \version.

\vspace{2ex}

\noindent Copyright \copyright 1994, 1995 Darren Erik Vengroff, 2002 Lars
Arge, Rakesh Barve, David Hutchinson, Octavian Procopiuc, Laura Toma,
Darren Erik Vengroff, Rajiv Wickeremesinghe, 2005 Lars Arge, Andrew
Danner, Thomas M\o lhave, Octavian Procopiuc, J\"{o}rg Rotthowe,
Laura Toma, Jan Vahrenhold, Markus Vogel.


\vspace{2ex}

This file is part of TPIE.

TPIE is free software: you can redistribute it and/or modify
it under the terms of the GNU Lesser General Public License as published by
the Free Software Foundation, either version 3 of the License, or
(at your option) any later version.

TPIE is distributed in the hope that it will be useful,
but WITHOUT ANY WARRANTY; without even the implied warranty of
MERCHANTABILITY or FITNESS FOR A PARTICULAR PURPOSE.  See the
GNU Lesser General Public License for more details.

You should have received a copy of the GNU Lesser General Public License
along with TPIE.  If not, see \texttt{http://www.gnu.org/licenses/}.


\end{titlepage}

\tableofcontents

\chapter*{Introduction}
\addcontentsline{toc}{chapter}{Introduction}

This manual describes TPIE, a Transparent Parallel I/O Environment,
designed to assist programmers in writing high performance
I/O-efficient programs for a variety of platforms.\comment{LA: Before
  distribution add a note about block collection stuff not documented
  yet}

\emph{This manual, like the whole of the TPIE project, is work in
  progress. The authors are making it available in its current state
  in the hopes that it will be useful, but without any warranty
  whatsoever. Refer to the copyright page at the beginning of this
  manual for full details. Please send comments, bug reports\index{bug
    reports}, etc., to \path"tpie@cs.duke.edu".}

%% Copyright 2008, The TPIE development team
%% 
%% This file is part of TPIE.
%% 
%% TPIE is free software: you can redistribute it and/or modify it under
%% the terms of the GNU Lesser General Public License as published by the
%% Free Software Foundation, either version 3 of the License, or (at your
%% option) any later version.
%% 
%% TPIE is distributed in the hope that it will be useful, but WITHOUT ANY
%% WARRANTY; without even the implied warranty of MERCHANTABILITY or
%% FITNESS FOR A PARTICULAR PURPOSE.  See the GNU Lesser General Public
%% License for more details.
%% 
%% You should have received a copy of the GNU Lesser General Public License
%% along with TPIE.  If not, see <http:%%www.gnu.org/licenses/>

\chapter*{Acknowledgements}
\addcontentsline{toc}{chapter}{Acknowledgements}

The development of TPIE was supported in part by the National Science
Foundation under grants CCR-9007851 and EIA-9870734 and by the U.S.
Army Research Office under grants DAAL03-91-G-0035 and
DAAH04-96-1-0013.\comment{LA: Update before next distribution}

The authors would like to thank the following people for their
contributions to the development of TPIE; Jeff Vitter, Paul Natsev,
Eddie Grove, Roberto Tamassia, Yi-Jen Chiang, Mike Goodrich, Jyh-Jong
Tsay, Tom Cormen, Len Wisniewski, Liddy Shriver, David Kotz, Owen
Astrachan, Lipyeow Lim, Vasilis Samoladas, and Min Wang.

\comment{LA: There is some more text from Darren's ack. in here.}

%I would like to thank the following people for helpful discussions
%concerning algorithms and implementation techniques which influenced
%the development of TPIE: 
%\htmladdnormallink{Jeff Vitter}{%
%\begin{rawhtml}
%  http://www.cs.duke.edu/~jsv/HomePage.html
%\end{rawhtml}%
%}, 
%\index{Vitter, Jeff}
%\htmladdnormallink{Eddie Grove}{%
%\begin{rawhtml}
%  http://www.cs.duke.edu/cgi-bin/facinfo?efg
%\end{rawhtml}%
%}, 
%\index{Grove, Eddie}
%\htmladdnormallink{Roberto Tamassia}{%
%\begin{rawhtml}
%  http://www.cs.brown.edu/people/rt/
%\end{rawhtml}%
%}, 
%\index{Tamassia, Roberto}
%\htmladdnormallink{Yi-Jen Chiang}{%
%\begin{rawhtml}
%  http://www.cs.brown.edu/people/yjc/
%\end{rawhtml}%
%}, 
%\index{Chiang, Yi-Jen}
%\htmladdnormallink{Mike Goodrich}{%
%\begin{rawhtml}
%  http://www.cs.jhu.edu/goodrich/home.html
%\end{rawhtml}%
%}, 
%Jyh-Jong Tsay, 
%\index{Tsay, Jyh-Jong}
%\htmladdnormallink{Lars Arge}{%
%\begin{rawhtml}
%  http://www.daimi.aau.dk/~large/
%\end{rawhtml}%
%},
%\index{Arge, Lars}
%\htmladdnormallink{Tom Cormen}{%
%\begin{rawhtml}
%  http://www.cs.dartmouth.edu/faculty/cormen.html
%\end{rawhtml}%
%}, 
%\index{Cormen, Tom}
%Len Wisniewski, 
%\index{Wisniewski, Len}
%Liddy Shriver,
%\index{Shriver, Liddy}
%and 
%\htmladdnormallink{David Kotz}{%
%\begin{rawhtml}
%  http://www.cs.dartmouth.edu/faculty/kotz.html
%\end{rawhtml}%
%}.
%\index{Kotz, David}

%I would also like to thank the following people and institutions for
%providing access to the hardware on which TPIE design and development
%are ongoing: 
%\htmladdnormallink{Brown University Department of Computer
%  Science}{http://www.cs.brown.edu}, 
%\index{Brown University!Department of Computer Science}
%for Sun Sparc 10s running
%Solaris; 
%\htmladdnormallink{Duke University Department of Computer
%  Science}{http://www.cs.duke.edu}, 
%\index{Duke University!Department of Computer Science}
%for a variety of Sun workstations
%running SunOS and for DEC Alphas running OSF/1; 
%Yale Patt\index{Patt, Yale} and the ACAL Lab in the 
%\htmladdnormallink{Department of Electrical Engineering and
%  Computer Science}{http://www.eecs.umich.edu} at the University of
%Michigan, 
%\index{University of Michigan!ACAL Lab}
%for a DEC Alpha running OSF/1 and for an HP 9000 running
%HP-UX; 
%\htmladdnormallink{David
%  Kotz}{http://www.cs.dartmouth.edu/faculty/kotz.html} and the
%\htmladdnormallink{Dartmouth College Department of Computer
%  Science}{http://www.cs.dartmouth.edu},
%\index{Kotz, David}\index{Dartmouth College!Department of Computer Science}
% for MIPS based DECstations
%running Ultrix.  

%Finally, I would like to thank 
%\htmladdnormallink{Owen Astrachan}{http://www.cs.duke.edu/\~ola/HomePage.html}
%\index{Astrachan, Owen}
%for his helpful discussions on some of the finer points of the C++
%language.
 % Chapter: Acknowledgments

\part{User Manual}
  %%
%% $Id: user_manual.tex,v 1.10 1999-06-21 00:30:51 rajiv Exp $
%%
\chapter{Overview}

As of today, gigabyte computer systems exist on desktops, and terabyte
systems are not unheard of.  In the not too distant future, systems
designed to manage petabytes of information
will come on-line.  The most important characteristic of such vast
amounts of data is that they cannot possibly be stored in the primary
memories of even the most powerful computers.  Instead, they must be
stored on secondary memory, such as magnetic disks, or tertiary
memory, such as tapes and optical memory.  Compared to CPUs and solid
state random access memory, these devices are extraordinarily slow;
the difference in access time is typically 2 to 5 orders of magnitude.
Because of the low speed of secondary storage, good performance in the
Input/Output (I/O) system that links secondary storage to main memory
and the CPU or CPUs is critical if good performance is to be achieved
overall.  Performance can be further improved if many disks can be
efficiently used in parallel.  Unfortunately, existing I/O systems
generally do not perform adequately~\cite{patt:computer}.

In recent years, computer science theorists have studied the problem
of efficiently using parallel disks\index{parallel disks}
\index{disks!parallel|see{parallel disks}} to solve a variety
of computational problems.  At the same time, a number of parallel I/O
systems have become available, though in most cases they have failed
to take adequate advantage of the insights theorists have had to offer
\cite{cormen:integrate-tr}.  TPIE, a transparent parallel I/O
environment, is designed to bridge the gap between the theory and
practice of parallel I/O systems.  It is intended to demonstrate that
a parallel I/O system can do all of the following simultaneously:
\begin{itemize}
\item Abstract away the details of how I/O is performed so that
  programmers need only deal with a simple high level interface.
\item Implement I/O-optimal paradigms for large scale computation that
  are efficient not only in theory, but also in practice.
\item Remain flexible, allowing programmers to specify the functional
  details of computation taking place within the supported paradigms.
  This will allow a wide variety of algorithms to be implemented
  within the system.
\item Be portable across a variety hardware platforms.
\item Be extensible, so that new features can be easily added later.
\end{itemize}

TPIE is implemented as a set of templated classes and functions in
C++.\index{C++} It also includes a small library and a set of test and
sample applications.

\section{Hardware Platforms}
\index{hardware platforms}

TPIE has been tested on a variety of hardware platforms with a variety
of flavors of UNIX operating systems.  Combinations that have been
tested include:
\begin{itemize}
\item Sun Sparcstation/SunOS 4.x 
\item Sun Sparcstation/Solaris 5.x
\item DEC Alpha/OSF/1 1.x and 2.x
\item HP 9000/HP-UX
\item Intel Pentium/Linux 1.x
\end{itemize}

\chapter{Obtaining and Installing TPIE}

\section{Licensing}

TPIE is available under the terms of the GNU General Public License,
\index{license}
version 2.  A copy of this license appears in Appendix~\ref{app:gpl}.

\section{Where to get TPIE}

The latest version of TPIE, \version, is an alpha test version.  It is
available through the \htmladdnormallink{TPIE WWW Home Page}{%
\begin{rawhtml}
http://www.cs.duke.edu/TPIE/
\end{rawhtml}%
}%
\begin{latexonly}
at URL \verb|http://www.cs.duke.edu/TPIE/|.
\end{latexonly}

To obtain the TPIE source distribution\index{source distribution}, follow the
pointers from the home page to the distribution itself, which consists
of a gzipped tar file named {\tt tpie-\version.tgz}.  Your Web
browser should be capable of downloading this file to your local
machine.

\section{Prerequisites}
\index{GNU software}
\label{sec:gnu-software}

To uncompress and unarchive the distribution, you will need either the GNU
\verb|tar| utility, or \verb|gzip| and a \verb|tar| program. (the GNU
version can decompress and untar at the same time with the '\verb|z|' option).

TPIE is dependent on the compiler used, mainly because of the template
syntax. It currently requires the GNU C++ compiler, \verb|g++|,
version~\gxxversion. We
expect that it will also be compatible with future version of this
compiler.  TPIE has also been successfully compiled using
\verb|egcs|, version 2.91.66.

The GNU \verb|make| utility is required. This is usually located in
\verb|/usr/local/bin/make|, or is called \verb|gmake|.

In general, invoking the tools with the single command line argument 
\verb|--version| will indicate whether they are compatible.
Information on where and how to obtain and install GUN software is
available from 
{\tt http://www.gnu.org/software/software.html}.

\section{Installation}
\index{installation}

Once you have obtained the TPIE source distribution file
{\tt tpie-\version.tgz}, you must decide where to install it.
\verb|/usr/local/tpie/| is a typical place.

Place {\tt tpie-\version.tgz} in the directory in which TPIE is to
be installed, \verb|cd| into that directory, and execute the command

\begin{flushleft}
{\tt tar xzf tpie-\version.tgz}
(or {\tt gunzip -c tpie-\version.tgz | tar xvf -} )  
\end{flushleft}


This will produce a directory {\tt tpie-\version} with subdirectories
\verb|include|, \verb|lib|, \verb|test|, and \verb|doc|.  Enter the
directory {\tt tpie-\version}.  You must now configure TPIE for your
particular system.  To do this, use the command

\begin{verbatim}
./configure
\end{verbatim}

\index{configuration}The configuration program will take some time to
examine the parameters of your system.  Once it has done so, it will
produce the various Makefiles and configuration files required to
build TPIE on your system.  When this is done, simply invoke your version
of GNU \verb|make|:

\begin{verbatim}
make all
\end{verbatim}

to build the complete TPIE system.  This will build the following
components:

\begin{description}
\item[\verb|include|] The TPIE header files.\index{header files}
\item[\verb|lib|] The TPIE library.  This is relatively small, as most
  of the TPIE system remains in the form of templated header
  files.\index{library}
\item[\verb|test|] A series of test applications designed to verify
  that TPIE is operating correctly.  This directory also includes the
  code to the example applications described in
  Chapter~\ref{ch:examples}.\index{test applications}
\item[\verb|doc|] Complete documentation for TPIE, consisting of the
  document you are reading right now in various formats: HTML, and DVI and
  Postscript(TM) for printing.\index{documentation}
\end{description}





\section{Customization}
\index{Customization}

It is possible to customize the installation by providing arguments to
the {\tt configure} script.\index{configuration:options} None of these
arguments are necessary, and the first time you build TPIE, you should
probably not need any of them.  The arguments recognized are as
follows:
\begin{description}
\item[\verb|--enable-log-lib|] 
  \index{enable-log-lib@{\tt --enable-log-lib}}
  Enable logging in TPIE library code.
  This can also be accomplished at compile time by defining the macro
  \verb|TP_LOG_LIB| using the syntax \verb|make lib TP_LOG_LIB=1|.
  This is useful for debugging the TPIE library, but slows it down.
  This option works by defining \verb|TPL_LOGGING|
  \index{TPL_LOGGING@{\tt TPL\_LOGGING}} (see Section~\ref{sec:macros})
  when compiling the library. 
  Section \ref{sec:logging} discusses TPIE logging.
\item[\verb|--enable-assert-lib|]  
  \index{enable-assert-lib@{\tt --enable-assert-lib}}
  Enable assertions in the TPIE library code for debugging purposes.
  This can also be accomplished at compile time by defining the macro
  \verb|TP_ASSERT_LIB| using the syntax \verb|make lib TP_ASSERT_LIB=1|.
  This option works by defining \verb|DEBUG_ASSERTIONS|
  \index{DEBUG_ASSERTIONS@{\tt DEBUG\_ASSERTIONS}} 
  (see Section~\ref{sec:macros})
  when compiling the library.
\item[\verb|--enable-log-apps|]  and
\item[\verb|--enable-assert-apps|]  
  \index{enable-assert-apps@{\tt --enable-assert-apps}}
  \index{enable-log-apps@{\tt --enable-log-apps}}
  Similar to {\tt --enable-log-lib} and {\tt --enable-assert-lib}, but
  they apply to the test application code.  Running \verb|make test|
  with the options \verb|TP_LOG_APPS=1| and/or \verb|TP_ASSERT_APPS=1|
  accomplishes the same thing.
\item[\verb|--enable-expand-ami-scan|]  Expand the macros in the file
  \verb|ami_scan.h| when making the include directory with the
command {\tt make include} (or {\tt make all}).  This is mainly useful for
debugging the code in \verb|ami_scan.h| itself, and is not normally
needed by TPIE programmers.  It may make compilation of TPIE programs
slightly faster because the macro processor of the C++ compiler will
have less work to do.  In addition to the standard GNU tools mentioned
in Section~\ref{sec:gnu-software}, this requires \verb|perl|.
\item[\verb|--disable-*|]  Any of the options above can be explicitly
  disabled  by using this syntax.  For example
  \verb|--disable-expand-ami-scan|. 
\end{description}

\chapter{A Taste of TPIE via a Sample Program}
\label{ch:samplepgm}
\input{samplepgm.tex}

\chapter{Tutorial}
\label{ch:tutorial}

\section{Introduction}

This tutorial is designed to introduce new users to the TPIE system.
It introduces the fundamental paradigms of computation that TPIE
supports, giving source code examples of each.  The majority of the
code presented in the tutorial is available in the test
applications\index{test applications} directory of the distribution, 
{\tt tpie-\version/test/}.

For the sake of brevity, much of the code presented in this tutorial
is incomplete, in the sense that necessary header files 
\index{header files} and macros\index{macros} are omitted.  Details 
concerning how to write your own complete TPIE code is presented at
the end of the tutorial in Section~\ref{sec:complete}.

\section{C++}

If you would like the use TPIE but are not familiar with the
C++\index{C++} language, a number of good books are available.  If you
are familiar with C\index{C}, \cite{pohl:c++} is a good place to
start.  A more basic, but very comprehensive book
is~\cite{deitel:c++}.  Once you have mastered the basics,
\cite{meyers:effective} an excellent source of information on
intermediate and advanced C++.  Finally, \cite{ellis:arm} is the
definitive book on C++, though not necessarily the bast place for new
programmers to start.


\section{Streams}

\index{stream}\index{structure!of streams}
Conceptually, TPIE programs work with streams of data stored on
external memory.  A stream is an ordered collection of objects of a
particular type.  Various paradigms of computation are defined on
these streams, though the functional details of the computation
performed within these paradigms is left to the TPIE programmer to
specify.  These details are specified using an operation management
object,\index{operation management object} which is an object with
member functions designed to work with the particular paradigm being
used.  Operation management objects are also known as operation
managers.\index{operation manager|see{operation management object}}.

Creating a stream of objects in TPIE is very much like creating any other
object in C++.  The only difference is that data placed in the stream,
whether explicitly, or as is more commonly the case, implicitly, is
stored on disk.  For example, to create a stream of integers, we could
use either of the following:
\begin{verbatim}
AMI_STREAM<int> stream0;

AMI_STREAM<int> *pstream0 = new AMI_STREAM<int>;
\end{verbatim}

The {\tt AMI} in {\tt AMI\_STREAM} stands for Access Method
Interface\index{access method interface}, which is the level of TPIE
that most applications interact with.  
{\tt AMI\_STREAM} is actually a macro that evaluates to the name of a
particular implementation of streams at the AMI level, but for now it
is safe to assume that it is simply a class.

The {\tt AMI\_STREAM} constructor does not actually put anything into
the stream; it simply creates the necessary data structures to keep
track of the contents of the stream when data is actually put into it.
Data is typically put into streams using \verb|AMI_scan()|, which is
described in the next section.

\section{Scanning}
\label{sec:scanning}

\index{scanning|(} \index{AMI_scan()@{\tt AMI\_scan()}}
The simplest paradigm available in TPIE is scanning.  Scanning can be
used to produce streams, examine the contents of streams, or transform
streams.  

\subsection{Basic Scanning}

The most basic thing a scan can do is write a series of objects to a
stream.  In the following example, we create a stream of integers
consisting of the first 10000 natural numbers.

\begin{verbatim}
class scan_count : AMI_scan_object {
private:
    int maximum;
public:
    int ii;

    scan_count(int max = 1000) : maximum(max), ii(0) {};

    AMI_err initialize(void) 
    {
        ii = 0;
        return AMI_ERROR_NO_ERROR;
    };

    AMI_err operate(int *out1, AMI_SCAN_FLAG *sf)
    {
        *out1 = ++ii;
        return (*sf = (ii <= maximum)) ? AMI_SCAN_CONTINUE : 
            AMI_SCAN_DONE;
    };
};

scan_count sc(10000);
AMI_STREAM<int> amis0;    

void f()
{
    AMI_scan(&sc, &amis0);
}
\end{verbatim}

The class \verb|scan_count| is a class of scan management
object\index{operation management object!scan}.  It has two member
functions, \verb|initialize()| and \verb|operate()|, which TPIE calls
when asked to perform a scan.  The first member function,
\verb|initialize()| is called at the beginning of the scan.  TPIE
expects that a call to this member function will cause the object to
initialize any internal state it may maintain in preparation for
performing a scan.  The second member function, \verb|operate()|, is
called repeatedly during the scan to create objects to go into the
output stream.  \verb|operate()| sets the flag \verb|*sf| to indicate
whether it generated output or not.  Only when \verb|operate()|
returns either an error or \verb|AMI_SCAN_DONE| does TPIE stop calling
it.

The call to \verb|AMI_scan| behaves as the following pseudo-code:

\begin{verbatim} 
AMI_err AMI_scan(scan_count &sc, AMI_STREAM<int> *pamis)
{
    int ii;
    AMI_err ae;    
    AMI_SCAN_FLAG sf;

    sc.initialize();    
    while ((ae = sc.operate(&ii, &sf)) == AMI_SCAN_CONTINUE) {
        if (sf) {
            write ii to *pamis;
        }
    }

    if (ae != AMI_SCAN_DONE) {
        handle error conditions;
    }

    return AMI_ERROR_NO_ERROR;
}
\end{verbatim}

Thus, after the function \verb|f()| in the original example code is
called, the stream \verb|amis0| contains the integers from 1 to 10000
in order.

Now that we have produced a stream, there are a variety of things we
can do with it.  One of the simplest things we can do with a stream of
objects is scan it in order to transform it in some way.  As an
example, suppose we wanted to square every integer in the stream
\verb|amis0|.  We could do so using the following code:

\begin{verbatim}
class scan_square : AMI_scan_object {
public:
    AMI_err initialize(void)
    {
        return AMI_ERROR_NO_ERROR;
    };

    AMI_err operate(const int &in, AMI_SCAN_FLAG *sfin,
                    int *out, AMI_SCAN_FLAG *sfout)
    {
        if (*sfout = *sfin) {
            *out = in * in;
            return AMI_SCAN_CONTINUE;
        } else {
            return AMI_SCAN_DONE;
        }
    };
};

scan_square ss;
AMI_STREAM<int> amis1;    

void g() 
{
    AMI_scan(&amis0, &ss, &amis1);
}
\end{verbatim}

Notice that the call to \verb|AMI_scan()| in \verb|g()| differs from
the one we used in \verb|f()| in that it takes two stream pointers and
a scan management object.  By convention, the stream \verb|amis0| is
an input stream, because it appears before the scan management object
\verb|ss| in the argument list.  By similar convention, \verb|amis1|
is an output stream.  Because the call to \verb|AMI_scan| has one
input stream and one output stream, TPIE expects the \verb|operate()|
member function of \verb|ss| to have one input argument (which is
called \verb|in| in the example above) and one output argument (called
\verb|out| in the example above).  Note that the \verb|operate()|
member function of the class \verb|square_scan| also takes two
pointers to flags, one for input (\verb|sfin|) and one for output
(\verb|sfout|).  \verb|*sfin| is set by TPIE to indicate that there is
more input to be processed.  \verb|*sfout| is set by the scan
management object to indicate when output is generated.
If a scan management object has no polymorph of \verb|operate()| that
takes the appropriate type number of arguments for the invocation of
\verb|AMI_scan()| that uses it then a compile-time error is generated.

A call to \verb|AMI_scan| with one input stream and one output stream
behaves as the following pseudo-code:

\begin{verbatim} 
AMI_err AMI_scan(AMI_STREAM<int> *instream, scan_square &ss, 
        AMI_STREAM<int> *outstream)
{
    int in, out;
    AMI_err ae;    
    AMI_SCAN_FLAG sfin, sfout;

    sc.initialize();

    while (1) {
        {
             read in from *instream;
             sfin = (read succeeded);
        }
        if ((ae = ss.operate(in, &sfin, &out, &sf)) == 
            AMI_SCAN_CONTINUE) {
            if (sfout) {
                write out to *outstream;
            }
            if (ae == AMI_SCAN_DONE) {
                return AMI_ERROR_NO_ERROR;
            }
            if (ae != AMI_SCAN_CONTINUE) {
                handle error conditions;
            }
        }
    }
}
\end{verbatim}

More complicated invocations of \verb|AMI_scan()| can operate on up
to four input streams and four output streams.  Here is an example
that takes two input streams of values, \verb|x| and \verb|y|, and
produces four output streams, 
one consisting of the running sum of the
\verb|x| values,
one consisting of the running sum of the
\verb|y| values,
one consisting of the running sum of the
squares of the \verb|x| values,
and
one consisting of the running sum of the
squares of the \verb|y| values.

\begin{verbatim}
class scan_sum : AMI_scan_object {
private:
    double sumx, sumx2, sumy, sumy2;
public:
    AMI_err initialize(void)
    {
        sumx = sumy = sumx2 = sumy2 = 0.0;
        return AMI_ERROR_NO_ERROR;
    };

    AMI_err operate(const double &x, const double &y, 
                    AMI_SCAN_FLAG *sfin,
                    double *sx, double *sy, 
                    double *sx2, double *sy2, 
                    AMI_SCAN_FLAG *sfout)
    {
        if (sfout[0] = sfout[2] = sfin[0]) {
            *sx = (sumx += x);
            *sx2 = (sumx2 += x * x);
        }
        if (sfout[1] = sfout[3] = sfin[1]) {
            *sy = (sumx += y);
            *sy2 = (sumy2 += y * y);
        }        
        return (sfin[0] || sfin[1]) ? AMI_SCAN_CONTINUE : AMI_SCAN_DONE;
    };
};

AMI_STREAM<double> xstream, ystream;

AMI_STREAM<double> sum_xstream, sum_ystream, sum_x2stream, sum_y2stream;

scan_sum ss;

void h()
{
    AMI_scan(&xstream, &ystream, &ss, 
             &sum_xstream, &sum_ystream, &sum_x2stream, &sum_y2stream);
}
\end{verbatim}

\subsection{ASCII Input/Output} \label{sec:ascii-io}

\index{ASCII I/O|see{scanning, ASCII I/O}}
\index{scanning!ASCII I/O|(}
TPIE provides a number of predefined scan management objects.  Among
the most useful are instances of the template classes
\verb|cxx_ostream_scan<T>| and \verb|cxx_ostream_scan<T>|, which are
used for reading ASCII data into streams and writing the contents of
streams in ASCII respectively.  This is done in conjunction with the
\verb|iostream| facilities provided in the standard C++ library.  Any
class \verb|T| for which the operators \verb|ostream
&operator<<(ostream &s, T &t)| and \verb|istream &operator>>(T &t)|
are defined can be used with this mechanism.

As an example, suppose we have a file called \verb|input_nums.txt|
containing one integer per line, such as

\begin{verbatim}
17
289
4195835
3145727
.
.
.
\end{verbatim}

To read this file into a TPIE stream of integers, square each, and
write them out to the file \verb|output_nums.txt| we could use the
following code:

\begin{verbatim}
void f()
{
    ifstream in_ascii("input_nums.txt");
    ofstream out_ascii("input_nums.txt");
    cxx_istream_scan<int> in_scan(in_ascii);
    cxx_ostream_scan<int> out_scan(out_ascii);
    AMI_STREAM<int> in_ami, out_ami;
    scan_square ss;    

    // Read them.
    AMI_scan(&in_scan, &in_ami);

    // Square them.
    AMI_scan(&in_ami, &ss, &out_scan);
    
    // Write them.
    AMI_scan(&out_ami, out_scan);

}    
\end{verbatim}

In order to read from an input file using the scan object
\verb|in_scan|, \verb|AMI_scan()| repeatedly calls
\verb|in_scan->operate()|, just as it would for any scan object.  Each
time \verb|in_scan->operate()| is called, it uses the \verb|>>|
operator to read a single integer from the input file.  When the input
file is exhausted, \verb|in_scan->operate()| returns
\verb|AMI_SCAN_DONE|, and \verb|AMI_scan()| returns to its caller.
The behaviour of \verb|out_scan| is similar to that of \verb|in_scan|,
except that it writes to a file instead of reading from one.
\index{scanning!ASCII I/O|)}

\subsection{Multi-Type Scanning}

\index{scanning!multi-type|(}

In all of the examples presented up to this point, scanning has been
done on streams of objects that are all of the same type.
\verb|AMI_scan()| is not limited to such scans, however.  In the
following example, we have a scan management class that takes two
streams of \verb|double|s and returns a stream of complex numbers.

\begin{verbatim}
class complex {
public:
    complex(double real_part, imaginary_part);
    ...
};

class scan_build_complex : AMI_scan_object {
public:
    AMI_err initialize(void) {};
    AMI_err operate(const double &r, const double &i, 
                    AMI_SCAN_FLAG *sfin,
                    complex *out, AMI_SCAN_FLAG *sfout)
    {
        if (*sfout = (sfin[0] || sfin[1])) {
            *out = complex((sfin[0] ? r : 0.0), (sfin[1] ? i : 0.0));
            return AMI_SCAN_CONTINUE;
        } else {
            return AMI_SCAN_DONE;
        }   
    };
};
\end{verbatim}
\index{scanning!multi-type|)}

\subsection{Out of Step Scanning}
\label{sec:out-of-step}

\index{scanning!out of step|(}
In all the examples up to this point, every call to the
\verb|operate()| member function of a scan management object has been
called with each object in the input stream(s) exactly once.  In this
section, we introduce the concept of out of step scanning, which
allows a scan management object to reject certain inputs and ask that
they be resubmitted in subsequent calls to the \verb|operate()| member
function.

Suppose we have two streams of integers, each of which we know is
sorted in ascending order.  We would like to merge the two streams
into a single output stream consisting of all the integers in the two
input streams, in sorted order.  In order to do this with a scan, we
must have the ability to look at the next integer from each stream,
choose the smaller of the two and write it to the output stream, and
then ask for the next number from the stream from which it was taken.
Luckily, there is a simple mechanism for doing this.  The same flags
that TPIE uses to tell the scan management object which inputs are
available can be used by the scan management object to indicate which
inputs were used and which should be presented again.

Consider the following example of a scan management object class which
performs exactly the sort of binary
merge\index{merge!binary}\index{merge sort!binary} described in the
preceding paragraph:

\begin{verbatim}
class scan_binary_merge : AMI_scan_object {
public:
    AMI_err initialize(void) {};
    
    AMI_err operate(const int &in0, const int &in1, AMI_SCAN_FLAG *sfin,
                    int *out, AMI_SCAN_FLAG *sfout) 
    {
        if (sfin[0] && sfin[1]) {
            if (in0 < in1) {
                sfin[1] = false;
                *out = in0;
            } else {
                sfin[0] = 0;
                *out = in1;
            }
        } else if (!sfin[0]) {
            if (!sfin[1]) {
                *sfout = 0;
                return AMI_SCAN_DONE;
            } else {
                *out = in1;
            }
        } else {
            *out = in0;
        }
        *sfout = 1;
        return AMI_SCAN_CONTINUE;
    }    
};
\end{verbatim}

In the operate method, we first check that both inputs are valid by
looking at the flags pointed to by \verb|sfin|.  If both are valid,
then we select the smaller of the inputs and copy it to the output.
We then clear the other input flag to let TPIE know that we did not
use that input, but we will need it later and it should be resubmitted
on the next call to operate.  The remainder of the function handles
the cases when one of more of the input streams in empty.
\index{scanning!out of step|)}
\index{scanning|)}

\section{Merging} \label{sec:merging}
\index{merging|(}

The binary merging scan management class presented in the previous
section could be used recursively to implement a merge
sorting\index{merge sorting!binary} algorithm.  We could simply divide
the input stream into sub-streams small enough to fit into main
memory, read each sub-stream into memory and sort it, and then merge
pairs of streams, then pairs of merged pairs of streams, and so on,
until we had merged all the input back into one completely sorted
stream.  While this approach would correctly sort the input, it would
not be nearly as efficient as possible on most machines.  The reason
is that we typically have enough main memory available to merge many
streams together at one time.

Taking advantage of all available main memory can be difficult, since
we must explicitly keep track to the space needed for input blocks
form each of the streams being merged, as well as the overhead of any
data structures needed for the merge.  Luckily, TPIE provides a
mechanism that does most of the work for us.  The function
\verb|AMI_partition_and_merge()| divides an input stream into
sub-streams just small enough to fit into main memory, operates on
each in main memory, then merges them back into a single output
stream, using intermediate streams if memory constraints dictate.  As
was the case with \verb|AMI_scan()|, the functional details of
\verb|AMI_partition_and_merge()| are specified via an operation
management object,\index{operation management object} as shown in the
following example:

\begin{verbatim}
class my_merger : AMI_merge_manager {
public:
    AMI_err initialize(arity_t arity, const T * const *in,
                       AMI_merge_flag *taken_flags,
                       int &taken_index);
    AMI_err operate(const T * const *in, AMI_merge_flag *taken_flags,
                    int &taken_index, T *out);
    AMI_err main_mem_operate(T* mm_stream, size_t len);
    size_t space_usage_overhead(void);
    size_t space_usage_per_stream(void);
};

AMI_STREAM<T> instream, outstream;

void f() 
{
    my_merger mm;    
    AMI_partition_and_merge(&instream, &outstream, &mm);
}
\end{verbatim}

The class members are as follows:

\begin{description}
\item[\verb|initialize()|] Tells the object how many streams it should
  merge (\verb|arity|) and what the first item from each stream is
  (\verb|in|).  \verb|taken_flags| and \verb|taken_index| provide two
  mechanisms for the merge manager to tell TPIE what objects it took
  from the input streams.  These are discussed in more detail in 
  the context of a merge sorting example in Section~\ref{sec:mergesort}.
\item[\verb|operate()|]
Just as in scanning, this member function is called repeatedly to process input objects.
\item[\verb|main\_mem\_operate()|]
Operates on an array of data in main memory when a sub-stream is small enough to fit entirely in main memory.
\item[\verb|space\_usage\_overhead()|]
Called by TPIE prior to initialization to asses how much main memory this object will use.
\item[\verb|space\_usage\_per\_item()|]
Called by TPIE prior to initialization to asses how much main memory may be used per input stream.  Merge management objects are allowed to use main memory space linear in the number of input streams.
\end{description}

\verb|AMI_partition_and_merge()| behaves as indicated by the following
pseudo-code.  Note that for simplicity of presentation, boundary
conditions are not covered.

\begin{verbatim}
AMI_err AMI_partition_and_merge(instream, outstream, mm)
{
    max_ss = max # of items that can fit in main memory;
    partition instream into num_substreams substreams of size max_ss;

    foreach substream[i] {
        read substream[i] into main memory;
        mm->main_mem_operate(substream[i]);
        write substream[i];
    }

    call mm->space_usage_overhead() and mm->space_usage_per_stream;
    
    compute merge_arity; // Maximum # of streams we can merge.     

    while (num_substreams > 1) {
        for (i = 0; i < num_substreams; i += merge_arity) {
            merge substream[i] .. substream[i+merge_arity-1];
        }
        num_substreams /= merge_arity;
        max_ss *= merge_arity;
    }

    write single remaining substream to outstream;
        
    return AMI_ERROR_NO_ERROR;
}
\end{verbatim}

\subsection{Implementing Mergesort: An Extended Example}
\label{sec:mergesort}

Here is an example of the implementation and use of a merge management
object for merge sorting integers.  First, we declare the class:

\begin{verbatim}
class s_merge_manager : public AMI_merge_base<int> {
private:
    arity_t input_arity;
    pqueue *pq;
public:
    s_merge_manager(void);
    virtual ~s_merge_manager(void);
    AMI_err initialize(arity_t arity, const int * const *in,
                       AMI_merge_flag *taken_flags,
                       int &taken_index);
    AMI_err operate(const int * const *in, AMI_merge_flag *taken_flags,
                    int &taken_index, int *out);
    AMI_err main_mem_operate(int* mm_stream, size_t len);
    size_t space_usage_overhead(void);
    size_t space_usage_per_stream(void);
};
\end{verbatim}

In addition to the standard class members for a merge management
object, we have the following:

\begin{description}
\item[\verb|input\_arity|]
The number of input streams the merge management object must handle.
\item[\verb|pq|]
A priority queue into which items will be placed.
\item[\verb|s\_merge\_manger()|]
A constructor.
\item[\verb|~s\_merge\_manger()|]
A destructor.
\end{description}

Construction and destruction are fairly straightforward.  At
construction time, we have no priority queue because we do not yet
know how big the priority queue should be.  \verb|pq| will be set up
when \verb|initialize| is called.  The destructor checks whether
\verb|pq| is valid, and deletes it if it is.  The constructor and
destructor are implemented as follows:

\begin{verbatim}
s_merge_manager::s_merge_manager(void)
{
    pq = NULL;
}

s_merge_manager::~s_merge_manager(void)
{
    if (pq != NULL) {
        delete pq;
    }
}
\end{verbatim}

When \verb|AMI_merge()| is called with a merge management object of
type \verb|s_merge_manager|, the first member functions called are
\verb|space_usage_overhead()| and \verb|space_usage_per_stream()|.
These return the number of bytes of main memory that the merge
management object will allocate when initialized.  
\verb|space_usage_overhead()|'s return value indicates that space will
be needed for a priority
queue.  
\verb|space_usage_per_stream()|'s return value indicates that for each
input stream, space (which is to be allocated when the priority
queue is constructed) will be needed for an integer and an arity type.

\begin{verbatim}
size_t s_merge_manager::space_usage_overhead(void)
{
    return sizeof(pqueue<arity_t,int>);
}


size_t s_merge_manager::space_usage_per_stream(void)
{
    return sizeof(arity_t) + sizeof(int);
}
\end{verbatim}

The next member function called by \verb|AMI_merge()| is
\verb|main_mem_operate()|, which is called to handle the initial
substreams that are small enough to fit in main
memory.  Since we are sorting, we will simply use
quicksort.

\begin{verbatim}
AMI_err s_merge_manager::main_mem_operate(int* mm_stream, size_t len)
{
    qsort(mm_stream, len, sizeof(int), c_int_cmp);
    return AMI_ERROR_NO_ERROR;
}
\end{verbatim}

Having sorted all of the initial substreams, \verb|AMI_merge()| begins
to merge them.  Before merging a set of substreams, the merge
management object's member function \verb|initialize()| is called to
inform the merge management object of the number of streams it should
be prepared to merge.  The object is also provided with the first
object from each of the streams to be merged.  For objects of the
class \verb|s_merge_manager|, the \verb|initialize()| member function
is as follows:

\begin{verbatim}
AMI_err s_merge_manager::initialize(arity_t arity, CONST int * CONST *in,
                                          AMI_merge_flag *taken_flags,
                                          int &taken_index)
{
    arity_t ii;

    input_arity = arity;

    if (pq != NULL) {
        delete pq;
    }

    // Construct a priority queue that can hold arity items.
    pq = new pqueue_heap_op(arity);

    for (ii = arity; ii--; ) {
        if (in[ii] != NULL) {
            taken_flags[ii] = 1;
            pq->insert(ii,*in[ii]);
        } else {
            taken_flags[ii] = 0;
        }
    }

    taken_index = -1;
    return AMI_MERGE_READ_MULTIPLE;
}
\end{verbatim}

Note the use of the return value \verb|AMI_MERGE_READ_MULTIPLE|.  This
indicates that the flags pointed to by \verb|*taken_flags| are set to
indicate which of the inputs were used and should not be presented
again.  This is very similar to the use of input flags to indicate
which inputs were used by a scan management object as described in
Section~\ref{sec:out-of-step}.  The reason that we have a special
return value to indicate when these flags are used to increase
performance.  In order for \verb|AMI_scan()| to determine which inputs
were taken, it must examine all the flags.  In a many way merge, this
might be time consuming.  In cases where only one item is taken, its
index can be returned in \verb|taken_index| in order to save the time
that would be spent scanning the flags.  This technique is used in the
\verb|operate()| member function, whose implementation is as follows:

\begin{verbatim}
AMI_err s_merge_manager::operate(CONST int * CONST *in,
                                       AMI_merge_flag *taken_flags,
                                       int &taken_index,
                                       int *out)
{
    // If the queue is empty, we are done.  There should be no more
    // inputs.
    if (!pq->num_elts()) {
        return AMI_MERGE_DONE;
    } else {
        arity_t min_source;
        int min_t;

        pq->extract_min(min_source,min_t);
        *out = min_t;
        if (in[min_source] != NULL) {
            pq->insert(min_source,*in[min_source]);
            taken_index = min_source;
        } else {
            taken_index = -1;
        }
        return AMI_MERGE_OUTPUT;
    }
}
\end{verbatim}
\index{merging|)}

\section{Distribution} \label{sec:distribution}

Distribution has not been implemented in the current version of TPIE.
It is primarily useful for parallel disks, and will be implemented in
the parallel disk version of TPIE.  On a single disk, merging should
be adequate for all applications where distribution might be
considered.

On a single disk, distribution will tend to result in algorithms that
take roughly twice as long as similar algorithms that use merging.
This is because distribution is done to the square root of the number
of streams that can be buffered in main memory rather than the full
number.  This results in recursion that is twice as deep.

\section{Permutation}

\subsection{General Permutation}

Permutation is a basic building block for many I/O algorithms.
Routing a general permutation in the I/O model is asymptotically as
complex as sorting, though for some important classes of permutations,
such as BMMC permutations (See Section~\ref{sec:bit-permuting}) faster
algorithms are possible.  In this section, we discuss
\verb|AMI_general_permute()|, which routes arbitrary permutations, but
always takes as long as sorting, regardless of whether the particular
permutation can be done more quickly or not.

General permutations are routed using the function
\verb|AMI_general_permute()|.  Like other AMI functions,
\verb|AMI_general_permute()| relies on an operation management
object\index{operation management object} to determine its precise
behavior.  Unlike functions covered up to now, however, the type of
the operation management object\index{operation management object}
need not depend on the type of object in the stream being permuted.

A general permutation management object must provide two member
functions, \verb|initialize()| and \verb|destination|.
\verb|initialize()| is called to inform the general permutation object
of the length of the stream to be permuted.  \verb|destination()| is
then called repeatedly to determine the destination for each object in
the stream based on it's initial position.

Here is an example of using general permutation to reverse the order
of the items in a stream.

\begin{verbatim}
class reverse_order : public AMI_gen_perm_object {
private:
    off_t total_size;
public:
    AMI_error initialize(off_t ts) { 
        total_size = ts; 
        return AMI_ERROR_NO_ERROR;
    };
    off_t destination(off_t source) {
        return total_size - 1 - source;
    };
};

AMI_STREAM<int> amis0, amis1;    

void f()
{
    reverse_order ro;

    AMI_general_permute(&amis0, &amis1, (AMI_gen_perm_object *)&ro);
}
\end{verbatim}

\subsection{Bit Permutation}
\label{sec:bit-permuting}

Bit permuting is a permutation technique in which the destination
address of a given item is computed by manipulating the bits of its
source address.  The particular class of bit permutations that TPIE
supports is the set of bit matrix multiply complement (BMMC)
permutations.  These permutations are defined on sets of objects whose
size is a power of 2.

Suppose we have an input consisting of $N = 2^n$ objects.  A BMMC
permutation on the input is defined by a nonsingular $n @times n$ bit
matrix $A$ and an $n$ element column vector $c$ of bits.  Source and
destination addresses are interpreted as column vectors of bits, with
the low order bit of the address at the top.  The destination address
$x'$ corresponding to a given source address $x$ is computed as
$$x' = Ax + c$$ where addition and multiplication of matrix elements
is done over GF(2).  For a detailed description of BMMC permutations,
see~\cite{cormen:integrate-tr}.
%\htmladdnormallink{Dartmouth College Technical Report PCS-TR94-223}%
%{ftp://cs.dartmouth.edu:/pub/CS-techreports/TR94-223.ps.Z}.

Routing BMMC permutations in TPIE is done using the
\verb|AMI_BMMC_permute()| entry point, which takes an input stream,
and output stream, and a pointer to a bit permutation management
object.  In the following example, we route a permutation that simply
reverses the order of the source address bits to produce the
destination address.

First, we construct the matrices the permutation will use.
\index{bit_matrix@{\tt bit\_matrix}}
\begin{verbatim}
    bit_matrix A(n,n);
    bit_matrix c(n,1);

    {
        unsigned int ii,jj;

        for (ii = n; ii--; ) {
            c[ii][0] = 0;
            for (jj = n; jj--; ) {
                A[n-1-ii][jj] = (ii == jj);
            }
        }
    }
\end{verbatim}
Now we simply construct a permutation management object from the
matrices and perform the permutation.
\begin{verbatim}
    AMI_bit_perm_object bpo(A,c);
    
    ae = AMI_BMMC_permute(&amis0, &amis1, (AMI_bit_perm_object *)&bpo);
\end{verbatim}


\section{Sorting}

\subsection{Comparison Sorting} \label{sec:cmp-sorting}

\index{sorting!comparison|(}
Sorting is a common primitive operation in many algorithms.  It can be
performed in a variety of ways, such as by merging (See
Section~\ref{sec:merging}), distribution (See
Section~\ref{sec:distribution}), Sharesort~\cite{aggarwal:optimal},
which combines elements of both along with simple bit permutations
(See Section~\ref{sec:bit-permuting}).  Because the best choice of
sorting algorithm varies from one I/O system to the next, TPIE
provides a single function \verb|AMI_sort()|, which selects an
appropriate algorithm based on the underlying hardware
characteristics.

\subsubsection{AMI\_sort()}
\verb|AMI_sort()| has two polymorphs.  The first works on streams of
objects for which the operator \verb|<| is defined.  It is invoked as
follows:

\begin{verbatim}
AMI_STREAM<int> instream;
AMI_STREAM<int> outstream;

void f()
{
    AMI_sort(&instream, &outstream);
}
\end{verbatim}

The second polymorph of \verb|AMI_sort()| uses an explicit function to
determine the relative order of two objects in the input stream.  This
is useful in cases where we may want to sort a stream of objects in
several different ways.  For example, the following code sorts a
stream of complex numbers in two ways, by their real parts and by
their imaginary parts.

\begin{verbatim}
class complex {
public:
    complex(double real_part, imaginary_part);
    double re(void);
    double im(void);
    ...
};

int compare_re(const complex &c1, const complex &c2)
{
    return (c1.re() < c2.re()) ? -1 :
           ((c1.re() > c2.re()) ? 1 : 0);
}

int compare_im(const complex &c1, const complex &c2)
{
    return (c1.im() < c2.im()) ? -1 :
           ((c1.im() > c2.im()) ? 1 : 0);
}

AMI_STREAM<complex> instream;
AMI_STREAM<complex> outstream_re;
AMI_STREAM<complex> outstream_im;

void f()
{
    AMI_sort(&instream, &outstream_re, compare_re);
    AMI_sort(&instream, &outstream_im, compare_im);
}
\end{verbatim}

\subsubsection{Other Sorting Functions}
TPIE also offers three other sorting functions, with different approaches used
in implementation. Section~\ref{sec:ref-ami-merge} and Section~\ref{sec:ref-imp-ami-sort}
provide more details with regards to the difference in the implementation details.

The first function called \verb|AMI_partition_and_merge_stream()| provides input
and functionality identical to the first \verb|AMI_sort()| polymorph mentioned above,
that is it assumes that the input items have a well-defined \verb| <| operator.
So we can sort by
\begin{verbatim}
AMI_STREAM<int> instream;
AMI_STREAM<int> outstream;

void f()
{
    AMI_partition_and_merge_stream(&instream, &outstream);
}
\end{verbatim}

The second function,  \verb|AMI_partition_and_merge_stream_cmp()|
on the other hand is identical in functionality to the second \verb|AMI_sort()| 
polymorph mentioned above, and as in that case, uses an explicit function to
determine the relative order of two objects in the input stream. Thus with 
the class \verb|complex|, the functions \verb|compare_re| and \verb|compare_im|,
and streams \verb|instream|, \verb|outstream_re| and \verb|outstream_im| as 
defined above, we'd do a 
\begin{verbatim}}

void f()
{
    AMI_partition_and_merge_stream_cmp(&instream, &outstream_re, compare_re);
    AMI_partition_and_merge_stream_cmp(&instream, &outstream_im, compare_im);
}
\end{verbatim}
to  perform the same operations.

The last function is based on the assumption that the items being sorted have 
a KEY field with  well-defined (possibly via overloading) \verb| <|, \verb| >|,
operators etc. and the items are to be sorted in the order imposed on them
by the KEY fields. So for instance consider the class \verb|rectangle|
below meant for axis parallel rectangles:
\begin{verbatim}}
class rectangle{
double northEast_x;
double northEast_y;
double southWest_x;
double southWest_y;
}
\end{verbatim}
and suppose that we want to sort a stream of rectangles in descending order according
to their \verb|southWest_y| co-ordinate. Then assuming that the size of each 
\verb|double| is 8 bytes, we simply sort a stream \verb|instream| of rectangles as
follows:
 
\begin{verbatim}
AMI_STREAM<rectangle> instream;
AMI_STREAM<rectangle> outstream;
double dummyKey;
void f()
{
    AMI_partition_and_merge_Key(&instream, &outstream, 24, dummyKey );
}
\end{verbatim}

The third argument above is basically the offset into the item of the field
\verb|southWest_y| which is the KEY according to which the stream is to be sorted.
The third argument is the offset within the item of the key according to which the items are to
be sorted. And the fourth argument is a dummy argument having the same type as the
KEY.



\index{sorting!comparison|)}

\subsection{Key Bucket Sorting}
\label{sec:kb-sorting}

\index{sorting!key bucket|(}
\index{sorting!key bucket|)}

\section{Matrix Operations}
\label{sec:matrix}

\index{matrices|(}

In addition to streams, which are linearly ordered collections of
objects, the AMI provides a mechanism for storing large matrices in
external memory.  Matrices are a subclass of streams, and thus can be
used with any of the stream operations discussed above.  When a matrix
is treated as a stream its elements appear in row major order.  In
addition to stream operations, matrices support three simple
arithmetic operations, addition, subtraction, and multiplication.

It is assumed that the class \verb|T| of the elements in a matrix
forms a quasiring with the operators \verb|+| and \verb|*|.
Furthermore, the object \verb|T((int)0)| is assumed to be an identity
for \verb|+|.  At the moment, it is not assumed that the operator
\verb|-| in an inverse of \verb|+|, and therefore no reduced
complexity matrix multiplication algorithms analogous to Strassen's
algorithm are used.

There are two different classes of matrices that TPIE provides, dense,
and sparse.

\subsection{Dense Matrix Operations}
\label{sec:dense-mat}

\index{matrices!dense|(}

Dense matrices are implemented by the templated class
\verb|AMI_matrix|,\index{AMI_matrix@{\tt AMI\_matrix}}
 which is a subclass of
\verb|AMI_STREAM|.\index{AMI_STREAM@{\tt AMI\_STREAM}}

Dense matrices can be filled using \verb|AMI_scan()|, though typically
they are filled using the function \verb|AMI_matrix_fill()|, which
uses a scan management object that is given the row and column of each
element of the matrix and asked to fill them in.  In the following
example, we create a 1000 by 1000 upper triangular matrix of ones and
zeroes:

\begin{verbatim}
template<class T>
class fill_upper_tri : public AMI_matrix_filler<T> {
    AMI_err initialize(unsigned int rows, unsigned int cols)
    {
        return AMI_ERROR_NO_ERROR;
    };
    T element(unsigned int row, unsigned int col)
    {
        return (row <= col) ? T(1) : T(0);
    };
};

AMI_matrix m(1000, 1000);

void f()
{
    fill_upper_tri<double> fut;

    AMI_matrix_fill(&em, (AMI_matrix_filler<T> *)&fut);
}
\end{verbatim}

Arithmetic on dense matrices is performed in a straightforward way
using the functions \verb|AMI_matrix_add()|,
\verb|AMI_matrix_subtract()|, and \verb|AMI_matrix_multiply()|, as is
the following example:

\begin{verbatim}
AMI_matrix m0(1000, 500), m1(500, 2000), m2(1000, 2000);
AMI_matrix m3(1000, 500), m4(1000, 500);

void f()
{
    // Add m3 to m4 and put the result in m0.
    AMI_matrix_add(em3, em4, em0);
   
    // Multiply m0 by em1 to get m2.
    AMI_matrix_mult(em0, em1, em2);

    // Subtract m4 from m3 and put the result in m0.
    AMI_matrix_subtract(em3, em4, em0);        
}
\end{verbatim}

\index{matrices!dense|)}

\subsection{Sparse Matrices}
\label{sec:sparse-mat}

\index{matrices!sparse|(}
\index{matrices!sparse|)}

\subsection{Elementwise Arithmetic}
\label{sec:elementwise}

\index{arithmetic!elementwise|see{elementwise arithmetic}}
\index{elementwise arithmetic|(} 
The functions \verb|AMI_matrix_add()|
and \verb|AMI_matrix_subtract()| defined in
Section~\ref{sec:dense-mat} perform elementwise arithmetic on
matrices.  At times, we might also wish to perform elementwise
multiplication or division, or perform a scalar arithmetic operation
on all elements of a matrix.  TPIE provides mechanisms for doing this
not only on matrices, but on arbitrary streams, so long as they are of
objects for which the appropriate arithmetic operators (i.e. {\tt +},
{\tt -}, {\tt *}, {\tt /}) are defined.

Elementwise arithmetic is done with scan management objects
\index{operation management objects!scan} of the classes
\verb|AMI_scan_add|, \verb|AMI_scan_sub|, \verb|AMI_scan_mult| and
\verb|AMI_scan_div|, which are defined in the file
\verb|AMI_stream_arith.h|.  Here is an example that performs
elementwise division on the elements of two streams.

\begin{verbatim}
#include <ami_stream_arith.h>

void foo()
{
    AMI_STREAM<int> amis0;
    AMI_STREAM<int> amis1;
    AMI_STREAM<int> amis2;

    AMI_scan_div<int> sd;

    // Divide each element of amis0 by the corresponding element of
    // amis1 and put the result in amis2.
    AMI_scan(&amis0, &amis1, &sd, &amis2);
}
\end{verbatim}
\index{elementwise arithmetic|)}

\index{matrices|)}

\section{Configuration, Compiler Flags, and Environment Variables.}
\label{sec:complete}
\index{Configurarions, compiler flags and environment variables}

The fragments of code presented in this tutorial are valuable for
instructive purposes; however, they are incomplete.  In order to
successfully compile, link, and run TPIE applications, some additional
code and appropriate compilation and run-time environments are needed.
These are discussed below.  The recommended way for a novice TPIE programmer
to begin is to first look directly at the sample program
of Chapter~3 or at the source code provided in
the {\tt test} directory, then set up the configuration
parameters (in file {\tt
  tpie-\version/test/app\_config.h}),  compiler flags, 
and various environment variables as described below, and then
proceed to compiling, linking and running the programs.



\subsection{Configuring TPIE for the Application}
\label{sec:macros}
\index{macros|(}

The exact behavior or TPIE at run time is controlled be several macros
that can be defined before including any TPIE headers.  In the test
application code\index{test applications} distributed with TPIE, these
are set in the header file {\tt
  tpie-\version/test/app\_config.h}.\index{app_config@{\tt app\_config.h}}
The macros are as follows:
\begin{description}
\item[{\verb|TPL\_LOGGING|}] \index{TPL_LOGGING@{\tt TPL\_LOGGING}} Set
  to a non-zero value to enable logging of TPIE's internal behavior.
  By default, information is logged to the log file\index{log file}
  \verb|/tmp/TPLOG_XXXXXX| where \verb|XXXXXX| is a unique system
  dependent identifier.  Typically it encodes the process ID of the
  TPIE process that produced it in some way.  See
  Section~\ref{sec:logging} of the reference manual for information on
  exactly what TPIE writes to the log file.
\item[{\verb|DEBUG\_ASSERTIONS|}] 
  \index{DEBUG_ASSERTIONS@{\tt DEBUG\_ASSERTIONS}}
  \index{debugging!TPIE}
  Set to a non-zero value to enable
  TPIE assertions.  These assertions check for inconsistent or
  erroneous conditions within TPIE itself.  They are primarily
  intended to aid in the debugging of TPIE.  Some overhead is added to
  programs compiled with this macro set.
\item[{\verb|DEBUG\_CERR|}] \index{DEBUG_CERR@{\tt DEBUG\_CERR}}
  \index{debugging!TPIE} Setting this macro to a non-zero value tells
  TPIE to write all internal assertion messages to the C++ standard
  error stream \verb|cerr| in addition to the TPIE log file.
\item[{\verb|DEBUG\_STR|}] 
  \index{DEBUG_STR@{\tt DEBUG\_STR}} \index{debugging!TPIE}
  Setting this macro to a non-zero value enables certain debugging
  messages that report on the internal behavior of TPIE but do not
  necessarily indicate error conditions.  In some cases this can
  increase the size of the log dramatically.
\item[{\verb|AMI\_VIRTUAL\_BASE|}] 
  \index{AMI_VIRTUAL_BASE@{\tt AMI\_VIRTUAL\_BASE}}
  \index{virtual base class!AMI} Setting this macro to a non-zero
  value makes the base class declares a large number of virtual
  functions for the class \verb|AMI_base_stream|, which is the base
  class of all implementations of AMI streams.  This is useful for
  debugging new AMI stream implementations, but many compilers cannot
  properly inline virtual functions, so it slows the system down
  significantly.  Normally, TPIE applications programmers would never
  set this flag.
\item[{\verb|BTE\_VIRTUAL\_BASE|}] 
  \index{BTE_VIRTUAL_BASE@{\tt BTE\_VIRTUAL\_BASE}}
  \index{virtual base class!BTE}  Similar to
  \verb|AMI_VIRTUAL_BASE|, but for the BTE layer.
  Normally, TPIE applications programmers would never
  set this flag.

\item[{\verb|AMI\_IMP\_*|}]
  \index{AMI_IMP_*@{\tt AMI\_IMP\_*}}
  \index{access method interface!implementation}
  \index{implementation!AMI}
  A macro of this form is used to tell TPIE which of the available
  access method interface implementations to use.  Version \version of
  TPIE is distributed only one AMI implementation, which stores the
  contents of a given stream on a single disk.  This implementation is
  selected by setting 
  {\tt AMI\_IMP\_SINGLE}.
  \index{implementation!AMI!single disk}

\item[{\verb|BTE\_IMP\_*|}]
  \index{BTE_IMP_*@{\tt BTE\_IMP\_*}}
  \index{block transfer engine!implementation}
  \index{implementation!BTE}
  These are important macros for the TPIE user.
  The macros of this form are used to tell TPIE which of the available
  block transfer engine implementations to use.  Version \version ~of
  TPIE is distributed with three implementations.  
  %
  An implementation
  based on the UNIX {\tt stdio} library
  \index{stdio (UNIX library)@{\tt stdio} (UNIX library)} is selected
  by setting {\tt BTE\_IMP\_STDIO}.  
  \index{implementation!BTE!{\tt stdio} library}
  %
  An implementation based on blocked UNIX {\tt read/write} calls 
  \index{read write} is selected by setting 
  {\tt BTE\_IMP\_UFS}.
  \index{implementation!BTE!read!write}
  %
  An implementation based on memory
  mapped I/O\index{memory mapped I/O} is selected by setting 
  {\tt BTE\_IMP\_MMB}.
  \index{implementation!BTE!memory mapped I/O}
  

\end{description}

See also the TPIE implementation section for details on each BTE.

The following flags are only used by the memory mapped I/O BTE (BTE\_MMB).

\begin{description}

\item[{\verb|BTE\_MMB\_READ\_AHEAD|}]
  \index{BTE_MMB_READ_AHEAD@{\tt BTE\_MMB\_READ\_AHEAD}}
  \index{read ahead}
  When the memory mapped I/O implementation of the BTE layer is
  selected, setting this flag tells the BTE to optimize for sequential
  read speed by reading blocks into main memory before the
  data they contain is actually needed. This version provides two methods
  of read-ahead. The default behaviour uses {\tt mmap}. If the USE\_LIBAIO
  flag is also set, read ahead is done using the asynchronous I/O library.
  
\item[{\verb|USE\_LIBAIO|}] \index{USE\_LIBAIO} If the
  \verb|BTE_MMB_READ_AHEAD| flag is also set, setting this flag performs read
  ahead using the asynchronous I/O library.  This feature requires the
  asynchronous I/O library {\tt libaio}.\index{libaio library@{\tt libaio}
    library.}. If {\verb|USE_LIBAIO|} flag is not set, read ahead is done
  by default using {\tt mmap} and double buffering.

\end{description}

There is one more flag used only by \verb|BTE_mmb| and \verb|BTE_ufs|, the
logical block-size factor:
\begin{description}
  
\item[{\verb|BTE\_\*\_LOGICAL\_BLOCKSIZE\_FACTOR|}] This flag sets the logical
  blocksize used by BTEs, which can be a multiple of the physical
  blocksize. Value 1 indicates that the logical blocksize is same as
  physical blocksize. Typical values are 16,32,64.
\end{description}

\index{macros|)}

\index{header files|(}
One the appropriate macros have been set, TPIE's templated classes and
functions are included by including the header file {\tt ami.h} from
the {\tt include} directory.  Normally, this directory is pointed to
by a {\tt -I} argument to the compiler.  This and other compiler flags
are discussed in more detail in Section~\ref{sec:comp-flags} and are
illustrated by their use un the {\tt Makefile} in the
{\tt test} directory.
\index{header files|)}


A sample \verb|app_config.h| file is included below.

\begin{verbatim}
// Use the single BTE stream version of AMI streams.
#define AMI_IMP_SINGLE

// Pick a version of BTE streams.
#define BTE_IMP_MMB
//#define BTE_IMP_STDIO
//#define BTE_IMP_UFS


#ifdef BTE_IMP_MMB

#ifndef BTE_MMB_LOGICAL_BLOCKSIZE_FACTOR
#define BTE_MMB_LOGICAL_BLOCKSIZE_FACTOR 32
#endif

#define BTE_MMB_READ_AHEAD 1
//#define USE_LIBAIO

#endif 


#ifdef BTE_IMP_UFS

#ifndef BTE_UFS_LOGICAL_BLOCKSIZE_FACTOR
#define BTE_UFS_LOGICAL_BLOCKSIZE_FACTOR 16
#endif

#define BTE_UFS_READ_AHEAD 0
#define DOUBLE_BUFFER 0
#define USE_LIBAIO 0

#define BTE_IMPLICIT_FS_READAHEAD 0
#endif


// Use logs if requested.
#if TP_LOG_APPS
#define TPL_LOGGING 1
#endif
// Enable assertions if requested.
#if TP_ASSERT_APPS
#define DEBUG_ASSERTIONS 1
#define DEBUG_CERR 1
#define DEBUG_STR 1
#endif

// Don't use virtual interface.
#ifndef AMI_VIRTUAL_BASE
#define AMI_VIRTUAL_BASE 0
#endif
#ifndef BTE_VIRTUAL_BASE 
#define BTE_VIRTUAL_BASE 0
#endif

\end{verbatim}

 
%\subsection{Template Instantiation}

%{\bf Important Note:} Much of the information in this section is
%likely to change as the template instantiation mechanism of the 
%{\tt g++}\index{g++@{\tt g++}} compiler improves.  If you are
%interested in the nitty gritty details of template instantiation,
%consult~\cite{ellis:arm} or one of the frequent discussions on the
%topic in the newsgroup {\tt comp.lang.c++}
%\index{comp.lang.c++@{\tt comp.lang.c++}}.

%\index{templates!instantiation|(}
%\noindent Most of the classes and functions TPIE defines are
%templated.  Furthermore, many user written operation management
%object\index{operation management objects!user supplied} classes are
%likely to be templated; many of those supplied with the test and
%sample applications are.

%Unfortunately, many C++\index{C++} compilers do not properly implement
%templated function and/or classes.  In particular, the GNU C++
%compiler, {\tt g++}\index{g++@{\tt g++}}, version \gxxversion, which
%was used in the development of TPIE has some deficiencies when it
%comes to template instantiation.  It also has a well defined mechanism
%for working around these deficiencies, which TPIE takes significant
%advantage of.  This mechanism prevents the compiler from implicitly
%instantiating any template.  Thus, all templates used by a program
%must be explicitly instantiated at compile time or they will not be
%available at link time and linking will fail.

%In order to tell {\tt g++}\index{g++@{\tt g++}} not to implicitly
%instantiate any templates, the {\tt -fno-implicit-templates} flag is
%used.  Additionally, the macro {\tt NO\_IMPLICIT\_TEMPLATES} should be
%defined on the command line, using {\tt -D}.  This macro informs TPIE
%that it should not rely on the presence of implicit template
%instantiation.  In response to the fact that this macro is set, TPIE
%defines a series of new macros with names of the form {\tt
%  TEMPLATE\_INSTANTIATE\_*}.  
%\index{TEMPLATE_INSTANTIATE_*@{\tt TEMPLATE\_INSTANTIATE\_*}|(}
%Each of these macros can be used to
%actually instantiate some set of functions and/or classes that TPIE
%needs to provide a given operation.  These macros should be used at
%the end of your source file in order to perform the proper
%instantiations.

%The {\tt TEMPLATE\_INSTANTIATE\_*} macros likely to be needed by TPIE
%programmers are as follows:
%\begin{description}
%\item[{\tt TEMPLATE\_INSTANTIATE\_STREAMS(T)}] Instantiate AMI and
%  BTE level streams of objects of type {\tt T}.  If your
%  application uses streams of several types, this macro must be called
%  once for each of them.
%\item[{\tt TEMPLATE\_INSTANTIATE\_ISTREAM(T)}]
%\item[{\tt TEMPLATE\_INSTANTIATE\_OSTREAM(T)}] Instantiate ASCII
%  input and output scan management objects for the type {\tt T}.
%  See Section~\ref{sec:ascii-io} for details on these objects.
%  \index{scanning!ASCII I/O}
%\item[{\tt TEMPLATE\_INSTANTIATE\_AMI\_MERGE}] Instantiate merging entry
%  points for streams of objects of type {\tt T}.  Merging is described
%  in Section~\ref{sec:merging}.
%\item[{\tt TEMPLATE\_INSTANTIATE\_SORT\_OP(T)}]
%\item[{\tt TEMPLATE\_INSTANTIATE\_SORT\_CMP(T)}]
%\item[{\tt TEMPLATE\_INSTANTIATE\_SORT\_OBJ(T)}] Instantiate
%  respectively operator, comparison function, and comparison object
%  based sorting of objects of type {\tt T}.  See
%  Section~\ref{sec:cmp-sorting} for details on these types of sorting.
%\item[{\tt TEMPLATE\_INSTANTIATE\_KB\_SORT(T)}] 
%\item[{\tt TEMPLATE\_INSTANTIATE\_KB\_SORT\_KEY(T,K)}] Instantiate key
%  bucket distribution sorting of objects of type {\tt T}.  The latter
%  form uses key {\tt K} for sorting.  Section~\ref{sec:kb-sorting}
%  describes key bucket sorting.
%\item[{\tt TEMPLATE\_INSTANTIATE\_STREAM\_ADD(T)}]
%\item[{\tt TEMPLATE\_INSTANTIATE\_STREAM\_SUB(T)}]
%\item[{\tt TEMPLATE\_INSTANTIATE\_STREAM\_MULT(T)}]
%\item[{\tt TEMPLATE\_INSTANTIATE\_STREAM\_DIV(T)}]
%  Instantiate elementwise arithmetic operations on streams of objects
%  of type {\tt T} as described in Section~\ref{sec:elementwise}.
%\item[{\tt TEMPLATE\_INSTANTIATE\_AMI\_MATRIX}]
%  Instantiate dense matrices of objects of type {\tt T} and the
%  standard operations on them.  Dense
%  matrices are described in
%  Section~\ref{sec:dense-mat}.\index{matrices!dense}
%\item[{\tt TEMPLATE\_INSTANTIATE\_AMI\_SPARSE\_MATRIX}]
%  Instantiate sparse matrices of objects of type {\tt T} and the
%  standard operations on them.  Sparse
%  matrices are described in
%  Section~\ref{sec:dense-mat}.\index{matrices!sparse}
%\end{description}
%\index{TEMPLATE_INSTANTIATE_*@{\tt TEMPLATE\_INSTANTIATE\_*}|)}

%In addition to instantiating functions and classes using the macros
%described above, it is often necessary to explicitly instantiate
%particular instances of AMI entry points for user supplied operation
%management objects.  For example, suppose we declare a scan management
%object class such as
%\begin{verbatim}
%class my_scan_class : AMI_scan_object {
%public:
%    AMI_err initialize(void);
%    AMI_err operate(const int &in1, const int &in2, AMI_SCAN_FLAG *sfin,
%                    float *out, AMI_SCAN_FLAG *sfout); 
%}
%\end{verbatim}
%Then, in order to explicitly instantiate \verb|AMI_scan()| to use
%objects of this type, we would use the following code:
%\begin{verbatim}
%template AMI_err AMI_scan(AMI_STREAM<int> *, AMI_STREAM<int> *, 
%                          my_scan_class *, AMI_STREAM<float> *); 
%\end{verbatim}
%This instantiates an instance of \verb|AMI_scan()| that takes two input
%streams of \verb|int|s, operates on them with an object of type
%\verb|my_scan_class|, and produces an output stream of \verb|float|s.  
%Note the correspondence between the types of input and output streams
%and the types of the operands to the \verb|operate()| member function
%of the class \verb|my_scan_class|.
%\index{templates!instantiation|)}

\subsection{Environment Variables and Compiler Flags}
\label{sec:comp-flags}
\index{compiler flags|(}
\index{compiler flags|)}


In version \version of TPIE there is only one environment variable that the
user needs to set before running TPIE applications. The variable is called
\verb|AMI_SINGLE_DEVICE| and defines where are placed the TPIE streams
created during the execution of the program . The defaut location is
\verb|/var/tmp|. If the user wants a different location, he must set the
\verb|AMI_SINGLE_DEVICE| accordingly, for example (in C-shell):

\begin{verbatim}
setenv AMI_SINGLE_DEVICE /usr/project/tmp/
\end{verbatim}

Before compiling an application the user should set up the desired
configuration in  file {\tt tpie-\version/test/app\_config.h}) (see
previous subsection).

TPIE test programs come with a Makefile and can be compiled in the usual
way by invoking \verb|make| with the program name. For instance, in order to
compile the test program \verb|test_ami_sort.cpp| the command is:

\begin{verbatim}
make test_ami_sort
\end{verbatim}


If the user writes new TPIE applications, in order to compile the options
are:
\begin{itemize}
\item either create an application directory, modify the Makefile
  accordingly and call \verb|make| with the new application name as
  follows:
\begin{verbatim}
make sample_appl
\end{verbatim}

  
\item or invoke directly the compiler as follows:

\begin{verbatim}
g++  sample_appl.cpp -I ../include/ -L ../lib/ -ltpie -o sample_appl
\end{verbatim}
\end{itemize}


%\subsection{}
%\label{sec:env-variables}
%\index{environment variables|(}
%\index{environment variables!AMI_SINGLE_DEVICE_ENV@{\tt AMI\_SINGLE\_DEVICE\_ENV}}
%\index{environment variables|)}







\chapter{Additional Examples} \label{ch:examples}
\index{examples}

This chapter contains some additional annotated examples of 
TPIE application code.

\section{Convex Hull}
\label{sec:convex-hull}
\index{convex hull|(}

The convex hull of a set of points in the plane is the smallest convex
polygon which encloses all of the points.  Graham's scan is a simple
algorithm for computing convex hulls.  It should be discussed in any
introductory book on computational geometry, such as~\cite{preparata:cg}.  Although Graham's scan was not originally designed for
external memory, it can be implemented optimally in this setting.
What is interesting about this implementation is that external memory
stacks are used within the implementation of a scan management object.

First, we need a data type for storing points.  We use the following
simple class, which is templated to handle any numeric type.

\begin{verbatim}
template<class T>
class point {
public:
    T x;
    T y;
    point() {};
    point(const T &rx, const T &ry) : x(rx), y(ry) {};
    ~point() {};

    inline int operator==(const point<T> &rhs) const {
        return (x == rhs.x) && (y == rhs.y);
    }
    inline int operator!=(const point<T> &rhs) const {
        return (x != rhs.x) || (y != rhs.y);
    }

    // Comparison is done by x.
    int operator<(const point<T> &rhs) const {
        return (x < rhs.x);
    }

    int operator>(const point<T> &rhs) const {
        return (x > rhs.x);
    }
    
    friend ostream& operator<<(ostream& s, const point<T> &p);
    friend istream& operator>>(istream& s, point<T> &p);
};
\end{verbatim}

Once the points are s by their $x$ values, we simply scan them to
produce the upper and lower hulls, each of which are stored as a stack
pointed to by the scan management object.  We then concatenate the
stacks to produce the final hull.  The code for computing the convex
hull of a set of points is thus

\begin{verbatim}
template<class T>
AMI_err convex_hull(AMI_STREAM< point<T> > *instream,
                    AMI_STREAM< point<T> > *outstream)
{
    AMI_err ae;

    point<T> *pt;

    AMI_stack< point<T> > uh((unsigned int)0, instream->stream_len());
    AMI_stack< point<T> > lh((unsigned int)0, instream->stream_len());

    AMI_STREAM< point<T> > in_sort;
        
    // Sort the points by x.

    ae = AMI_sort(instream, &in_sort);
    
    // Compute the upper hull and lower hull in a single scan.

    scan_ul_hull<T> sulh;

    sulh.uh_stack = &uh;
    sulh.lh_stack = &lh;
    
    ae = AMI_scan(&in_sort, &sulh);

    // Copy the upper hull to the output.

    uh.seek(0);
    
    while (1) {
        ae = uh.read_item(&pt);
        if (ae == AMI_ERROR_END_OF_STREAM) {
            break;
        } else if (ae != AMI_ERROR_NO_ERROR) {
            return ae;
        }

        ae = outstream->write_item(*pt);
        if (ae != AMI_ERROR_NO_ERROR) {
            return ae;
        }
    }
    
    // Reverse the lower hull, concatenating it onto the upper hull.

    while (lh.pop(&pt) == AMI_ERROR_NO_ERROR) {
        ae = outstream->write_item(*pt);
        if (ae != AMI_ERROR_NO_ERROR) {
            return ae;
        }
    }

    return AMI_ERROR_NO_ERROR;
}
\end{verbatim}

The only thing that remains is to define a scan management object that
is capable of producing the upper and lower hulls by scanning the
points.  According to the Graham's scan algorithm, we produce the
upper hull by moving forward in the $x$ direction, adding each
point we encounter to the upper hull, until we find one that induces a
concave turn on the surface of the hull.  We then move backwards
through the list of points that have been added to the hull,
eliminating points until a convex path is reestablished.  This process
is made efficient by storing the points on the hull so far in a stack.
The code for the scan management object, which relies on the function
\verb|ccw()| to actually determine whether a corner is
convex or not, is as follows:

\begin{verbatim}
template<class T>
class scan_ul_hull : AMI_scan_object {
public:
    AMI_stack< point <T> > *uh_stack, *lh_stack;

    scan_ul_hull(void);
    virtual ~scan_ul_hull(void);
    AMI_err initialize(void);
    AMI_err operate(const point<T> &in, AMI_SCAN_FLAG *sfin);
};

template<class T>
scan_ul_hull<T>::scan_ul_hull(void) : uh_stack(NULL), lh_stack(NULL)
{
}

template<class T>
scan_ul_hull<T>::~scan_ul_hull(void)
{
}

template<class T>
AMI_err scan_ul_hull<T>::initialize(void)
{
    return AMI_ERROR_NO_ERROR;
}


template<class T>
AMI_err scan_ul_hull<T>::operate(const point<T> &in,
                                 AMI_SCAN_FLAG *sfin)
{
    AMI_err ae;

    // If there is no more input we are done.
    if (!*sfin) {
        return AMI_SCAN_DONE;
    }

    if (!uh_stack->stream_len()) {

        // If there is nothing on the stacks then put the first point
        // on them.
        ae = uh_stack->push(in);
        if (ae != AMI_ERROR_NO_ERROR) {
            return ae;
        }

        ae = lh_stack->push(in);
        if (ae != AMI_ERROR_NO_ERROR) {
            return ae;
        }

    } else {

        // Add to the upper hull.

        {
            // Pop the last two points off.

            point<T> *p1, *p2;

            tp_assert(uh_stack->stream_len() >= 1, "Stack is empty.");
            
            uh_stack->pop(&p2);

            // If the point just popped is equal to the input, then we
            // are done.  There is no need to have both on the stack.
            
            if (*p2 == in) {
                uh_stack->push(*p2);
                return AMI_SCAN_CONTINUE;
            }
            
            if (uh_stack->stream_len() >= 1) {
                uh_stack->pop(&p1);
            } else {
                p1 = p2;
            }
            
            // While the turn is counter clockwise and the stack is
            // not empty pop another point.
            
            while (1) {                
                if (ccw(*p1,*p2,in) >= 0) {
                    // It does not turn the right way.  The points may
                    // be colinear.
                    if (uh_stack->stream_len() >= 1) {
                        // Move backwards to check another point.
                        p2 = p1;
                        uh_stack->pop(&p1);
                    } else {
                        // Nothing left to pop, so we can't move
                        // backwards.  We're done.
                        uh_stack->push(*p1);
                        if (in != *p1) {
                            uh_stack->push(in);
                        }
                        break;
                    }
                } else {
                    // It turns the right way.  We're done.
                    uh_stack->push(*p1);
                    uh_stack->push(*p2);
                    uh_stack->push(in);
                    break;
                }
            }
        }

        // Add to the lower hull.

        {
            // Pop the last two points off.

            point<T> *p1, *p2;

            tp_assert(lh_stack->stream_len() >= 1, "Stack is empty.");
            
            lh_stack->pop(&p2);

            // If the point just popped is equal to the input, then we
            // are done.  There is no need to have both on the stack.
            
            if (*p2 == in) {
                lh_stack->push(*p2);
                return AMI_SCAN_CONTINUE;
            }
            
            if (lh_stack->stream_len() >= 1) {
                lh_stack->pop(&p1);
            } else {
                p1 = p2;
            }
            
            // While the turn is clockwise and the stack is
            // not empty pop another point.
            
            while (1) {                
                if (ccw(*p1,*p2,in) <= 0) {
                    // It does not turn the right way.  The points may
                    // be colinear.
                    if (lh_stack->stream_len() >= 1) {
                        // Move backwards to check another point.
                        p2 = p1;
                        lh_stack->pop(&p1);
                    } else {
                        // Nothing left to pop, so we can't move
                        // backwards.  We're done.
                        lh_stack->push(*p1);
                        if (in != *p1) {
                            lh_stack->push(in);
                        }
                        break;
                    }
                } else {
                    // It turns the right way.  We're done.
                    lh_stack->push(*p1);
                    lh_stack->push(*p2);
                    lh_stack->push(in);
                    break;
                }
            }
        }       
    }

    return AMI_SCAN_CONTINUE;    
}
\end{verbatim}

The function \verb|ccw()| computes twice the signed area of a triangle in
the plane by evaluating a 3 by 3 determinant.  The result is positive
if and only if the the three points in order form a counterclockwise
cycle.

\begin{verbatim}
template<class T>
T ccw(const point<T> &p1, const point<T> &p2, const point<T> &p3)
{
    T sa;
    
    sa = ((p1.x * p2.y - p2.x * p1.y) -
          (p1.x * p3.y - p3.x * p1.y) +
          (p2.x * p3.y - p3.x * p2.y));

    return sa;
}
\end{verbatim}
\index{convex hull|)}

\section{List-Ranking}
\label{sec:list-ranking}
\index{list ranking|(}

List ranking is a fundamental problem in graph theory.  The problem is
as follows: We are given the directed edges of a linked list in some
arbitrary order.  Each edge is an ordered pair of node ids.  The first
is the source of the edge and the second is the destination of the
edge.  Our goal is to assign a weight to each edge corresponding to
the number of edges that would have to be traversed to get from the
head of the list to that edge.

The code given below solves the list ranking problem using a simple
randomized algorithm due to Chiang, Goodrich, Grove, Tamassia,
Vengroff, and Vitter, which appears in the proceedings of SODA '95.
As was the case in the code examples in the tutorial in
Chapter~\ref{ch:tutorial}, \verb|#include| statements
for header files and definitions of some classes and functions as well
as some error and consistency checking code are left out so that the
reader can concentrate on the more important details of how TPIE is
used.  A complete ready to compile version of this code is included in
the TPIE source distribution.

First, we need a class to represent edges.  Because the algorithm will
set a flag for each edge and then assign weights to the edges, we
include fields for these values.

\begin{verbatim}
class edge {
public:
    unsigned long int from;        // Node it is from
    unsigned long int to;          // Node it is to
    unsigned long int weight;      // Position when ranked.
    bool flag;            // A flag used to randomly select some edges.

    friend ostream& operator<<(ostream& s, const edge &e);
};    
\end{verbatim}

As the algorithm runs, it will sort the edges.  At times this will be
done by their sources and at times by their destinations.  The
following simple functions are used to compare these values:

\begin{verbatim}
int edgefromcmp(CONST edge &s, CONST edge &t)
{
    return (s.from < t.from) ? -1 : ((s.from > t.from) ? 1 : 0);
}
  
int edgetocmp(CONST edge &s, CONST edge &t)
{
    return (s.to < t.to) ? -1 : ((s.to > t.to) ? 1 : 0);
}
\end{verbatim}

The first step of the algorithm is to assign a randomly chosen flag,
whose value is 0 or 1 with equal probability, to each edge.  This is
done using \verb|AMI_scan()| with a scan management object of the
class \verb|random_flag_scan|, which is defined as follows:

\begin{verbatim}
class random_flag_scan : AMI_scan_object {
public:
    AMI_err initialize(void);  
    AMI_err operate(const edge &in, AMI_SCAN_FLAG *sfin,
                    edge *out, AMI_SCAN_FLAG *sfout);
};

AMI_err random_flag_scan::initialize(void) {
    return AMI_ERROR_NO_ERROR;
}

AMI_err random_flag_scan::operate(const edge &in, AMI_SCAN_FLAG *sfin,
                                  edge *out, AMI_SCAN_FLAG *sfout)
{ 
    if (!(sfout[0] = *sfin)) {
        return AMI_SCAN_DONE;
    }
    *out = in;
    out->flag = (random() & 1);
    
    return AMI_SCAN_CONTINUE;
}
\end{verbatim}

The next step of the algorithm is to separate the edges into an active
list and a cancel list.  In order to do this, we sort one copy of the
edges by their sources (using \verb|edgefromcmp|) and sort another copy by
their destinations (using \verb|edgetocmp|).  We then call
\verb|AMI_scan()| to scan the two lists and produce an active list and
a cancel list.  A scan management object of class
\verb|separate_active_from_cancel|, which is defined below, is used.

\begin{verbatim}
////////////////////////////////////////////////////////////////////////
// separate_active_from_cancel
//
// A class of scan object that takes two edges, one to a node and one 
// from it, and writes an active edge and possibly a canceled edge.
//
// Let e1 = (x,y,w1,f1) be the first edge and e2 = (y,z,w2,f2) the second.
// If e1's flag (f1) is set and e2's (f2) is not, then we write 
// (x,z,w1+w2,?) to the active list and e2 to the cancel list.  The
// effect of this is to bridge over the node y with the new active edge.
// f2, which was the second half of the bridge, is saved in the cancellation
// list so that it can be ranked later after the active list is recursively 
// ranked.
//
// Since all the flags should have been set randomly before this function
// is called, the expected size of the active list is 3/4 the size of the
// original list.
////////////////////////////////////////////////////////////////////////
class separate_active_from_cancel : AMI_scan_object {
public:
    AMI_err initialize(void);
    AMI_err operate(CONST edge &e1, CONST edge &e2, AMI_SCAN_FLAG *sfin,
                    edge *active, edge *cancel, AMI_SCAN_FLAG *sfout);
};

AMI_err separate_active_from_cancel::initialize(void)
{
    return AMI_ERROR_NO_ERROR;
}

// e1 is from the list of edges sorted by where they are from.
// e2 is from the list of edges sorted by where they are to.
AMI_err separate_active_from_cancel::operate(CONST edge &e1,
                                             CONST edge &e2, 
                                             AMI_SCAN_FLAG *sfin,
                                             edge *active, edge *cancel, 
                                             AMI_SCAN_FLAG *sfout)
{
    // If we have both inputs.
    if (sfin[0] && sfin[1]) {
        // If they have a node in common we may be in a bridging situation.
        if (e2.to == e1.from) {
            // We will write to the active list no matter what.
            sfout[0] = 1;
            *active = e2;
            if (sfout[1] = (e2.flag && !e1.flag)) {
                // Bridge.  Put e1 on the cancel list and add its
                // weight to the active output.
                active->to = e1.to;
                active->weight += e1.weight;
                *cancel = e1;
                sfout[1] = 1;
            } else {
                // No bridge.
                sfout[1] = 0;
            }
        } else {
            // They don't have a node in common, so one of them needs
            // to catch up with the other.  What happened is that
            // either e2 is the very last edge in the list or e1 is
            // the very first or we just missed a bridge because of
            // flags.
            sfout[1] = 0;                
            if (e2.to > e1.from) {
                // e1 is behind, so just skip it.
                sfin[1] = 0;
                sfout[0] = 0;
            } else {
                // e2 is behind, so put it on the active list.
                sfin[0] = 0;
                sfout[0] = 1;
                *active = e2;
            }
        }
        return AMI_SCAN_CONTINUE;
    } else {
        // If we only have one input, either just leave it active.
        if (sfin[0]) {
            *active = e1;
            sfout[0] = 1;
            sfout[1] = 0;
            return AMI_SCAN_CONTINUE;
        } else if (sfin[1]) {
            *active = e2;
            sfout[0] = 1;
            sfout[1] = 0;
            return AMI_SCAN_CONTINUE;
        } else {
            // We have no inputs, so we're done.
            sfout[0] = sfout[1] = 0;            
            return AMI_SCAN_DONE;
        }
    }
}
\end{verbatim}

The next step of the algorithm is to strip the cancelled edges away
from the list of all edges.  The remaining active edges will form a
recursive subproblem.  Again, we use a scan management object, this
time of the class \verb|strip_active_from_cancel|, which is defined as
follows:

\begin{verbatim}
////////////////////////////////////////////////////////////////////////
//
// strip_cancel_from_active
//
// A scan management object to take an active list and remove the
// smaller weighted edge of each pair of consecutive edges with the
// same destination.  The purpose of this is to strip edges out of the
// active list that were sent to the cancel list.
//
////////////////////////////////////////////////////////////////////////
class strip_cancel_from_active : AMI_scan_object {
private:
    bool holding;
    edge hold;
public:
    AMI_err initialize(void);  
    AMI_err operate(const edge &active, AMI_SCAN_FLAG *sfin,
                    edge *out, AMI_SCAN_FLAG *sfout);
};

AMI_err strip_cancel_from_active::initialize(void) {
    holding = false;
    return AMI_ERROR_NO_ERROR;
}

// Edges should be sorted by destination before being processed by
// this object.
AMI_err strip_cancel_from_active::operate(const edge &active,
                                  AMI_SCAN_FLAG *sfin,
                                  edge *out, AMI_SCAN_FLAG *sfout)
{
    // If no input then we're done, except that we might still be
    // holding one.
    if (!*sfin) {
        if (holding) {
            *sfout = 1;
            *out = hold;
            holding = false;
            return AMI_SCAN_CONTINUE;
        } else {
            *sfout = 0;
            return AMI_SCAN_DONE;
        }
    }

    if (!holding) {
        // If we are not holding anything, then just hold the current
        // input.
        hold = active;
        holding = true;
        *sfout = 0;
    } else {
        *sfout = 1;
        
        if (active.to == hold.to) {
            if (active.weight > hold.weight) {
                *out = active;
            } else {
                *out = hold;
            }

            holding = false;
        } else {
            *out = hold;
            hold = active;
        }
    }

    return AMI_SCAN_CONTINUE;
}
\end{verbatim}

After recursion, we must patch the cancelled edges back into the
recursively ranked list of active edges.  This is done using a scan
with a scan management object of the class
\verb|interleave_active_cancel|, which is implemented as follows:

\begin{verbatim}
////////////////////////////////////////////////////////////////////////
// interleave_active_cancel
//
// This is a class of merge object that merges two lists of edges
// based on their to fields.  The first list of edges should be active
// edges, while the second should be cancelled edges.  When we see two
// edges with the same to field, we know that the second was cancelled
// when the first was made active.  We then fix up the weights and
// output the two of them, one in the current call and one in the next
// call.
//
// The streams this operates on should be sorted by their terminal
// (to) nodes before AMI_scan() is called.
// 
////////////////////////////////////////////////////////////////////////

class patch_active_cancel : AMI_scan_object {
private:
    bool holding;
    edge hold;
public:
    AMI_err initialize(void);
    AMI_err operate(CONST edge &active, CONST edge &cancel,
                    AMI_SCAN_FLAG *sfin,
                    edge *patch, AMI_SCAN_FLAG *sfout);
};

AMI_err patch_active_cancel::initialize(void)
{
    holding = false;
    return AMI_ERROR_NO_ERROR;
}

AMI_err patch_active_cancel::operate(CONST edge &active, CONST edge &cancel,
                                     AMI_SCAN_FLAG *sfin,
                                     edge *patch, AMI_SCAN_FLAG *sfout)
{
    // Handle the special cases that occur when holding an edge and/or
    // completely out of input.
    if (holding) {
        sfin[0] = sfin[1] = 0;
        *patch = hold;
        holding = false;
        *sfout = 1;
        return AMI_SCAN_CONTINUE;
    } else if (!sfin[0]) {
        *sfout = 0;
        return AMI_SCAN_DONE;
    }

    if (!sfin[1]) {
        // If there is no cancel edge (i.e. all have been processed)
        // then just pass the active edge through.
        *patch = active;
    } else {
        if (holding = (active.to == cancel.to)) {
            patch->from = active.from;
            patch->to = cancel.from;
            patch->weight = active.weight - cancel.weight;
            hold.from = cancel.from;
            hold.to = active.to;
            hold.weight = active.weight;
        } else {
            *patch = active;
            sfin[1] = 0;
        }
    }

    *sfout = 1;
    return AMI_SCAN_CONTINUE;

}
\end{verbatim}

Finally, here is the actual function to rank the list.

\begin{verbatim}
////////////////////////////////////////////////////////////////////////
// list_rank()
//
// This is the actual recursive function that gets the job done.
// We assume that all weights are 1 when the initial call is made to
// this function.
//
// Returns 0 on success, nonzero otherwise.
////////////////////////////////////////////////////////////////////////

int list_rank(AMI_STREAM<edge> *istream, AMI_STREAM<edge> *ostream)
{
    AMI_err ae;
    
    off_t stream_len = istream->stream_len();

    AMI_STREAM<edge> *edges_rand;
    AMI_STREAM<edge> *active;
    AMI_STREAM<edge> *active_2;
    AMI_STREAM<edge> *cancel;
    AMI_STREAM<edge> *ranked_active;

    AMI_STREAM<edge> *edges_from_s;
    AMI_STREAM<edge> *cancel_s;
    AMI_STREAM<edge> *active_s;
    AMI_STREAM<edge> *ranked_active_s;

    // Scan/merge management objects.
    random_flag_scan my_random_flag_scan;
    separate_active_from_cancel my_separate_active_from_cancel;
    strip_cancel_from_active my_strip_cancel_from_active;
    patch_active_cancel my_patch_active_cancel;
    
    // Check if the recursion has bottomed out.  If so, then read in the
    // array and rank it.

    {
        size_t mm_avail;
        
        MM_manager.available(&mm_avail);

        if (stream_len * sizeof(edge) < mm_avail / 2) {
            edge *mm_buf = new edge[stream_len];
            istream->seek(0);
            istream->read_array(mm_buf,&stream_len);
            main_mem_list_rank(mm_buf,stream_len);
            ostream->write_array(mm_buf,stream_len);
            return 0;
        }
    }
    
    // Flip coins for each node, setting the flag to 0 or 1 with equal
    // probability.

    edges_rand = new AMI_STREAM<edge>;
    
    AMI_scan(istream, &my_random_flag_scan, edges_rand);

    // Sort one stream by source.  The original input was sorted by
    // destination, so we don't need to sort it again.

    edges_from_s = new AMI_STREAM<edge>;

    ae = AMI_sort(edges_rand, edges_from_s, edgefromcmp);

    // Scan to produce and active list and a cancel list.

    active = new AMI_STREAM<edge>;
    cancel = new AMI_STREAM<edge>;

    ae = AMI_scan(edges_from_s, edges_rand,
                  &my_separate_active_from_cancel,
                  active, cancel);

    delete edges_from_s;
    delete edges_rand;
    
    // Strip the edges that went to the cancel list out of the active list.

    active_s = new AMI_STREAM<edge>;

    ae = AMI_sort(active, active_s, edgetocmp);

    delete active;

    active_2 = new AMI_STREAM<edge>;

    ae = AMI_scan(active_s,
                  &my_strip_cancel_from_active,
                  active_2);

    delete active_s;

    // Recurse on the active list.  The list we pass in is sorted by
    // destination.  The recursion will return a list sorted by
    // source.

    ranked_active = new AMI_STREAM<edge>;
    
    list_rank(active_2, ranked_active);

    delete active_2;

    cancel_s = new AMI_STREAM<edge>;

    AMI_sort(cancel, cancel_s, edgetocmp);

    delete cancel;

    // The output of the recursive call is not necessarily sorted by
    // destination.  We'll make it so before we try to merge in the
    // cancel list.

    ranked_active_s = new AMI_STREAM<edge>;

    AMI_sort(ranked_active, ranked_active_s, edgetocmp);

    delete ranked_active;
    
    // Now merge the recursively ranked active list and the sorted 
    // cancel list.

    ae = AMI_scan(ranked_active_s, cancel_s,
                  &my_patch_active_cancel, ostream);

    delete ranked_active_s;
    delete cancel_s;
    
    return 0;
}
\end{verbatim}

Our recursion bottoms out when the problem is small enough to fit
entirely in main memory, in which case we read it in and call a
function to rank a list in main memory.  The details of this function
are omitted here.

\begin{verbatim}
////////////////////////////////////////////////////////////////////////
// main_mem_list_rank()
//
// This function ranks a list that can fit in main memory.  It is used
// when the recursion bottoms out.
//
////////////////////////////////////////////////////////////////////////

int main_mem_list_rank(edge *edges, size_t count)
{
    // Rank the list in main memory

    ...
        
    return 0;  
}
\end{verbatim}
\index{list ranking|)}

\section{NAS Parallel Benchmarks}

Code designed to implement external memory versions of a number of the
NAS parallel benchmarks is included with the TPIE distribution.
Examine this code for examples of how the various primitives TPIE
provides can be combined into powerful applications capable of solving
real-world problems.

Detailed descriptions of the NAS parallel benchmarks are available
from the \htmladdnormallink{NAS Parallel Benchmark Home Page}%
{http://www.nas.nasa.gov/NAS/NPB/}
\begin{latexonly}
at URL \verb|http://www.nas.nasa.gov/NAS/NPB/|.
\end{latexonly}


\part{Reference}
  %%
%% $Id: ami_reference.tex,v 1.16 2003-11-20 15:49:50 tavi Exp $
%%
\chapter{TPIE Programmer's Reference}
\plabel{cha:reference}

%%%%%%% Memory Manager %%%%%%%%%
\mysection{Registration-based Memory Manager}
\plabel{sec:mm-ref}
\index{memory manager|(}

\subsection{Files}
  \btabb
    \entry{\#include <mm\_register.h>} {Note that there is no need to
include this file when using the AMI entry points, since it is included by
all AMI header files.}
  \etabb

\subsection{Class Declaration}
  \btabb
    \entry{class \textbf{MM\_register};} {}
  \etabb

\subsection{Global Variables}
  \btabb
    \entry{MM\_register \textbf{MM\_manager};} {This is the only instance of
the \myverb{MM\_register} class that should exist in a program.}
  \etabb

\subsection{Description}
The TPIE memory manager \myverb{MM\_manager}, the only instance of class
\myverb{MM\_register}, traps memory allocation and deallocation requests in
order to monitor and enforce memory usage limits. The actual memory
allocation requests are done using the standard C++ operators \myverb{new}
and \myverb{delete}, which have been replaced with in-house versions that
interact with the memory manager.

\subsection{Public Member Functions}
  \btabb

    \entry{MM\_err \textbf{enforce\_memory\_limit}();} {Instruct TPIE to
    abort computation when the memory limit is exceeded.}

    \entry{MM\_err \textbf{ignore\_memory\_limit}();} {Instruct TPIE to
    ignore the memory limit set using \myverb{set\_memory\_limit}.}

    \entry{size\_t \textbf{memory\_available}();} {Return the number of
    bytes of memory which can be allocated before the user-specified limit
    is reached.}

    \entry{size\_t \textbf{memory\_limit}();} {Return the memory limit as
    set by the last call to method \myverb{set\_memory\_limit}.}

    \entry{size\_t \textbf{memory\_used}();} {Return the number of bytes
    of memory currently allocated.}

    \entry{MM\_err \textbf{set\_memory\_limit}(size\_t size);} {Set the
    application's memory limit. The memory limit is set to \myverb{size}
    bytes. If the specified mamory limit is greater than or equal to the
    amount of memory already allocated, \myverb{set\_memory\_limit} returns
    \myverb{MM\_ERROR\_NO\_ERROR}, otherwise it returns
    \myverb{MM\_ERROR\_EXCESSIVE\_ALLOCATION}. By default, successive calls
    to operator \myverb{new} will cause the program to abort if the
    resulting memory usage would exceed \myverb{size} bytes. This behaviour
    can be controlled explicitly by the use of methods
    \myverb{enforce\_memory\_limit}, \myverb{warn\_memory\_limit} and
    \myverb{ignore\_memory\_limit}.}

    \entry{int \textbf{space\_overhead}();} {TPIE imposes a small space
    overhead on each memory allocation request received by operator
    \myverb{new}. This involves increasing each allocation request by a
    fixed number of bytes. The precise size of this increase is machine
    dependent, but typically 8 bytes. Method \myverb{space\_overhead}
    returns the size of this increase.}

    \entry{MM\_err \textbf{warn\_memory\_limit}();} {Instruct TPIE to
    issue a warning when the memory limit is exceeded.}

  \etabb
\index{memory manager|)}

%%%%%%% AMI Stream %%%%%%%%%%
\mysection{Streams}
\index{streams!AMI|(}\plabel{sec:ref-ami-stream}
\index{AMI_STREAM@{\tt AMI\_STREAM}}

\subsection{Files}
  \btabb
    \entry{\#include <ami\_stream.h>} {}
  \etabb

\subsection{Class Declaration}
  \btabb
    \entry{template<class T> class \textbf{AMI\_STREAM};} {}
  \etabb

\subsection{Description}
An \myverb{AMI\_STREAM<T>} object stores an ordered collection of objects of
type {\tt T} on external memory.

\index{AMI_STREAM@{\tt AMI\_STREAM}!stream types|(}
The stream type of an \myverb{AMI\_STREAM} indicates what
operations are permitted on the stream.
An \myverb{AMI\_STREAM<T>} object can have one of four different types:
\begin{itemize}
    
    \item \myverb{AMI\_READ\_STREAM}: Input operations on
    the stream are permitted, but output is not permitted.
    
    \item \myverb{AMI\_WRITE\_STREAM}: Output operations are
    permitted, but input operations are not permitted. 
    
    \item \myverb{AMI\_APPEND\_STREAM}: Output is appended
    to the end of the stream. Input operations are not
    permitted. This is similar to
    \myverb{AMI\_WRITE\_STREAM} except that if the stream is
    constructed on a file containing an existing stream,
    objects written to the stream will be appended at the
    end of the stream.\comment{LA: It this true?
    DH: seems ok}

    \item \myverb{AMI\_READ\_WRITE\_STREAM}: Both input and output
    operations are permitted.
\end{itemize}
\index{AMI_STREAM@{\tt AMI\_STREAM}!stream types|)}

\subsection{Constructors, Destructor and Related Functions}
  \btabb

    \entry{\textbf{AMI\_STREAM}();} {A new stream of type
    \myverb{AMI\_READ\_WRITE\_STREAM} is constructed on a file with a
    randomly generated name.}
 
    \entry{\textbf{AMI\_STREAM}(const char *path\_name);} {A stream of type
    \myverb{AMI\_READ\_WRITE\_STREAM} is constructed on the file whose path
    name is given. If the file does not already exist, a new stream is
    constructed on a newly created file with the specified file name. If
    the file already exists, it is checked if it contains a valid stream,
    and if so, the new stream is constructed on this file. If the file does
    not contain a valid stream, the status flag is set to
    \myverb{AMI\_STREAM\_STATUS\_INVALID}.}

    \entry{\textbf{AMI\_STREAM}(const char *path\_name, AMI\_stream\_type st);} {A
    stream of type \noiv{st} is constructed on the file whose pathname is
    given.}

    \entry{\textbf{AMI\_STREAM}(BTE\_STREAM<T> \&bs);} {A stream is constructed from
    an existing \myverb{BTE\_STREAM} (see Section~\ref{sec:ref-bte}). This
    constructor will not normally be used by a TPIE application
    programmer\comment{LA: Why do we have it then? DH: for
    completeness}. The new \myverb{AMI\_STREAM} gets the same type as the
    \myverb{BTE\_STREAM}.}

    \entry{\textbf{$\sim$AMI\_STREAM}();} {Destructor. Free the memory
    buffer and close the file. IF the persistence flag is
    \myverb{PERSIST\_DELETE}, also remove the file.}

    \index{new_substream()@{\tt new\_substream()}!AMI|)}     
    \entry{AMI\_err \textbf{new\_substream}(AMI\_stream\_type    st,
                            off\_t              sub\_begin,
                            off\_t              sub\_end,
                            AMI\_base\_stream<T> **sub\_stream );} 
    {A substream is an AMI stream that is part of another AMI stream. More
    precisely, a substream $B$ of a stream $A$ is defined as a contiguous
    range of objects from the ordered collection of objects that make up
    the stream $A$.  If desired, one can construct substreams of substreams
    of substreams {\em ad infinitum}. Since a substream is a stream in its
    own right, many of the stream member functions can be applied to a
    substream. A substream can be created via the
    pseudo-constructor\footnote{The reason we do not use a real constructor
    is to get around the fact that constructors can not be virtual.  Please
    see Section~\ref{cha:implementation} for more details.}
    \myverb{new\_substream()}. Here, \myverb{st} specifies the type of the
    substream, and the offsets \myverb{sub\_begin} and \myverb{sub\_end}
    define the positions in the original stream $A$ where the new substream
    $B$ will begin and end. Upon completion, \myverb{*sub\_stream} points
    to the newly created substream.}
    \index{new_substream()@{\tt new\_substream()}!AMI|)} 
  \etabb

\subsection{Public Member Functions}
  \btabb

    \entry{bool \textbf{operator!}() const;} {Return \noiverb{true} if the
    status of the stream is not
    \myverb{AMI\_STREAM\_STATUS\_VALID}, \noiverb{false} otherwise. See
    also \myverb{is\_valid()} and \myverb{status()}.}

    \entry{off\_t \textbf{chunk\_size}() const;} {Return the maximum number
    of items (of type \noiverb{T}) that can be stored in one block.}

    \entry{static const tpie\_stats\_stream\& \textbf{gstats}();}
    {Return an object containing the statistics of all streams opened by the
    application (global statistics). See also \myverb{stats()}.}

    \entry{bool \textbf{is\_valid}() const;} {Return \noiverb{true} if the
    status of the stream is \myverb{AMI\_STREAM\_STATUS\_VALID},
    \noiverb{false} otherwise. See also \myverb{status()}.}

    \index{main_memory_usage()@{\tt main\_memory\_usage()}!AMI|(}
    \entry{AMI\_err \textbf{main\_memory\_usage}(size\_t *usage, MM\_stream\_usage
usage\_type);} {This function is used for obtaining the amount of main memory used by an
\noiv{AMI\_STREAM<T>} object (in bytes). \myverb{usage\_type} is one of the
following:
\begin{description}
  \item[\myverb{MM\_STREAM\_USAGE\_CURRENT}] Total amount of memory currently
used by the stream.
  \item[\myverb{MM\_STREAM\_USAGE\_MAXIMUM}] Max amount of memory that will
ever be used by the stream.
  \item[\myverb{MM\_STREAM\_USAGE\_OVERHEAD}] The amount of memory used by
the object itself, without the data buffer.
  \item[\myverb{MM\_STREAM\_USAGE\_BUFFER}]  The amount of memory used by the
data buffer.
  \item[\myverb{MM\_STREAM\_USAGE\_SUBSTREAM}] The additional amount of
memory that will be used by  each substream created.
\end{description}
}
    \index{main_memory_usage()@{\tt main\_memory\_usage()}!AMI|)}

    \index{name()@{\tt name()}!AMI|(}
    \entry{AMI\_err \textbf{name}(char **stream\_name);} {Store the path to the UNIX
file name holding the stream, in newly allocated memory.}
    \index{name()@{\tt name()}!AMI|)}

    \entry{void \textbf{persist}(persistence p)} {Set the persistence flag to \noiv{p},
    which can have one of two values: \myverb{PERSIST\_DELETE} and
    \myverb{PERSIST\_PERSISTENT}.}

    \index{read_array()@{\tt read\_array()}!AMI|(}
    \entry{AMI\_err \textbf{read\_array}(T *mm\_array, off\_t *len);} {Read \noiv{*len} items from
the current position of the stream into the array \noiv{mm\_array}. The
``current item'' pointer is increased accordingly. }
    \index{read_array()@{\tt read\_array()}!AMI|)}

    \index{read_item()@{\tt read\_item()}!AMI|(} 
    \entry{AMI\_err
    \textbf{read\_item}(T **elt);} {Read the current item from the stream and
    advance the ``current item'' pointer to the next item. The item read is
    pointed to by \noiv{*elt}. If no error has occured, return
\myverb{AMI\_ERROR\_NO\_ERROR}. If the ``current item'' pointer is beyond
the last item in the stream, return \myverb{AMI\_ERROR\_END\_OF\_STREAM}.}
    \index{read_item()@{\tt read\_item()}!AMI|)}
    
    \index{seek()@{\tt seek()}!AMI|(}
    \entry{AMI\_err \textbf{seek}(off\_t off);} {Move the current position to \noiv{off}.}
    \index{seek()@{\tt seek()}!AMI|)}

    \entry{const tpie\_stats\_stream\& \textbf{stats}() const;} {Return an
    object containing the statistics of this stream. The types of
    statistics computed for a collection are tabulated below. See also
    \myverb{gstats()}.\\[1mm] \begin{tabular}{|l|l|} \hline \myverb{BLOCK\_READ}
    & Number of block reads\\ \myverb{BLOCK\_WRITE} & Number of block
    writes \\ \myverb{ITEM\_READ} & Number of item reads\\
    \myverb{ITEM\_WRITE} & Number of item writes\\ \myverb{ITEM\_SEEK} &
    Number of item seek operations\\ \myverb{STREAM\_OPEN} & Number of
    stream open operations\\ \myverb{STREAM\_CLOSE} & Number of stream
    close operations \\ \myverb{STREAM\_CREATE} & Number of stream create
    operations\\ \myverb{STREAM\_DELETE} & Number of stream delete
    operations \\ \myverb{SUBSTREAM\_CREATE} & Number of substream create
    operations\\ \myverb{SUBSTREAM\_DELETE} & Number of substream delete
    operations \\ \hline \end{tabular} }

    \entry{AMI\_stream\_status \textbf{status}() const;} {Return the status
    of the stream instance. The result is either
    \myverb{AMI\_STREAM\_STATUS\_VALID} or
    \myverb{AMI\_STREAM\_STATUS\_INVALID}. The only operation that can
    leave the stream invalid is the constructor (if that happens, the log
    file contains more information). No items should be read from or
    written to an invalid stream.}

    \index{stream_len()@{\tt stream\_len()}!AMI|(}
    \entry{off\_t \textbf{stream\_len}(void);} {Return the number of items stored in
the stream.}
    \index{stream_len()@{\tt stream\_len()}!AMI|)}

    \index{truncate()@{\tt truncate()}!AMI|(}
    \entry{AMI\_err \textbf{truncate}(off\_t off);} {Resize the stream to
\noiv{off} items. If \noiv{off} is less than the number of objects in the
stream, \noiv{truncate()} truncates the stream to
\noiv{off} objects. If \noiv{off} is more than the
number of objects in the stream, \noiv{truncate()} extends
the stream to the specified number of objects. In either
case, the ``current item'' pointer will be moved to the new end of
the stream.}
    \index{truncate()@{\tt truncate()}!AMI|)}

    \index{write_array()@{\tt write\_array()}!AMI|(}
    \entry{AMI\_err \textbf{write\_array}(const T *mm\_array, off\_t len);} {Write
\noiv{len} items from array \noiv{mm\_array} to the stream, starting in the
current position. The ``current item'' pointer is increased accordingly.}
    \index{write_array()@{\tt write\_array()}!AMI|)}

    \index{write_item()@{\tt write\_item()}!AMI|(}
    \entry{AMI\_err \textbf{write\_item}(const T \&elt);} {Write \noiv{elt} to the
stream in the current position. Advance the ``current item'' pointer to the
next item. If no error has occured, return \myverb{AMI\_ERROR\_NO\_ERROR}.}
    \index{write_item()@{\tt write\_item()}!AMI|)}
        
  \etabb
\index{streams!AMI|)}


%%%%%%% Scanning %%%%%%%%
\mysection{Scanning}
\plabel{sec:ref-ami-scan}
\index{scanning|(}
\index{AMI_scan()@{\tt AMI\_scan()}|(}

\subsection{Files}
  \btabb
    \entry{\#include <ami\_scan.h>} {}
  \etabb

\subsection{Function Declaration}
  \btabb
    \entry{template<class T1, class T2, ..., class ST, class U1, class U2,
...> AMI\_err \textbf{AMI\_scan}(AMI\_STREAM<T1> *in1, AMI\_STREAM<T2>
*in2, ..., ST *smo, AMI\_STREAM<U1> *out1, AMI\_STREAM<U2> *out2, ...);} {}
  \etabb

\subsection{Description}

\noiv{AMI\_scan()} reads zero, one or multiple input streams (up to
four), each potentially of a different type, and writes zero, one or
multiple output streams (up to four), each potentially of a different type.
\noiv{smo} is a pointer to a {\em scan management object} of user-defined
class \noiv{ST}, as described below.  
%\noiv{ST} should provide member functions {\tt
%AMI\_err initialize(void)} and {\tt AMI\_err operate(const T1 \&in1,
%const T2 \&in2, ..., AMI\_SCAN\_FLAG *sfin, U1 *out1, U2 *out2, ...,
%AMI\_SCAN\_FLAG  *sfout)}.

\subsection{Scan Management Objects}

\index{operation management objects!scan|(}

A scan management object class must inherit from \myverb{AMI\_scan\_object}:
\begin{verbatim}
template<class T1, class T2,..., class U1, class U2,...>
class ST: public AMI_scan_object;
\end{verbatim}
In addition, it must provide two member
functions for \myverb{AMI\_scan()} to call: \myverb{initialize()} and \myverb{operate()}.
\index{initialize()@{\tt initialize()}|(}
\begin{verbatim}
    AMI_err initialize(void);
\end{verbatim}
    Initializes a scan management object to prepare
    it for a scan.  This member function is called once by
    each call to \myverb{AMI\_scan()} in order to initialize
    the scan management object before any data processing
    takes place.  This function should return
    \myverb{AMI\_ERROR\_NO\_ERROR} if successful, or an
    appropriate error otherwise. See
    Section~\ref{sec:ami-errors} for a list of error codes.
\index{initialize()@{\tt initialize()}|)}
    
\index{operate()@{\tt operate()}|(}
    Most of the work of a scan is typically done in the scan
    management object's \myverb{operate()} member function:
\begin{verbatim}
    AMI_err operate(const T1 &in1, const T2 &in2,...,  AMI_SCAN_FLAG *sfin,
                    U1 *out1, U2 *out2,..., AMI_SCAN_FLAG *sfout);
\end{verbatim}
    
    One or more input objects or one or more output
    parameters must be specified.  These must correspond in
    number and type to the streams passed to the polymorph
    of \myverb{AMI\_scan()} with which this scan management
    object is to be used.
    
    If present, the inputs \noiv{*in1, ...} are application
    data items of type \noiv{T1}, and \myverb{sfin} points
    to an array of flags, one for each input.  On entry to
    \myverb{operate()}, flags that are set (non-zero)
    indicate that the corresponding inputs contain data.  If
    on exit from \myverb{operate()}, the input flags are
    left untouched, \myverb{AMI\_scan()} assumes that the
    corresponding inputs were processed.  If one or more
    input flags are cleared (set to zero) then
    \myverb{AMI\_scan()} assumes that the corresponding
    inputs were not processed and should be presented again
    on the next call to \myverb{operate()}.  This permits
    out of step scanning\index{scanning!out of step}, as
    illustrated in Section~\ref{sec:tut-out-of-step}.
    
    If present, the outputs \noiv{*out1, ...} are
    application data items of type \noiv{U1}, and
    \myverb{sfout} points to an array of flags, one for each
    output. On exit from \myverb{operate()}, the outputs
    should contain any objects to be written to the output
    streams, and the output flags must be set to indicate to
    \myverb{AMI\_scan()} which outputs are valid and should
    be written to the output streams.
    
    The return value of \myverb{operate()} will normally be
    one of the following:
    \begin{itemize}
        \item \myverb{AMI\_SCAN\_CONTINUE}:
        \index{AMI_SCAN_CONTINUE@{\tt AMI\_SCAN\_CONTINUE}} indicates that
        the function should be called again with any
        ``taken'' inputs replaced by the next objects from
        their respective streams
        \item \myverb{AMI\_SCAN\_DONE}: 
        \index{AMI_SCAN_DONE@{\tt AMI\_SCAN\_DONE}}
        indicates that the
        scan is complete and no more input needs to be
        processed.
    \end{itemize}
    
    Note that \myverb{operate()} is permitted to return
    \myverb{AMI\_SCAN\_CONTINUE} even when the input flags
    indicate that there is no more input to be processed.
    This is useful if the scan management object maintains
    some internal state that must be written out after all
    input has been processed.  

\index{operate()@{\tt operate()}|)} 

Examples of the use of scan management objects are given in
Section~\ref{sec:tut-scanning} as well as in the test applications that
appear in the TPIE distribution.

\index{operation management objects!scan|)}
\index{AMI_scan()@{\tt AMI\_scan()}|)}
\index{scanning|)}


%%%%%%%%% Scanning from a C++ stream %%%%%%%%%%
\mysection{Scanning from a C++ stream}
\plabel{sec:ref-cxx-stream-input}

\subsection{Files}
  \btabb
    \entry{\#include <ami\_scan\_utils.h>} {}
  \etabb


\subsection{Class Declaration}
  \btabb
    \entry{template<class T> class \textbf{cxx\_istream\_scan};} {}
  \etabb

\subsection{Description}
A scan management class template for reading the contents of an
ordinary C++ input stream into a TPIE stream.  It works with
streams of any type for which a \myverb{>>} operator is defined for C++
stream input.

\subsection{Constructor}
  \btabb
    \entry{\textbf{cxx\_istream\_scan}(istream *instr = \&cin);} {Create a
scan management object for scanning the contents of C++ stream
\noiv{*instr}. The actual scanning is done using \myverb{AMI\_scan} with
no input streams and one output stream.}
  \etabb


%%%%%%%%% Scanning into a C++ stream %%%%%%%%%%
\mysection{Scanning into a C++ stream}
\plabel{sec:ref-cxx-stream-output}

\subsection{Files}
  \btabb
    \entry{\#include <ami\_scan\_utils.h>} {}
  \etabb


\subsection{Class Declaration}
  \btabb
    \entry{template<class T> class \textbf{cxx\_ostream\_scan};} {}
  \etabb

\subsection{Description}
A scan management class template for writing the contents of a TPIE stream
into an ordinary C++ output stream.  It works with
streams of any type for which a \myverb{<<} operator is defined for C++
stream output.

\subsection{Constructor}
  \btabb
    \entry{\textbf{cxx\_ostream\_scan}(istream *outstr = \&cout);} {Create a
scan management object for scanning into C++ stream
\noiv{*outstr}. The actual scanning is done using \myverb{AMI\_scan} with
one input stream and no output streams.}
  \etabb


%%%%%%%% Stream Merging %%%%%%%%%
\mysection{Stream Merging}
\index{merging|(}
\index{AMI_merge()@{\tt AMI\_merge()}|(}
\plabel{sec:ref-ami-merge}

\subsection{Files}
  \btabb
    \entry{\#include <ami\_merge.h>} {}
  \etabb

\subsection{Function Declarations}
  \btabb
    \entry{template<class T>\\  AMI\_err \textbf{AMI\_merge}(AMI\_STREAM<T> **instreams,
                      arity\_t arity, AMI\_STREAM<T> *outstream);} {}
    \entry{template<class T>\\  AMI\_err \textbf{AMI\_merge}(AMI\_STREAM<T> **instreams,
                      arity\_t arity, AMI\_STREAM<T> * outstream,
                      int (*cmp)(const T\&, const T\&));} {}
    \entry{template<class T>\\ AMI\_err \textbf{AMI\_merge}(AMI\_STREAM<T> **instreams,
                      arity\_t arity, AMI\_STREAM<T> * outstream, CmpObj *co);} {}
    \entry{template<class T, class KEY>\\ AMI\_err \textbf{AMI\_merge}(AMI\_STREAM<T> **instreams,
                   arity\_t arity, AMI\_STREAM<T> *outstream, int keyoff, KEY dummy);} {}
  \etabb

\subsection{Description}

TPIE provides several merge entry points for merging sorted streams to
produce a single, interleaved output stream. \myverb{AMI\_merge} has
four polymorphs, described below. We will refer to these as the (1)
comparison operator, (2) comparison function, (3) comparison class and (4) key-based
versions of \myverb{AMI\_merge}. 
 The comparison operator version tends to
be the fastest and most straightforward to use. The
comparison class version is comparable in speed (maybe
slightly slower), but somewhat more flexible, as it can support
multiple, different merges on the same keys. The comparison
function version is slightly easier to use than the
comparison class version, but typically it is measureably slower.

\index{AMI_merge()@{\tt AMI\_merge()}|)}
\index{merging|)}


%%%%%%%% Generalized Stream Merging %%%%%%%%%
\mysection{Generalized Stream Merging}
\index{merging!generalized|(}
\index{AMI_generalized_merge()@{\tt AMI\_generalized\_merge()}|(}

\subsection{Files}
  \btabb
    \entry{\#include <ami\_merge.h>} {}
  \etabb

\subsection{Function Declaration}
  \btabb
    \entry{template<class T, class MergeMgr>\\ 
           AMI\_err \textbf{AMI\_generalized\_merge}(AMI\_STREAM<T> **instreams, 
           arity\_t arity, AMI\_STREAM<T> *outstream, MergeMgr *mo);} {}
  \etabb

\subsection{Description}
TPIE entry point \myverb{AMI\_generalized\_merge()} allows an
arbitrary number of streams to be merged into one stream in one pass,
subject to the available main memory.  TPIE will attempt to read the
first block of each stream into the internal memory, and will update
the contents of these buffers as the merge progresses. At least one
block buffer is also required for the output stream from the merge.
The function takes four arguments:
\myverb{instreams} is an array of pointers to the input streams, all of
  which are of type \myverb{AMI\_STREAM<T>},
\myverb{arity} is the number of input streams,
\myverb{outstream} is the output stream, of type \myverb{AMI\_STREAM<T>}, and
\myverb{mo} points to a merge management object that controls the
  merge (merge management objects are described below).

If the merge cannot be completed in one pass due to insufficient
memory, the function fails and it returns
\myverb{AMI\_ERROR\_INSUFFICIENT\_MAIN\_MEMORY}. Otherwise, it returns
\myverb{AMI\_ERROR\_NO\_ERROR}.

\index{AMI_generalized_merge()@{\tt AMI\_generalized\_merge()}|)}


\subsection{Merge Management Objects}\label{ssec:mmo}
\index{operation management objects!merge|(} 

A merge management object class must inherit from
\myverb{AMI\_generalized\_merge\_base}:
\begin{verbatim}
template<class T>
class MergeMgr: public AMI_generalized_merge_base;
\end{verbatim}
In addition, 
a merge management object must provide \myverb{initialize()}
and \myverb{operate()} member functions, whose purposes are
analogous to their namesakes for scan management objects.

\index{initialize()@{\tt initialize()}|(}
    The user's \myverb{initialize()} member function is
    called by the merge function once so that
    application-specific data structures (if any) can be
    initialized.
\begin{verbatim}
    AMI_err initialize(arity_t arity, const T * const *in,
                       AMI_merge_flag *taken_flags,
                       int &taken_index);
\end{verbatim}
 
    where
    \begin{itemize}
        \item \myverb{arity} is the number of input streams
        in the merge,
        \item \myverb{in} is a pointer to an array of
        pointers to input objects, each of which is the
        first objects appearing in one of the input streams,    
        \item \myverb{taken\_flags} an array of flags
        indicating which of the inputs are present (i.e.
        which of the input streams is not empty), and a
        pointer to an output object.
    \end{itemize}
    
    The typical behavior of \myverb{initialize()} is to
    place all the input objects into a data structure and
    then return \myverb{AMI\_MERGE\_READ\_MULTIPLE} to
    indicate that it used (and is now finished with) all of
    the inputs which were indicated to be valid by
    \myverb{taken\_flags}.  \myverb{initialize} need not
    process all inputs; it can turn off any flags in
    \myverb{taken\_flags} corresponding to inputs that
    should be presented to \myverb{operate()}.
    Alternatively, it can set \myverb{taken\_index} to the
    index of a single input it processed and return
    \myverb{AMI\_MERGE\_CONTINUE}.
\index{initialize()@{\tt initialize()}|)}

\index{operate()@{\tt operate()}|(}
    When performing a merge, TPIE relies on the application
    programmer to provide code to determine the order of any
    two application data elements, and certain other
    application-specific processing. By convention, TPIE
    expects these decisions to be made by the
    \myverb{operate()} function:
\begin{verbatim}
    AMI_err operate(const T * const *in, AMI_merge_flag *taken_flags,
                    int &taken_index, T *out);
\end{verbatim}
    The \myverb{operate()} member function is called
    repeatedly to process input objects.  Typically,
    \myverb{operate()} will choose a single input object to
    process, and set \myverb{taken\_index} to the index of
    the pointer to that object in the input array.  This
    object is then typically added to a dynamic data
    structure maintained by the merge management object.  If
    output is generated, for example by removing an object
    from the dynamic data structure, \myverb{operate()}
    should return \myverb{AMI\_MERGE\_OUTPUT}, otherwise, it
    returns either \myverb{AMI\_MERGE\_CONTINUE} to indicate
    that more input should be presented, or
    \myverb{AMI\_MERGE\_DONE} to indicate that the merge has
    completed.
    
    Alternatively, \myverb{operate()} can clear the elements
    of \myverb{taken\_flags} that correspond to inputs it
    does not currently wish to process, and then return
    \myverb{AMI\_MERGE\_READ\_MULTIPLE}.  This is generally
    undesirable because, if only one input is taken, it is
    far slower than using \myverb{taken\_index} to indicate
    which input was taken.  The merge management object must
    clear all other flags, and then TPIE must test all the
    flags to see which inputs were or were not processed.
\index{operate()@{\tt operate()}|)}


\index{operation management objects!merge|)}
\index{merging!generalized|)}


%%%%%%%% Stream Partitioning and Merging %%%%%%%%%
\mysection{Stream Partitioning and Merging}
\index{AMI_partition_and_merge()@{\tt AMI\_partition\_and\_merge()}|(}
\plabel{sec:ref-ami-pmerge}

\subsection{Files}
  \btabb
    \entry{\#include <ami\_merge.h>} {}
  \etabb

\subsection{Function Declaration}
  \btabb
    \entry{template<class T>\\  AMI\_err \textbf{AMI\_partition\_and\_merge}(AMI\_STREAM<T> **instreams,
                      arity\_t arity, AMI\_STREAM<T> *outstream);} {}
    \entry{template<class T>\\  AMI\_err \textbf{AMI\_partition\_and\_merge}(AMI\_STREAM<T> **instreams,
                      arity\_t arity, AMI\_STREAM<T> *outstream,
                      int (*cmp)(const T\&, const T\&));} {}
    \entry{template<class T, class KEY>\\ AMI\_err \textbf{AMI\_partition\_and\_merge}(AMI\_STREAM<T> **instreams,
                      arity\_t arity, AMI\_STREAM<T> *outstream, int keyoffset, KEY dummy);} {}
  \etabb

\subsection{Description}
Each of these functions partitions a stream into substreams small
enough to fit in main memory, sorts them in main memory, and then
merges them together, possibly in several passes if low memory conditions
dictate. The difference between the three polymorphs is the comparison
method: in the first polymorph, comparison is done using the comparison
operator of class \myverb{T}; in the second polymorph, comparison is done
using the comparison function \myverb{cmp}; in the third polymorph,
comparison is done using a key of type \myverb{KEY}, extracted from objects
of type \myverb{T} at byte offset \myverb{keyoffset}.

In order to complete the merge successfully, these functions need sufficient
memory for a binary merge. If not enough memory is available, the function
fails and it returns
\myverb{AMI\_ERROR\_INSUFFICIENT\_MAIN\_MEMORY}. Otherwise, it returns
\myverb{AMI\_ERROR\_NO\_ERROR}.
\index{AMI_partition_and_merge()@{\tt AMI\_partition\_and\_merge()}|)}

%%%%%%%% Generalized Stream Partitioning and Merging %%%%%%%%%
\mysection{Generalized Stream Partitioning and Merging}
\index{AMI_generalized_partition_and_merge()@{\tt AMI\_generalized\_partition\_and\_merge()}|(}
\plabel{sec:ref-ami-gpmerge}

\subsection{Files}
  \btabb
    \entry{\#include <ami\_merge.h>} {}
  \etabb

\subsection{Function Declaration}
  \btabb
    \entry{template<class T, class MergeMgr>\\
           AMI\_err \textbf{AMI\_generalized\_partition\_and\_merge}(AMI\_STREAM<T> *instream, AMI\_STREAM<T> *outstream, MergeMgr *mo);} {}
  \etabb

\subsection{Description}
This function partitions a stream into substreams small
enough to fit in main memory, operates on each in main memory, and then
merges them together, possibly in several passes if low memory conditions
dictate. This function takes three arguments:
\myverb{instream} points to the input stream,
\myverb{outstream} points to the output stream, and
\myverb{mo} points to a merge management object that controls the merge.
This function takes care of all the details of determining how much main
memory is available, how big the initial substreams can be, how many
streams can be merged at a time, and how many levels of merging must take
place.

In order to complete the merge successfully, the function needs sufficient
memory for a binary merge. If not enough memory is available, the function
fails and it returns
\myverb{AMI\_ERROR\_INSUFFICIENT\_MAIN\_MEMORY}. Otherwise, it returns
\myverb{AMI\_ERROR\_NO\_ERROR}.
\index{AMI_generalized_partition_and_merge()@{\tt AMI\_generalized\_partition\_and\_merge()}|)}

\subsection{Merge Management Objects}
\index{operation management objects!merge|(} 
The \myverb{AMI\_partition\_and\_generalized\_merge()} entry point requires
a merge management object similar to the one described in
Section~\ref{ssec:mmo}. The following three additional member functions
must also be provided.

\begin{itemize}
    \item \index{main_mem_operate()@{\tt main\_mem\_operate()}}
\begin{verbatim}
    AMI_err main_mem_operate(T* mm_stream, size_t len);
\end{verbatim}
\noindent
where
    \begin{itemize}
        \item \myverb{mm\_stream} is a pointer to an array
        of objects that have been read into main memory,
        \item \myverb{len} is the number of objects in the
        array.
    \end{itemize}
    
    This function is called by
    \myverb{AMI\_partition\_and\_merge()} when a substream of
    the data is small enough to fit into main memory, and
    the (application-specific) processing of this subset of
    the data can therefore be completed in internal memory.

    
    \item \index{space_usage_per_stream()@{\tt space\_usage\_per\_stream()}}
\begin{verbatim}
    size_t space_usage_per_stream(void);
\end{verbatim}
    This function should return the amount of main memory
    that the merge management object will need per per input
    stream. Merge management objects are allowed to maintain
    data structures whose size is linear in the number of
    input streams being processed.

    \item \index{space_usage_overhead()@{\tt space\_usage\_overhead()}}
\begin{verbatim}
    size_t space_usage_overhead(void);
\end{verbatim}
    This function should return an upper bound on the number
    of bytes of main memory\comment{LA: In bytes? DH: yes.}
    the merge management object will allocate in addition to
    the portion that is linear in the number of streams.
    
\end{itemize}

\index{operation management objects!merge|)}


%%%%%%%%% Merge Sorting %%%%%%%%
\mysection{Merge Sorting}\plabel{sec:ref-ami-sort}
\index{sorting!merge|)} 
\index{AMI_key_sort@{\tt AMI\_key\_sort}|(}
\index{AMI_ptr_sort@{\tt AMI\_ptr\_sort}|(}
\index{AMI_sort@{\tt AMI\_sort}|(}

\subsection{Files}
  \btabb
     \entry{\#include <ami\_sort.h>} {}
  \etabb

\subsection{Function Declarations}

  \btabb
     \entry{template<class T>\\
AMI\_err \textbf{AMI\_sort}(AMI\_STREAM<T> *instream, AMI\_STREAM<T> *outstream);} {}     \entry{template<class T>\\
AMI\_err \textbf{AMI\_sort}(AMI\_STREAM<T> *instream, AMI\_STREAM<T> *outstream,
                 int (*cmp)(const T\&, const T\&));} {}
     \entry{template<class T, class CMPR>\\
AMI\_err \textbf{AMI\_sort}(AMI\_STREAM<T> *instream, AMI\_STREAM<T> *outstream,
                 CMPR *cmp);} {}

     \entry{template<class T>\\
AMI\_err \textbf{AMI\_ptr\_sort}(AMI\_STREAM<T> *instream, AMI\_STREAM<T>
*outstream);} {}
     \entry{template<class T>\\
AMI\_err \textbf{AMI\_ptr\_sort}(AMI\_STREAM<T> *instream, AMI\_STREAM<T> *outstream,
                 int (*cmp)(const T\&, const T\&));} {}
     \entry{template<class T, class CMPR>\\
AMI\_err \textbf{AMI\_ptr\_sort}(AMI\_STREAM<T> *instream, AMI\_STREAM<T> *outstream, CMPR *cmp);} {}

     \entry{template<class T, class KEY, class CMPR>\\
AMI\_err \textbf{AMI\_key\_sort}(AMI\_STREAM<T> *instream, AMI\_STREAM<T> *outstream,
         KEY dummykey, CMPR *cmp) ;} {}
  \etabb

\subsection{Description}
TPIE offers several entry points for sorting which use
merging as their underlying paradigm. Please see
Section~\ref{sec:tut-mrg-sorting} for more details of this
approach.

Currently, TPIE offers three merge sorting variants. The
user must decide which variant is most appropriate for their
circumstances.  All accomplish the same goal, but the
performance can vary depending on the situation. They differ
mainly in the way they perform the merge phase of merge
sort, specifically how they maintain their heap data
structure used in the merge phase. The three variants are as
follows:
\begin{itemize}
    \item \myverb{AMI\_sort}: keeps the (entire) first record
    of each sorted run (each is a stream) in a heap. This
    approach is most suitable when the record consists
    entirely of the record key.
    
    \item \myverb{AMI\_ptr\_sort}: keeps a pointer to the
    first record of each stream in the heap. This approach
    works best when records are very long and the key
    field(s) take up a large percentage of the record.

    \item \myverb{AMI\_key\_sort}: keeps the key field(s) and
    a pointer to the first record of each stream in the
    heap. This approach works best when the key field(s) are
    small in comparison to the record size.
\end{itemize}

Any of these variants will accomplish the task of sorting an
input stream in an I/O efficient way, but there can be
noticeable differences in processing time between the
variants. As an example, \myverb{AMI\_key\_sort} appears to be
more cache-efficient than the others in many cases, and
therefore often uses less processor time, despite extra data
movement relative to \myverb{AMI\_ptr\_sort}.

In addition to the three variants discussed above, there are
multiple choices within each variant regarding how the
actual comparison operations are to be performed. These
choices are described in detail for \myverb{AMI\_sort}, below.

\subsubsection{AMI\_sort()}
\myverb{AMI\_sort()} has three polymorphs, described below.
We will refer to these as the (1) comparison operator, (2)
comparison function, and (3) comparison class versions of
\myverb{AMI\_sort}. The comparison operator version tends to
be the fastest and most straightforward to use. The
comparison class version is comparable in speed (maybe
slightly slower), but somewhat more flexible, as it can support
multiple, different sorts on the same keys. The comparison
function version is slightly easier to use than the
comparison object version, but typically it is measureably slower.
Please refer to Section~\ref{sec:tut-mrg-sorting} for examples
of the use of these versions of \myverb{AMI\_sort()}.

\paragraph{Comparison operator version.} This version works on streams of
objects for which the operator \myverb{<} is defined. 

\paragraph{Comparison function version.}
This version uses an explicit function to determine the
relative order of two objects in the input stream. The function takes two
arguments of type \noiv{T} and returns $-1$, $0$, or $1$, if the first
object is less than, equal or greater than the second object in the desired
order. This is useful in cases where we may want to sort a stream of
objects in several different ways.  

\paragraph{Comparison class version.} 
This version of \myverb{AMI\_sort()} is similar to the
comparison function version, except that the comparison
function is now a method of a user-defined comparison
object. This object must have a public member function named
\myverb{compare}, having the following prototype:
\begin{verbatim}
   inline int compare (const KEY & k1, const KEY & k2);
\end{verbatim}

The user-written \myverb{compare} function computes the
order of the two user-defined keys \noiv{k1} and
\noiv{k2}, and returns $-1$, $0$, or $+1$ to indicate that
$k1<k2$, $k1==k2$, or $k1>k2$ respectively.
It will be called by the internals of \noiv{AMI\_key\_sort} to
determine the relative order of records during the sort.


\index{AMI_sort@{\tt AMI\_sort}|)}

\subsubsection{AMI\_ptr\_sort()}

The \myverb{AMI\_ptr\_sort} variant of merge sort in TPIE
keeps only a pointer to each record in the heap used to
perform merging of runs. Similar to \myverb{AMI\_sort}
above, it offers comparison operator, comparison function,
and comparison class polymorphs. The syntax is identical to
that illustrated in the \myverb{AMI\_sort} examples; simply
replace \myverb{AMI\_sort} by \myverb{AMI\_ptr\_sort}.

\index{AMI_ptr_sort@{\tt AMI\_ptr\_sort}|)}

\subsubsection{AMI\_key\_sort()}

The \myverb{AMI\_key\_sort} variant of TPIE merge sort keeps
the key field(s) plus a pointer to the corresponding record
in an internal heap during the merging phase of merge sort.
It requires a sort management object with member functions
\myverb{compare} and \myverb{copy}.
The \noiv{dummyKey} argument of \myverb{AMI\_key\_sort()} is a a
dummy argument having the same type as the user key, and
\noiv{*smo} is the sort management
object, having user-defined \myverb{compare} and
\myverb{copy} member functions as described below.

The \myverb{compare} member function has the following
prototype:
\begin{verbatim}
   inline int compare (const KEY & k1, const KEY & k2);
\end{verbatim}

The user-written \myverb{compare} function computes the
order of the two user-defined keys \noiv{k1} and
\noiv{k2}, and returns $-1$, $0$, or $+1$ to indicate that
$k1<k2$, $k1==k2$, or $k1>k2$ respectively.
It will be called by the internals of \noiv{AMI\_key\_sort} to
determine the relative order of records during the sort.

The \myverb{copy} member function has the following
prototype:
\begin{verbatim}
   inline void copy (KEY *key, const T &record);
\end{verbatim}

The user-written \myverb{copy} function constructs the
user-defined key \noiv{*key} from the contents of the
user-defined record \noiv{record}. It will be called by the
internals of \noiv{AMI\_key\_sort} to make copies of record
keys as necessary during the sort.

\index{AMI_key_sort@{\tt AMI\_key\_sort}|)}
\index{sorting!merge|)} 

%%%%%%%%% Internal Memory Sorting %%%%%%%%
\mysection{Internal Memory Sorting}
\plabel{sec:ref-ami-memsort}
\index{sorting!internal memory|(}

\subsection{Files}
  \btabb
        \entry{\#include <quicksort.h>} {}
  \etabb

\subsection{Function Declarations}

  \btabb
     \entry{template<class T>\\
     void \textbf{quick\_sort\_op}(T *data, size\_t len);} {}

     \entry{template<class T>\\
     void \textbf{quick\_sort\_cmp}(T *data, size\_t len, 
     int (*cmp)(const T\&, const T\&));} {}

     \entry{template<class T>\\
     void \textbf{quick\_sort\_obj}(T *data, size\_t len, 
     CMPR *cmp);} {}
  \etabb

\subsection{Description}

These are internal memory in-place sorting routines that implement the
quicksort\index{quicksort} algorithm (randomized).  These routines are used by the external
memory sorting routines (see Section~\ref{sec:ref-ami-sort}) on streams
that are small enough to fit in memory.
The three polymorphs use different comparison methods:
\noiverb{quick\_sort\_op} uses the comparison operator $<$,
\noiverb{quick\_sort\_cmp} uses the comparison function \noiverb{cmp}, and
\noiverb{quick\_sort\_obj} uses a comparison object of type \noiverb{CMPR}.
\index{sorting!internal memory|)}

%%%%%%%%% Stacks %%%%%%%%%
\mysection{Stacks}
\plabel{sec:ref-ami-stack}
\index{AMI_stack@{\tt AMI\_stack}|(}
\index{stacks|(}

\subsection{Files}
  \btabb
        \entry{\#include <ami\_stack.h>} {}
  \etabb

\subsection{Class Declaration}
   \btabb
        \entry{template<class T> class \textbf{AMI\_stack};} {}
   \etabb

\subsection{Description}
External stacks are implemented through the templated class
\myverb{AMI\_stack<T>}, 
which is a subclass of \myverb{AMI\_STREAM<T>}. As a consequence, it
inherits all public members of \myverb{AMI\_STREAM<T>}, including its
constructors. See Section~\ref{sec:ref-ami-stream}.

\subsection{Public Member Functions}
   \btabb 

      \entry{AMI\_err \textbf{push}(const \&T t);} {Insert a copy of the
      object \myverb{t} to the top of the stack, increasing its length by
      one.}

       \entry{AMI\_err \textbf{pop}(T **ppt);} {Remove the top object from
       the stack, decreasing its length by one and returning the address of
       a pointer to the popped object in \myverb{ppt}.}

   \etabb

\index{stacks|)}
\index{AMI_stack@{\tt AMI\_stack}|)}

%%%%%%%%% Blocks %%%%%%%%%
\mysection{Blocks}
\index{AMI_block@{\tt AMI\_block}|(}

\subsection{Files}
  \btabb
        \entry{\#include <ami\_block.h>} {}
  \etabb

\subsection{Class Declaration}
   \btabb
        \entry{template<class E, class I> class \textbf{AMI\_block}; }{The types
        {\tt E} and {\tt I} should have a default constructor, a copy
        constructor and an assignment operator. The size returned by {\tt
        sizeof(E)} and {\tt sizeof(I)} should be the total size of the
        items copied by the copy constructor/assignment operator.}

   \etabb

\subsection{Description}

An instance of class {\tt AMI\_block<E,I>} is a typed
view of a logical block, which is the unit amount of data transfered
between external storage and main memory. 

The {\tt AMI\_block} class serves a dual purpose: (a) to provide an
interface for seamless transfer of blocks between disk and main memory,
and (b) to provide a structured access to the contents of the block.
The first purpose is achieved through internal mechanisms, transparent
to the user. When creating an instance of class {\tt AMI\_block}, the
constructor is responsible for making the contents of the block
avilable in main memory. When the object is deleted, the destructor is
responsible for writing back the data, if necessary, and freeing the
memory. Consequently, during the life of an {\tt AMI\_block} object, the
contents of the block is available in main memory.
The second purpose is achieved by partitioning the contents of the block
into three fields:
\begin{itemize}
\item[] Links: an array of pointers to other blocks, represented as
block identifiers, of type {\tt AMI\_bid};
\item[] Elements: an array of elements of parameter type {\tt E};
\item[] Info: an info field of parameter type {\tt I}, used to store a 
constant amount of administrative data;
\end{itemize}

The number of elements and links that can be stored is set during
construction: the number of links is passed to the constructor, and the
number of elements is computed using the following formula:
\[\mbox{\rm number\_of\_elements} = \left\lfloor\frac{\mbox{\rm
block\_size} - (\mbox{\tt sizeof}(I) + \mbox{\tt sizeof(AMI\_bid)} *
\mbox{\rm number\_of\_links})}{\mbox{\tt sizeof}(E)}\right\rfloor \]

\subsection{Constructors and Destructor}

   \btabb 

        \entry{\textbf{AMI\_block}(AMI\_COLLECTION *pcoll, unsigned int l, AMI\_bid
        bid);} {{\em Read the block} with id {\tt bid} from block collection
        {\tt *pcoll} in newly allocated memory and format it using the
        template types and the maximum number of links {\tt l}. Persistency
        is set to {\tt PERSIST\_PERSISTENT}.}

        \entry{\textbf{AMI\_block}(AMI\_COLLECTION *pcoll, unsigned int l);} {{\em
        Create a new block} in collection {\tt *pcoll}, allocate memory for
        it, and format it using the template types and the maximum number
        of links {\tt l}. Persistency is set to {\tt
        PERSIST\_PERSISTENT}. The id of the block can be inquired using the
        access member function {\tt bid()}.}

        \entry{\textbf{$\sim$AMI\_block}();} {Destructor. If persistency 
        is {\tt PERSIST\_DELETE}, remove the block from the collection. 
        If it is {\tt PERSIST\_PERSISTENT}, write the block to the 
        collection. Deallocate the memory.}

   \etabb

\subsection{Public Member Objects}

   \btabb

        \entry{b\_vector<E> \textbf{el};} {Access to the elements is done through
        this object, using the public methods of the {\tt b\_vector} class
        (described below).}

        \entry{b\_vector<AMI\_bid> \textbf{lk};} {Access to the links is done
        through this object, using the public methods of the {\tt
        b\_vector} class (described below).}

   \etabb

\subsection{Public Member Functions}

   \btabb
        
        \entry{AMI\_block<E,I>\& \textbf{operator=}(AMI\_block<E,I>\& B);} {Copy
        block {\tt B} into the current block, if both blocks are associated
        with the same collection. Returns a reference to this block.}

        \entry{bool \textbf{operator!}() const;} {Return {\tt true} if the block's status 
        is not {\tt AMI\_BLOCK\_STATUS\_VALID}. See also {\tt is\_valid()} and 
        {\tt status()}.}

        \entry{AMI\_bid \textbf{bid}() const;} {Return the block id.}

        \entry{size\_t \textbf{block\_size}() const;} {Return the size of this block
        in bytes.}

        \entry{char\& \textbf{dirty}();} {Return a reference to the dirty bit. The
        dirty bit is used to optimize writing in some implementations of
        the block collection class. It should be set to $1$ whenever the
        block data is modified. See the implementation details for more.}

        \entry{char \textbf{dirty}() const;} {Return the value of the dirty bit.}

        \entry{static size\_t \textbf{el\_capacity}(size\_t bsz, size\_t l);} 
        {Return the capacity of the {\tt el} vector of a block with size 
        {\tt bsz} and number of links {\tt l}.}

        \entry{I *\textbf{info}();} {Return a pointer to the info element.}
        
        \entry{const I* \textbf{info}() const;} {Return a const pointer to the 
        info element.}

        \entry{bool \textbf{is\_valid}() const;} {Return {\tt true} if the block's 
        status is {\tt AMI\_BLOCK\_STATUS\_VALID}. See also {\tt status()}.}

        \entry{void \textbf{persist}(persistence p);} {Set the persistency flag to
        {\tt p}. The possible values for {\tt p} are {\tt
        PERSIST\_PERSISTENT} and {\tt PERSIST\_DELETE}.}

        \entry{persistence \textbf{persist}() const;} {Return the value of the 
        persistency flag.} 

        \entry{AMI\_block\_status \textbf{status}() const;} {Return the status of the
        block. The result is either {\tt AMI\_BLOCK\_STATUS\_VALID} or {\tt
        AMI\_BLOCK\_STATUS\_INVALID}. The status of an {\tt AMI\_block}
        instance is set during construction. The methods of an invalid
        block can give erroneous results or fail.}

        \entry{AMI\_err \textbf{sync}();} {Synchronize the in-memory image of the
        block with the one stored in external storage.}

   \etabb

\subsection{The  b\_vector class}

The {\tt b\_vector} class stores an array of objects of a templated type
{\tt T}. It has a fixed maximum size, or capacity, which is set during
construction (since instances of this class are created only by the
{\tt AMI\_block} class, the constructors are not part of the public
interface). The items stored can be accessed through the array operator.

\subsubsection{Class Declaration}
   \btabb 
        \entry{template<class T> class \textbf{b\_vector};} {The type {\tt T}
        should have a default constructor, as well as copy constructor and
        assignment operator.}
   \etabb

%\subsection{Constructors and Destructor}
%   \begin{tabbing}
%   \hspace*{.3in} \= \hspace{.5in} \= \\ 
%
%      \> {\tt b\_vector(T* p, size\_t cap)}\\
%      \>\>\parbox[t]{5.5in}{Create a b\_vector instance using {\tt p}
%      as the underlying array, of capacity {\tt cap}.}
%
%   \end{tabbing}

\subsubsection{Member Functions}

   \btabb

      \entry{T\& \textbf{operator[]}(size\_t i);}{Return
      a reference to the $i$th item.}

      \entry{const T\& \textbf{operator[]}(size\_t i) const;}{Return a const reference to the $i$th      item.}

      \entry{size\_t \textbf{capacity}() const;}{Return the
      capacity (i.e., maximum number of {\tt T} elements) of this {\tt
      b\_vector}.}

      \entry{size\_t \textbf{copy}(size\_t start, size\_t length,
      b\_vector<T>\& src, size\_t src\_start = 0);}{Copy {\tt length} items from the {\tt src}
      vector, starting with item {\tt src\_start}, to this vector,
      starting with item {\tt start}. Return the number of items
      copied. Source can be {\tt *this}.}

      \entry{size\_t \textbf{copy}(size\_t start, size\_t length, const T* src);}{Copy {\tt length} items from the array {\tt
      src} to this vector, starting in position {\tt start}. Return the
      number of items copied.}

      \entry{void \textbf{insert}(const T\& t, size\_t pos);}{Insert item {\tt t} in position {\tt pos}; all
      items from position {\tt pos} onward are shifted one position higher;
      the last item is lost.}

      \entry{void \textbf{erase}(size\_t pos);}{Erase
      the item in position {\tt pos} and shift all items from position
      {\tt pos+1} onward one position lower; the last item becomes
      identical with the next to last item.}

   \etabb

\index{AMI_block@{\tt AMI\_block}|)}


%%%%%%%% AMI Block Collection %%%%%%%%%
\mysection{Block Collections}
\index{AMI_COLLECTION@{\tt AMI\_COLLECTION}|(}

\subsection{Files}
   \btabb
       \entry{\#include <ami\_coll.h>} {}
   \etabb

\subsection{Class declaration}

   \btabb
        \entry{class \textbf{AMI\_COLLECTION};} {}
   \etabb

\subsection{Description}

A block collection is a set of fixed size blocks. Each block inside the
collection is identified by a block ID, of type \myverb{AMI\_bid}.

\subsection{Constructors and Destructor}

   \btabb
        
        \entry{\textbf{AMI\_COLLECTION}(size\_t lbf = 1);} {Create a {\em new}
        collection with access type {\tt AMI\_WRITE\_COLLECTION} using
        temporary file names. The files are created in a directory given by
        the {\tt AMI\_SINGLE\_DEVICE} environment variable (or {\tt
        ``/var/tmp/''} if that variable is not set).  The {\tt lbf}
        (logical block factor) parameter determines the size of the blocks
        stored (the block size is {\tt lbf} times the operating system
        page size). The persistency of the collection is set to {\tt
        PERSIST\_DELETE}.}

        \entry{\textbf{AMI\_COLLECTION}(char *base\_file\_name,
        AMI\_collection\_type t = AMI\_READ\_WRITE\_COLLECTION, size\_t lbf
= 1);} {Create a new or open an 
        existing collection using {\tt base\_file\_name} to find the
        necessary files. The access type is set to {\tt t}. It has one of
        the following values:
        \begin{itemize} 
          \item[]{\tt AMI\_READ\_COLLECTION} Open an existing collection
          read-only;
          \item[]{\tt AMI\_WRITE\_COLLECTION} If the files specified by
          {\tt base\_file\_name} exist, open a collection using those files
          for reading and writing. If the files do not exist, create a new
          collection with read and write acces;
        \end{itemize}
        The {\tt lbf} (logical block factor) parameter determines the size
        of the blocks stored (the block size is {\tt lbf} times the
        operating system page size). The persistency of the collection is
        set to {\tt PERSIST\_PERSISTENT}.}

      \entry{\textbf{$\sim$AMI\_COLLECTION}();} {Destructor.
      Closes all files. If persistency is set to {\tt PERSIST\_DELETE}, it
      also removes the files. There should be no blocks in memory. If the
      destructor detects in-memory blocks, it issues a warning in the TPIE
      log file (if logging is turned on). The memory held by those blocks
      is lost to this program.}

   \etabb

\subsection{Member Functions}

   \btabb

        \entry{bool \textbf{operator!}() const;} {Return \noiverb{true} if
        the status of the collection is not
        \myverb{AMI\_COLLECTION\_STATUS\_VALID}, \noiverb{false}
        otherwise. See also \myverb{is\_valid()} and \myverb{status()}.}

        \entry{size\_t \textbf{block\_factor}() const;} {Return the logical
        block factor. The block size is obtained by multiplying the
        operating system page size by this value.}

        \entry{size\_t \textbf{block\_size}() const;} {Return the size of a block
        stored in this collection, in bytes (all blocks in a collection
        have the same size).}

        \entry{static const tpie\_stats\_collection\& \textbf{gstats}();}
        {Return an object containing the statistics of all collections
        opened by the application (global statistics). See also
        \myverb{stats()}.}

        \entry{bool \textbf{is\_valid}() const;} {Return \noiverb{true} if the status
        of the collection is \myverb{AMI\_COLLECTION\_STATUS\_VALID}, \noiverb{false}
        otherwise. See also \myverb{status()}.}

        \entry{void \textbf{persist}(persistence p);} {Set the persistency flag to
        {\tt p}. The possible values for {\tt p} are {\tt
        PERSIST\_PERSISTENT} and {\tt PERSIST\_DELETE}.}

        \entry{persistence \textbf{persist}() const;} {Return the value of the 
        persistency flag.}

        \entry{size\_t \textbf{size}() const;} {Return the number of blocks in the
        collection.}

        \entry{const tpie\_stats\_collection\& \textbf{stats}() const;}
        {Return an object containing the statistics of this collection. The
        types of statistics computed for a collection are tabulated
        below. See also \myverb{gstats()}.\\[1mm] \begin{tabular}{|l|l|} \hline \myverb{BLOCK\_GET} & Number
        of block reads\\ \myverb{BLOCK\_PUT} & Number of block writes \\
        \myverb{BLOCK\_NEW} & Number of block creates\\
        \myverb{BLOCK\_DELETE} & Number of block deletes\\
        \myverb{BLOCK\_SYNC} & Number of block sync operations\\
        \myverb{COLLECTION\_OPEN} & Number of collection open operations\\
        \myverb{COLLECTION\_CLOSE} & Number of collection close
        operations\\ \myverb{COLLECTION\_CREATE} & Number of collection
        create operations\\ \myverb{COLLECTION\_DELETE} & Number of
        collection delete operations\\ \hline \end{tabular} }

        \entry{AMI\_collection\_status \textbf{status}() const;} {Return
        the status of the collection. The result is either
        \myverb{AMI\_COLLECTION\_STATUS\_VALID} or
        \myverb{AMI\_COLLECTION\_STATUS\_INVALID}. The only operation that
        can leave the collection invalid is the constructor (if that
        happens, the log file contains more information). No blocks should
        be read from or written to an invalid collection.}

        \entry{void *\textbf{user\_data}();} {Return a pointer to a 512-byte array
        stored in the header of the collection. This can be used by the
        application to store initialization information (e.g., the id of
        the block containing the root of a B-tree).}

   \etabb
\index{AMI_COLLECTION@{\tt AMI\_COLLECTION}|)}


%%%%%%%%%% AMI B+-tree %%%%%%%%%%%
\mysection{B+-tree}
\index{AMI_btree@{\tt AMI\_btree}|(}

\subsection{Files}
\btabb
   \entry{\#include <ami\_btree.h>} {}
\etabb

\subsection{Class Declaration}

\btabb
   \entry{template<class Key, class Value, class Compare, class
   KeyOfValue>\\ class \textbf{AMI\_btree};} {}
\etabb

\subsection{Description}

The {\tt AMI\_btree<Key, Value, Compare, KeyOfValue>} class implements the
behavior of a dynamic B+-tree or $(a,b)$-tree storing fixed-size data
items. All data elements (of type {\tt Value}) are stored in the leaves of
the tree, with internal nodes containing keys (of type {\tt Key}) and links
to other nodes. The keys are ordered using the {\tt Compare} function
object, which should define a strict weak ordering (as in the STL sorting
algorithms). Keys are extracted from the {\tt Value} data elements using
the {\tt KeyOfValue} function object.

\subsection{Constructors and Destructor}

\btabb

   \entry{\textbf{AMI\_btree}(const AMI\_btree\_params \&params = btree\_params\_default);}
   {Construct an empty AMI\_btree using temporary files. The tree is stored in a
   directory given by the {\tt AMI\_SINGLE\_DEVICE} environment variable (or {\tt
   "/var/tmp/"} if that variable is not set). The persistency flag is set to
   {\tt PERSIST\_DELETE}. The {\tt params} object contains the
   user-definable parameters (see Appendix for an explanation of the {\tt
   AMI\_btree\_params} class and the default values).}

   \entry{\textbf{AMI\_btree}(const char *bfn, BTE\_collection\_type t =
   BTE\_WRITE\_COLLECTION, const AMI\_btree\_params \&params =
   btree\_params\_default);}{Construct a B-tree
   using the files given by {\tt bfn} (base file name).  The files
   created/used by a Btree instance are outlined in the following
   table.\\[1mm] \begin{tabular}{|l|l|} \hline {\em bfn}{\tt .l.blk} &
   Contains the leaves block collection.\\ \hline {\em bfn}{\tt .l.stk} &
   Contains the free blocks stack for the leaves block collection.\\ \hline
   {\em bfn}{\tt .n.blk} & Contains the nodes block collection.\\ \hline
   {\em bfn}{\tt .n.stk} & Contains the free blocks stack for the nodes
   block collection.\\ \hline \end{tabular}\\[2mm] The persistency flag is
   set to {\tt PERSIST\_PERSISTENT}. The {\tt params} object contains the
   user-definable parameters (see Appendix for an explanation of the {\tt
   AMI\_btree\_params} class and the default values).}

   \entry{\textbf{$\sim$AMI\_btree}();} {Destructor. Either remove or close the supporting
   files, depending on the persistency flag (see method {\tt persist()}).}

\etabb

\subsection{Member functions}

\btabb

   \entry{bool \textbf{erase}(const Key\& k);} {Delete the element with key {\tt k}
   from the tree. Return true if succeded, false otherwise (key not
   found).}

   \entry{bool \textbf{find}(const Key\& k, Value\& v);} {Find an element based on
   its key. If found, store it in {\tt v} and return true.}

   \entry{size\_t \textbf{height}() const;} {Return the height of the tree, including
   the leaf level. A value of $0$ represents an empty tree.}

   \entry{bool \textbf{insert}(const Value\& v);} {Insert an element {\tt v} into the
   tree. Return true if the insertion succeded, false otherwise (duplicate
   key).}

   \entry{bool \textbf{is\_valid}() const;} {Return \noiverb{true} if the status
   of the tree is \myverb{AMI\_BTREE\_STATUS\_VALID}, \noiverb{false}
   otherwise. See also \myverb{status()}.}

   \entry{AMI\_err \textbf{load}(AMI\_STREAM<Value>* is, float lf = 0.7,
   float nf = 0.5)} {Bulk load from the stream {\tt is} of elements. Leaves
   are filled to {\tt lf}$\times$capacity, and nodes are filled to {\tt
   nf}$\times$capacity.}

   \entry{AMI\_err \textbf{load}(AMI\_btree<Key, Value, Compare,
   KeyOfValue>* bt, float leaf\_fill = .7, float node\_fill =
   .5);}{Bulk load from another B-tree. This is a means of reoganizing a
   B-tree after a lot of updates. A newly loaded structure may use less
   space and may answer range queries faster.}

   \entry{AMI\_err \textbf{load\_sorted}(AMI\_STREAM<Value>* is, float lf =
   0.7, float nf = 0.5);} {Same as {\tt load()} above, but bypasses the
   expensive sorting step, by assuming that the stream {\tt is} is sorted.}

   \entry{const AMI\_btree\_params\& \textbf{params}() const;} {Return a const
   reference to the {\tt AMI\_btree\_params} object used by the B-tree. This
   object contains the true values of all parameters (unlike the object
   passed to the constructor, which may contain $0$-valued parameters to
   indicate default behavior; see Section~\ref{ssec:params} below).}

   \entry{void \textbf{persist}(persistence p);} {Set the persistency flag to {\tt
   p}. The persistency flag dictates the behavior of the destructor of
   this AMI\_btree object. If {\tt p} is {\tt PERSIST\_DELETE}, all files
   associated with the tree will be removed, and all the elements stored in
   the tree will be lost after the destruction of this AMI\_btree object. If
   {\tt p} is {\tt PERSIST\_PERSISTENT}, all files associated with the tree
   will be closed during the destruction of this AMI\_btree object, and all the
   information needed to reopen this tree will be saved.}

   \entry{bool \textbf{pred}(const Key\& k, Value\& v);} {Find the highest
   element stored in the tree whose key is lower than \noiverb{k}. If such
   an element exists, return \noiverb{true} and store the result in
   \noiverb{v}. Otherwise, return \noiverb{false}.}

   \entry{void \textbf{range\_query}(const Key\& k1, const Key\& k2,
   AMI\_STREAM<Value>* os);} {Find all elements within the range given by
   keys {\tt k1} and {\tt k2} and write them to stream {\tt os}.}

   \entry{size\_t \textbf{size}() const;} {Return the number of elements
   stored in the leaves of this tree.}

   \entry{AMI\_err \textbf{sort}(AMI\_STREAM<Value>* is,
   AMI\_STREAM<Value>* \&os);} {As a convenience, this function sorts the
   stream {\tt is} and stores the result in {\tt os}. If the value of {\tt
   os} passed to the function is {\tt NULL}, a new stream is created and
   {\tt os} points to it.}

   \entry{AMI\_btree\_status \textbf{status}() const;} {Return the status
   of the collection. The result is either
   \myverb{AMI\_BTREE\_STATUS\_VALID} or
   \myverb{AMI\_BTREE\_STATUS\_INVALID}. The only operation that can leave
   the tree invalid is the constructor (if that happens, the log file
   contains more information).}

   \entry{bool \textbf{succ}(const Key\& k, Value\& v);} {Find the lowest
   element stored in the tree whose key is higher than \noiverb{k}. If such
   an element exists, return \noiverb{true} and store the result in
   \noiverb{v}. Otherwise, return \noiverb{false}.}

   \entry{AMI\_err \textbf{unload}(AMI\_STREAM<Value>* s);} {Write all
   elements stored in this tree to the given stream, in sorted order. No
   changes are performed on the tree.}

%   \> {\tt bool defragment()}\\ \>\>\parbox[t]{5.5in}{Rearrange the nodes
%   and leaves of the tree so that they take the minimum disk space. This is
%   a very time-consuming operation. Return true if completed
%   successfully.}\\[3mm]
\etabb

\subsection{The {\tt AMI\_btree\_params} Class}\label{ssec:params}
\index{AMI_btree_params@{\tt AMI\_btree\_params}|(}
The {\tt AMI\_btree\_params} class encapsulates all user-definable B-tree
parameters. These parameters dictate the layout of the tree and its
behavior under insertions and deletions. An instance of the class created
using the default constructor gives default values to all parameters. Each
paramter can then be changed independently.

\subsubsection{Class Declaration}

\btabb

  \entry{class \textbf{AMI\_btree\_params};} {}

\etabb

\subsubsection{Constructor}

\btabb

  \entry{\textbf{AMI\_btree\_params}()} {Initialize a {\tt Btree\_params} object with
  default values. The default values are given in the following table.\\[1mm]
  \begin{tabular}{|l|c|}
    \hline
    {\em Parameter} & {\em Value} \\ \hline
    {\tt leaf\_size\_min} & 0 \\ \hline
    {\tt node\_size\_min} & 0 \\ \hline
    {\tt leaf\_size\_max} & 0 \\ \hline
    {\tt node\_size\_max} & 0 \\ \hline
    {\tt leaf\_block\_factor} & 1 \\ \hline
    {\tt node\_block\_factor} & 1 \\ \hline
    {\tt leaf\_cache\_size} & 5 \\ \hline
    {\tt node\_cache\_size} & 10 \\ \hline
  \end{tabular}
  }

\etabb

\subsubsection{Public Member Objects}

\btabb

  \entry{size\_t \textbf{leaf\_size\_min}} {Minimum number of elements in a leaf. A
  value of $0$ tells the class to use the default B+-tree behavior. This
  parameter is a guideline. To improve performance, some leaves may have
  fewer elements.}

  \entry{size\_t \textbf{node\_size\_min}} {Minimum number of keys in an internal
  node. A value of $0$ tells the class to use the default B+-tree
  behavior. As above, this parameter is a guideline.}

  \entry{size\_t \textbf{leaf\_size\_max}} {Maximum number of elements in a leaf. A
  value of $0$ tells the class to fill a leaf to capacity. This value is
  strictly enforced.}

  \entry{size\_t \textbf{node\_size\_max}} {Maximum number of keys in an internal
  node. A value of $0$ tells the class to fill a node to capacity. This
  value is strictly enforced.}

  \entry{size\_t \textbf{leaf\_block\_factor}} {The size (in bytes) of a leaf block
  is {\tt leaf\_block\_factor$\times$ os\_block\_size}, where {\tt
  os\_block\_size} is the operating-system specific page size.}

  \entry{size\_t \textbf{node\_block\_factor}} {The size (in bytes) of an internal
  node block is {\tt node\_block\_factor$\times$ os\_block\_size}.}

  \entry{size\_t \textbf{leaf\_cache\_size}} {The size (in number of leaf blocks) of
  the leaf block cache. The cache implements an LRU replacement policy.}

  \entry{size\_t \textbf{node\_cache\_size}} {The size (in number of node blocks) of
  the node block cache. The cache implements an LRU replacement policy.}

\etabb
\index{AMI_btree_params@{\tt AMI\_btree\_params}|)}
\index{AMI_btree@{\tt AMI\_btree}|)}

%%%%%%%% AMI Cache Manager %%%%%%%%%%
\mysection{Cache Manager}

\subsection{Files}
   \btabb
      \entry{\#include <ami\_cache.h>} {}
   \etabb

\subsection{Class Declaration}
   \btabb
      \entry{template<class T, class W> class \textbf{AMI\_CACHE\_MANAGER};} {}
   \etabb

\subsection{Description}

\subsection{Constructors and Destructor}
   \btabb
      \entry{\textbf{AMI\_CACHE\_MANAGER}(size\_t capacity);}{Construct a fully-associative cache manager with the given capacity.}
      \entry{\textbf{AMI\_CACHE\_MANAGER}(size\_t capacity, size\_t assoc);}{Construct a cache manager with the given capacity and associativity.}
      \entry{\textbf{$\sim$AMI\_CACHE\_MANAGER}();}{Destructor. Write out all items still in the cache.}
   \etabb

\subsection{Member Functions}
   \btabb
       \entry{bool \textbf{read}(size\_t k, T \& item);}{Read an item from the cache based on key {\tt k} and store it in {\tt item}. If found, the item is removed from the cache. Return true if the key was found.}
      \entry{bool \textbf{write}(size\_t k, const T \& item);}{Write an item in the cache based on the given key {\tt k}. If the cache was full, the least recently used item is writen out using the {\tt W} function object, and it is removed from the cache.}
      \entry{bool \textbf{erase}(size\_t k);}{Erase an item from the cache based on the given key {\tt k}. Return true if the key was found.}
   \etabb
 % Chapter: AMI Programmer's Reference
  %%
%% $Id: implementation.tex,v 1.2 1999-06-12 15:49:48 rbarve Exp $
%%
\chapter{The Implementation of TPIE}

This chapter discusses the implementation of TPIE.  It is primarily
targeted at those who might wish to port TPIE to additional platforms
or implement similar systems.  

\section{The Structure of TPIE}
\index{structure!of TPIE}
\index{components!of TPIE}

TPIE has three main components, the Access Method Interface
(AMI)\index{access method interface}, the Block Transfer Engine
(BTE)\index{block transfer engine}, and the Memory Manager
(MM)\index{memory manager}.  The BTE handles block transfer for a single
processor.  The MM performs low level memory management across all the
processors in the system.  The AMI works on top of the MM and one or more
BTEs, each running on a single processor, to provide a uniform interface
for application programs.  Applications that use this interface are
portable across hardware platforms, since they never have to deal with the
underlying details of how I/O is performed on a particular machine.

The BTE is intended to bridge the gap between the I/O hardware and the
rest of our system.  It works alongside the traditional buffer
cache\index{buffer cache} in a UNIX system.  Unlike the buffer
cache\index{buffer cache}, which must support concurrent access to
files from multiple address spaces, the BTE is specifically designed
to support high throughput processing of data from secondary memory
through a single user level address space.  In order to efficiently
support the merging, distribution, and scanning paradigms, the BTE
provides stream oriented buffer replacement policies.  To further
improve performance, some implementations of the BTE move data from
disk directly into user space rather than using a kernel level buffer
cache\index{buffer cache}.  This saves both main memory space and
copying time.  Although the BTE runs on a single processor, it can
support concurrent access to multiple disks\index{parallel disks},
allocating and managing buffer space for all of them concurrently.

The MM\index{memory manager} manages random access memory on behalf of
TPIE.  It is the most architecture-dependent component of the system.
On a single processor or multiprocessor system with a single global
address space, the MM is relatively simple; its task is to allocate
and manage the physical memory used by the BTE.  On a distributed
memory system, the MM has the additional task of coordinating
communication between processors and memory modules in order to
support the primitives that the AMI provides.

The AMI\index{access method interface} is a layer between the BTE and
user level processes.  It implements fundamental access methods, such
as scanning, permutation routing, merging, distribution, and batch
filtering. It also provides a consistent, object-oriented interface to
application programs.  The details of how these access methods are
implemented depends on the hardware on which the system is running.
For example, recursive distribution will be done somewhat differently
on a parallel disk machine than on a single disk machine.  The AMI
abstracts this fact away, allowing an application program that calls a
function such as
\verb|AMI_partition_and_merge()|\index{AMI_partition_and_merge@{\tt 
AMI\_partition\_and\_merge}} to work correctly
regardless of the underlying I/O system.

The key to keeping the AMI simple and flexible is the fact that its
user accessible functions serve more as templates for computation than
as actual problem solving functions.  The details of how a computation
proceeds within the template is up to the application programmer, who
is responsible for providing the functions that the template applies
to data.

\section{The Access Method Interface (AMI)}
\index{access method interface|(}
\index{AMI|see{access method interface}}
\label{sec:ref-ami}
To use TPIE, the programmer has to create, modify or delete items each of
which belongs to a stream. From the programmer's perspective,
computation simply requires her to \emph{choreograph the movement of
items in various streams}: The I/O operations, the buffering required
and memory management are performed in TPIE itself. In order to specify
the choreography of streams, the user can make heavy use of inbuilt
TPIE templates and functions; it is not often that the user has to write
special-purpose custom-made TPIE functions. The reason why TPIE's unique
approach is attractive for external memory computation is the well
known fact that most external memory algorithms consist of sewing
together a small number of paradigms such as \emph{scanning},
\emph{merging}, \emph{distributing}, in a well-coordinated manner
exploiting all available memory and I/O bandwidth. A large variety of 
indexing data structures and methods related to those data structures 
can be implemented using the above paradigms in conjunction with a 
basic tree template: While indexing data structures have been
implemented using TPIE streams, work on tree templates and related
methods to further ease implementation of indexing data structures is
currently ongoing.

Each datum in TPIE is an item of a certain type.
The programmer can access data by using the
\emph{Access Method Interface (AMI)} to access items in 
\verb|AMI_STREAM|s. Items of type T constitute an AMI\_STREAM of 
type T. Each \verb|AMI_STREAM| is a wrapper around  what is known
as a \verb|BTE_STREAM|, where BTE stands for \emph{block transfer
engine.} Each \verb|BTE_STREAM| is a stream built on top of a Unix file, with 
additional features such as typing,  and automatic, efficient,
``under-the-hoods'' I/O and buffering so that the programmer does
not bear this burden. TPIE can be configured to use one of three
different \verb|BTE_STREAM| implementations: The implementations vary
fundamentally in the way they do buffering and I/O. The implementation
of choice depends, to some extent, on the operating system on which 
TPIE runs. Later we describe the different BTE implementations and
how to determine the best implementation on any given system; for now,
we focus on the \verb|AMI_STREAM| interface since that is most directly
relevant to a programmer/user.


Note that the string \verb|AMI_STREAM| has
 been \verb|#define|d to be the string \verb|AMI_stream_single| in the
 header file \verb|ami_imps.h| to facilitate alternative implementations
of \verb|AMI_STREAM| if necessary in the future.
The member functions of the \verb|AMI_STREAM| class,
templated on the type T of its items,  are defined in the file \verb|ami_single.h|;
some important related enumerated types that the user will be
 concerned with have been defined in \verb|ami_base.h|.


An important consequence of the construction of an \verb|AMI_STREAM| is a
reduction in the total amount of memory that the user has at her
disposal since a certain amount of memory is consumed now by buffers
dedicated to that \verb|AMI_STREAM| and its related data
structures.\footnote{The total amount of memory that the user has
available, divided by the memory consumed by every \verb|AMI_STREAM| thus
is an upper bound on the maximum number of \verb|AMI_STREAM|s that the user
can have.}
 
In order to effectively utilize memory, the user needs to keep track
of the amount of memory she has at her disposal: Sometimes, 
when one of several algorithms may be usable to solve a given problem,
it may be useful to choose the algorithm depending on the problem
size and available memory size. The total amount of memory consumed
by an \verb|AMI_STREAM| often plays an important role while making such 
a choice. The memory manager \verb|mm.h| is responsible for keeping
track of the amount of memory avaoable to the user at any time.

When an \verb|AMI_STREAM| is destroyed using a destructor, that
amount of memory becomes is added to the available memory for the
user. 


Sometimes the
items in an \verb|AMI_STREAM| will never be used  once the \verb|AMI_STREAM| is
destructed whereas sometimes an ``idle'' \verb|AMI_STREAM| may be destroyed 
to prevent it from hogging memory only to be constructed later on
when really required. In the former case, it is desirable to 
destroy or delete the Unix file backing the \verb|AMI_STREAM| to save
disk space whereas in the latter case we need the Unix file backing
the \verb|AMI_STREAM| to exist on disk even when the \verb|AMI_STREAM| is
destroyed. The  \verb|AMI_STREAM| destructor looks at the persistence 
flag to decide whether it need only close the backing Unix file or
delete it from disk.



\index{access method interface|)}

\section{AMI Entry Points}
\label{sec:ref-entry}

\subsection{Scanning}
\label{sec:ref-imp-ami-scan}

\index{scanning|(}
\index{scanning|)}

\subsection{Merging}
\label{sec:ref-imp-ami-merge}
\index{merging|(}

\subsubsection{Merge Management Objects}
Merge management objects based on the super class \verb|AMI_merge_base| 
defined in the file \verb|ami_merge.h| can be used conveniently by
appropriately defining the member functions of the class.

\subsubsection{External Merging using Merge Management Objects}
The external merging routine \verb|AMI_single_merge()|
and related functions (see Section~\ref{sec:ref-ami-merge}) 
that are based on merge management objects are in the \verb|ami_merge.h| file.

The implementation of  \verb|AMI_single_merge()| is simple: The merge management
object's \verb|operate()| function
is repeatedly used to output the smallest (or largest, depending on the desired
merge order) item of the set of items consisting of the leading item of each input 
stream. A priority queue implementation, in the file \verb|pqueue_heap.h|, is
used by the merge management object for this purpose. The output of the  \verb|operate()| 
function function is repeatedly appended to the output stream. Since the \verb|AMI_STREAM| interface ensures that the current block of each stream is buffered in memory, it results 
in an efficient merging implementation.

The implementation of  \verb|AMI_partition_and_merge()| first involves 
a ``run-formation'' stage in which a memory-load of the input stream is 
read into memory, sorted in memory and then written as a substream into
an intermediate stream. The merge management object uses an appropriate 
quicksort implementation from the file \veb|quicksort.h| to do the internal
memory sorting. Thereafter, the \verb|AMI_partition_and_merge()| repeatedly
merges together the maximum possible number of sub-streams at a time using the 
\verb|AMI_single_merge()| function. An important point is that all the
substreams input to \verb|AMI_single_merge()| during the execution of 
\verb|AMI_partition_and_merge()| all originate from the same underlying
parent \verb|AMI_STREAM|; thus filesystem accesses are inherently non-sequential.
Throughout the execution of \verb|AMI_partition_and_merge()|, there are only
two active \verb|AMI_STREAM|s at any time: One which stores the substreams
being merged at that stage and one which stores the substreams output by
that stage.

\subsubsection{External Merging without Merge Management Objects}
The entry points \verb|MIAMI_single_merge_*()| described in Section~\ref{sec:ref-ami-merge} 
do not use merge management objects. Instead  these merge functions, in file
\verb|optimized_merge.h|,  directly use appropriate simple heap implementations 
defined in file \verb|mergeheap.h|. The heap implementations are somewhat less general
than the priority queue implementations in \verb|pqueue_heap.h|. The 
\verb|MIAMI_single_merge_Key()| function ensures that the heap is used to
store and process the key values as opposed to using the entire items. When
the item size is large compared to key size, performance improvement on account
of this technique is often tangible.

The three \verb|AMI_partition_and_merge_*()| implementations (See Section~\ref{sec:ref-ami-merge}, one corresponding to each one of the \verb|MIAMI_single_merge_*()| functions) are
fundamentally  different from the \verb|AMI_partition_and_merge()| function in the way they store
their  sub-streams. As before, the number, say $R$,  of streams being merged together 
at any time is the maximum possible for that amount of memory. In contrast to the 
\verb|AMI_partition_and_merge()| function which has two \verb|AMI_STREAM|s active at
any time, the \verb|AMI_partition_and_merge_*()| implementations have $R$  \verb|AMI_STREAM|s 
active at any time: While the $R$ sub-streams
sent for merging by the \verb|AMI_partition_and_merge()| function at any time 
all have the same parent stream, the $R$ sub-streams sent for merging  by the 
 \verb|AMI_partition_and_merge_*()| implementations all reside in different
parent streams. Thus, each stream involved is accessed in a sequential manner.
The in-memory sorting routines used during run-formation are once more the 
quicksort routines in \verb|quicksort.h|. During run-formation, for each memoryload, 
the \verb|AMI_partition_and_merge_Key()| routine 
first extricates the keys from the items in a separate table and then sorts them 
in memory and then permutes in memory the items based on the sorted order.

\index{merging|)}

\subsection{Comparison Sorting}
\label{sec:ref-imp-ami-sort}

\index{sorting!comparison|(}
The comparison sorts \verb|AMI_sort()| are defined in the file \verb|ami_sort_single.h|.
They are implemented using the \verb|AMI_partition_ane_merge()| routine of 
\verb|ami_merge.h| and the corresponding merge management objects.

The comparison sorts \verb|AMI_partition_and_merge_stream()|, \verb|AMI_partition_and_merge_Key()| and  \verb|AMI_partition_and_merge_stream_cmp()| are defined in the file
\verb|optimized_merge.h|, and are implemented as described in the previous 
Section.


\index{sorting!comparison|)}

\subsection{Key Bucket Sorting}
\label{sec:ref-imp-ami-kb-sort}

\index{sorting!key bucket|(}
\index{sorting!key bucket|)}

\subsection{General Permuting}
\label{sec:ref-imp-ami-gp}

\index{permutation!general|(}
\index{permutation!general|)}

\subsection{Bit Permuting}
\label{sec:ref-imp-ami-bp}

\index{permutation!bit|(}
\index{permutation!bit|)}

\subsection{Dense Matrices}
\label{sec:ref-imp-ami-matrix}

\index{matrices!dense|(}
\index{matrices!dense|)}

\subsection{Sparse Matrices}
\label{sec:ref-imp-ami-sm}

\index{matrices!sparse|(}
\index{matrices!sparse|)}

\subsection{Stacks}
\label{sec:ref-imp-ami-stack}

\index{stacks|(}
\index{stacks|)}

\subsection{Elementwise Arithmetic}
\label{sec:ref-imp-ami-arith}

\index{elementwise arithmetic|(}
\index{elementwise arithmetic|)}

\section{The Block Transfer Engine (BTE)}
\label{sec:ref-bte}
\index{block transfer engine|(}
\index{BTE|see{block transfer engine}}

The BTE is lowest layer of TPIE.  It is the layer that is ultimately
responsible for moving blocks of data from physical disk devices to main
memory and back.  We hope that in most cases it will be possible for the
BTE to work with device drivers provided by the machine vendor's operating
system.  In some cases, however, new drivers may have to be written.  The
BTE is also responsible for maintaining the integrity of streams striped
across multiple disks attached to a single CPU, which it will do as
described in \cite{vitter:parmem1}.  The BTE is not, however, responsible
for coordinating the actions of multiple CPU's and the disks attached to
them.  A separate instance of the BTE will run on each such CPU, and their
actions will be coordinated by a single multi-threaded MM running at a
higher level.  The reason for the functional split between the two levels
is that it will likely be advantageous to be able to use a single BTE
written for a specific piece of hardware with more than one MM, for
example, one MM written for a homogeneous environment and one for a
heterogeneous environment.

The user should set one of the flags in \verb|app_config.h| corresponding
to the BTE that he wants to use for the application.  Multiple
implementations are allowed to coexist, with some restrictions. Version
\version of TPIE is distributed with three BTE implementations and the
flags used to select them are as follows:

An implementation based on the UNIX {\tt stdio} library \index{stdio (UNIX
  library)@{\tt stdio} (UNIX library)} is selected by setting {\tt
  BTE\_IMP\_STDIO}.  \index{implementation!BTE!{\tt stdio} library}
 
An implementation based on blocked UNIX {\tt read/write} calls \index{read
  write} is selected by setting {\tt BTE\_IMP\_UFS}.
\index{implementation!BTE!read!write}

An implementation based on memory mapped I/O\index{memory mapped I/O} is
selected by setting {\tt BTE\_IMP\_MMB}.  \index{implementation!BTE!memory
  mapped I/O}

If none of the above flags is specified {\tt BTE\_IMP\_STDIO} is used by
default and a warning is generated.

Implementations of BTE streams are written as subclasses of the class
\verb|BTE_base_stream|, which contains the following abstract public methods:


\subsubsection{new\_substream}
\begin{verbatim}
    BTE_err new_substream(BTE_stream_type st,
                          off_t sub_begin, off_t sub_end,
                          BTE_base_stream<T> **sub_stream);
\end{verbatim}
A virtual psuedo-constructor for substreams. The arguments \verb|sub_begin| and
\verb|sub_end| are item offsets.


\subsubsection{read\_item}
\begin{verbatim}
    BTE_err read_item(T **elt);
\end{verbatim}
Read the next item from the stream. If block boundaries are crossed the
read ahead mechanism is called, but this is invisible to the user.

\subsubsection{write\_item}
\begin{verbatim}
    BTE_err write_item(const T &elt);
\end{verbatim}
Write the item to stream.


\subsubsection{seek}
\begin{verbatim}
    BTE_err seek(off_t offset);
\end{verbatim}
Seek to the item offset in the stream.


\subsubsection{truncate}
\begin{verbatim}
    BTE_err truncate(off_t offset);
\end{verbatim}
Truncate/extend the stream to the specified number of items. The file
pointer will be moved to the end of the stream.


\subsubsection{main\_memory\_usage}
\begin{verbatim}
    BTE_err main_memory_usage(size_t *usage,
                              MM_stream_usage usage_type);
\end{verbatim}
Query memory usage.



\subsubsection{get\_status}
\begin{verbatim}
    BTE_stream_status get_status(void);
\end{verbatim}
Returns the status of the stream as one
of the following:
\begin{itemize}
\item \verb|BTE_STREAM_STATUS_NO_STATUS|
\item \verb|BTE_STREAM_STATUS_INVALID|
\item \verb|BTE_STREAM_STATUS_EOS_ON_NEXT_CALL|
\item \verb|BTE_STREAM_STATUS_END_OF_STREAM|
\end{itemize}


\subsubsection{stream\_len}
\begin{verbatim}
    off_t stream_len(void);
\end{verbatim}
Returns the current number of items in the stream.


\subsubsection{name}
\begin{verbatim}
    BTE_err name(char **stream_name);
\end{verbatim}
Returns the path name of the file backing the stream. The name will be
stored in newly allocated space.


\subsubsection{read\_only}
\begin{verbatim}
    int read_only(void);
\end{verbatim}
Returns true if the stream is read\_only.

    
\subsubsection{available\_streams}
\begin{verbatim}
    int available_streams(void);    
\end{verbatim}
Returns the number of currently available streams.

\subsubsection{chunk\_size}
\begin{verbatim}
    off_t chunk_size(void);
\end{verbatim}
Not clear what this does...


\subsubsection{persistence}
\begin{verbatim}
    void persist(persistence);
\end{verbatim}
Set the persistence of the stream to one of the following:
\begin{itemize}
\item \verb|PERSIST_DELETE:| Delete the stream from the disk when it is
  destructed.
\item \verb|PERSIST_PERSISTENT:| Do not delete the stream from the disk when
  it is destructed.
\item \verb|PERSIST_READ_ONCE:| Delete each block of data from the disk as
  it is read.
\end{itemize}

By default, all streams are deleted at destruction time
(\verb|PERSIST_DELETE|).  
\\ \\
All BTE stream implementations inherit from class \verb|BTE_base_stream|
and must support all the abstract member functions as declared above.




\subsection{BTE\_stdio}

\verb|BTE_stdio| streams are streams in a special format that are designed
to be stored as ordinary files in a UNIX file system.  The read/write
primitives of \verb|BTE_stdio| streams are implemened using system calls
\verb|fread| and \verb|fwrite|. The underlying operating system blocking
and prefetching assure that stream accesses are done in blocks and
prefetching is therefore automatic and invisible to the TPIE developer.

\verb|BTE_stdio| streams are stored as ordinary UNIX files with a header
with the following structure:

\begin{verbatim}
typedef struct BTE_stdio_header_v1 { 
    unsigned int magic_number;  // Set to BTE_STDIO_HEADER_MAGIC_NUMBER
    unsigned int version;       // Should be 1 for current version.
    unsigned int length;        // # of bytes in this structure.
    unsigned int block_length;  // # of bytes in a block.
    size_t item_size;           // The size of each item in the stream.
} BTE_stdio_header;
\end{verbatim}

\verb|BTE_stdio| class inherits from \verb|BTE_base_stream|:
\begin{verbatim}
class BTE_stream_stdio : public BTE_base_stream {
  private:
     FILE  *file;          
     BTE_stdio_header      header;
     ...
}  
\end{verbatim}

\verb|BTE_stdio| defines the abstract methods inherited from
\verb|BTE_base_stream| presented in the previous section. In addition to
these it defines its own constructors:
\begin{verbatim}
     BTE_stream_stdio(const char *dev_path, const BTE_stream_type st); 
     BTE_stream_stdio(const BTE_stream_type st); 
     BTE_stream_stdio(const BTE_stream_stdio<T> &s);
\end{verbatim}

For implementation details please consult the code in
\verb|/include/bte_stdio.h|.



\subsection{BTE\_mmb}

Just like \verb|BTE_stdio| streams presented in previous section,
\verb|BTE_mmb| streams are streams in a special format that are designed to
be stored as ordinary files in a UNIX file system. What distinguishes them
from the \verb|BTE_stdio| streams is the way stream input/output is
implemented. The \verb|BTE_mmb| uses the memory map paradigm which allows
the user to memory map (\verb|mmap|) blocks of a file and work on them as
if the file were in memory. The \verb|BTE_mmb| primitives maintain the
currently accessed block of the file \verb|mmaped| in memory. When a file
offset ouside current block boundaries is requested, the current block is
unmapped and a new one is mapped from the source file.

The \verb|BTE_mmb| header structure is very similar to the \verb|BTE_stdio|
one:
\begin{verbatim}
struct mmap_stream_header { 
  public:
    unsigned int magic_number;  // Set to MMB_HEADER_MAGIC_NUMBER
    unsigned int version;       // Should be 1 for current version.
    unsigned int length;        // # of bytes in this structure.
    off_t item_logical_eof;     // The number of items in the stream.
    size_t item_size;           // The size of each item in the stream.
    size_t block_size;          // The size of a physical block on the device
                                // where this stream resides.
    unsigned int items_per_block;
};
\end{verbatim}

\verb|BTE_mmb| class inherits from \verb|BTE_base_stream|:
\begin{verbatim}
class BTE_stream_mmb : public BTE_base_stream {
  private:
     // descriptor of the mapped file.  
     int fd;   
     // A pointer to the mapped in header block for the stream. 
     mmap_stream_header *header;
     ...
}  
\end{verbatim}

\verb|BTE_mmb| defines the abstract methods inherited from
\verb|BTE_base_stream| presented in a previous section. In addition to
these it defines its own constructors:
\begin{verbatim}
  BTE_stream_mmb(const char *dev_path, BTE_stream_type st); 
  BTE_stream_mmb(BTE_stream_type st); 
  BTE_stream_mmb(BTE_stream_mmb<T> &s); 
  
  // A substream constructor.
  BTE_stream_mmb(BTE_stream_mmb *super_stream,
                 BTE_stream_type st,
                 off_t sub_begin, off_t sub_end);
\end{verbatim}

For implementaion details please consult the code in
\verb|/include/bte_mmb.h|.

While with \verb|BTE_stdio| prefetching is implicitely done by the
operating system, \verb|BTE_mmb| has to implement its own prefething
scheme. \verb|BTE_mmb| prefetching can be turned on by setting the flag
\verb|BTE_MMB_READ_AHEAD| in the header file {\tt
  tpie-\version/test/app\_config.h}.\index{app_config@{\tt app\_config.h}}.
Setting this flag tells the BTE to optimize for sequential read speed by
reading blocks into main memory before the data they contain is actually
needed. This version provides two methods of read-ahead:
\begin{itemize}
\item If the \verb|USE_LIBAIO| flag is set (and \verb|BTE_MMB_READ_AHEAD|
  is set), read ahead is done using the asynchronous I/O library.  This
  feature requires the asynchronous I/O library {\tt libaio}.\index{libaio
    library@{\tt libaio} library.}
\item If the \verb|USE_LIBAIO| flag is not set (and
  \verb|BTE_MMB_READ_AHEAD| is set), read ahead is done using \verb|mmap|
  calls to map the next block of the source file in memory.
\end{itemize}

By default \verb|BTE_MMB_READ_AHEAD| is set, \verb|USE_LIBAIO| is not set.


\subsection{BTE\_UFS}

In addition to \verb|BTE_stdio| and  \verb|BTE_mmb| implementations,
TPIE provides another implementation for BTE streams, called
\verb|BTE_ufs|. As in the previously described BTE stream
implementations, \verb|BTE_ufs| streams are essentially Unix files
that have been specially formated to facilitate TPIE-specific stream
operations. Actually, barring the value of one certain header field, the 
stream format of \verb|BTE_ufs| streams is identical to
\verb|BTE_mmb| streams. \verb|BTE_ufs| streams differ from 
\verb|BTE_mmb| streams in the particular system calls used to
implement I/O and buffering: While \verb|BTE_mmb| streams use the memory
map paradigm to implement I/O,  \verb|BTE_ufs| streams use 
\verb|read()|/\verb|write()| calls to implement their I/O. The
motivation behind implementing  \verb|BTE_ufs| streams is because of
empirically observed inefficiency\footnote{The inefficiency can occur
on account of various reasons. On one system,  \verb|mmap()| calls
were implemented on top of  \verb|stdio| interface instead of a direct
I/O implementation, resulting in extra overhead.} 
in \verb|mmap()|-based (and hence
\verb|BTE_mmb| stream) implementations on some systems. 
In such situations, a stream implementation that,
like  \verb|BTE_mmb| streams, have the potential of  
exploiting large sized blocks and buffers is needed.


Actually speaking, the \verb|BTE_ufs| stream implementation simulates
the \verb|BTE_mmb| stream implementation: Whenever the latter maps in
(via  \verb|mmap()|) a new block, the former reads in a new block (via
\verb|read()|) and whenver the latter unmaps a block (via
\verb|munmap()|), the latter attains the same result as unmapping via
a \verb|write()| call. But  the \verb|BTE_ufs| implementation involves
explicitly keeping track in the BTE code various things which are 
``under the hood'' in \verb|mmap()| implementations. In fact, the code
in the  \verb|BTE_ufs| implementation can be said to amount to a
(very rudimentary) \verb|mmap()| implementation.

The name of the class implementing  \verb|BTE_ufs| is \verb|BTE_single_disk|.

The \verb|BTE_ufs| header structure is identical to the \verb|BTE_mmb|
one:
\begin{verbatim}
struct mmap_stream_header { 
  public:
    unsigned int magic_number;  // Set to UFS_HEADER_MAGIC_NUMBER
    unsigned int version;       // Should be 1 for current version.
    unsigned int length;        // # of bytes in this structure.
    off_t item_logical_eof;     // The number of items in the stream.
    size_t item_size;           // The size of each item in the stream.
    size_t block_size;          // The size of a physical block on the device
                                // where this stream resides.
    unsigned int items_per_block;
};
\end{verbatim}

\verb|BTE_ufs| class inherits from \verb|BTE_base_stream|:
\begin{verbatim}
class BTE_single_disk : public BTE_base_stream {
  private:
     // descriptor of the mapped file.  
     int fd;   
     // A pointer to the mapped in header block for the stream. 
     mmap_stream_header *header;
     ...
}  
\end{verbatim}

\verb|BTE_ufs| defines the abstract methods inherited from
\verb|BTE_base_stream| presented in a previous section. In addition to
these it defines its own constructors:
\begin{verbatim}
  BTE_single_disk(const char *dev_path, BTE_stream_type st); 
  BTE_single_disk(BTE_stream_type st); 
  BTE_single_disk(BTE_single_disk<T> &s); 
  
  // A substream constructor.
  BTE_single_disk(BTE_single_disk *super_stream,
                 BTE_stream_type st,
                 off_t sub_begin, off_t sub_end);
\end{verbatim}

For implementation details please consult the code in
\verb|/include/bte_ufs.h|.

As in  \verb|BTE_stdio|, prefetching can be done implicitly by the
filesystem underlying TPIE. In fact, nowadays, in the case of 
sequential accesses, most filesystems almost surely implement
readahead prefetching, which should suffice for the purpose of
streaming operations in TPIE. (In the case of non-sequential acceses,
the next block to be accessed is more often than not dependent on
the processing of the contents of the current block, so prefetching
is difficult to implement or impossible.) In \verb|BTE_ufs|
streams, when the asynchronous I/O library {\tt libaio}\index{libaio
library@{\tt libaio} library.} is available, there is a provision
to do (user-level) prefetching within \verb|BTE_ufs| streams but we
do not recommend its use on account of the implicit filesystem readahead.


Following is a description of the portion relevant to  \verb|BTE_ufs|
streams in  the header file {\tt
tpie-\version/test/app\_config.h}.\index{app_config@{\tt app\_config.h}}.


\begin{verbatim}
/* ********************************************************************** */
/* BTE_UFS configuration options */
/* ********************************************************************** */

#ifdef BTE_IMP_UFS

// The blocksize (corresp to the theoretical I/O model) is 
// BTE_UFS_LOGICAL_BLOCKSIZE_FACTOR * os blocksize 
#ifndef BTE_UFS_LOGICAL_BLOCKSIZE_FACTOR
#define BTE_UFS_LOGICAL_BLOCKSIZE_FACTOR 32
#endif

//In the current version of TPIE, BTE_UFS_READ_AHEAD should be
//defined as 0 and DOUBLE_BUFFER should be defined 0. 
#define BTE_UFS_READ_AHEAD 0
#define DOUBLE_BUFFER 0

// USE_LIBAIO can be set to 1 to trigger off a certain kind of 
// readahead on Solaris machines, but we suggest keeping this 0 as well.
#define USE_LIBAIO 0

// Very often bte_ufs will be used to sequentially access a file;
//for instance this happens with mergesort and scanning. Typical
//filesystems in such situations tend to carryout sequential readahead.
//When BTE_IMPLICIT_FS_READAHEAD is set to 1, we try to account for the
//amount of memory used up by the read-ahead portion (in the filesystem
//buffer cache) by assuming (quick and dirty guess) that the amount of
//read-ahead at any time is equal to the blocksize (corresp to theoretical
//I/O model). If set to 0, we essentially cheat by not accounting at all
//for memory used by readahead. So in applications in which you sequentially
//access streams, BTE_IMPLICIT_FS_READAHEAD shd be set to 1; otherwise for
//tree-like accesses etc. it should be set to 0.

#define BTE_IMPLICIT_FS_READAHEAD 1
#endif
\end{verbatim}



\section{The Memory Manager (MM)}
\label{sec:ref-mm}
\index{memory manager|(}
\index{MM|see{memory manager}}

The MM is the layer of TPIE that sits between the AMI interface and the
BTE.  Its primary role is managing main memory, including memory that
may be distributed across multiple physical machines.  The performance
of many of the AMI stream operations, such as sorting, permuting,
merging, and distribution depend critically on the efficient use of
main memory.  The first thing the MM will have to do to achieve this
is bypass the virtual memory system provided by UNIX and related
operating systems.  The second thing it has to do is bypass the
traditional UNIX buffer cache and take charge of managing the blocks
of data provided by the BTE.  In some cases, operating system kernels
will have to be modified in order for the MM to do its job.  In modern
micro-kernel operating systems, however, the MM may be able to operate
entirely as a user level process.

In multiple CPU environments, the job of the MM will be complicated by
the need to manage multiple banks of memory.  In tightly coupled
homogeneous parallel environments, this task is likely to be made far
simpler by existing hardware and operating system support.  In
distributed, and in particular in heterogeneous environments, the MM
will have to work with various network protocols and drivers to
accomplish its task.

Some comments on the current simple MM that we have and some OS issues
that come up in attempting to make it more robust.
\index{memory manager|)}

\section{TPIE Logging}\index{logging}
\label{sec:logging}

When logging is turned on (see Section \ref{sec:macros}), TPIE creates a log file\index{log file} with the name \verb|/tmp/TPLOG_XXXXXX|, where \verb|XXXXXX| is a unique system dependent identifier. TPIE writes into this file using a \verb|logstream| class, which is derived from \verb|ofstream| and has the additional functionality of setting a priority and a threshold for this priority. If the priority of a message is smaller than the threshold, the message is not logged. There are three priority levels defined in TPIE, as follows.
\begin{description}
\item[\verb|TP\_LOG\_FATAL|] is the highest level and is used for all kinds of errors that would normally impair subsequent computations; errors are always logged;
\item[\verb|TP\_LOG\_WARNING|] is the next lowest and is used for warnings;
\item[\verb|TP\_LOG\_DEBUG\_INFO|] is the lowest level and is used for debugging information and any other information that might be useful for the developer.
\end{description}
By default, the threshold of the log is set to the lowest level, \verb|TP_LOG_DEBUG_INFO|.

To simplify and unify logging, three macros are provided for writing into the log: 
\begin{quote}
\verb|LOG_FATAL|({\em msg})

\verb|LOG_WARNING|({\em msg})

\verb|LOG_DEBUG_INFO|({\em msg}),
\end{quote}
where {\em msg} is the information to be logged; {\em msg} can be any type that is supported by the C++ \verb|fstream| class. Each of these macros sets the corresponding priority and sends {\em msg} to the log stream.

{\em Logging should always be done using one of the above macros.} Any other method of logging could hinder the ability of TPIE to turn off logging and, as a result, could affect performance.

% Explain what the TPIE library writes into the log.
% I Need input from bte developers.
 % Chapter: The Implementation of TPIE
  %%
%% $Id: tuning.tex,v 1.6 1999-11-15 18:01:30 hutchins Exp $
%%
\chapter{Configuration and Performance Tuning}
\label{sec:tuning}\index{Configuration}\label{sec:configuration}

\comment{LA: Should there be something about compiling in this chapter (and
about e.g including ami.h)?}

\section{TPIE Configuration}
Certain behaviours of TPIE at run-time are controlled by
compile-time variables, whose values should be defined
before including any TPIE headers. Depending on the options
desired, the values of these
variables can be specified as early as when TPIE is
installed, or as late as when an individual application
program is compiled. Section \ref{sec:customization}
described the options available at installation
time. Section \ref{sec:appconfig} describes how TPIE can be
configured differently for individual
TPIE applications.

\subsection{Installation Options}
\index{Customization} \label{sec:customization}

\comment{LA: We need to change this when we change logging}
It is possible to customize the TPIE by providing
arguments to the {\tt configure}
script when TPIE is first installed (see Section \ref{sec:installation}.\index{configuration:options} None of these
 arguments are necessary and the first time you build TPIE
 you should not need any of them. The arguments
recognized are as follows:
\begin{description}
\item[\verb|--enable-log-lib|] 
  \index{enable-log-lib@{\tt --enable-log-lib}}
  Enable logging in TPIE library code.
  This can also be accomplished at compile time by defining the macro
  \verb|TP_LOG_LIB| using the syntax \verb|make lib TP_LOG_LIB=1|.
  This is useful for debugging the TPIE library, but slows it down.
  This option works by defining \verb|TPL_LOGGING|
  \index{TPL_LOGGING@{\tt TPL\_LOGGING}}
  when compiling the library. 
  Section \ref{sec:logging} discusses TPIE logging.
\item[\verb|--enable-assert-lib|]  
  \index{enable-assert-lib@{\tt --enable-assert-lib}}
  Enable assertions in the TPIE library code for debugging purposes.
  This can also be accomplished at compile time by defining the macro
  \verb|TP_ASSERT_LIB| using the syntax \verb|make lib TP_ASSERT_LIB=1|.
  This option works by defining \verb|DEBUG_ASSERTIONS|
  \index{DEBUG_ASSERTIONS@{\tt DEBUG\_ASSERTIONS}} 
  when compiling the library.
\item[\verb|--enable-log-apps|]  and
\item[\verb|--enable-assert-apps|]  
  \index{enable-assert-apps@{\tt --enable-assert-apps}}
  \index{enable-log-apps@{\tt --enable-log-apps}}
  Similar to {\tt --enable-log-lib} and {\tt --enable-assert-lib}, but
  they apply to the test application code.  Running \verb|make test|
  with the options \verb|TP_LOG_APPS=1| and/or \verb|TP_ASSERT_APPS=1|
  accomplishes the same thing.
\item[\verb|--enable-expand-ami-scan|]  Expand the macros in the file
  \verb|ami_scan.h| when making the include directory with the
command {\tt make include} (or {\tt make all}).  This is mainly useful for
debugging the code in \verb|ami_scan.h| itself, and is not normally
needed by TPIE programmers.  It may make compilation of TPIE programs
slightly faster because the macro processor of the C++ compiler will
have less work to do.  In addition to the standard GNU tools mentioned
in Section~\ref{sec:gnu-software}, this requires \verb|perl|.
\item[\verb|--disable-*|]  Any of the options above can be explicitly
  disabled  by using this syntax.  For example
  \verb|--disable-expand-ami-scan|. 
\end{description}

\subsection{Configuring TPIE for Individual Applications}\label{sec:appconfig}

Certain TPIE configuration options can be selected by
setting compile-time variables in the file
\verb|app_config.h|\index{app_config.h@{\tt app\_config.h}}
which is then included in an application program. A typical
example of this file can be found in the \verb|test|
directory. Selected parts of the file are shown
and discussed below. \comment{LA: Something general about TPIE
   configuration as set up by the configure-script needs to
   be included here (e.g. discuss config.h file).}

\subsection{app\_config.h}

\begin{verbatim}
// Get the configuration as set up by the TPIE configure script.
#include <config.h>

/* ********************************************************************** */
/*                      developer use                                     */
/* ********************************************************************** */


/* ********************************************************************** */
/*                       choose BTE                                       */
/* ********************************************************************** */

/* Pick a version of BTE streams; default is BTE_IMP_UFS */
//#define BTE_IMP_MMB
//#define BTE_IMP_STDIO
#define BTE_IMP_UFS


/* ********************************************************************** */
/*                      configure BTE                                     */
/* ********************************************************************** */


/* ********************************************************************** */
/* BTE_MMB configuration options */
/* ********************************************************************** */
#ifdef BTE_IMP_MMB

/* define logical blocksize; default is 32 * operating system blocksize */
#define BTE_MMB_LOGICAL_BLOCKSIZE_FACTOR 32

/* enable/disable TPIE read ahead; default is enabled (set to 1) */
#define BTE_MMB_READ_AHEAD 1

/* read ahead method, ignored unless BTE_MMB_READ_AHEAD is set to 1;
   if USE_LIBAIO is enabled, use asynchronous IO read ahead; otherwise
   use use mmap-based read ahead; default is mmap-based read ahead
   (USE_LIBAIO not defined) */
//#define USE_LIBAIO

#endif


/* ********************************************************************** */
/* BTE_UFS configuration options */
/* ********************************************************************** */
#ifdef BTE_IMP_UFS

/* define logical blocksize; default is 32 * operating system blocksize */
#define BTE_UFS_LOGICAL_BLOCKSIZE_FACTOR 32

/* enable/disable TPIE read ahead; default is disabled (set to 0) */
#define BTE_UFS_READ_AHEAD 0

/* read ahead method, ignored unless BTE_UFS_READ_AHEAD is set to 1;
   if USE_LIBAIO is set to 1, use asynchronous IO read ahead;
   otherwise no TPIE read ahead is done; default is disabled (set to
   0) */
#define USE_LIBAIO 0

#endif
/********************************************************************/


/********************************************************************/
/*  THE FOLLOWING MACROS ARE NORMALLY NOT MODIFIED BY USER           */
/********************************************************************/

/* Use the single BTE stream version of AMI streams; in the current
   option this is the only option */
#define AMI_IMP_SINGLE

/* enable/disable virtual interface; normally disabled */
#ifndef AMI_VIRTUAL_BASE
#define AMI_VIRTUAL_BASE 0
#endif
#ifndef BTE_VIRTUAL_BASE 
#define BTE_VIRTUAL_BASE 0
#endif


/********************************************************************/
/*                            logging;                              */
/*              this should NOT be modified by user!!!              */
/*       in order to enable/disable library/application logging,    */
/*     run tpie configure script with appropriate options           */
/********************************************************************/
// Use logs if requested.
#if TP_LOG_APPS
#define TPL_LOGGING 1
#endif

#include <tpie_log.h>

// Enable assertions if requested.
#if TP_ASSERT_APPS
#define DEBUG_ASSERTIONS 1
#define DEBUG_CERR 1
#define DEBUG_STR 1
#endif
#include <tpie_assert.h>
/********************************************************************/
\end{verbatim}

\subsection{Macros in app\_config.h}

\begin{description}
\item[{\verb|BTE\_IMP\_*|}] Used to choose which of the available Block
Transfer Engine (see Section~\ref{sec:ref-bte}) implementations to
use. Version \version~of TPIE is distributed with three BTEs and the
desired BTE is chosen by defining \verb|BTE_IMP_STDIO|, \verb|BTE_IMP_MMB|
or \verb|BTE_IMP_UFS|. See Section~\ref{sec:ref-bte} for a discussion of
the implementation details in these BTEs. The next section discusses how to
choose an appropriate BTE for a given application in order to obtain
maximal performance.

  \index{BTE_IMP_*@{\tt BTE\_IMP\_*}}
  \index{block transfer engine!implementation}
  \index{implementation!BTE}
\end{description}

\noindent
If \verb|BTE_IMP_MMB| or \verb|BTE_IMP_UFS| is defined, the following macros
are used to control BTE options (how to set the options for maximal
performance is discussed in the next section):

  \begin{description}

  \item[{\verb|BTE\_\*\_LOGICAL\_BLOCKSIZE\_FACTOR|}] This macro sets the
  logical blocksize used by the BTE in units of the physical block size
  (refer to Section~\ref{sec:ref-bte}). Value 1 indicates that the logical
  blocksize is the same as the physical blocksize of the OS.

  \item[{\verb|BTE\_*\_READ\_AHEAD|}] Defining this macro instructs the
  BTE to optimize for sequential read speed by reading blocks into main
  memory before the data they contain is actually needed.

  \index{BTE_MMB_READ_AHEAD@{\tt BTE\_MMB\_READ\_AHEAD}}
  \index{read ahead}
  
  \item[{\verb|USE\_LIBAIO|}] If \verb|BTE_MMB_READ_AHEAD| is defined,
  defining this macro results in the read ahead being performed using the
  asynchronous I/O library \verb|libaio|. If the macro {\verb|USE_LIBAIO|}
  is not defined the read ahead is done using \verb|mmap| and double
  buffering in the case of \verb|BTE_IMP_MMB| and not done at all in the
  case of \verb|BTE_IMP_UFS| (refer to Section~\ref{sec:ref-bte}).

  \index{libaio library@{\tt libaio}}
  \index{USE\_LIBAIO }
  \end{description}

\noindent
The rest of the macros are normally not modified by TPIE application
programmer:

\begin{description}
\item[{\verb|AMI\_IMP\_*|}] This macro controls which Access Method
Interface implementations (see Section~\ref{sec:ref-bte}) to use. Version
\version~of TPIE is only distributed with one AMI implementation, which stores
the contents of a given stream on a single disk. This implementation is
selected by defining \verb|AMI_IMP_SINGLE|.

  \index{AMI_IMP_*@{\tt AMI\_IMP\_*}}
  \index{access method interface!implementation}
  \index{implementation!AMI}
  \index{implementation!AMI!single disk}

\item[{\verb|AMI\_VIRTUAL\_BASE|}] Defining this macro makes the base class
declares a large number of virtual functions for the class
\verb|AMI_base_stream|, which is the base class of all implementations of
AMI streams. This is useful for debugging new AMI stream implementations,
but many compilers cannot properly inline virtual functions, so it slows
the system down significantly.

  \index{AMI_VIRTUAL_BASE@{\tt AMI\_VIRTUAL\_BASE}}
  \index{virtual base class!AMI} 

\item[{\verb|BTE\_VIRTUAL\_BASE|}] Similar to \verb|AMI_VIRTUAL_BASE|, but
for the BTE layer.
  
  \index{BTE_VIRTUAL_BASE@{\tt BTE\_VIRTUAL\_BASE}}
  \index{virtual base class!BTE} 

%All the AMI streams in TPIE are instances of a class derived from the base
%stream class \verb|AMI_base_stream|. Although this base class has no
%functionality in itself, it can provide the minimum set of methods that
%need to be implemented by a derived class in order to have the required
%stream functionality. This is done by using abstract virtual methods. To
%tell TPIE to declare these methods in the base stream class you need to
%set \verb|AMI_VIRTUAL_BASE| to 1 (either in the application's
%configuration, or during compilation by using the -D flag). Similarly,
%\verb|BTE_VIRTUAL_BASE| controls the behavior of the BTE streams.
%
%By default, both BTE and AMI streams use a non-virtual base. By doing this,
%the performance of the stream methods is improved dramatically, in many
%cases. The biggest gain is obtained from inlining short methods (virtual
%methods cannot be inlined). 
%
%We believe this setup provides a good compromise between the need for high
%performance one one hand, and clarity, modularity and debugging support on
%the other hand.


\item[{\verb|TPL\_LOGGING|}] Set to a non-zero value to enable logging of
TPIE's internal behavior.\comment{LA: Is this correct?} By default,
information is logged to the log file\index{log file} \verb|/tmp/TPLOG_XXX|
where \verb|XXX| is a unique system dependent identifier. Typically it
encodes the process ID of the TPIE process that produced it in some
way. See Section~\ref{sec:logging} for information on exactly what TPIE
writes to the log file.

  \index{TPL_LOGGING@{\tt TPL\_LOGGING}}

\item[{\verb|DEBUG\_ASSERTIONS|}] Define to enable TPIE assertions. These
assertions check for inconsistent or erroneous conditions within TPIE
itself. They are primarily intended to aid in the debugging of TPIE. Some
overhead is added to programs compiled with this macro set.

  \index{DEBUG_ASSERTIONS@{\tt DEBUG\_ASSERTIONS}}
  \index{debugging!TPIE}

\item[{\verb|DEBUG\_CERR|}] Defining this macro tells TPIE to write all
internal assertion messages to the C++ standard error stream \verb|cerr| in
addition to the TPIE log file.

 \index{DEBUG_CERR@{\tt DEBUG\_CERR}}
 \index{debugging!TPIE}

\item[{\verb|DEBUG\_STR|}] Defining this macro enables certain debugging
messages that report on the internal behavior of TPIE but do not
necessarily indicate error conditions. In some cases this can increase the
size of the log dramatically.

  \index{DEBUG_STR@{\tt DEBUG\_STR}}
  \index{debugging!TPIE}

\end{description}


\subsection{Environment Variables}
\index{Environment variables}\label{sec:environment}

In version \version~of TPIE there is only one environment variable. The
variable is called \verb|AMI_SINGLE_DEVICE| and defines where TPIE places
temporary streams. The default location is \verb|/var/tmp|. If a different
location is desired, \verb|AMI_SINGLE_DEVICE| must be set accordingly. For
example (in C-shell): setenv AMI\_SINGLE\_DEVICE /usr/project/tmp/.
\index{configuration|)}


\section{TPIE Performance Tuning}
\index{performance tuning}

\subsection{Choosing and Configuring a BTE Implementation}
\label{sec:choosingbte}

\comment{LA: This needs an overhaul!}

Choosing an appropriate BTE implementation (and BTE parameter settings) for
best performance is both application and system dependent. (See
section~\ref{sec:ref-bte} for a description of the three BTEs TPIE is
currently distributed with). Theoretically, \verb|BTE_mmb| should have the
best performance for most applications, because space and copy time is
saved relative to \verb|BTE_stdio| and \verb|BTE_ufs| as steam objects do
not have to pass through kernel level buffer space when accessed. On the
other hand, buffering and prefetching has to be explicitly implemented in
\verb|BTE_mmb| whereas it is (typically) done by the OS in \verb|BTE_stdio|
and \verb|BTE_ufs|. Also theoretically, \verb|BTE_ufs| (and \verb|BTE_mmb|)
should perform better than \verb|BTE_stdio| because of fewer kernel calls
and because of the (possible) larger logical block size. However, in
practice the performance of the three BTE's are very system (and
application) dependent. This is for example due to different
implementations of the \verb|fread()|, \verb|fwrite()|, \verb|read()|,
\verb|write()|, \verb|mmap()|, and \verb|munmap()| calls on different
machines.\comment{LA: Other reasons?}

The most important BTE configuration parameter is the logical block size
setting of the \verb|BTE_mmb| and \verb|BTE_ufs|. As discussed in
Section~\ref{sec:ref-bte}, a large setting increases performance due to the
less kernel calls and due to the (track) buffering and prefetching in the
disk controller, while a too large setting results in decreased performance
due to the BTE's use of main memory. Thus this parameter should be chosen
carefully. As far at the other BTE configuration parameters (prefetching)
are concerned, the default settings in the \verb|app_config.h| file in the
\verb|test| directory are normally the best.

%One criterion that should be carefully considered while setting BTE
%parameters is the \verb|BTE_LOGICAL_BLOCKSIZE_FACTOR|. This parameter
%determines the unit of I/O and buffering used by the BTE stream
%implementation. In case of \verb|BTE_mmb| or \verb|BTE_ufs|
%implementations, the amount of memory dedicated to a stream is either
%\verb|BTE_LOGICAL_BLOCKSIZE_FACTOR| times the operating system blocksize
%(no prefetching) or twice \verb|BTE_LOGICAL_BLOCKSIZE_FACTOR| times the
%operating system blocksize (in the presence of prefetching). So the value
%of the \verb|BTE_LOGICAL_BLOCKSIZE_FACTOR| parameter, together with
%available memory, determines the number of BTE streams (and hence AMI
%streams) that can be active or ``alive'' at any time. This is an upper
%bound on the arity of a multi-way merge or a multi-way distribution
%operation that can be undertaken by a TPIE application; which can have a
%crucial impact on (say, the number of passes required in external sorting
%and hence the) net running time. The size of each buffer and the size of
%each I/O in the BTE stream is \verb|BTE_LOGICAL_BLOCKSIZE_FACTOR| times the
%operating system blocksize, so this roughly corresponds to the amount of
%data brought in or written out at the cost of a single disk operation.  The
%latter observation suggests that \verb|BTE_LOGICAL_BLOCKSIZE_FACTOR| should
%be set to a high value; but a high value for this parameter inhibits the
%number of streams active at a time and hence can result in an increase in
%the number of passes required in sorting.
%%In the case of external memory indexing data structures based on trees, the
%value of the \verb|BTE_LOGICAL_BLOCKSIZE_FACTOR| for any BTE stream (or BTE
%block collection, in future TPIE versions) used to implement the external
%memory data structure should be made as close to the size of the tree node
%as possible.

In order to help in deciding which BTE to choose for a given
application/system, as well as deciding on what logical block size to use
(in \verb|BTE_mmb| and \verb|BTE_ufs|), we have included a C program in the
\verb|test| directory of the TPIE distribution called \verb|bte_test.c|.
This program can be used to determine the streaming speeds attained by
\verb|BTE_stdio|, \verb|BTE_mmb|, and \verb|BTE_ufs| streams on a given
system. The program simulates the buffering and I/O mechanisms used by each
of the BTE stream implementations so that the ``raw'' (in the sense that
there is no TPIE layer between the program and the filesystem) streaming
speed of an I/O-buffering mechanism combination can be determined. To use
the program, define one of \verb|MMAP_TEST|,
\verb|READ_WRITE_TEST|\comment{LA: Why not UFS\_TEST?} or \verb|STDIO_TEST|
in the program depending on whether you want to test the streaming speed of
\verb|BTE_mmb|, \verb|BTE_ufs| or \verb|BTE_stdio|. Also define the
\verb|BLOCKSIZE_BASE| parameter to be equal to the underlying operating
system blocksize.\comment{LA: Why not automatic?} Compile the program using
a C compiler. In order to test the streaming performance of BTE streams of
objects of size \verb|ItemSize|, the program first writes out some
specified number \verb|NumStreams| of BTE streams containing a specified
number \verb|NumItems| of items. Then it carries out a perfect
\verb|NumStreams|-way interleaving of the streams via a simple merge like
process, writing the output to an output stream. During the computation,
each of the \verb|NumStreams| streams input to the merge, as well as the
stream being output by the merge uses either one (when
\verb|READ_WRITE_TEST| or \verb|STDIO_TEST| are set to 1) or two (when
\verb|MMAP_TEST| is set to 1) buffers.  In case of \verb|STDIO_TEST|, the
buffers are not maintained in the program but by the stdio library. In the
case of \verb|MMAP_TEST| or \verb|READ_WRITE_TEST|, each buffer is set to
be of size \verb|block_factor| times \verb|BLOCKFACTOR_BASE|, and each I/O
operation corresponds to a buffer-sized operation. To test the streaming
performance of a BTE stream with \verb|items_in_block| items in each block
simply execute:
\begin{verbatim}
bte_test NumItems ItemSize NumStreams block_factor items_in_block DataFile 
\end{verbatim}
The output of the program (streaming speed) is appended to the file
\verb|DataFile|. The streaming speed, alternatively called I/O Bandwidth,
is given in units of MB/s, and can be used to decide which BTE to use and
how to configure it.

\subsection{Other Factors Affecting Performance}

In addition to the choice (and configuration) of BTE, a number of other
factors, not all of which are TPIE specific, can effect the performance of
a TPIE application.

\begin{description}
\item[Inlining operation management object methods] Failing to inline the
\verb|operate()| method of operation management objects can be a major
source of lackluster performance of an application, since \verb|operate()|
is called once for every object in a stream being scanned. Inlining of
\verb|operate()| is, of course, just a suggestion to the compiler, which
can choose to ignore it. In order to maximize the likelihood of inlining,
it is a good idea to keep the function  short and simple. One way of
doing this is to wrap complex pieces of code that are called less often in
separate functions.
\item[\verb|gcc| optimization] We recommend using the \verb|-O2| level of
optimization of \verb|gcc| in order to obtain the best overall
performance. Although better performance can normally be obtained using
\verb|-O3|, this optimization leads to increased program size which can
potentially result in decreased performance.
\item[Memory size] To insure that no disk swapping is done by the OS, the
size of main memory used by TPIE (set by \verb|MM_manager.resize_heap()|,
see Section\ref{sec:compiling} and Section~\ref{sec:mm-ref}) should be set
to a realistic value. The best value is usually much smaller than the size
of the memory installed in the computer (due to memory use of operating
system resources and daemons).
\end{description}

%\section{Using Multiple Physical Devices}


\section{TPIE Logging}\index{logging}
\label{sec:logging}

\comment{LA: The whole logging thing needs to be reevaluated and changed}
When logging is turned on (see Section \ref{sec:configuration}), TPIE
creates a log file\index{log file} in \verb|/tmp/TPLOG_XXXXXX|, where
\verb|XXXXXX| is a unique system dependent identifier. TPIE writes into
this file using a \verb|logstream| class, which is derived from
\verb|ofstream| and has the additional functionality of setting a priority
and a threshold for logging. If the priority of a message is below
 the threshold, the message is not logged. There are four priority
levels defined in TPIE, as follows.
\begin{description}
\item[\verb|TP\_LOG\_FATAL|] is the highest level and is used for all kinds
of errors that would normally impair subsequent computations. Errors are
always logged;
\item[\verb|TP\_LOG\_WARNING|] is the next lowest and is used for warnings.
\item[\verb|TP\_LOG\_APP\_DEBUG|] is used by applications built on top of TPIE, for logging debugging information.
\item[\verb|TP\_LOG\_DEBUG\_INFO|] is the lowest level and is used by the TPIE library for logging debugging information.
\end{description}
By default, the threshold of the log is set to the lowest level, \verb|TP_LOG_DEBUG_INFO|. To change the threshold level, the following macro is provided:
\begin{quote}
\verb|LOG_SET_THRESHOLD(|{\em level}\verb|)|
\end{quote}
The threshold level can be reset as many times as needed in a program. This enables the developer to focus the debugging effort on a certain part of the program.

Four macros are provided for writing into the log:
\begin{quote}
\verb|LOG_FATAL|({\em msg})

\verb|LOG_WARNING|({\em msg})

\verb|LOG_APP_DEBUG|({\em msg})

\verb|LOG_DEBUG_INFO|({\em msg}),
\end{quote}
where {\em msg} is the information to be logged; {\em msg} can be any type
that is supported by the \verb|C++| \verb|fstream| class. Each of these
macros sets the corresponding priority and sends {\em msg} to the log
stream.

%{\em Logging should always be done using one of the above macros.} Any
%other method of logging could hinder the ability of TPIE to turn off
%logging and, as a result, could affect performance.


%\subsection{Template Instantiation}

%{\bf Important Note:} Much of the information in this section is
%likely to change as the template instantiation mechanism of the 
%{\tt g++}\index{g++@{\tt g++}} compiler improves.  If you are
%interested in the nitty gritty details of template instantiation,
%consult~\cite{ellis:arm} or one of the frequent discussions on the
%topic in the newsgroup {\tt comp.lang.c++}
%\index{comp.lang.c++@{\tt comp.lang.c++}}.

%\index{templates!instantiation|(}
%\noindent Most of the classes and functions TPIE defines are
%templated.  Furthermore, many user written operation management
%object\index{operation management objects!user supplied} classes are
%likely to be templated; many of those supplied with the test and
%sample applications are.

%Unfortunately, many C++\index{C++} compilers do not properly implement
%templated function and/or classes.  In particular, the GNU C++
%compiler, {\tt g++}\index{g++@{\tt g++}}, version \gxxversion, which
%was used in the development of TPIE has some deficiencies when it
%comes to template instantiation.  It also has a well defined mechanism
%for working around these deficiencies, which TPIE takes significant
%advantage of.  This mechanism prevents the compiler from implicitly
%instantiating any template.  Thus, all templates used by a program
%must be explicitly instantiated at compile time or they will not be
%available at link time and linking will fail.

%In order to tell {\tt g++}\index{g++@{\tt g++}} not to implicitly
%instantiate any templates, the {\tt -fno-implicit-templates} flag is
%used.  Additionally, the macro {\tt NO\_IMPLICIT\_TEMPLATES} should be
%defined on the command line, using {\tt -D}.  This macro informs TPIE
%that it should not rely on the presence of implicit template
%instantiation.  In response to the fact that this macro is set, TPIE
%defines a series of new macros with names of the form {\tt
%  TEMPLATE\_INSTANTIATE\_*}.  
%\index{TEMPLATE_INSTANTIATE_*@{\tt TEMPLATE\_INSTANTIATE\_*}|(}
%Each of these macros can be used to
%actually instantiate some set of functions and/or classes that TPIE
%needs to provide a given operation.  These macros should be used at
%the end of your source file in order to perform the proper
%instantiations.

%The {\tt TEMPLATE\_INSTANTIATE\_*} macros likely to be needed by TPIE
%programmers are as follows:
%\begin{description}
%\item[{\tt TEMPLATE\_INSTANTIATE\_STREAMS(T)}] Instantiate AMI and
%  BTE level streams of objects of type {\tt T}.  If your
%  application uses streams of several types, this macro must be called
%  once for each of them.
%\item[{\tt TEMPLATE\_INSTANTIATE\_ISTREAM(T)}]
%\item[{\tt TEMPLATE\_INSTANTIATE\_OSTREAM(T)}] Instantiate ASCII
%  input and output scan management objects for the type {\tt T}.
%  See Section~\ref{sec:ascii-io} for details on these objects.
%  \index{scanning!ASCII I/O}
%\item[{\tt TEMPLATE\_INSTANTIATE\_AMI\_MERGE}] Instantiate merging entry
%  points for streams of objects of type {\tt T}.  Merging is described
%  in Section~\ref{sec:merging}.
%\item[{\tt TEMPLATE\_INSTANTIATE\_SORT\_OP(T)}]
%\item[{\tt TEMPLATE\_INSTANTIATE\_SORT\_CMP(T)}]
%\item[{\tt TEMPLATE\_INSTANTIATE\_SORT\_OBJ(T)}] Instantiate
%  respectively operator, comparison function, and comparison object
%  based sorting of objects of type {\tt T}.  See
%  Section~\ref{sec:cmp-sorting} for details on these types of sorting.
%\item[{\tt TEMPLATE\_INSTANTIATE\_KB\_SORT(T)}] 
%\item[{\tt TEMPLATE\_INSTANTIATE\_KB\_SORT\_KEY(T,K)}] Instantiate key
%  bucket distribution sorting of objects of type {\tt T}.  The latter
%  form uses key {\tt K} for sorting.  Section~\ref{sec:kb-sorting}
%  describes key bucket sorting.
%\item[{\tt TEMPLATE\_INSTANTIATE\_STREAM\_ADD(T)}]
%\item[{\tt TEMPLATE\_INSTANTIATE\_STREAM\_SUB(T)}]
%\item[{\tt TEMPLATE\_INSTANTIATE\_STREAM\_MULT(T)}]
%\item[{\tt TEMPLATE\_INSTANTIATE\_STREAM\_DIV(T)}]
%  Instantiate elementwise arithmetic operations on streams of objects
%  of type {\tt T} as described in Section~\ref{sec:elementwise}.
%\item[{\tt TEMPLATE\_INSTANTIATE\_AMI\_MATRIX}]
%  Instantiate dense matrices of objects of type {\tt T} and the
%  standard operations on them.  Dense
%  matrices are described in
%  Section~\ref{sec:dense-mat}.\index{matrices!dense}
%\item[{\tt TEMPLATE\_INSTANTIATE\_AMI\_SPARSE\_MATRIX}]
%  Instantiate sparse matrices of objects of type {\tt T} and the
%  standard operations on them.  Sparse
%  matrices are described in
%  Section~\ref{sec:dense-mat}.\index{matrices!sparse}
%\end{description}
%\index{TEMPLATE_INSTANTIATE_*@{\tt TEMPLATE\_INSTANTIATE\_*}|)}

%In addition to instantiating functions and classes using the macros
%described above, it is often necessary to explicitly instantiate
%particular instances of AMI entry points for user supplied operation
%management objects.  For example, suppose we declare a scan management
%object class such as
%\begin{verbatim}
%class my_scan_class : AMI_scan_object {
%public:
%    AMI_err initialize(void);
%    AMI_err operate(const int &in1, const int &in2, AMI_SCAN_FLAG *sfin,
%                    float *out, AMI_SCAN_FLAG *sfout); 
%}
%\end{verbatim}
%Then, in order to explicitly instantiate \verb|AMI_scan()| to use
%objects of this type, we would use the following code:
%\begin{verbatim}
%template AMI_err AMI_scan(AMI_STREAM<int> *, AMI_STREAM<int> *, 
%                          my_scan_class *, AMI_STREAM<float> *); 
%\end{verbatim}
%This instantiates an instance of \verb|AMI_scan()| that takes two input
%streams of \verb|int|s, operates on them with an object of type
%\verb|my_scan_class|, and produces an output stream of \verb|float|s.  
%Note the correspondence between the types of input and output streams
%and the types of the operands to the \verb|operate()| member function
%of the class \verb|my_scan_class|.
%\index{templates!instantiation|)}

%%% Local Variables: 
%%% mode: latex
%%% TeX-master: t
%%% End: 
 % Chapter: TPIE Performance Tuning
\part{Appendices}
\appendix
  %%
%% $Id: applications.tex,v 1.6 1999-10-13 21:05:41 hutchins Exp $
%%
\chapter{Test and Sample Applications}

\section{General Structure and Operation}

The test and sample applications distributed with TPIE are in the
\verb|test| directory.  The test programs are designed primarily to
test the operation of the system to verify that it has been installed
correctly and is as bug free as possible.  These applications all have
names of the form \verb|test_*|.  The sample applications are designed
to demonstrate the use of TPIE in the solution of non-trivial
problems.

The test and sample applications all share a small amount of common
initialization and argument parsing code.  They all include the header file
\verb|app_config.h|, which selects a particular implementation of streams
at the AMI and BTE levels. They also all use the same argument parsing
function \verb|parse_args()|, which parses certain default arguments and
then uses a callback function for arguments specific to one particular
application.

Much of the functionality provided by the common initialization and
argument passing code is intended to eventually be subsumed by
operating system provided services.  For example, the amount of main
memory a particular application is permitted to use can be set via a
command line argument.  It is up to the user to be sure that this
number is reasonable and does not exceed the true amount of main
memory available to the application.  In the future, it is hoped that
this information will be provided by the operating system.

\verb|parse_args()| is declared as follows:

\begin{verbatim}
    parse_args();
\end{verbatim}


The following is a summary of the common command line arguments that
are parsed by \verb|parse_args()|.
\begin{description}
\item[\verb|-t testsize|]
Set the size of the test to be run to \verb|testsize|.  Typically this
is the number of objects to be put into the application produced input
stream.  In matrix tests, however, it is the number of rows and
columns is the test matrices.  If this argument is not passed, then
the default value of 8 Meg is used.
\item[\verb|-m memsize|]
The number of bytes of main memory that the application is permitted to
use.  The MM\index{memory manager} will ensure that no more than this amount is
used.  If this option is not specified, then a default value of 2 Mb
is used.
\item[\verb|-z randomseed|]
Seed the random number generator with the value \verb|randomseed|.
This is useful for debugging or testing, when we want several runs of
the application to rely on the same series of pseudo-random numbers.
For applications that do not generate test data randomly, this has no
effect. 
\item[\verb|-v|]
Turns on verbose mode.  When running in verbose mode, report major
actions of the running program to verb|stdout|.
\end{description}

Each application specific argument appears in the string pointed to by
\verb|aso| as a single character, possibly followed by the single
character `\verb|:|', indicating that the argument requires a value.
For example, if \verb|aso| pointed to the string ``\verb|ax:z|'' then the
following command line arguments would all be parsed correctly:
\begin{description}
\item[\verb|-a|]
\item[\verb|-x 123|]
\item[\verb|-a -x 123|]
\item[\verb|-ax123|]
\item[\verb|-x123 -a|]
\end{description}
In each case, \verb|parse_app()| would be called to take some
application specific action for each of the arguments.  It would be
called once with \verb|opt| set to `\verb|a|' and \verb|optarg| set to
\verb|NULL|, and/or once with \verb|opt| set to `\verb|x|' and
\verb|optarg| pointing to the string ``\verb|123|.''  When multiple arguments
are present on the command line, they are parsed from left to right.

The following is an example of how a test application, in this case
\verb|test_ami_sort|, can use application specific command line
arguments to set up it's global state.

\begin{verbatim}
static const char as_opts[] = "R:S:rsao";
void parse_app_opt(char c, char *optarg)
{
    switch (c) {
        case 'R':
            rand_results_filename = optarg;
        case 'r':
            report_results_random = true;
            break;
        case 'S':
            sorted_results_filename = optarg;
        case 's':
            report_results_sorted = true;
            break;
        case 'a':
            sort_again = !sort_again;
            break;
        case 'o':
            use_operator = !use_operator;
            break;
    }
}

int main(int argc, char **argv)
{
    parse_args(argc,argv,as_opts,parse_app_opt);

    ...

    return 0;
}
\end{verbatim}

\section{Test Programs}

The test programs include with TPIE are as follows:\comment{LA: Is this
still correct?}

\begin{description}
\item[\verb|test\_ami\_merge|] Test fixed way merging with direct
  calls to \verb|AMI_merge()|, as described in 
  Section~\ref{sec:ref-ami-merge}.
\item[\verb|test\_ami\_pmerge|] Test many-way merging using 
  \verb|AMI_partition_and_merge()|, as described in 
  Section~\ref{sec:ref-ami-merge}.
\item[\verb|test\_ami\_sort|] Test sorting using \verb|AMI_sort()| as
  described in Section~\ref{sec:ref-ami-sort}.
\item[\verb|test\_ami\_gp|] Test general permutation using
  \verb|AMI_general_permute()| as described in Section~\ref{sec:ref-ami-gp}.
The program generates an input stream
consisting of sequential integers, and outputs a stream consisting of 
the same integers, in reverse order.
\item[\verb|test\_ami\_bp|] Test bit permutations using
  \verb|AMI_BMMC_permute()| as described in
  Section~\ref{sec:ref-ami-bp}. The program generates an input stream
consisting of sequential integers, and outputs a stream consisting of 
a permutation of these integers, as described in the example given in the Tutorial, Section~\ref{sec:bit-permuting}.
\item[\verb|test\_matrix|]
\item[\verb|test\_bit\_matrix|] Test main memory matrix manipulation
  and arithmetic.  This is used both by the bit permuting code
  described in Section~\ref{sec:ref-ami-bp} and the dense matrix
  multiplication code described in Section~\ref{sec:ref-ami-matrix}
  for internal manipulation of sub-matrices of external memory
  matrices.
\item[\verb|test\_ami\_matrix\_pad|] Test padding of external
  matrices.  This is the preprocessing step for the external dense
  matrix multiplication algorithm TPIE uses, which is described in 
  Section~\ref{sec:ref-ami-matrix}. 
\item[\verb|test\_ami\_matrix|] Test external dense matrix arithmetic
  as described in Section~\ref{sec:ref-ami-matrix}.
\item[\verb|test\_ami\_sm|] Test external sparse matrix arithmetic
  as described in Section~\ref{sec:ref-ami-sm}.
\item[\verb|test\_ami\_stack|] Test external memory stacks as
  described in Section~\ref{sec:ref-ami-stack}.
\item[\verb|test\_ami\_arith|] Test element-wise arithmetic on
  external memory streams as described in Section~\ref{sec:ref-ami-arith}.
The program generates an input stream
consisting of sequential integers, squares them, and performs 
elementwise division between the resulting stream and the input stream.
\end{description}

\section{Sample Applications}

The sample applications included with TPIE are as follows:

\begin{description}
\item[\verb|ch2|] Two dimensional convex hull\index{convex hull}
  program using Graham's scan.  It is implemented using a scan
  management object that maintains the upper and lower hull internally
  as external memory stacks.  Much of the code in this application
  appears in Section~\ref{sec:convex-hull}.
\item[\verb|lr|] An implementation of an asymptotically optimal list
  ranking \index{list ranking} algorithm.  The idea of geometrically
  decreasing computation is used.  Much of the code in this
  application appears in Section~\ref{sec:list-ranking}.
\item[\verb|nas\_ep|] An I/O-efficient implementation of the NAS EP
  parallel benchmark.  This benchmark generates pairs of independent
  Gaussian random variates.
\item[\verb|nas\_is|] An I/O-efficient implementation of the NAS IS
  parallel benchmark.  This benchmark sorts integers using one of a
  variety of approaches.
\end{description}

Detailed descriptions of the NAS parallel benchmarks are available
from the \htmladdnormallink{NAS Parallel Benchmark Home Page}%
{http://www.nas.nasa.gov/NAS/NPB/}
\begin{latexonly}
at URL \verb|http://www.nas.nasa.gov/NAS/NPB/|.
\end{latexonly}

%%% Local Variables: 
%%% mode: latex
%%% TeX-master: t
%%% End: 
 %  Chapter: Test and Sample Applications
  \chapter{Additional Examples} \label{ch:examples}
\index{examples}

This chapter contains some additional annotated examples of 
TPIE application code.\comment{LA: Is this chapter still ok?}

\section{Convex Hull}
\label{sec:convex-hull}
\index{convex hull|(}

The convex hull of a set of points in the plane is the smallest convex
polygon which encloses all of the points.  Graham's scan is a simple
algorithm for computing convex hulls.  It should be discussed in any
introductory book on computational geometry, such
as~\cite{preparata:computational}.  Although Graham's scan was not
originally designed for external memory, it can be implemented optimally in
this setting.  What is interesting about this implementation is that
external memory stacks are used within the implementation of a scan
management object.

First, we need a data type for storing points.  We use the following
simple class, which is templated to handle any numeric type.

\lstinputlisting[numbers=left,basicstyle=\ttfamily\small,firstline=15,lastline=47,caption={Code taken from \texttt{tpie\_\version/apps/convex\_hull/point.h}}]{../apps/convex_hull/point.h}

Once the points are s by their $x$ values, we simply scan them to
produce the upper and lower hulls, each of which are stored as a stack
pointed to by the scan management object.  We then concatenate the
stacks to produce the final hull.  The code for computing the convex
hull of a set of points is thus

\lstinputlisting[numbers=left,basicstyle=\ttfamily\small,firstline=204,lastline=258,caption={Code taken from \texttt{tpie\_\version/apps/convex\_hull/convex\_hull.cpp}}]{../apps/convex_hull/convex_hull.cpp}

The only thing that remains is to define a scan management object that
is capable of producing the upper and lower hulls by scanning the
points.  According to the Graham's scan algorithm, we produce the
upper hull by moving forward in the $x$ direction, adding each
point we encounter to the upper hull, until we find one that induces a
concave turn on the surface of the hull.  We then move backwards
through the list of points that have been added to the hull,
eliminating points until a convex path is reestablished.  This process
is made efficient by storing the points on the hull so far in a stack.
The code for the scan management object, which relies on the function
\lstinline|ccw()| to actually determine whether a corner is
convex or not, is as follows:

\lstinputlisting[numbers=left,basicstyle=\ttfamily\small,firstline=30,lastline=199,caption={Code taken from \texttt{tpie\_\version/apps/convex\_hull/convex\_hull.cpp}}]{../apps/convex_hull/convex_hull.cpp}

The function \lstinline|ccw()| computes twice the signed area of a triangle in
the plane by evaluating a 3 by 3 determinant.  The result is positive
if and only if the the three points in order form a counterclockwise
cycle.

\lstinputlisting[numbers=left,basicstyle=\ttfamily\small,firstline=61,lastline=74,caption={Code taken from \texttt{tpie\_\version/apps/convex\_hull/point.h}}]{../apps/convex_hull/point.h}

\index{convex hull|)}

\section{List-Ranking}
\label{sec:list-ranking}
\index{list ranking|(}

List ranking is a fundamental problem in graph theory.  The problem is
as follows: We are given the directed edges of a linked list in some
arbitrary order.  Each edge is an ordered pair of node ids.  The first
is the source of the edge and the second is the destination of the
edge.  Our goal is to assign a weight to each edge corresponding to
the number of edges that would have to be traversed to get from the
head of the list to that edge.

The code given below solves the list ranking problem using a simple
randomized algorithm due to Chiang {\em et al}.~\cite{chiang:external}.
As was the case in the code examples in the tutorial in
Chapter~\ref{ch:tutorial}, \lstinline|#include| statements
for header files and definitions of some classes and functions as well
as some error and consistency checking code are left out so that the
reader can concentrate on the more important details of how TPIE is
used.  A complete ready to compile version of this code is included in
the TPIE source distribution.

First, we need a class to represent edges.  Because the algorithm will
set a flag for each edge and then assign weights to the edges, we
include fields for these values.

\lstinputlisting[numbers=left,basicstyle=\ttfamily\small,firstline=16,lastline=24,caption={Code taken from \texttt{tpie\_\version/apps/list\_rank/list\_edge.h}}]{../apps/list_rank/list_edge.h}

As the algorithm runs, it will sort the edges.  At times this will be
done by their sources and at times by their destinations.  The
following simple functions are used to compare these values:

\lstinputlisting[numbers=left,basicstyle=\ttfamily\small,firstline=28,lastline=36,caption={Code taken from \texttt{tpie\_\version/apps/list\_rank/list\_edge.h}}]{../apps/list_rank/list_edge.h}

The first step of the algorithm is to assign a randomly chosen flag,
whose value is 0 or 1 with equal probability, to each edge.  This is
done using \lstinline|AMI_scan()| with a scan management object of the
class \lstinline|random_flag_scan|, which is defined as follows:

\lstinputlisting[numbers=left,basicstyle=\ttfamily\small,firstline=199,lastline=220,caption={Code taken from \texttt{tpie\_\version/apps/list\_rank/lr.cpp}}]{../apps/list_rank/lr.cpp}

The next step of the algorithm is to separate the edges into an active
list and a cancel list.  In order to do this, we sort one copy of the
edges by their sources (using \lstinline|edgefromcmp|) and sort another copy by
their destinations (using \lstinline|edgetocmp|).  We then call
\lstinline|AMI_scan()| to scan the two lists and produce an active list and
a cancel list.  A scan management object of class
\lstinline|separate_active_from_cancel| is used.

\lstinputlisting[numbers=left,basicstyle=\ttfamily\small,firstline=222,lastline=315,caption={Code taken from \texttt{tpie\_\version/apps/list\_rank/lr.cpp}}]{../apps/list_rank/lr.cpp}

The next step of the algorithm is to strip the cancelled edges away
from the list of all edges.  The remaining active edges will form a
recursive subproblem.  Again, we use a scan management object, this
time of the class \lstinline|strip_active_from_cancel|, which is defined as
follows:

\lstinputlisting[numbers=left,basicstyle=\ttfamily\small,firstline=317,lastline=385,caption={Code taken from \texttt{tpie\_\version/apps/list\_rank/lr.cpp}}]{../apps/list_rank/lr.cpp}

After recursion, we must patch the cancelled edges back into the
recursively ranked list of active edges.  This is done using a scan
with a scan management object of the class
\lstinline|interleave_active_cancel|, which is implemented as follows:

\lstinputlisting[numbers=left,basicstyle=\ttfamily\small,firstline=388,lastline=464,caption={Code taken from \texttt{tpie\_\version/apps/list\_rank/lr.cpp}}]{../apps/list_rank/lr.cpp}

Finally, here is the actual function to rank the list.

\lstinputlisting[numbers=left,basicstyle=\ttfamily\small,firstline=468,lastline=656,caption={Code taken from \texttt{tpie\_\version/apps/list\_rank/lr.cpp}}]{../apps/list_rank/lr.cpp}

Our recursion bottoms out when the problem is small enough to fit
entirely in main memory, in which case we read it in and call a
function to rank a list in main memory.  The details of this function
are omitted here.

\begin{lstlisting}[basicstyle=\ttfamily\small,caption={Code taken from \texttt{tpie\_\version/apps/list\_rank/lr.cpp}}]
////////////////////////////////////////////////////////////////////////
// main_mem_list_rank()
//
// This function ranks a list that can fit in main memory.  It is used
// when the recursion bottoms out.
//
////////////////////////////////////////////////////////////////////////

int main_mem_list_rank(edge *edges, size_t count)
{
    // Rank the list in main memory

    ...
        
    return 0;  
}
\end{lstlisting}
\index{list ranking|)}

\section{NAS Parallel Benchmarks}

\tobeextended

Code designed to implement external memory versions of a number of the
NAS parallel benchmarks is included with the TPIE distribution.
Examine this code for examples of how the various primitives TPIE
provides can be combined into powerful applications capable of solving
real-world problems.

Detailed descriptions of the parallel benchmarks are available
from the NAS Parallel Benchmark Report at URL \href{http://www.nas.nasa.gov/Research/Reports/Techreports/1994/HTML/npbspec.html}{\path"http://www.nas.nasa.gov/Research/Reports/Techreports/1994/HTML/npbspec.html"}.

\section{Spatial Join}

\tobewritten

\comment{LA: Distribution sweeping, SSSJ, ect}

\comment{LA: Someting about R-tree building and drainage networks at
  some point}

%%% Local Variables: 
%%% mode: latex
%%% TeX-master: "tpie"
%%% End: 

  %% Copyright 2008, The TPIE development team
%% 
%% This file is part of TPIE.
%% 
%% TPIE is free software: you can redistribute it and/or modify it under
%% the terms of the GNU Lesser General Public License as published by the
%% Free Software Foundation, either version 3 of the License, or (at your
%% option) any later version.
%% 
%% TPIE is distributed in the hope that it will be useful, but WITHOUT ANY
%% WARRANTY; without even the implied warranty of MERCHANTABILITY or
%% FITNESS FOR A PARTICULAR PURPOSE.  See the GNU Lesser General Public
%% License for more details.
%% 
%% You should have received a copy of the GNU Lesser General Public License
%% along with TPIE.  If not, see <http:%%www.gnu.org/licenses/>

\chapter{TPIE Error Codes and Management Object Return Values}

\comment{LA: I added the managment object stuff in title since it was
  there. Is there other stuff that needs to go in this appendix?}

\section{AMI Error Codes}
\label{sec:ami-errors}
\index{error codes|(}


AMI entry points typically return error codes of the enumerated type
\lstinline|AMI_err|.  Member functions of operation management
objects\index{operation management objects} also typically return this
type.  Possible values for error codes include those listed below.  It
is expected that in future releases of TPIE, many of these error codes
will be replaced by exceptions.
%Exceptions are not currently used by TPIE because the {\tt
%g++} compiler does not fully support them.

%%
%%  Note: Do not attempt to add ``[]'' to the items, otherwise
%%        \lstinline will not work. (JV)
%%
\begin{description}\index{AMI\_ERROR\_*@{\tt AMI\_ERROR\_*}|(}
\item\lstinline|AMI_ERROR_NO_ERROR:| No error occurred.  The call the the
  entry point returned normally.
\item\lstinline|AMI_ERROR_IO_ERROR:| A low level I/O error occurred.
\item\lstinline|AMI_ERROR_END_OF_STREAM:| An attempt was made to read
  past the end of a stream or write past the end of a substream.
\item\lstinline|AMI_ERROR_READ_ONLY:| An attempt was made to write to a
  read-only stream.
\item\lstinline|AMI_ERROR_OS_ERROR:|  An unexpected operating system
  error occurred.  Details should appear in the log file if logging is
  enabled.  See Section~\ref{sec:logging}.
\item\lstinline|AMI_ERROR_BASE_METHOD:| An attempt was made to call a
  member function of the virtual base class of \lstinline|AMI_STREAM|.  This
  indicates a bug in the implementation of AMI streams.
\item\lstinline|AMI_ERROR_BTE_ERROR:| An error occurred at the BTE
  level.  
\item\lstinline|AMI_ERROR_MM_ERROR:| An error occurred within the memory
  manager.
\item\lstinline|AMI_ERROR_OBJECT_INITIALIZATION:| An AMI entry point was
  not able to properly initialize the operation management object that
  was passed to it.  This generally indicates a bug in the operation
  management object's initialization code.
%\item\lstinline|AMI_ERROR_PERMISSION_DENIED:|
\item\lstinline|AMI_ERROR_INSUFFICIENT_MAIN_MEMORY:| The MM could not
  make adequate main memory available to complete the requested
  operation.  Many operations adapt themselves to use whatever main
  memory is available, but in some cases, when memory is extremely
  tight, they may not be able to function.
\item\lstinline|AMI_ERROR_INSUFFICIENT_AVAILABLE_STREAMS:|
  The AMI could not allocate enough intermediate streams to perform
  the requested operation.  Certain operating system restrictions
  limit the number of streams that can be created on certain
  platforms.  Only in unusual circumstances, such as when the
  application itself has a very large number of open streams, will
  this error occur. 
\item\lstinline|AMI_ERROR_ENV_UNDEFINED:|
  An environment variable necessary to initialize the AMI was not defined.
\item\lstinline|AMI_ERROR_BIT_MATRIX_BOUNDS:|
  A bit matrix larger than the number of bits in an offset into a
  stream was passed to \lstinline|ami_gp()|.
\item\lstinline|AMI_ERROR_NOT_POWER_OF_2:|
  The length of a stream on which a bit permutation was to be
  performed is not a power of two.
\item\lstinline|AMI_MATRIX_BOUNDS:| An attempt was made to perform a
  matrix operation on matrices whose bounds did not match appropriately.
\index{AMI\_ERROR\_*@{\tt AMI\_ERROR\_*}|)}
\end{description}

\section{Management Object Return Values}
\index{Management object return values|(}

\subsection{Return Values for Scan Management Objects}


More information on the precise semantics of these values appears in
Section~\ref{sec:ref-ami-scan}.
\begin{description}
\item\lstinline|AMI_SCAN_CONTINUE:| \index{AMI\_SCAN\_CONTINUE@{\tt
AMI\_SCAN\_CONTINUE}} Tells \lstinline|AMI_scan()|\index{AMI\_scan()@{\tt AMI\_scan()}} to continue
  to call the \lstinline|operate()| member function of the scan management
  object with more data.
\item\lstinline|AMI_SCAN_DONE:| \index{AMI\_SCAN\_DONE@{\tt AMI\_SCAN\_DONE}} Tells \lstinline|AMI_scan()| that the scan is
  complete. 
\end{description}

\subsection{Return Values for Merge Management Objects}

More information on the precise semantics of these values appears in
Section~\ref{sec:ref-ami-merge}.
\begin{description}\index{AMI\_MERGE\_*@{\tt AMI\_MERGE\_*}}
\item\lstinline|AMI_MERGE_CONTINUE:| Tells
\lstinline|AMI_merge()|\index{AMI\_merge()@{\tt AMI\_merge()}} to continue
  to call the \lstinline|operate()| member function of the scan management
  object with more data.
\item\lstinline|AMI_MERGE_DONE:| Tells \lstinline|AMI_merge()| that the scan is
  complete. 
\item\lstinline|AMI_MERGE_OUTPUT:|  Tells \lstinline|AMI_merge()| that the
  last call generated output for the output stream.
\item\lstinline|AMI_MERGE_READ_MULTIPLE:|  Tells \lstinline|AMI_merge()| that
  more than one input object was consumed and thus the input flags
  should be consulted.
\end{description}

\index{error codes|)}

%%% Local Variables: 
%%% mode: latex
%%% TeX-master: "tpie"
%%% End: 

  \renewcommand{\labelenumii}{\alph{enumii})}
\renewcommand{\labelenumiii}{\arabic{enumiii})}

\chapter{GNU Lesser General Public License}\label{app:lgpl}


\begin{center}
{\parindent 0in

Copyright \copyright\  2007 Free Software Foundation, Inc. \texttt{http://fsf.org/}

\bigskip
Everyone is permitted to copy and distribute verbatim copies of this

license document, but changing it is not allowed.}

\end{center}


  This version of the GNU Lesser General Public License incorporates
the terms and conditions of version 3 of the GNU General Public
License, supplemented by the additional permissions listed below.

\begin{enumerate}
\addtocounter{enumi}{-1}  % start at 0

\item Additional Definitions.

  As used herein, ``this License'' refers to version 3 of the GNU Lesser
General Public License, and the ``GNU GPL'' refers to version 3 of the GNU
General Public License.

  ``The Library'' refers to a covered work governed by this License,
other than an Application or a Combined Work as defined below.

  An ``Application'' is any work that makes use of an interface provided
by the Library, but which is not otherwise based on the Library.
Defining a subclass of a class defined by the Library is deemed a mode
of using an interface provided by the Library.

  A ``Combined Work'' is a work produced by combining or linking an
Application with the Library.  The particular version of the Library
with which the Combined Work was made is also called the ``Linked
Version''.

  The ``Minimal Corresponding Source'' for a Combined Work means the
Corresponding Source for the Combined Work, excluding any source code
for portions of the Combined Work that, considered in isolation, are
based on the Application, and not on the Linked Version.

  The ``Corresponding Application Code'' for a Combined Work means the
object code and/or source code for the Application, including any data
and utility programs needed for reproducing the Combined Work from the
Application, but excluding the System Libraries of the Combined Work.

\item Exception to Section 3 of the GNU GPL.

  You may convey a covered work under sections 3 and 4 of this License
without being bound by section 3 of the GNU GPL.

\item Conveying Modified Versions.

  If you modify a copy of the Library, and, in your modifications, a
facility refers to a function or data to be supplied by an Application
that uses the facility (other than as an argument passed when the
facility is invoked), then you may convey a copy of the modified
version:

   \begin{enumerate}
   \item under this License, provided that you make a good faith effort to
   ensure that, in the event an Application does not supply the
   function or data, the facility still operates, and performs
   whatever part of its purpose remains meaningful, or

   \item under the GNU GPL, with none of the additional permissions of
   this License applicable to that copy.
   \end{enumerate}

\item Object Code Incorporating Material from Library Header Files.

  The object code form of an Application may incorporate material from
a header file that is part of the Library.  You may convey such object
code under terms of your choice, provided that, if the incorporated
material is not limited to numerical parameters, data structure
layouts and accessors, or small macros, inline functions and templates
(ten or fewer lines in length), you do both of the following:

   \begin{enumerate}
   \item Give prominent notice with each copy of the object code that the
   Library is used in it and that the Library and its use are
   covered by this License.

   \item Accompany the object code with a copy of the GNU GPL and this license
   document.
   \end{enumerate}

\item Combined Works.

  You may convey a Combined Work under terms of your choice that,
taken together, effectively do not restrict modification of the
portions of the Library contained in the Combined Work and reverse
engineering for debugging such modifications, if you also do each of
the following:

   \begin{enumerate}
   \item Give prominent notice with each copy of the Combined Work that
   the Library is used in it and that the Library and its use are
   covered by this License.

   \item Accompany the Combined Work with a copy of the GNU GPL and this license
   document.

   \item For a Combined Work that displays copyright notices during
   execution, include the copyright notice for the Library among
   these notices, as well as a reference directing the user to the
   copies of the GNU GPL and this license document.

   \item Do one of the following:

       \begin{enumerate}
       \addtocounter{enumiii}{-1}  % start at 0
       \item Convey the Minimal Corresponding Source under the terms of this
       License, and the Corresponding Application Code in a form
       suitable for, and under terms that permit, the user to
       recombine or relink the Application with a modified version of
       the Linked Version to produce a modified Combined Work, in the
       manner specified by section 6 of the GNU GPL for conveying
       Corresponding Source.

       \item Use a suitable shared library mechanism for linking with the
       Library.  A suitable mechanism is one that (a) uses at run time
       a copy of the Library already present on the user's computer
       system, and (b) will operate properly with a modified version
       of the Library that is interface-compatible with the Linked
       Version. 
       \end{enumerate}

   \item Provide Installation Information, but only if you would otherwise
   be required to provide such information under section 6 of the
   GNU GPL, and only to the extent that such information is
   necessary to install and execute a modified version of the
   Combined Work produced by recombining or relinking the
   Application with a modified version of the Linked Version. (If
   you use option 4d0, the Installation Information must accompany
   the Minimal Corresponding Source and Corresponding Application
   Code. If you use option 4d1, you must provide the Installation
   Information in the manner specified by section 6 of the GNU GPL
   for conveying Corresponding Source.)
   \end{enumerate}

\item Combined Libraries.

  You may place library facilities that are a work based on the
Library side by side in a single library together with other library
facilities that are not Applications and are not covered by this
License, and convey such a combined library under terms of your
choice, if you do both of the following:

   \begin{enumerate}
   \item Accompany the combined library with a copy of the same work based
   on the Library, uncombined with any other library facilities,
   conveyed under the terms of this License.

   \item Give prominent notice with the combined library that part of it
   is a work based on the Library, and explaining where to find the
   accompanying uncombined form of the same work.
   \end{enumerate}

\item Revised Versions of the GNU Lesser General Public License.

  The Free Software Foundation may publish revised and/or new versions
of the GNU Lesser General Public License from time to time. Such new
versions will be similar in spirit to the present version, but may
differ in detail to address new problems or concerns.

  Each version is given a distinguishing version number. If the
Library as you received it specifies that a certain numbered version
of the GNU Lesser General Public License ``or any later version''
applies to it, you have the option of following the terms and
conditions either of that published version or of any later version
published by the Free Software Foundation. If the Library as you
received it does not specify a version number of the GNU Lesser
General Public License, you may choose any version of the GNU Lesser
General Public License ever published by the Free Software Foundation.

  If the Library as you received it specifies that a proxy can decide
whether future versions of the GNU Lesser General Public License shall
apply, that proxy's public statement of acceptance of any version is
permanent authorization for you to choose that version for the
Library.

\end{enumerate}


\bibliographystyle{abbrv}
\newpage
\addcontentsline{toc}{part}{Bibliography}
\bibliography{tpie}
\newpage
\addcontentsline{toc}{part}{Index}
\printindex
\end{document}


