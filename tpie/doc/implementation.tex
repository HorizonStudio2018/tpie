%%
%% $Id: implementation.tex,v 1.13 2000-10-05 19:27:24 hutchins Exp $
%%
\chapter{The Implementation of TPIE}
\plabel{cha:implementation}

%This chapter discusses the implementation of TPIE.
%It is primarily targeted at those who might wish to port TPIE to
%additional platforms or implement similar systems.

\section{The Structure of TPIE}
\index{structure!of TPIE} \index{components!of TPIE} TPIE
has three main components, the Access Method Interface
(AMI)\index{access method interface}, a Block Transfer
Engine (BTE)\index{block transfer engine} component, and a
Memory Manager (MM)\index{memory manager} component.
Various Block Transfer Engines (BTEs) can be chosen for
handling disk block transfers, perhaps more than one in a
single application.  The MM component provides low level
memory management services such as allocating, deallocating,
and accounting of internal memory. The AMI works on top of
the Memory Manager and one or more BTEs to provide a uniform
interface for application programs. Applications that use
this interface are portable across hardware platforms, since
they never have to deal with the underlying details of how
I/O is performed on a particular machine. This chapter
describes the design decisions, algorithms, and
implementation decisions that were used to build the MM, BTE
and AMI components of TPIE. The Reference Section of this
manual contains a description of the AMI and Memory Manager
entry points that an application programmer might normally
use. Typically, an application programmer will not request
services from a BTE directly. For this reason, the BTE
services are not presented in the Reference Section of this
manual, but are presented here for those readers who wish to
understand the implementation details of TPIE.


%Although the BTE runs on a single processor, it can support concurrent
%access to multiple disks\index{parallel disks}, allocating and managing
%buffer space for all of them concurrently.

The MM\index{memory manager} manages random access memory on behalf of
TPIE.
%It is the most architecture-dependent component of the
%system. 
Currently, TPIE is distributed with an MM designed for a single processor, or
multiprocessor system with a single global address space. This MM is
relatively simple; its task is to allocate and manage the physical memory
used by the BTE component. 

%On a distributed memory system, the MM will have the
%additional task of coordinating communication between processors and memory
%modules in order to support the primitives that the AMI provides.

The AMI\index{access method interface} is an interface layer between the BTE and user
level processes.  It implements fundamental access methods, such as
scanning, permutation routing, merging, and distribution. It also provides
a consistent, object-oriented interface to application programs.
%The details of how these access methods are implemented depends on the
%hardware on which the system is running.  For example, recursive
%distribution will be done somewhat differently on a parallel disk machine
%than on a single disk machine. The AMI abstracts this fact away, allowing
%an application program that calls a function such as
%\myverb{AMI\_partition\_and\_merge()}\index{AMI\_partition\_and\_merge@{\tt
%AMI\\_partition\\_and\_merge}} to work correctly regardless of the underlying
%I/O system.
The key to keeping the AMI simple and flexible is the fact that its
user accessible functions serve more as templates for computation than
as actual problem solving functions.  The details of how a computation
proceeds within the template is up to the application programmer, who
is responsible for providing the functions that the template applies
to data.

\section{The Block Transfer Engine (BTE)}
\plabel{sec:ref-bte}
\index{block transfer engine|(}
\index{BTE|see{block transfer engine}}

%\tobeextended 

The BTE component is intended to bridge the gap between the
I/O hardware and the rest of the system. It is the layer
that is ultimately responsible for moving blocks of data
from physical disk devices to main memory and back. It works
alongside the traditional buffer cache\index{buffer cache}
in a UNIX system.  Unlike the buffer cache, which must
support concurrent access to files from multiple address
spaces, the BTE is specifically designed to support high
throughput processing of data from secondary memory through
a single, user level address space.  In order to efficiently
support the merging, distribution, and scanning paradigms,
several BTEs support stream-oriented input and output of
data blocks.  To further improve performance, some BTE
implementations move data from disk directly into user space
rather than using a kernel-level buffer cache\index{buffer
   cache}.  This saves both main memory space and copying
time. In the future, TPIE will also provide a BTE
implementation with support for random-access to disk
blocks. Such functionality is useful when implementing
external data structures (indexes).  Currently, all TPIE
BTEs are designed for processing streams on a single disk.
Planned BTE implementations will also support streams stored
on multiple disks.

\comment{LA: This whole section needs an overhaul - discuss
   implementation!}
%We hope that in most cases it will be possible for the BTE to work with
%device drivers provided by the machine vendor's operating system.  In some
%cases, however, new drivers may have to be written.
%  The BTEs will not, however, be responsible for coordinating the
% actions of multiple CPU's and the disks attached to them. In such cases, a
% separate instance of a BTE will run on each such CPU, and their actions
% will be coordinated by a single multi-threaded MM running at a higher
% level. The reason for the functional split between the two levels is that
% it will likely be advantageous to be able to use a single BTE written for a
% specific piece of hardware with more than one MM, for example, one MM
% written for a homogeneous environment and one for a heterogeneous
% environment. 

Streams in TPIE are implemented as sequentially accessed
files, and each BTE offers public member functions that are
analogous to, and implemented via corresponding functions
available in the underlying \emphd{file access method}. The main
difference is that TPIE maintains a typed view of streams,
where user-defined data elements (i.e. objects) are accessed
in a stream, rather than the untyped view of the data
offered by the corresponding file
access primitives. \comment{%
   LA: Some intro about streams, files, ect - like in AMI
   reference. Also something about; block access, read
   ahead, double buffering, logical blocks, file pointer,
   stream types, persistence, ect.?}


Version \version~of TPIE supports the following three BTE
implementations:
\begin{enumerate}
    \item \myverb{BTE\_stdio}, which uses the
    UNIX \myverb{stdio} library as its file access method.
    
    \item \myverb{BTE\_ufs}, which performs I/O via UNIX {\tt
       read/write} calls,
    
    \item \myverb{BTE\_mmb}, which uses the UNIX
    \myverb{mmap} call to perform I/O
\end{enumerate}

The choice of BTE in a given application is controlled by
compile-time variables in the \myverb{app\_config.h} file.
The UNIX \myverb{stdio} implementation is selected by
defining the variable \myverb{BTE\_IMP\_STDIO}, the
\myverb{read/write} implementation by defining
\myverb{BTE\_IMP\_UFS}, and the memory mapped implementation
by defining \myverb{BTE\_IMP\_MMB}.  For example, the
following code (taken from \myverb{app\_config.h} selects
the \myverb{BTE\_ufs} Block Transfer Engine (and does not
select the others):

\begin{verbatim}
//#define BTE_IMP_MMB
//#define BTE_IMP_STDIO
#define BTE_IMP_UFS
\end{verbatim}

% If the including module did not explicitly ask for multiple
% implementations but requested more than one implementation,
% issue a warning.  
Multiple implementations are allowed to coexist, with some
restrictions, e.g. declarations of streams must use explicit subclasses
of \myv{AMI\_base\_stream} to specify what type of streams they
are. 
\comment{LA: This needs to be extended somewhere.}
\comment{DH: I am skeptical that it works with multiple impl., but
   I don't see why we need this feature with single disks.
   Multiple impls. in an app may be possible with separate
   compiles? We may need more flexibility with multi disk
   implementations however.} The best choice of BTE for a
given application is both application and system dependent.
Section~\ref{sec:choosingbte} discuss how to choose an
appropriate BTE (also refer to the more detailed
descriptions of the BTE's in the next three subsections). If
none of the available BTEs is selected,
\myverb{BTE\_IMP\_STDIO}\comment{DH: We should change the default
   to BTE\_IMP\_UFS} is defined by default and a warning
is generated at compile time.


The first block in a TPIE stream, referred to as the
\emphd{header block}\index{streams!header block}, contains
contextual information for the stream. Currently, the format
of the header block is BTE-dependent. Please refer to the
descriptions of the individual BTEs in
Sections~\ref{sec:ref-bte-stdio},\ref{sec:ref-bte-mmb}, and
\ref{sec:ref-bte-ufs} for more details.

\subsection{BTE Common Functionality}
\plabel{sec:imp-bte-base}

All BTE stream implementations inherit from class
\myverb{BTE\_base\_stream} (in file
\myverb{bte\_base\_stream.h}) and must support the member
functions listed below. 


%In addition to the data elements stored in a stream, a TPIE
%stream contains a ``header block'' which contains
%BTE-dependent information, as well as context information
%such as the length of the stream, the size of each item in
%the stream, etc. 
%A stream in TPIE is implemented via a file, and the
%individual BTE choices available correspond to different
%access methods for such a file. 

Since the details of each access method (ie
\myverb{BTE\_ufs}, \myverb{BTE\_stdio} or \myverb{BTE\_mmb} are
different, the BTEs generally have different member function
implementations, including constructors (construction
involves opening the appropriate stream file, etc.).
The \myverb{BTE\_ufs}, \myverb{BTE\_stdio}, and
\myverb{BTE\_mmb} implementations are defined in the
respective files \myverb{include/bte\_ufs.h},
\myverb{include/bte\_stdio}, and \myverb{include/bte\_mmb.h}.
\comment{LA: Some note about construction being BTE
dependent - why are they? DH: DONE}
Functionally, the member functions for each BTE are similar.

The common BTE member functions\footnote{%
   These functions can be defined as \myverb{virtual} to
   assist in debugging a new BTE if the compile-time
   variable \myverb{BTE\_VIRTUAL\_BASE} is defined to be
   non-zero (see also file \myverb{app\_config.h}). For best
   run-time performance, the default value for
   \myverb{BTE\_VIRTUAL\_BASE} is zero.  } 
and their usage are outlined below:

\subsubsection{Stream Constructors}
\begin{verbatim}
    BTE_stream_mmb   ( const char *dev_path, BTE_stream_type st );
    BTE_stream_stdio ( const char *dev_path, BTE_stream_type st );
    BTE_stream_ufs   ( const char *dev_path, BTE_stream_type st );
\end{verbatim}

These constructors each create a stream which is backed by
the file whose path is pointed to by \myverb{dev\_path}.  The
stream can be opened in one of several modes, depending on
the value of \myverb{st}:

\begin{itemize}

    \item \myverb{BTE\_READ\_STREAM}: the stream can be read,
    but not written. The underlying file must have been
    created previously in this case.
    \item \myverb{BTE\_WRITE\_STREAM} or
    \myverb{BTE\_WRITEONLY\_STREAM}: the stream can be
    written, but not read. This will allow a new stream to
    be created or overwrite a previously-created one.

    \item \myverb{BTE\_APPEND\_STREAM}: the stream is opened,
    and positioned to the end of stream so that new data can
    be appended. This is valid only if the stream was
    previously created. 
\end{itemize}

\subsubsection{new\_substream}\index{new_substream()@{\tt new\_substream()}}
\index{substreams!of BTE streams}
\begin{verbatim}
    BTE_err new_substream(BTE_stream_type st, off_t sub_begin, off_t
                          sub_end, BTE_base_stream<T> **sub_stream);
\end{verbatim}

Constructs a substream of the current stream.  A substream
is a logical contiguous subset of a stream, i.e. from the
implementation point of view, it is a contiguous subset of
the data elements in a file.  The parameters of
\myverb{new\_substream()} are as follows:
\begin{itemize}
    \item \myverb{sub\_begin} and \myverb{sub\_end} are
    object offsets into the current stream, indicating the
    limits of the new substream.
    
    \item \myverb{st} is the stream type of the new
    substream. Please refer to the discussion of substream
    constructors above for a list of the valid (sub)stream
    types.

    \item \myverb{sub\_stream} will be given the address of
    a pointer to the new substream.
\end{itemize}

While it might seem logical to explicitly define a substream
in TPIE as inheriting (in the C++ sense) from a stream, this
would imply a need for both to have a constructor, and hence
for the stream constructor to be \myverb{virtual}. This is a
problem in C++, and so the ``pseudo-constructor'' approach
described here is used instead.\comment{LA: Extend like in
   AMI ref.  Stream type?}
Because \myverb{new\_substream()} is not a constructor, but rather a
function each particular implementation of which calls an appropriate
constructor, it can be a pure virtual function in the stream base class,
which forces it to be defined for all actual stream
implementations.
\comment{LA: I don't understand this paragraph. DH:
   reworked. Clearer?} 

\subsubsection{read\_item}\index{read_item()@{\tt read\_item()}!BTE}
\begin{verbatim}
    BTE_err read_item(T **elt);
\end{verbatim}
Returns the address of a pointer to the next element in the
stream in \myverb{elt}. Since the application data elements are blocked,
this often does not require a physical I/O operation. If the
current block has been exhausted, the first element of the
next block will be accessed. In some cases the BTE will
try to prefetch the next block and it may already be in
memory. (See descriptions of the individual BTEs for more
details about prefetching or read-ahead.)

\subsubsection{write\_item}\index{write_item()@{\tt write\_item()}!BTE}
\begin{verbatim}
    BTE_err write_item(const T &elt);
\end{verbatim}
Writes the data element pointed to by \myverb{elt} to the
current position in the stream. Streams are normally written
in sequential order, and so the current position is usually
directly after the previous element written. The current
position (which we will refer to as the value of the
\emphd{file pointer}) can be modified via the \myverb{seek}
command below.  \comment{LA: Extend! DH:DONE}


\subsubsection{seek}\index{seek()@{\tt seek()}!BTE}
\begin{verbatim}
    BTE_err seek(off_t offset);
\end{verbatim}
Seeks to the object offset \noiverb{offset} in the
stream. The BTE \myverb{seek} member function is similar to
and utilizes the \noiverb{seek} function supported by the
underlying file I/O system being used. The difference is
that TPIE performs a seek to the requested application data
element in the stream rather than to a byte offset within a
file of untyped data. 


\subsubsection{truncate}\index{truncate()@{\tt truncate()}!BTE}
\begin{verbatim}
    BTE_err truncate(off_t offset);
\end{verbatim}
Truncates/extends the stream to the specified number of
application data elements. The file
pointer\comment{LA: ? DH: defined above} will be moved to
the end of the stream. This is analogous to the
\myverb{truncate} function of the underlying file access
method. Note: \myverb{truncate} is not supported for substreams.

\subsubsection{main\_memory\_usage}
\index{main_memory_usage()@{\tt main\_memory\_usage()}!BTE}
\begin{verbatim}
    BTE_err main_memory_usage(size_t *usage,
                              MM_stream_usage usage_type);
\end{verbatim}
Queries how much memory is used by the stream for
\myverb{usage\_type} purposes. On return, \myverb{usage}
points to the value used. The valid values for \myverb{usage\_type}
are as follows:\comment{LA: ? DH: DONE}
\begin{itemize}
\item \myverb{MM\_STREAM\_USAGE\_OVERHEAD}: the size in
bytes of the stream object plus, if this is not a substream,
the stream header block.
\item \myverb{MM\_STREAM\_USAGE\_BUFFER}: the number of
bytes consumed by block buffers (blocks of elements in main memory).
\item \myverb{MM\_STREAM\_USAGE\_CURRENT}:
\myverb{MM\_STREAM\_USAGE\_OVERHEAD} plus the number of
bytes currently consumed by block buffers for this stream. 
\item \myverb{MM\_STREAM\_USAGE\_MAXIMUM}:
\myverb{MM\_STREAM\_USAGE\_OVERHEAD} plus the maximum number of
bytes that will be consumed by block buffers for this
stream.
\item \myverb{MM\_STREAM\_USAGE\_SUBSTREAM}:
The same as \myverb{MM\_STREAM\_USAGE\_OVERHEAD} but assumes
that this is a substream.
\end{itemize}

\subsubsection{get\_status}\index{get_status()@{\tt get\_status()}}
\begin{verbatim}
    BTE_stream_status get_status(void);
\end{verbatim}
Returns the status of the stream as one
of the following values:\comment{LA: Is this really used
   anywhere? DH: Yes, appears in the code.}
\index{BTE\_STREAM\_STATUS_*@{\tt BTE\_STREAM\_STATUS\_*}}
\begin{itemize}
\item \myverb{BTE\_STREAM\_STATUS\_NO\_STATUS}: No status
information exists for the stream.  
\item \myverb{BTE\_STREAM\_STATUS\_INVALID}: An error was
encountered while manipulating the stream and the stream
can no longer be used.
%UNUSED \item \myverb{BTE\_STREAM\_STATUS\_EOS\_ON\_NEXT\_CALL}:
\item \myverb{BTE\_STREAM\_STATUS\_END\_OF\_STREAM}: The end
of stream was encountered.
\end{itemize}
The \myverb{get\_status} member function is one of the few
that is implemented in \myverb{bte\_base\_stream.h} and its
implementation is therefore common to all BTEs. The status
is stored in the private variable \myverb{status} of a BTE object. 

\subsubsection{stream\_len}\index{stream_len()@{\tt stream\_len()}!BTE}
\begin{verbatim}
    off_t stream_len(void);
\end{verbatim}
Returns the current number of application data elements in the stream.


\subsubsection{name}\index{name()@{\tt name()}!BTE}
\begin{verbatim}
    BTE_err name(char **stream_name);
\end{verbatim}
Returns the path name of the file backing the stream. The name will be
stored in newly allocated space.

\subsubsection{read\_only}\index{read_only()@{\tt read\_only()}}
\begin{verbatim}
    int read_only(void);
\end{verbatim}
Returns true if the stream mode is BTE\_READ\_STREAM. The file mode
is set when creating a
stream. See BTE constructors, above for details.\comment{LA:? DH: DONE}

\subsubsection{available\_streams}
\index{available_streams()@{\tt available\_streams()}!BTE}
\begin{verbatim}
    int available_streams(void);    
\end{verbatim}
Returns the number of additional streams that can be
activated (i.e. created) before an
operating system limit on the number of open files would be
exceeded. This reflects only the operating system limit on
number of open files. The TPIE memory requirements for
active streams are not reflected in this limit.\comment{LA: Extend DH:DONE}

\subsubsection{chunk\_size}
\begin{verbatim}
    off_t chunk_size(void);
\end{verbatim}
%Not clear what this does...
Returns the number of application data elements that fit in
a logical block of the current stream.\comment{LA: ?}

\subsubsection{persist}\index{persistence!BTE}
\index{persist()@{\tt persist()}|see{persistence}}
\begin{verbatim}
    void persist(persistence);
\end{verbatim}
Sets the persistence attribute of the current stream to the
value of \myverb{persistence}. This may be one of the
following values:
\begin{itemize}
\item \myverb{PERSIST\_DELETE}: Delete the stream from the
disk when the stream 
  destructor is called.
\item \myverb{PERSIST\_PERSISTENT}: Do not delete the stream from the disk when
  the stream destructor is called.
%\item \myverb{PERSIST\_READ\_ONCE}: Delete each block of data from the disk as
%  it is read.
\end{itemize}

By default, all streams are deleted when their destructors
are called.

\subsection{BTE \emphd{stdio}}
\index{BTE stdio|see \tt{BTE\_stream\_stdio}}
\plabel{sec:ref-bte-stdio}

The \emphd{stdio} BTE is implemented by the class
\myv{BTE\_stream\_stdio}, defined in file
\myverb{/include/bte\_stdio.h}.  \myverb{BTE\_stream\_stdio}
streams are stored as ordinary UNIX files which are
manipulated via the \emphd{stdio} file access method, i.e.
via the standard \myverb{C} I/O library. The read/write
primitives of \myverb{BTE\_stream\_stdio} streams are
implemented using the system calls \myverb{fread} and
\myverb{fwrite}. The underlying operating system blocking
and prefetching assure that stream accesses are done in
blocks and so both blocking and prefetching are therefore
automatic and invisible to the TPIE developer.  Note that
this means that a (OS) kernel call is incurred every time a
stream object is accessed, and that every object passes
through kernel level buffer space on its way to user space.
\comment{DH: this seems incorrect to me.}
\comment{LA: Extend. Also something about no logical block size
concept.}

The \myverb{BTE\_stream\_stdio} class inherits from
\myverb{BTE\_base\_stream} (in file
\myverb{/include/bte\_base\_stream.h}):\index{BTE_base_stream@{\tt
      BTE\_base\_stream}}
\begin{verbatim}
class BTE_stream_stdio : public BTE_base_stream {
  private:
     FILE  *file;          
     BTE_stdio_header      header;
     ...
}  
\end{verbatim}

\myverb{BTE\_stream\_stdio} defines the public member
functions defined for \myverb{BTE\_base\_stream} (please
see Section~\ref{sec:imp-bte-base} for a list of these
member functions and their semantics).  \comment{LA: Extend
   - discuss implementation!} In addition to these,
\myverb{BTE\_stream\_stdio} defines its own
constructors:\index{BTE\_stream\_stdio@{\tt BTE\_stream\_stdio}!constructors}

\begin{verbatim}
     BTE_stream_stdio(const char *dev_path, const BTE_stream_type st); 
     BTE_stream_stdio(const BTE_stream_type st); 
     BTE_stream_stdio(const BTE_stream_stdio<T> &s);
\end{verbatim}

The first block of a TPIE \emphd{stdio} stream is a header
block\index{BTE_stream_stdio@{\tt BTE\_stream\_stdio}!header}
\index{streams!header block} with the following structure:

\index{BTE_stdio_header@{\tt BTE\_stdio\_header}}
\begin{verbatim}
typedef struct BTE_stdio_header_v1 { 
    unsigned int magic_number;  // Set to BTE_STDIO_HEADER_MAGIC_NUMBER
    unsigned int version;       // Should be 1 for current version.
    unsigned int length;        // # of bytes in this structure.
    unsigned int block_length;  // # of bytes in a block.
    size_t item_size;           // The size of each item in the stream.
} BTE_stdio_header;
\end{verbatim}


\index{BTE_stdio@{\tt BTE\_stdio}|)}


\index{BTE_stream_mmb@{\tt BTE\_stream\_mmb}|(} 
\subsection{BTE \emphd{mmb}}
\plabel{sec:ref-bte-mmb}
\index{BTE mmb|see {\tt BTE\_stream\_mmb}}
\comment{LA: Why not called BTE\_mmp?}

The \emphd{mmb} BTE is implemented by the class
\myv{BTE\_stream\_mmb}, defined in file
\myverb{/include/bte\_stdio.h}.  \myverb{BTE\_stream\_mmb}
streams are stored as ordinary UNIX files.  The TPIE
\myverb{BTE\_stream\_mmb} implementation uses the Unix
\myverb{mmap} and \myverb{munmap} calls to perform I/O. The
UNIX \myverb{mmap} call allows a part of a file to be
associated with a corresponding section of internal memory.
When the memory is accessed, the corresponding portion of
the file is copied directly into that memory buffer. The
\myverb{BTE\_stream\_mmb} primitives explicitly maintain the
currently accessed block of the file mapped into memory.
When an object outside the current block boundaries is
requested, the current block is unmapped and a new one is
mapped from the source file.

The \myverb{BTE\_stream\_mmb} is not limited to mapping in
blocks of size equal to the physical block size of the OS
(typically 8K bytes). Often improved performance can be
obtained by mapping blocks of much larger size (for example
256KB\footnote{We use the notation KB to mean kilobytes.}).
This is typically due to (track) buffering and prefetching
performed in the disk controller. The (logical) block size
used by \myverb{BTE\_stream\_mmb} can be set using the macro
\myverb{BTE\_MMB\_LOGICAL\_BLOCKSIZE\_FACTOR} in the
\myverb{app\_config.h} file. However, choosing a large
(logical) block size limits the amount of main memory
available for an application program and thus the (logical)
block size should be chosen very carefully in order to
obtain maximal performance.

Unlike in \myverb{BTE\_stream\_stdio}, where a function call
is incurred every time a stream object is accessed, the
\myverb{BTE\_stream\_mmb} only incurs such a call on every
(logical) block. This can lead to improved performance,
especially on systems with a relatively slow CPU compared to
the disk.  However, while prefetching of disk blocks is
implicitly done by the operating system in
\myverb{BTE\_stream\_stdio}, \myverb{BTE\_stream\_mmb} has
to implement its own prefetching scheme. (Note on the other
hand that unlike in \myverb{BTE\_stream\_stdio} objects does
not pass through the kernel level buffer space on its way to
user space).  \myverb{BTE\_stream\_mmb} prefetching can be
turned on by defining the macro
\myverb{BTE\_MMB\_READ\_AHEAD} in the \myverb{app\_config.h}
file. If this compile-time variable is defined,
\myverb{BTE\_stream\_mmb} optimize for sequential read speed
by reading (mapping) blocks into main memory before the data
they contain is actually needed. Two methods of read-ahead
is provided:

\begin{itemize}

    \item If the \myverb{USE\_LIBAIO} macro is undefined
    (and \myverb{BTE\_MMB\_READ\_AHEAD} is set), read ahead
    is done using \myverb{mmap} calls.

    \item If the \myverb{USE\_LIBAIO} macro is defined (and
    \myverb{BTE\_MMB\_READ\_AHEAD} is set), read ahead is
    done using the asynchronous I/O library. This feature
    requires the asynchronous I/O library {\tt
       libaio}.\index{libaio library@{\tt libaio} library.}

\end{itemize}
By default \myverb{BTE\_MMB\_READ\_AHEAD} is defined,
\myverb{USE\_LIBAIO} is not.\comment{LA: Discuss the two and
   how they differ.}

\myverb{BTE\_stream\_mmb} class inherits from \myverb{BTE\_base\_stream}:\index{BTE_base_stream@{\tt BTE\_base\_stream}}
\begin{verbatim}
class BTE_stream_mmb : public BTE_base_stream {
  private:
     // descriptor of the mapped file.  
     int fd;   
     // A pointer to the mapped in header block for the stream. 
     mmap_stream_header *header;
     ...
}  
\end{verbatim}

\myverb{BTE\_stream\_mmb} defines the public member
functions defined for \myverb{BTE\_base\_stream} (please see
Section~\ref{sec:imp-bte-base} for a list of these member
functions and their semantics).  In addition to these it
defines its own constructors: 
\index{BTE_stream_mmb@{\tt BTE\_stream\_mmb}!constructors}
\comment{LA: Extend - discuss implementation.} 
\comment{LA: Why special substream constructor? DH:Previously discussed.}
\begin{verbatim}
  BTE_stream_mmb(const char *dev_path, BTE_stream_type st);
  BTE_stream_mmb(BTE_stream_type st); 
  BTE_stream_mmb(BTE_stream_mmb<T> &s); 
  
  // A substream constructor.
  BTE_stream_mmb(BTE_stream_mmb *super_stream,
                 BTE_stream_type st,
                 off_t sub_begin, off_t sub_end);
\end{verbatim}

The \myverb{BTE\_stream\_mmb} header\index{BTE_stream_mmb@{\tt
      BTE\_stream\_mmb}!header}\index{streams!header block}
structure is as follows:

\begin{verbatim}
struct mmap_stream_header { 
  public:
    unsigned int magic_number;  // Set to MMB_HEADER_MAGIC_NUMBER
    unsigned int version;       // Should be 1 for current version.
    unsigned int length;        // # of bytes in this structure.
    off_t item_logical_eof;     // The number of items in the stream.
    size_t item_size;           // The size of each item in the stream.
    size_t block_size;          // The size of a physical block on the device
                                // where this stream resides.
    unsigned int items_per_block;
};
\end{verbatim}


\index{BTE_stream_mmb@{\tt BTE\_stream\_mmb}|)}

\index{BTE_stream_ufs@{\tt BTE\_stream\_ufs}|(}

\subsection{BTE \emphd{ufs}}
\plabel{sec:ref-bte-ufs}
\index{BTE ufs|see {\tt BTE\_stream\_ufs}}

(!!! CHECK COMMENTS) 
The \emphd{ufs} BTE is implemented by the class
\myv{BTE\_stream\_ufs}, defined in file
\myverb{/include/bte\_stdio.h}.  \myverb{BTE\_stream\_ufs}
streams are stored as ordinary UNIX files.
\myverb{BTE\_ufs} streams use
\myverb{read()}/\myverb{write()} calls to implement their
I/O.


% The motivation behind implementing \myverb{BTE\_ufs} streams
% is an empirically observed inefficiency\footnote{The
%    inefficiency can occur on account of various reasons. On
%    one system, \myverb{mmap()} calls were implemented on top
%    of the \myverb{stdio} interface instead of a direct I/O
%    implementation, resulting in extra overhead.} of
% \myverb{mmap()} implementations in some systems.\comment{LA:
%    Say that its Solaris?} In such situations, another stream
% implementation that, like \myverb{BTE\_stream\_mmb} streams,
% have the potential of exploiting large sized blocks and
% buffers is needed.

% The \myverb{BTE\_ufs} stream implementation simulates the
% \myverb{BTE\_stream\_mmb} stream implementation: Whenever the latter
% maps in a new block, the former reads in a new block (via
% \myverb{read()}) and whenever the latter unmaps a block, the
% latter attains the same result via a \myverb{write()} call.
% But the \myverb{BTE\_ufs} implementation involves explicitly
% keeping track in the BTE code various things which are
% ``under the hood'' in \myverb{mmap()}
% implementations.\comment{LA: Like what?} In fact, the code
% in the \myverb{BTE\_ufs} implementation can be said to
% amount to a (very rudimentary) \myverb{mmap()}
% implementation. 

Like in the case of \myverb{BTE\_stream\_mmb} the
(logical) block size can be controlled using the macro
\myverb{BTE\_UFS\_LOGICAL\_BLOCKSIZE\_FACTOR}.
\myverb{BTE\_stream\_ufs} also has the same advantage as
\myverb{BTE\_stream\_mmb} over \myverb{BTE\_stdio} of only incurring
one kernel call per block. However, unlike \myverb{BTE\_stream\_mmb}
(but like \myverb{BTE\_stdio}), prefetching is done
implicitly by the filesystem underlying TPIE (and objects
have to pass through kernel level buffer space)
%In fact, nowadays, in the case of sequential accesses, most filesystems
%almost surely implement readahead prefetching, which should suffice for the
%purpose of streaming operations in TPIE. (In the case of non-sequential
%acceses, the next block to be accessed is more often than not dependent on
%the processing of the contents of the current block, so prefetching is
%difficult to implement or impossible.)
In \myverb{BTE\_ufs} streams, when the asynchronous I/O library
\myverb{libaio}\index{libaio library@{\tt libaio} library.} is available,
there is a provision to do (user-level) prefetching within \myverb{BTE\_ufs}
streams but we do not recommend its use on account of the implicit
filesystem prefetching.

\myverb{BTE\_stream\_ufs} class inherits from
\myverb{BTE\_base\_stream}:\comment{LA: Why called
   BTE\_single\_disk and not BTE\_stream\_ufs?? DH: Yes,
   this documentation is wrong, but the code should be fixed!!!}
\begin{verbatim}
class BTE_single_disk : public BTE_base_stream { 
  private:
     // descriptor of the mapped file.  
     int fd;
     // A pointer to the mapped in header block for
     the stream.  mmap_stream_header *header; ...  }
\end{verbatim}

\myverb{BTE\_stream\_ufs} defines the public member
functions defined for \myverb{BTE\_base\_stream} (please see
Section~\ref{sec:imp-bte-base} for a list of these member
functions and their semantics).  In addition to these it
defines its own constructors: 
\index{BTE_stream_ufs@{\tt BTE\_stream\_ufs}!constructors}
\comment{LA: Why a special substream constructer?}
\begin{verbatim}
  BTE_single_disk(const char *dev_path, BTE_stream_type st); 
  BTE_single_disk(BTE_stream_type st); 
  BTE_single_disk(BTE_single_disk<T> &s); 
  
  // A substream constructor.
  BTE_single_disk(BTE_single_disk *super_stream,
                 BTE_stream_type st,
                 off_t sub_begin, off_t sub_end);
\end{verbatim}
The \myverb{BTE\_ufs} header structure as follows:
\index{BTE\_stream\_ufs@{\tt BTE\_stream\_ufs}!header}\index{streams!header block}
\begin{verbatim}
struct mmap_stream_header { 
  public:
    unsigned int magic_number;  // Set to UFS_HEADER_MAGIC_NUMBER
    unsigned int version;       // Should be 1 for current version.
    unsigned int length;        // # of bytes in this structure.
    off_t item_logical_eof;     // The number of items in the stream.
    size_t item_size;           // The size of each item in the stream.
    size_t block_size;          // The size of a physical block on the device
                                // where this stream resides.
    unsigned int items_per_block;
};
\end{verbatim}


\index{BTE_ufs@{\tt BTE\_ufs}|)}
\comment{LA: Discuss implementation!}


\section{The Memory Manager (MM)}
\plabel{sec:ref-mm}
\index{memory manager|(}
\index{MM|see{memory manager}}
\index{MM_manager@{\tt MM\_manager}|see{memory manager}}

The Memory Manager components of TPIE provide services related to
the management of internal memory:
\begin{itemize}
\item allocation and deallocation of (internal)
memory as requested by the \myverb{new} and \myverb{delete} operators,
\item accounting of memory usage (when required),
\item enforcing of the user-specified internal memory usage
limit (when required),
\item logging of memory allocation requests (when required).
\end{itemize}

In version \version of TPIE, the memory manager
\myverb{MM\_manager}, is built from the source files
\myverb{mm\_register.cpp}, \myverb{mm\_base.cpp},
\myverb{mm\_register.h}, and \myverb{mm\_base.h}.  The memory
manager traps memory allocation and deallocation requests in
order to monitor and enforce memory usage limits. It
provides a number of user-callable functions and services,
which are documented in Chapter \ref{cha:reference}.

\tobewritten

\comment{LA: There is some text in here that I commented out.}

%The MM is the layer of TPIE that sits between the AMI interface and the
%BTE.  Its primary role is managing main memory, including memory that
%may be distributed across multiple physical machines.  The performance
%of many of the AMI stream operations, such as sorting, permuting,
%merging, and distribution depend critically on the efficient use of
%main memory.  The first thing the MM will have to do to achieve this
%is bypass the virtual memory system provided by UNIX and related
%operating systems.  The second thing it has to do is bypass the
%traditional UNIX buffer cache and take charge of managing the blocks
%of data provided by the BTE.  In some cases, operating system kernels
%will have to be modified in order for the MM to do its job.  In modern
%micro-kernel operating systems, however, the MM may be able to operate
%entirely as a user level process.

%In multiple CPU environments, the job of the MM will be complicated by
%the need to manage multiple banks of memory.  In tightly coupled
%homogeneous parallel environments, this task is likely to be made far
%simpler by existing hardware and operating system support.  In
%distributed, and in particular in heterogeneous environments, the MM
%will have to work with various network protocols and drivers to
%accomplish its task.

(Need some comments on the current simple MM that we have and some OS issues
that come up in attempting to make it more robust.)

\index{memory manager|)}


\section{The Access Method Interface (AMI)}
\index{access method interface|(}
\index{AMI|see{access method interface}}
\plabel{sec:imp-ami}

The AMI-level entry points provided by TPIE are documented
in the Reference section of this manual (see
Section~\ref{sec:ref-ami}), and a number of examples of
their use are given in the Tutorial section (see
Section~\ref{ch:tutorial}).  In this section we examine the
TPIE source files which compose the AMI and provide a brief
discussion of their purpose and relationship to to each
other. We also discuss the algorithmic decisions that were
made in constructing the various TPIE services such as
creation, scanning, merging, sorting, and permutation of
streams, and the services of the block collection class. The
presentation of this section is organized by TPIE service
(creation of streams, scanning of stream, etc.) and within
each service the relevant source files are itemized
(alphabetically) and discussed. The index of this manual
provides a more direct way to find the documentation for a
particular source file.

\subsection{Using Multiple BTE Implementations}
\plabel{sec:imp-multi-imp}


For instance, if a single implementation is needed, it is
sufficient for the application code to create a stream of
\noiv{int} as follows:
\begin{verbatim}
AMI_STREAM<int> aStream;
\end{verbatim}
However, if different implementations, say a \myv{ufs} and
an \myv{mmb} stream are desired, the code would be similar
to the following:
\begin{verbatim}
    BTE_stream_mmb   firstStream( const char *dev_path, BTE_stream_type st );
    BTE_stream_stdio secondStream( const char *dev_path, BTE_stream_type st );

\end{verbatim}

\subsection{General Considerations}

The following TPIE files are fundamental and therefore
involved in every TPIE program, no matter which TPIE
services are accessed:\comment{DH: There may be a better
   place for some of the stuff in this intro}

\begin{itemize}
    \item \myv{include/ami.h}: This file should be included
    (at compile time) in every TPIE application program that
    uses the AMI-level interface. It in turn inputs the
    definitions for the AMI-level services of TPIE. The
    files input by \myv{include/ami.h} are itemized below.

    \item \myv{test/app\_config.h}: This file contains TPIE
    flags and settings that can be customized to an
    individual application. The options available in this
    file are described in detail in
    Section~\ref{sec:tun-appconfig}.

    \item \myv{include/config.h}: This file contains flags and
    indicators that describe the machine and operating
    system on which TPIE is currently running. It is
    generated automatically by the TPIE installation
    process and is not intended to be modified.
    
    \item \myv{lib/libtpie.a}: This is the TPIE load
    library. It contains code for the memory manager
    \myv{MM\_manager}, TPIE logging, BTE statistics, and a
    small number of other services. While most TPIE code is
    in the form of templates, which generate code at
    compile-time, application programs must also link with
    \myv{lib/libtpie.a}. This is described further in
    Section~\ref{sec:tut-compiling}.
\end{itemize} 

\subsubsection{File: \myverb{include/ami.h}} This C++ include
file inputs the various include files required for
compilation of an application program with TPIE. The relevant portion
of \myverb{include/ami.h} is included below:

\begin{verbatim}
// Get the base class, enums, etc...
#include <ami_base.h>

// Get the device description class
#include <ami_device.h>

// Get an implementation definition
#include <ami_imps.h>

// Get templates for ami_scan().
#include <ami_scan.h>

// Get templates for ami_merge().
#include <ami_merge.h>

// Get templates for ami_sort().
#include <ami_sort.h>

// Get templates for general permutation.
#include <ami_gen_perm.h>

// Get templates for bit permuting.
#include <ami_bit_permute.h>
\end{verbatim} 

Each of these files is discussed in the sections which
follow.

\subsection{Creation of Streams}

AMI stream objects are created in a TPIE program via the
\myv{AMI\_STREAM} keyword.  \myverb{AMI\_STREAM} is a macro
that resolves to a class template invocation appropriate to
the declared target I/O architecture. This involves the
declared AMI implementation (see Section~\ref{sec:ref-ami}),
BTE implementation\index{block transfer engine} (see
Section~\ref{sec:ref-bte}) and whether one or multiple disks
are used.  In the current version of TPIE an
\myverb{AMI\_STREAM} is stored in a standard UNIX file on a
single disk. For most applications, an \myv{AMI\_STREAM}
will have type \myv{AMI\_stream\_single}, which is the TPIE
base class for a stream backed by a single file (normally on
a single disk). The string \myverb{AMI\_STREAM} is
\myverb{\#define}'d to be the string
\myverb{AMI\_stream\_single} in the header file
\myverb{ami\_imps.h}. This arrangement is intended to permit
alternative implementations of \myverb{AMI\_STREAM} if
necessary in the future.

The TPIE code associated with the creation and manipulation
of stream objects is contained mainly within the following
files:

\begin{itemize}
    \item \myv{include/ami\_base.h}: 
    \begin{enumerate}     
        \item defines the \myv{AMI\_err} codes returned by
        the AMI-level services (these and their meanings are
        listed in Appendix~\ref{sec:ami-errors}.

        \item defines codes for the AMI stream types
        (\myv{AMI\_READ\_STREAM}, \myv{AMI\_WRITE\_STREAM},
        \myv{AMI\_APPEND\_STREAM},
        \myv{AMI\_READ\_WRITE\_STREAM}.
  
        \item defines the member functions of
        \myv{AMI\_base\_stream} as \myv{virtual} if compile
        time variable \myv{AMI\_VIRTUAL\_BASE} is defined.
    \end{enumerate}    
    
    \item \myv{include/ami\_single.h}: This include file
    defines the class \myv{AMI\_stream\_single}, which is the
    base class for all AMI-level streams backed by a single
    Unix file.
    \begin{enumerate}

        \item the template parameter \noiverb{<T>}
        represents the type of the application data element
        in the stream.
        
        \item At construction time, an
        \myv{AMI\_stream\_single<T>} is mapped onto a
        \myv{BTE\_STREAM<T>} and the AMI stream type is
        mapped as follows:
        \begin{itemize}
            \item \myv{AMI\_READ\_STREAM} is mapped to
            \myv{BTE\_READ\_STREAM}
            \item \myv{AMI\_APPEND\_STREAM} is mapped to
            \myv{BTE\_APPEND\_STREAM}
            \item \myv{AMI\_WRITE\_STREAM} and
            \myv{AMI\_READ\_WRITE\_STREAM} are mapped to
            \myv{BTE\_WRITE\_STREAM}
        \end{itemize}
        
        \item The AMI-level services for streams are
        implemented by corresponding BTE-level services. The
        AMI member functions described in
        Sections~\ref{sec:ref-ami-stream} (see examples of
        use of some of these in Section~\ref{sec:tut-samplepgm})
        are implemented by calls to the BTE-level functions
        documented in Section~\ref{sec:ref-bte}. An AMI
        stream has little internal context information as a
        result. Two exceptions are the following:
        \begin{itemize}
            \item \myv{r\_only}: this flag is one (TRUE) if
            the stream is only readable, and not writeable,
            corresponding to the status
            \myv{AMI\_READ\_STREAM}. Maintaining this flag
            allows the AMI stream member functions to catch
            erroneous attempt by application-level code to
            write to a stream whose underlying file was
            opened for reading.
            
            \item \myv{destruct\_bte}: this flag is one
            (TRUE) if the destructor of the underlying BTE
            stream should be called by the AMI-level
            destructor (the normal situation). If the
            AMI-level stream was constructed from a
            pre-existing BTE stream via the constructor
\begin{verbatim}
    AMI_stream_single(BTE_STREAM<T> *bs);
\end{verbatim}
            then a \myv{destruct\_bte} value of zero is used
            to prevent the destructor from deleting the BTE
            stream (owned by another AMI stream).
        \end{itemize}
    \end{enumerate}

\end{itemize}


% The public member functions of the \myverb{AMI\_STREAM}
% class are defined in the file ``ami\_single.h''. Some
% important related enumerated types that the user will be
% concerned with are defined in ``ami\_base.h''.
% \myverb{AMI\_STREAM} is a macro that resolves to a class
% template declared to match the underlying semantics of the
% target I/O architecture by using an appropriate AMI
% implementation (see section~\ref{sec:ref-ami}), BTE
% implementation\index{block transfer engine} (see
% Section~\ref{sec:ref-bte}) and MM (see
% Section~\ref{sec:ref-mm}) implementation. 
\comment{LA: Talk about streams ect (built on top of
   BTE\_STREAM.) DH: Done}
\comment{LA: Remember to include something about r\_only
   flag somewhere. DH: Done}
\comment{LA: There is a lot of text in here I have commented out}

%To use TPIE, the programmer has to create, modify or delete items each of
%which belongs to a stream. From the programmer's perspective,
%computation simply requires her to \emph{choreograph the movement of
%items in various streams}: The I/O operations, the buffering required
%and memory management are performed in TPIE itself. In order to specify
%the choreography of streams, the user can make heavy use of inbuilt
%TPIE templates and functions; it is not often that the user has to write
%special-purpose custom-made TPIE functions. The reason why TPIE's unique
%approach is attractive for external memory computation is the well
%known fact that most external memory algorithms consist of sewing
%together a small number of paradigms such as \emphd{scanning},
%\emphd{merging}, \emphd{distributing}, in a well-coordinated manner
%exploiting all available memory and I/O bandwidth. A large variety of 
%indexing data structures and methods related to those data structures 
%can be implemented using the above paradigms in conjunction with a 
%basic tree template: While indexing data structures have been
%implemented using TPIE streams, work on tree templates and related
%methods to further ease implementation of indexing data structures is
%currently ongoing.

% Each datum in TPIE is an item of a certain type.  The
% programmer normally manipulates the data via the services of
% the \emphd{Access Method Interface (AMI)}, which accesses
% data elements stored in \myverb{AMI\_STREAM}s.  An
% \myverb{AMI\_STREAM} can be viewed as simply a wrapper
% around a \myverb{BTE\_STREAM} (See
% Section~\ref{BTEinterface} for a discussion of the BTE
% interface).

%as a \myverb{BTE\_STREAM}, where BTE stands for \emphd{block transfer
%engine.} Each \myverb{BTE\_STREAM} is a stream built on top
%of a Unix file, with  
%additional features such as typing,  and automatic, efficient,
% I/O and buffering so that the programmer does
%not bear this burden. TPIE can be configured to use one of three
%different \myverb{BTE\_STREAM} implementations: The implementations vary
%fundamentally in the way they do buffering and I/O. The implementation
%of choice depends, to some extent, on the operating system on which 
%TPIE runs. Later we describe the different BTE implementations and
%how to determine the best implementation on any given system; for now,
%we focus on the \myverb{AMI\_STREAM} interface since that
%is most directly 
%relevant to a programmer/user.

%The member functions of the \myverb{AMI\_STREAM} class,
%templated on the type T of its items,  are defined in the file \myverb{ami\_single.h};
%some important related enumerated types that the user will be
% concerned with have been defined in \myverb{ami\_base.h}.

%An important consequence of the construction of an \myverb{AMI\_STREAM} is a
%reduction in the total amount of memory that the user has at her
%disposal since a certain amount of memory is consumed now by buffers
%dedicated to that \myverb{AMI\_STREAM} and its related data
%structures.\footnote{The total amount of memory that the user has
%available, divided by the memory consumed by every AMI\_STREAM thus
%is an upper bound on the maximum number of AMI\_STREAMs that the user
%can have.}
 
%In order to effectively utilize memory, the user needs to keep track
%of the amount of memory she has at her disposal: Sometimes, 
%when one of several algorithms may be usable to solve a given problem,
%it may be useful to choose the algorithm depending on the problem
%size and available memory size. The total amount of memory consumed
%by an \myverb{AMI\_STREAM} often plays an important role while making such 
%a choice. The memory manager \myverb{mm.h} is responsible for keeping
%track of the amount of memory avaoable to the user at any time.

%When an \myverb{AMI\_STREAM} is destroyed using a destructor, that
%amount of memory becomes is added to the available memory for the
%user. 


%Sometimes the
%items in an \myverb{AMI\_STREAM} will never be used  once
%the \myverb{AMI\_STREAM} is 
%destructed whereas sometimes an ``idle''
%\myverb{AMI\_STREAM} may be destroyed  
%to prevent it from hogging memory only to be constructed later on
%when really required. In the former case, it is desirable to 
%destroy or delete the Unix file backing the \myverb{AMI\_STREAM} to save
%disk space whereas in the latter case we need the Unix file backing
%the \myverb{AMI\_STREAM} to exist on disk even when the
%   \myverb{AMI\_STREAM} is 
%destroyed. The  \myverb{AMI\_STREAM} destructor looks at the persistence 
%flag to decide whether it need only close the backing Unix file or
%delete it from disk.


\subsection{Scanning}
\index{scanning|(}
\index{scanning|)}

\tobeextended

\myverb{ami\_scan.h}

\subsection{Merging}
\plabel{sec:imp-ami-merge}
\plabel{sec:imp-ami-generalized-merge}
\index{merging|(}

\tobeextended

\myverb{ami\_merge\_base.h}

%\subsubsection{Merge Management Objects}
%
%Merge management objects based on the super class \myverb{AMI\_merge\_base} 
%defined in the file \myverb{ami\_merge.h} can be used conveniently by
%appropriately defining the member functions of the class.

\subsubsection{Generalized Merging: using Merge Management Objects}

\tobeextended

\myverb{ami\_merge.h}\comment{LA: Remember something about ami\_merge really
called ami\_single\_merge.}

\subsubsection{Specialized Merging: without Merge Management Objects}

\tobeextended

The \myverb{AMI\_merge()} polymorphs (defined in file
\myverb{ami\_optimized\_merge.h}) work without a merge
management object and perform standard (total
order/comparison based) merging. The elimination of the
merge management object object results in one less level of
function calls and thus in improved efficiency over a
similar comparison based merging based on a merge management
object. The merge is controlled by a priority queue (defined
in file \myverb{include/mergeheap.h}) which exploits the
fact that \myverb{delete-min} and \myverb{insert} are always
performed together. This improves efficiency. In
 \myverb{AMI\_key\_merge()} the heap stores the
key and not the entire object. When the object size is large
compared to the key size, this often leads to further
performance improvement.\comment{LA: Extend this stuff so it
   becomes readable}

\subsubsection{Partition and Merge}

\tobeextended

\myverb{AMI\_partition\_and\_merge()} repeatedly merges together the maximum
possible number of sub-streams using \myverb{AMI\_merge()}. An important point
is that the substreams input to \myverb{AMI\_merge()} during the execution of
\myverb{AMI\_partition\_and\_merge()} all originate from the same underlying
parent \myverb{AMI\_STREAM}; thus filesystem accesses are inherently
non-sequential. Throughout the execution of
\myverb{AMI\_partition\_and\_merge()}, there are only two active
\myverb{AMI\_STREAM}s at any time: One which stores the substreams being
merged at that stage and one which stores the substreams output by that
stage.\comment{LA: Extend this stuff so that it becomes readable!}


\subsection{Comparison Sorting}
\plabel{sec:ref-imp-ami-sort}
\index{sorting!comparison|(}


The TPIE source files involved in merge sorting include the following:

\begin{enumerate}

    \item \myv{include/ami\_sort.h}: This file simply
    includes several sorting-related files in the compilation:
    \begin{itemize}
        \item \myv{include/ami\_sort\_single.h}
        \item \myv{include/ami\_optimized\_sort.h}
        \item \myv{include/ami\_sort\_single\_dh.h}
    \end{itemize}
    \noindent
     These files are discussed briefly below.
     
     \item \myv{include/ami\_sort\_single.h}: This file
     contains a number of templates for outdated (version 1)
     sort routines:
    \begin{itemize}
        \item 
\begin{verbatim}
AMI_err AMI_sort_V1 ( AMI_STREAM<T> *instream, 
                      AMI_STREAM<T> *outstream ); 
\end{verbatim}
        \item 
\begin{verbatim}
AMI_err AMI_sort_V1 ( AMI_STREAM<T> *instream, 
                      AMI_STREAM<T> *outstream, 
                      int (*cmp)(CONST T&, CONST T&) );
\end{verbatim}

        \item 
\begin{verbatim}
AMI_err AMI_sort_V1 ( AMI_STREAM<T> *instream, 
                      AMI_STREAM<T> *outstream, 
                      CMPR *cmp)
\end{verbatim}
    \end{itemize}
    Except for their name (\myv{AMI\_sort\_V1}) these
    routines have the same syntax as the version 2 sort
    routines \myv{AMI\_sort} (see
    Section~\ref{sec:ref-ami-sort} for details). The version
    1 routines are generally slower than the version 2
    routines. They are included in TPIE for historical
    comparison purposes.
    The current versions (version 2) of these routines can
    be found in \myv{include/ami\_sort\_single\_dh.h}).

    
    The version 1 merge sort polymorphs of
    \myverb{AMI\_sort\_V1()} are implemented using
    \myverb{AMI\_partition\_and\_merge()} and a merge
    management object. The \myverb{merge\_sort\_manager}
    class, a base class for merge management objects needed in
    \myverb{AMI\_sort\_V1()}, is also defined in file
    \myv{include/ami\_sort\_single.h}.  The member functions
    \myverb{main\_mem\_operate()}, \myverb{initialize()},
    and \myverb{operate()} of the merge management object
    provide the sorting-specific details required for merge
    sorting:
    \begin{itemize}
        \item \myverb{main\_mem\_operate()} is simply an
        in-memory sorting algorithm based on quicksort (in
        file \myverb{quicksort.h}).
        
        \item Member function \myverb{operate()} selects the
        next output element during the merging of runs using
        a standard priority queue (in file
        \myverb{pqueue\_heap.h}).
        
        \item Member function \myverb{initialize()} just
        initializes this priority queue.\comment{LA: extend/rewrite?}
    \end{itemize}
    
    \item \myv{include/ami\_optimized\_sort.h}: Contains
    beta test versions of some of the sorting methods now
    accessed via file \myv{include/apm\_dh.h} (see below for
    details).

    \item \myv{include/apm\_dh.h}: Described in detail below.

    \item \myv{include/ami\_sort\_single\_dh.h}: Described in detail below.

\end{enumerate}

\subsubsection{File: \myv{include/ami\_sort\_single\_dh.h}}
     
This file contains the templates for the TPIE sort routines,
as described in Section~\ref{sec:ref-ami-sort}.  The
\myv{AMI\_sort} and \myv{AMI\_ptr\_sort} templates are:
\begin{verbatim}
AMI_err AMI_sort     ( AMI_STREAM<T> *, AMI_STREAM<T> *, sort_manager & );
AMI_err AMI_ptr_sort ( AMI_STREAM<T> *, AMI_STREAM<T> *, sort_manager & );
\end{verbatim}

The \myv{AMI\_key\_sort} template is:
\begin{verbatim}
AMI_err AMI_key_sort ( AMI_STREAM<T> *, AMI_STREAM<T> *, KEY, sort_manager & );
\end{verbatim}

Please refer to Section~\ref{sec:ref-ami-sort} for details
of these templates.
In all cases the first two arguments are input and output
streams. The \myv{sort\_manager} object is customized for
the particular sorting options implied by the template that
is used. All of these templates invoke
\myv{AMI\_partition\_and\_merge\_dh} (see file
\myv{include/apm\_dh.h} below for details) and pass their
customized sort management object to indicate the specific
details or options required.

   Also in this file are the following (templated) class definitions:

    \begin{itemize}
        
        \item class \myv{sort\_manager} is a base class for
        sort management objects. It is inherited by the
        following sort management classes, which are also
        defined in \myv{include/ami\_sort\_single\_dh.h}:

        \item \myv{sort\_manager\_op}: used by the operator
        variant of the \myv{AMI\_sort} and \myv{AMI\_ptr\_sort} templates.
        
        \item \myv{sort\_manager\_obj}: used by the
        comparison object variant of the \myv{AMI\_sort}
        and \myv{AMI\_ptr\_sort} templates.\comment{DH:
           ``comparison object'' should be called
           ``comparison management object''}
        
        \item \myv{sort\_manager\_cmp}: used by the
        comparison function variant of the \myv{AMI\_sort} and
        \myv{AMI\_ptr\_sort} templates.

%        \item \myv{sort\_manager\_kop}: used by the operator
%        variant of the \myv{AMI\_ptr\_sort} template.
%
%        \item \myv{sort\_manager\_kobj}: used by the
%        comparison function variant of the
%        \myv{AMI\_ptr\_sort} template.
        
        \item \myv{sort\_manager\_kcmp}: used by the
        \myv{AMI\_key\_sort} template (which uses a
        comparison object).
    \end{itemize}

All \myverb{sort\_manager} objects contain the following
member functions:
\begin{itemize}
    \item member function \myv{sort\_fits\_in\_memory}
    returns TRUE if there is sufficient internal memory to
    sort the input stream without external merging.

    \item member function \myv{main\_mem\_operate\_init}
    initializes any internal memory data structures needed
    for internal sorting (e.g. an array of pointers in the
    case of \myv{AMI\_ptr\_sort} and an array of keys in the
    case
    of \myv{AMI\_key\_sort}).

    \item member function \myv{main\_mem\_operate} performs
    an internal sort of the appropriate type (e.g. quicksort
    of records, quicksort of pointers to records, or
    quicksort of keys)

    \item member function \myv{main\_mem\_operate\_cleanup}
    cleans up after \myv{main\_mem\_operate} (e.g. frees the
    data structures used after an internal memory sort).

    \item member function \myv{single\_merge} merges a set
    of input streams into an output stream. This resolves
    to a call to \myv{AMI\_single\_merge\_dh} in all
    cases. However the merge heap object used in that call
    is customized according to which sort template was
    invoked by
    the application programmer.

\end{itemize}

All \myverb{sort\_manager} objects also contain a
\emphd{merge heap} object, which is custom-chosen by the
particular sort template selected by the application
programmer. The merge heap classes are defined in file
\myv{include/mergeheap\_dh.h}. The following table shows the
class of sort management object and merge heap object used
for each sorting template and variant:
\begin{figure}
\begin{center}
\begin{minipage}[hb]{1.0\linewidth}
\raggedright
\centering{
\begin{tabular}{l|p{2in}|l|l}
\hline
Sorting  & Comparison & Class of Sort & Class of Merge \\
Template & Variant    & Mgmt Object   & Mgmt Object    \\
\hline
\myv{AMI\_sort} & operator & \myv{sort\_manager\_op} & \myv{merge\_heap\_dh\_op} \\
\myv{AMI\_sort} & object   & \myv{sort\_manager\_obj} & \myv{merge\_heap\_dh\_obj} \\
\myv{AMI\_sort} & function & \myv{sort\_manager\_cmp} & \myv{merge\_heap\_dh\_cmp} \\
\myv{AMI\_ptr\_sort} & operator & \myv{sort\_manager\_op} & \myv{merge\_heap\_pdh\_op} \\
\myv{AMI\_ptr\_sort} & object   & \myv{sort\_manager\_obj} & \myv{merge\_heap\_pdh\_obj} \\
\myv{AMI\_ptr\_sort} & function & \myv{sort\_manager\_cmp} &
\myv{merge\_heap\_pdh\_cmp} \\
\myv{AMI\_key\_sort} &   & \myv{sort\_manager\_kobj} & \myv{merge\_heap\_dh\_kobj} \\
\hline
\end{tabular}}
\caption{\plabel{fig:imp-sort-variants} Customization of Arguments
   to \myv{AMI\_partition\_and\_merge\_dh}.}
\end{minipage}
\end{center}
\end{figure}


\subsubsection{File: \myv{include/apm\_dh.h}}

This file contains code for the following templates:
\begin{itemize}
    
    \item
\begin{verbatim}
template < class T, class M>
AMI_err AMI_single_merge_dh ( AMI_STREAM <T> **instreams,
                              arity_t arity,
                              AMI_STREAM <T> *outstream.
                              M MergeHeap );
\end{verbatim}
    Function: Merge an array \noiv{*instreams} of \noiv{arity} input
    streams using the services of merge heap object
    \noiv{Mergeheap}, to create the output stream
    \noiv{*outstream}.
    
    \item
\begin{verbatim}
template < class T, class M>
AMI_err AMI_partition_and_merge_dh ( AMI_STREAM <T> *instream,
                                     AMI_STREAM <T> *outstream.
                                     M mgmt_obj );
\end{verbatim}
    Function: Cut the input stream \noiv{*instream} into pieces that
    fit into memory, \myv{operate} on each of them (e.g.sort
    them) internally, and recursively merge them as memory
    permits until only a single output stream
    \noiv{*outstream} remains.
\end{itemize}

\noindent
Template \myv{AMI\_partition\_and\_merge\_dh} is invoked directly by
the user-callable sort templates. Its sort management object
\noiv{mgmt\_obj} contains the following:
\begin{itemize}
    \item a merge heap object, selected by the particular
    sorting template invoked by the application programmer,
    and instantiated from one of the classes defined in file
    \myv{include/mergeheap\_dh.h}.
    
    \item member function \myv{sort\_fits\_in\_memory}
    returns TRUE if there is sufficient internal memory to
    sort the input stream without external merging.

    \item member function \myv{main\_mem\_operate\_init}
    initializes any internal memory data structures needed
    for internal sorting (e.g. an array of pointers in the
    case of \myv{AMI\_ptr\_sort} and an array of keys in the case
    of \myv{AMI\_key\_sort}).

    \item member function \myv{main\_mem\_operate} performs
    an internal sort of the appropriate type (e.g. quicksort
    of records, quicksort of pointers to records, or
    quicksort of keys)
    
    \item member function \myv{main\_mem\_operate\_cleanup}
    cleans up after \myv{main\_mem\_operate} (e.g. frees the
    data structures used after an internal memory sort).

    \item member function \myv{single\_merge} merges a set
    of input streams into an output stream. This resolves
    to a call to \myv{AMI\_single\_merge\_dh} in all
    cases. However the merge heap object used in that call
    is customized according to which sort template was invoked by
    the application programmer.
  
\end{itemize} 

The merge sorting algorithm used by
\myv{AMI\_partition\_and\_merge\_dh} is outlined below:

\step{1.} {If \myv{sort\_fits\_in\_memory}
    \step{(a.)} { \myv{main\_mem\_operate\_init} }
    \step{(b.)} { \myv{main\_mem\_operate} }
    \step{(c.)} { \myv{main\_mem\_operate\_cleanup} }}
\step{}{else
    \step{2.}{ Partition the input into memoryloads, sort
    each one using 
    \step{(a.)} { \myv{main\_mem\_operate\_init} }
    \step{(b.)} { \myv{main\_mem\_operate} }
    \step{(c.)} { \myv{main\_mem\_operate\_cleanup} }
    and write them out into separate streams.
    \step{3.} { If sorting can be completed by a single
    pass of external merging, then do so using
       \step{(a.)} { \myv{single\_merge} }}
    \step{4.} { Recursively merge as many streams as memory
    constraints allow, using
       \step{(a.)} { \myv{single\_merge} }
    until only a single output stream remains.}}}

One important difference between the \myverb{AMI\_sort\_V1()}
and the \myverb{AMI\_sort()}, \myverb{AMI\_ptr\_sort()}, and
\myverb{AMI\_key\_sort()} implementations is in the way they
store the (sub-) streams to be merged after the initial
in-memory sorting is performed. The number, say $R$, of
streams (files) being merged together at any time is the maximum
possible for the amount of memory available. In contrast to
the \myverb{AMI\_sort\_V1()} implementations which only have
two streams (files) active at any time with the $R$ substreams
stored 'inside' them, the newer implementations have $R$
files active. While the $R$ sub-streams sent
for merging by the \myverb{AMI\_sort()} function all have
the same parent stream (file), the $R$ sub-streams sent for merging
by the new implementations all reside in different parent
streams (files). Thus, each stream (file) involved is accessed in a
sequential manner. This can greatly improve performance,
apparently because the operating system can perform
read-ahead more effectively on the individual
streams (files).\comment{LA: This section really needs to be
   rewritten. DH: Ok now?}


% One final difference between the two sorting algorithm
% implementation is that during the initial main memory sort
% the \myverb{AMI\_optimized\_sort()} polymorph which relies
% on the objects having a special key field first extracts the
% keys from the objects and sorts them to obtain the order of
% the objects in memory. The full objects are then permuted to
% reflect this order. Again this result in improved efficiency
% if the keys are much smaller than the objects.

% As the \myverb{AMI\_sort()} polymorphs are based on
% \myverb{AMI\_partition\_and\_merge()} which in turn are
% based on \myverb{AMI\_merge()} and a merge management
% object, the \myverb{AMI\_optimized\_sort()} polymorphs are
% based on the \myverb{AMI\_merge()} polymorphs that do not
% use a standard (as described in
% Section~\ref{sec:ref-ami-merge}) TPIE merge management
% object. Thus \myverb{AMI\_optimized\_sort()} benefits from
% the efficiency improvement obtained in
% \myverb{AMI\_optimized\_merge()} relative to
% \myverb{AMI\_merge()}. \myverb{AMI\_optimized\_sort()} is a
% merge sort implemented along the lines of
% \myverb{AMI\_partition\_and\_merge()} (in fact the
% implementation of \myverb{AMI\_optimized\_sort()} really is
% an \myverb{AMI\_partition\_and\_merge()} polymorphs defined
% in the file \myverb{ami\_optimized\_merge.h}).

\subsubsection{File: \myv{include/mergeheap\_dh.h}}

This file defines the following classes:
\begin{itemize}
    \item \myv{merge\_heap\_dh\_op}
    \item \myv{merge\_heap\_dh\_cmp}
    \item \myv{merge\_heap\_dh\_obj}
    \item \myv{merge\_heap\_pdh\_op}
    \item \myv{merge\_heap\_pdh\_cmp}
    \item \myv{merge\_heap\_pdh\_obj}
    \item \myv{merge\_heap\_dh\_kobj}
\end{itemize}

Each of the above classes implements the following member
functions:
\begin{itemize}
    \item \myv{allocate}: Allocates space for the heap.
    \item \myv{insert}: Inserts (copies) an element into
    the heap array.
    \item \myv{deallocate}: Deallocates the space used by the heap.
    \item \myv{initialize}: Makes the heap array into a heap
    by repeatedly calling internal member function
    \myv{Heapify}.\footnote{%
       See \cite{cormen:introduction} for more details if
       required.}
    \item \myv{get\_min\_run\_id}: Returns the number of the
    run which has the smallest element in the heap.
    \item \myv{delete\_min\_and\_insert}: Replaces the
    smallest item in the heap with a new element  and
    re-heapifies the heap.
    \item \myv{sizeofheap}: Returns the number of elements
    in the heap.
\end{itemize}

\comment{LA: Knuth forecasting stuff? }

\index{sorting!comparison|)}


\subsection{Distribution}
\plabel{sec:ref-imp-ami-distribution}

\tobewritten

\index{Distribution}

\subsection{Key Bucket Sorting}
\plabel{sec:ref-imp-ami-kb-sort}

\tobewritten

\index{sorting!key bucket|(}
\index{sorting!key bucket|)}

\subsection{General Permuting}
\plabel{sec:ref-imp-ami-gp}

\tobewritten

\index{permutation!general|(}
\index{permutation!general|)}

\subsection{Bit Permuting}
\plabel{sec:ref-imp-ami-bp}

\tobewritten

\index{permutation!bit|(}
\index{permutation!bit|)}

\subsection{Dense Matrices}
\plabel{sec:ref-imp-ami-matrix}

\tobewritten

\index{matrices!dense|(}
\index{matrices!dense|)}

\subsection{Sparse Matrices}
\plabel{sec:ref-imp-ami-sm}

\tobewritten

\index{matrices!sparse|(}
\index{matrices!sparse|)}

\subsection{Stacks}
\plabel{sec:ref-imp-ami-stack}

\tobewritten

\index{stacks|(}
\index{stacks|)}

\subsection{Elementwise Arithmetic}
\plabel{sec:ref-imp-ami-arith}

\tobewritten

\index{elementwise arithmetic|(}
\index{elementwise arithmetic|)}

\index{access method interface|)}


%%% Local Variables: 
%%% mode: latex
%%% TeX-master: t
%%% End: 





