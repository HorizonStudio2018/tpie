%%
%% $Id: applications.tex,v 1.4 1999-06-30 22:41:20 large Exp $
%%
\chapter{Test and Sample Applications}

\section{General Structure and Operation}

The test and sample applications distributed with TPIE are in the
\verb|test| directory.  The test programs are designed primarily to
test the operation of the system to verify that it has been installed
correctly and is as bug free as possible.  These applications all have
names of the form \verb|test_*|.  The sample applications are designed
to demonstrate the use of TPIE in the solution of non-trivial
problems.

The test and sample applications all share a small amount of common
initialization and argument parsing code.  They all include the header file
\verb|app_config.h|, which selects a particular implementation of streams
at the AMI and BTE levels. They also all use the same argument parsing
function \verb|parse_args()|, which parses certain default arguments and
then uses a callback function for arguments specific to one particular
application.

Much of the functionality provided by the common initialization and
argument passing code is intended to eventually be subsumed by
operating system provided services.  For example, the amount of main
memory a particular application is permitted to use can be set via a
command line argument.  It is up to the user to be sure that this
number is reasonable and does not exceed the true amount of main
memory available to the application.  In the future, it is hoped that
this information will be provided by the operating system.

\verb|parse_args()| is declared as follows:

\begin{verbatim}
    parse_args();
\end{verbatim}


The following is a summary of the common command line arguments that
are parsed by \verb|parse_args()|.
\begin{description}
\item[\verb|-t testsize|]
Set the size of the test to be run to \verb|testsize|.  Typically this
is the number of objects to be put into the application produced input
stream.  In matrix tests, however, it is the number of rows and
columns is the test matrices.  If this argument is not passed, then
the default value of 8 Meg is used.
\item[\verb|-m memsize|]
The number of bytes of main memory that the application is permitted to
use.  The MM\index{memory manager} will ensure that no more than this amount is
used.  If this option is not specified, then a default value of 2 Mb
is used.
\item[\verb|-z randomseed|]
Seed the random number generator with the value \verb|randomseed|.
This is useful for debugging or testing, when we want several runs of
the application to rely on the same series of pseudo-random numbers.
For applications that do not generate test data randomly, this has no
effect. 
\item[\verb|-v|]
Toggle verbose mode.  When running in verbose mode, report major
actions of the running program to verb|stdout|.
\end{description}

Each application specific argument appears in the string pointed to by
\verb|aso| as a single character, possibly followed by the single
character `\verb|:|', indicating that the argument requires a value.
For example, if \verb|aso| pointed to the string ``\verb|ax:z|'' then the
following command line arguments would all be parsed correctly:
\begin{description}
\item[\verb|-a|]
\item[\verb|-x 123|]
\item[\verb|-a -x 123|]
\item[\verb|-ax123|]
\item[\verb|-x123 -a|]
\end{description}
In each case, \verb|parse_app()| would be called to take some
application specific action for each of the arguments.  It would be
called once with \verb|opt| set to `\verb|a|' and \verb|optarg| set to
\verb|NULL|, and/or once with \verb|opt| set to `\verb|x|' and
\verb|optarg| pointing to the string ``\verb|123|.''  When multiple arguments
are present on the command line, they are parsed from left to right.

The following is an example of how a test application, in this case
\verb|test_ami_sort|, can use application specific command line
arguments to set up it's global state.

\begin{verbatim}
static const char as_opts[] = "R:S:rsao";
void parse_app_opt(char c, char *optarg)
{
    switch (c) {
        case 'R':
            rand_results_filename = optarg;
        case 'r':
            report_results_random = true;
            break;
        case 'S':
            sorted_results_filename = optarg;
        case 's':
            report_results_sorted = true;
            break;
        case 'a':
            sort_again = !sort_again;
            break;
        case 'o':
            use_operator = !use_operator;
            break;
    }
}

int main(int argc, char **argv)
{
    parse_args(argc,argv,as_opts,parse_app_opt);

    ...

    return 0;
}
\end{verbatim}

\section{Test Programs}

The test programs include with TPIE are as follows:\comment{LA: Is this
still correct?}

\begin{description}
\item[\verb|test\_ami\_merge|] Test fixed way merging with direct
  calls to \verb|AMI_merge()|, as described in 
  Section~\ref{sec:ref-ami-merge}.
\item[\verb|test\_ami\_pmerge|] Test many-way merging using 
  \verb|AMI_partition_and_merge()|, as described in 
  Section~\ref{sec:ref-ami-merge}.
\item[\verb|test\_ami\_sort|] Test sorting using \verb|AMI_sort()| as
  described in Section~\ref{sec:ref-ami-sort}.
\item[\verb|test\_ami\_gp|] Test general permutation using
  \verb|AMI_general_permute()| as described in Section~\ref{sec:ref-ami-gp}.
The program generates an input stream
consisting of sequential integers, and outputs a stream consisting of 
the same integers, in reverse order.
\item[\verb|test\_ami\_bp|] Test bit permutations using
  \verb|AMI_BMMC_permute()| as described in
  Section~\ref{sec:ref-ami-bp}. The program generates an input stream
consisting of sequential integers, and outputs a stream consisting of 
a permutation of these integers, as described in the example given in the Tutorial, Section~\ref{sec:bit-permuting}.
\item[\verb|test\_matrix|]
\item[\verb|test\_bit\_matrix|] Test main memory matrix manipulation
  and arithmetic.  This is used both by the bit permuting code
  described in Section~\ref{sec:ref-ami-bp} and the dense matrix
  multiplication code described in Section~\ref{sec:ref-ami-matrix}
  for internal manipulation of sub-matrices of external memory
  matrices.
\item[\verb|test\_ami\_matrix\_pad|] Test padding of external
  matrices.  This is the preprocessing step for the external dense
  matrix multiplication algorithm TPIE uses, which is described in 
  Section~\ref{sec:ref-ami-matrix}. 
\item[\verb|test\_ami\_matrix|] Test external dense matrix arithmetic
  as described in Section~\ref{sec:ref-ami-matrix}.
\item[\verb|test\_ami\_sm|] Test external sparse matrix arithmetic
  as described in Section~\ref{sec:ref-ami-sm}.
\item[\verb|test\_ami\_stack|] Test external memory stacks as
  described in Section~\ref{sec:ref-ami-stack}.
\item[\verb|test\_ami\_arith|] Test element-wise arithmetic on
  external memory streams as described in Section~\ref{sec:ref-ami-arith}.
The program generates an input stream
consisting of sequential integers, squares them, and performs 
elementwise division between the resulting stream and the input stream.
\end{description}

\section{Sample Applications}

The sample applications included with TPIE are as follows:

\begin{description}
\item[\verb|ch2|] Two dimensional convex hull\index{convex hull}
  program using Graham's scan.  It is implemented using a scan
  management object that maintains the upper and lower hull internally
  as external memory stacks.  Much of the code in this application
  appears in Section~\ref{sec:convex-hull}.
\item[\verb|lr|] An implementation of an asymptotically optimal list
  ranking \index{list ranking} algorithm.  The idea of geometrically
  decreasing computation is used.  Much of the code in this
  application appears in Section~\ref{sec:list-ranking}.
\item[\verb|nas\_ep|] An I/O-efficient implementation of the NAS EP
  parallel benchmark.  This benchmark generates pairs of independent
  Gaussian random variates.
\item[\verb|nas\_is|] An I/O-efficient implementation of the NAS IS
  parallel benchmark.  This benchmark sorts integers using one of a
  variety of approaches.
\end{description}

Detailed descriptions of the NAS parallel benchmarks are available
from the \htmladdnormallink{NAS Parallel Benchmark Home Page}%
{http://www.nas.nasa.gov/NAS/NPB/}
\begin{latexonly}
at URL \verb|http://www.nas.nasa.gov/NAS/NPB/|.
\end{latexonly}
